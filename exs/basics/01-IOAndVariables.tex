
\documentclass[12pt]{article}
\usepackage{fontspec}
\usepackage{polyglossia}
\usepackage{amsmath}
\usepackage{xcolor}
\usepackage{fancyhdr}

\pagestyle{fancy}

\setdefaultlanguage{arabic}
\setotherlanguage{english}
\newfontfamily\arabicfont[Script=Arabic]{Amiri}

\title{وظيفة 1 للصف العاشر 10}
\fancyhf{} % clear default
\fancyhead[L]{مدرسة التسامح الشاملة}
\fancyfoot[C]{\thepage}

\begin{document}

\maketitle
\thispagestyle{fancy}

1. اكتب برنامجًا يستقبل من المستخدم أطوال أضلاع مثلث ويطبع له محيط المثلث. \\
(محيط المثلث = مجموع أطوال الأضلاع) \\

2. اكتب برنامجًا يستقبل من المستخدم طول ضلع مربّع ويطبع له مساحة المربّع ومحيطه.
\[
    \text{Area} = side ^ 2
\]
\[
    \text{Perimeter} = 4 \cdot side
\]

3. اكتب برنامجا يستقبل من المستخدم طول قاعدة المثلث والارتفاع ويطبع له مساحة المثلث.
\[
    \text{Area} = width \cdot height
\]


4. اكتب برنامجًا يستقبل من المستخدم طول مستطيل وعرضه ويطبع له مساحة المستطيل ومحيطه.
\[
    \text{Area} = width \cdot height
\]
\[
    \text{Perimeter} = 2 \cdot width + 2 \cdot height
\]

5. اكتب برنامجًا يستقبل من المستخدم طول نصف قطر دائرة، ويطبع له محيطها ومساحتها.
\[
\text{Area} = 3.14 \cdot r^2
\]
\[
    \text{Perimeter} = 2 \cdot 3.14 \cdot r
\]

6. اكتب برنامجًا يستقبل من المستخدم علامات 3 مواضيع: اللغة العربية، الرياضيات والحاسوب، ويطبع له معدّل العلامات الثلاث.

7. اكتب برنامجًا يستقبل من المستخدم علامات 3 مواضيع: اللغة العربية، الرياضيات والحاسوب، ثم يستقبل منه عدد الوحدات لكل موضوع، ويطبع له معدّل العلامات الثلاث.

8. اكتب برنامجًا يستقبل من المستخدم طوله بالمتر، ووزنه بالكغم، ويحسب له درجة الـ BMI حسب المعادلة التالية:
\[
\text{BMI} = \frac{weight}{height^2}
\]

9. اكتب برنامجًا يستقبل من المستخدم درجة الحرارة بالدرجة المئوية، ويطبع له الدرجة بالفهرنهايت وبالكلفن حسب المعادلات التالية:
\[
    F = \frac{9}{5} \times C + 32
\]

\[
K = C + 273.15
\]

10. اكتب برنامجًا يستقبل من المستخدم درجة الحرارة بالفهرنهايت، ويطبع له الدرجة بالدرجة المئوية حسب المعادلة التالية:
\[
C = \frac{5}{9} \times (F - 32)
\]

11. اكتب برنامجًا يستقبل من المستخدم درجة الحرارة بالكلفن، ويطبع له الدرجة بالدرجة المئوية حسب المعادلة التالية:
\[
C = K - 273.15
\]

12. اكتب برنامجًا يستقبل من المستخدم درجة الحرارة بالفهرنهايت، ويطبع له الدرجة بالكلفن حسب المعادلة التالية:
\[
K = \frac{5}{9} \times (F - 32) + 273.15
\]

13. اكتب برنامجًا يستقبل من المستخدم درجة الحرارة بالكلفن، ويطبع له الدرجة بالفهرنهايت حسب المعادلة التالية:
\[
F = \frac{9}{5} \times (K - 273.15) + 32
\]

14. اكتب برنامجًا يستقبل من المستخدم 5 أرقام، واحدًا تلوَ الآخر. في كل مرة يدخل المستخدم رقمًا جديدًا، يطبع له البرنامج مجموع الأرقام التي أدخلها المستخدم حتى الآن.
\\

15. حل السؤال السابق باستخدام متغيّرين فقط.

\vspace{3cm}
\begin{flushleft}
أرجو لكم وقتًا ممتعًا.

الأستاذ محمود اغبارية.
\end{flushleft}

\end{document}
