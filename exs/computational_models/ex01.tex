\documentclass[12pt]{article}

% For Arabic & multiple languages
\usepackage{polyglossia}
\setmainlanguage{arabic}
\setotherlanguage{english}
\newfontfamily\arabicfont[Script=Arabic]{Amiri} % Or Scheherazade, Noto Naskh Arabic

% For math
\usepackage{amsmath, amssymb}

% For images
\usepackage{graphicx}

% For C# code
\usepackage{listings}
\lstset{
  language=[Sharp]C,
  numbers=left,
  stepnumber=1,
  numbersep=8pt,
  frame=single,
  basicstyle=\ttfamily\small,
  keywordstyle=\color{blue},
  stringstyle=\color{red},
  commentstyle=\color{green!50!black}
}

\newif\ifwithcode

% Colors
\usepackage{xcolor}

% Page layout
\usepackage{geometry}
\geometry{a4paper, margin=2.5cm}

\begin{document}

\section*{مقدمة}

هذا مثال لمذكرة محاضرة في مادة الحاسوب للمرحلة الثانوية.
يمكن أن نكتب نصوصًا بالعربية بشكل طبيعي ومحاذاة من اليمين إلى اليسار.

\section*{مثال رياضي}

نفترض أننا نريد حساب الجذر التربيعي لمعادلة تربيعية:

\begin{equation}
x = \frac{-b \pm \sqrt{b^2 - 4ac}}{2a}
\end{equation}

\section*{مثال برمجي بلغة C\#}

\ifwithcode
\begin{english}
\begin{lstlisting}
// C# Example
using System;

class Program {
    static void Main() {
        Console.WriteLine("Hello, world!");
    }
}
\end{lstlisting}
\end{english}
\fi

\section*{مثال على إدراج صورة}

\begin{center}
\includegraphics[width=0.5\textwidth]{example-image} % replace with your image
\end{center}

\end{document}
