\subsection*{سؤال 6 امتحان 899222 سنة 2019}
\addcontentsline{toc}{subsection}{سؤال 6 امتحان 899222 سنة 2019}
\addcontentsline{toc}{subsection}{سؤال 6 امتحان 899222 سنة 2019}

\noindent
%  \makebox[\textwidth][c]{\includegraphics[width=0.9\paperwidth,keepaspectratio]{ ../../../bagrut_questions/basics/loops_for_2019_899222_6.png }%}%

\begin{enumerate}
    \item اكتب عملية خارجية باسم \textenglish{IsTeen} تتلقى عددًا صحيحًا وأكبر من 0، يمثل عمر شخص. \\
    تُعيد العملية \textenglish{true} إذا كان العمر هو 15 حتّى 18 (بما في ذلك 15 و 18)، خلاف ذلك - تُعيد العملية \textenglish{false}.
    \\ \textbf{مثال:} للعمر 16 تعيد \textenglish{true}، وللعمر 20 تعيد \textenglish{false}.

    \item استمعَ في أحد الأيّام لقناة الأخبار 492 مُستمعاً. \\
    اكتب قطعة برنامج تستقبل بالنسبة لكلّ واحد من المُستمعين عمرهُ وعدد الساعات التي استمعَ فيها لقناة الأخبار. \\
    على البرنامج أن يطبع عدد المُستمعين أبناء 15-18 (بما في ذلك 15 و 18)، الذين استمعوا لقناة الأخبار \textbf{أكثر من 3 ساعات}. \\
    عليك استعمال العملية التي كتبتها في البند "أ".
\end{enumerate}
\clearpage

\ifwithsols
\begin{boxSolution}
\begin{english}
\begin{minted}{csharp}
public static bool IsTeen(int age)
{
    return age >= 15 && age <= 18;
}

static void Main(string[] args)
{
    int count = 0;
    int age, hours;
    for(int i = 1; i <= 492; i++)
    {
        age = int.Parse(Console.ReadLine());
        hours = int.Parse(Console.ReadLine());
        if (IsTeen(age) && hours > 3)
            count++;
    }
    Console.WriteLine(count);
}
\end{minted}
\end{english}
\end{boxSolution}
\clearpage
\fi
