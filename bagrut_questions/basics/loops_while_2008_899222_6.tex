\subsection*{سؤال 6 امتحان 899222 سنة 2008}
\addcontentsline{toc}{subsection}{سؤال 6 امتحان 899222 سنة 2008}

% \makebox[\textwidth][c]{\includegraphics[width=0.9\paperwidth,keepaspectratio]{ ../../../bagrut_questions/basics/loops_while_2008_899222_6.png }%}%
أعلنت شبكة محلات الأزياء "لباس لي" عن حملات لزبائنها:
\begin{itemize}
    \item الزبون الذي يشتري بمبلغ 800 شيقل وما فوق، يحصل على تخفيض قدره 50 شيقل عن كلّ 800 شيقل من التي عليه دفعها في هذه المرّة من الشراء.
    \item الزبون الذي يشتري 4 قطع ألبسة أو أكثر يحصل على كوبون (שובר) بقيمة 20 شيقل لاستعماله للشراء في المرّة القادمة.
\end{itemize}

\underline{أمثلة}:
\begin{itemize}
    \item الزبون الذي اشترى 3 قطع ألبسة ثمنها الكلّي 1100 شيقل، حصل على تخفيض قدره 50 شيقل ودفع مقابل مشترياته 1050 شيقل.
    \item الزبون الذي اشترى 5 قطع ألبسة ثمنها الكلّي 500 شيقل، حصل على كوبون بقيمة 20 شيقل لاستعماله للشراء في المرّة القادمة.
    \item الزبون الذي اشترى 9 قطع ألبسة ثمنها الكلّي 1800 شيقل، حصل على تخفيض قدره 100 شيقل، ودفع مقابل مشترياته 1700 شيقل. كذلك حصل الزبون على كوبون بقيمة 20 شيقل لاستعماله للشراء في المرّة القادمة.
\end{itemize}

اكتب عملية خارجية باسم \textenglish{CalculatePrice} تحسب السعر الذي على زبون واحد أن يدفعه.
على العملية أن تستقبل سعر كلّ واحدة من قطع الألبسة التي اشتراها الزبون.
الأسعار هي أعداد صحيحة.
استقبال الأسعار ينتهي عندما يُستقبل 0 بالنسبة للسعر.
بعد نهاية الاستقبال، على العملية أن:
\begin{itemize}
    \item تطبع عدد قطع الألبسة التي اشتراها الزبون ويحسب ثمنها الكلّي (بدون أي تخفيض).
    \item يحسب البرنامج التخفيض الذي يستحقه الزبون مقابل مشترياته هذه المرّة ويطبعه.
    \item يطبع البرنامج التخفيض والمبلغ الذي على الزبون دفعه مقابل مشترياته.
    \item إذا كان الزبون يستحقّ كوبون لاستعماله للشراء في المرّة القادمة، يطبع البرنامج رسالة \\ \textenglish{'You got 20 NIS Coupon'}.
\end{itemize}

\textbf{ملاحظة}: لا حاجة لفحص صحّة المدخلات.

\begin{boxExample}
لنفرض أن الزبون اشترى قطعًا بالأسعار التالية: $600, 400, 200, 500$. \\
عندها تكون المدخلات:
$$ 600, \quad 400, \quad 200, \quad 500, \quad 0 $$
(العدد 0 يشير إلى نهاية الاستقبال).

\begin{itemize}
    \item \textbf{عدد القطع:} 4 (بما أن العدد 4 أكبر أو يساوي 4، يستحق الزبون كوبون).
    \item \textbf{المجموع الكلي:} $600 + 400 + 200 + 500 = 1700$.
    \item \textbf{حساب التخفيض:} في المبلغ 1700 يوجد مرتان 800 (لأن $800 \times 2 = 1600$). \\
    لذلك التخفيض هو: $50 \times 2 = 100$.
    \item \textbf{المبلغ للدفع:} $1700 - 100 = 1600$.
\end{itemize}

\textbf{المخرجات التي سيطبعها البرنامج:}
\begin{boxCode}
\begin{english}
Total Items: 4 \\
Total Sum: 1700 \\
Discount: 100 \\
Final Payment: 1600 \\
You got 20 NIS Coupon
\end{english}
\end{boxCode}
\end{boxExample}

\clearpage
\ifwithsols
\begin{boxSolution}[سؤال 6 امتحان 899222 سنة 2008]
\begin{boxCode}
\begin{english}
\begin{minted}{csharp}
public static void CalculatePrice()
{
    int sum = 0;
    int count = 0;
    int price = int.Parse(Console.ReadLine());
    while (price != 0)
    {
        sum += price;
        count++;
        price = int.Parse(Console.ReadLine());
    }

    int discount = (sum / 800) * 50;
    int finalPayment = sum - discount;

    Console.WriteLine("Total Items: " + count);
    Console.WriteLine("Total Sum: " + sum);
    Console.WriteLine("Discount: " + discount);
    Console.WriteLine("Final Payment: " + finalPayment);
    if (count >= 4)
        Console.WriteLine("You got 20 NIS Coupon");
}
\end{minted}
\end{english}
\end{boxCode}

\end{boxSolution}
\clearpage
\fi
