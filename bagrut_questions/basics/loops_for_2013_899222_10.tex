\subsection*{سؤال 10 امتحان 899222 سنة 2013}

\noindent
    \makebox[\textwidth][c]{
        \includegraphics[width=0.9\paperwidth,keepaspectratio]{ ../../../bagrut_questions/basics/loops_for_2013_899222_10.png }%
    }%

\ifwithsols
\clearpage

\begin{boxSolution}[سؤال 10 امتحان 899222 سنة 2013]
\begin{enumerate}[itemsep=1.5em, label=\alph*.]
\item .
\begin{boxCode}
\begin{english}
\begin{minted}{csharp}
public static double CalculateTwoBooksPrice(double p1, double p2)
{
    double minPrice = Math.Min(p1, p2);
    double maxPrice = Math.Max(p1, p2);
    return maxPrice + (minPrice / 2.0);
}
\end{minted}
\end{english}
\end{boxCode}

\item .
\begin{boxCode}
\begin{english}
\begin{minted}{csharp}
public static double CalculateTotalPayment(int count)
{
    if (count == 2)
    {
        double p1 = double.Parse(Console.ReadLine());
        double p2 = double.Parse(Console.ReadLine());
        return CalculateTwoBooksPrice(p1, p2);
    }
    else
    {
        double p1 = double.Parse(Console.ReadLine());
        double p2 = double.Parse(Console.ReadLine());
        double p3 = double.Parse(Console.ReadLine());
        return Sum2max(p1, p2, p3);
    }
}
\end{minted}
\end{english}
\end{boxCode}
\end{enumerate}
\end{boxSolution}

\clearpage
\begin{boxSolution}[سؤال 10 امتحان 899222 سنة 2013 - تكملة]
\begin{enumerate}[itemsep=1.5em, label=\alph*., start=3]
\item .
\begin{boxCode}
\begin{english}
\begin{minted}{csharp}
static void Main(string[] args)
{
    for (int i = 1; i <= 142; i++)
    {
        int bookCount = int.Parse(Console.ReadLine());
        Console.WriteLine(CalculateTotalPayment(bookCount));
    }
}
\end{minted}
\end{english}
\end{boxCode}

\end{enumerate}
\end{boxSolution}
\clearpage
\fi
