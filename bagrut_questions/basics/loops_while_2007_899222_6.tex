\subsection*{سؤال 6 امتحان 899222 سنة 2007}
\addcontentsline{toc}{subsection}{سؤال 6 امتحان 899222 سنة 2007}

% \noindent
% \makebox[\textwidth][c]{\includegraphics[width=0.9\paperwidth,keepaspectratio]{ ../../../bagrut_questions/basics/loops_while_2007_899222_6.png }%}%

في معهد الأرصاد الجوّية يقيسون درجات الحرارة كلّ يوم.
اكتب برنامجًا يستقبل درجات الحرارة التي قيست خلال عدة أيّام، يستقبل قياسًا واحدًا بالضبط عن كلّ يوم.

نرمز بـ \textenglish{firstTemp} إلى درجة الحرارة التي استُقبلت عن أوّل يوم. \\
يمكن لدرجة الحرارة في أيّام أخرى أن تكون مطابقة لـ \textenglish{firstTemp}، فيكون عناك عدة أيّام كانت درجة الحرارة فيها مطابقة لدرجة حرارة أول يوم.

على البرنامج أن يحسب كم يومًا مرّ بين كلّ يومَين كانت درجة الحرارة في كل واحد منهما مطابقة لـ \textenglish{firstTemp} (لا يشمل الأيّام التي قيست فيها درجة حرارة مطابقة لـ \textenglish{firstTemp}).

المطلوب:
\begin{itemize}
    \item يطبع البرنامج عدد الأيّام \textbf{الأكبر} الذي مرّ بين قياسَيْن لدرجة حرارة مطابقة لـ \textenglish{firstTemp}.
    \item إذا لم تتكرّر درجة حرارة مطابقة لـ \textenglish{firstTemp} مرّة أخرى خلال القياسات، يطبع البرنامج $-1$.
    \item ينتهي استقبال المعطيات عندما تُستقبل درجة حرارة أعلى من 100.
    \item افترض أنّ درجة الحرارة \textenglish{firstTemp} ليست أعلى من 100.
\end{itemize}

\begin{boxExample}
لنفرض أن المدخلات هي كالتالي (من اليسار لليمين):
$$ 25, \quad 29, \quad 31, \quad 25, \quad 25, \quad 28, \quad 30, \quad 29, \quad 25, \quad 101 $$

\begin{enumerate}
    \item القيمة الأولى (\textenglish{firstTemp}) هي \textbf{25}.
    \item يظهر الرقم 25 مرة أخرى بعد يومين (الأرقام 29, 31). \textbf{الفجوة الحالية = 2}.
    \item يظهر الرقم 25 مرة تالية مباشرة. \textbf{الفجوة الحالية = 0}.
    \item يظهر الرقم 25 مرة أخيرة بعد ثلاثة أيام (الأرقام 28, 30, 29). \textbf{الفجوة الحالية = 3}.
    \item الرقم 101 ينهي البرنامج.
\end{enumerate}
أكبر فجوة تم تسجيلها بين يومين درجة حرارتهما $25$ هي \textbf{3}، لذا المخرج يكون: \textbf{3}.
\end{boxExample}

\ifwithsols
\begin{boxSolution}[سؤال 6 امتحان 899222 سنة 2007]
\begin{boxCode}
\begin{english}
\begin{minted}{csharp}
int firstTemp = int.Parse(Console.ReadLine());
int maxGap = -1;
int currentGap = 0;

int currentTemp = firstTemp;

while (currentTemp <= 100)
{
    currentTemp = int.Parse(Console.ReadLine());
    if (currentTemp == firstTemp)
    {
        if (currentGap > maxGap)
            maxGap = currentGap;
        currentGap = 0;
    }
    else
        currentGap++;
}
Console.WriteLine(maxGap);
\end{minted}
\end{english}
\end{boxCode}

\end{boxSolution}
\clearpage
\fi
