\subsection*{سؤال 8 امتحان 899222 سنة 2011}
\addcontentsline{toc}{subsection}{سؤال 8 امتحان 899222 سنة 2011}

\noindent
\makebox[\textwidth][c]{\includegraphics[width=0.9\paperwidth,keepaspectratio]{ ../../../bagrut_questions/basics/loops_for_2011_899222_8.png }%
}%
\begin{boxExample}
    \begin{itemize}
        \item زبون عدد سنوات عضويته 2، ونطاق مشترياته الشهري 3000 شيقل، لن يحصل على هدية لأنّ سنوات عضويته ليس أكثر من 3.
        \item زبون عدد سنوات عضويته 4، ونطاق مشترياته الشهري 1000 شيقل، لن يحصل على هدية لأنّ نطاق مشترياته ليس أكثر من 1200.
        \item زبون عدد سنوات عضويته 4، ونطاق مشترياته الشهري 1500 شيقل، سيحصل على قسيمة = $50 \times 4 = 200$.
        \item زبون عدد سنوات عضويته 10، ونطاق مشترياته الشهري 3000 شيقل، سيحصل على قسيمة = $100 \times 10 = 1000$.

    \end{itemize}
\end{boxExample}
\clearpage

\ifwithsols
\begin{boxSolution}[سؤال 8 امتحان 899222 سنة 2011 - فرع أ]
\begin{boxCode}
\begin{english}
\begin{minted}{csharp}
public static int CalculateGift(int years, int monthlyPurchase)
{
    if (years > 3 && monthlyPurchase > 1200)
    {
        if (years < 8)
            return years * 50;
        else
            return years * 100;
    }
    return 0;
}
\end{minted}
\end{english}
\end{boxCode}
\end{boxSolution}

\begin{boxSolution}[سؤال 8 امتحان 899222 سنة 2011 - فرع ب]
\begin{boxCode}
\begin{english}
\begin{minted}{csharp}
int count = 0;
int totalGifts = 0;
for (int i = 1; i <= 5000; i++)
{
    int years = int.Parse(Console.ReadLine());
    int purchase = int.Parse(Console.ReadLine());
    int giftAmount = CalculateGift(years, purchase);
    if (giftAmount > 0)
    {
        count++;
        totalGifts += giftAmount;
    }
}
Console.WriteLine(count);
Console.WriteLine(totalGifts);
\end{minted}
\end{english}
\end{boxCode}
\end{boxSolution}
\clearpage
\fi
