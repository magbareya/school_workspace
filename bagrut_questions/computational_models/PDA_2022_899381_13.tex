\subsection*{سؤال 13B امتحان 899381 سنة 2022}
\addcontentsline{toc}{subsection}{سؤال 13B امتحان 899381 سنة 2022}

\insertFullImg{../../../bagrut_questions/computational_models/PDA_2022_899381_13.png}

\ifwithsols
\begin{boxSolution}[سؤال 13B امتحان 899381 سنة 2022]
\begin{enumerate}[label=\alph*., itemsep=2em]
    \item
    \begin{enumerate}[label=(\arabic*), itemsep=2em]
    \item نحصل على أقصر كلمة عندما نعوّض أصغر قيمة لكل بارامتر، أي: $k = 1, n=1$. \\
    لذلك أقصر كلمة هي:
    $$ c^{1+1+1} b^1 a^{2 \cdot 1} \rightarrow c^3 b^1 a^2 \rightarrow cccbaa $$ \\
    كي نحدد إذا كانت كلمة ما موجودة في $L_2$ يجب أن يظهر الحرف $d$ مرة واحدة في الوسط بالضبط. \\
    وأيضًا، القسم الذي قبل الحرف $d$ يجب أن يكون كلمة في $L_1$. \\
    والقسم الذي بعد الحرف $d$ يجب ان يكون مقلوب كلمة في $L_1$ (ليس بالضرورة نفس الكلمة التي كانت قبل الحرف $d$).

    \item . % //TODO
    \end{enumerate}

    \item $L2 = L_1 \cdot d \cdot R(L_1) = \{ c^{1+k+n} b^k a^{2n} d a^{2m} b^q c^{1+q+m} \mid, m,n,k,q \geq 1 \}$
    \begin{itemize}
        \item \textenglish{cccbaaddaabccc} \\
        ليست موجودة في اللغة، لأنّ الحرف $d$  يظهر مرتين في الكلمة.

        \item \textenglish{cccbaadcccbaa} \\
        ليست موجودة في الكلمة لأنّ القسم الموجود بعد الحرف $d$ ليس مقلوب كلمة من اللغة $L_1$، بل هو نفسه موجود في اللغة $L_1$.

        \item $w = cccbaadaaaabcccc$ \\
        موجودة في اللغة $L_2$. لنوضح ذلك نعرّف كلمتين: \\
        $u = cccbaa = c^3 b^1 a^2$ و $v = aaaabcccc = a^4 b^1 c^4$. \\
        يمكننا كتابة:  $w = u \cdot d \cdot v$. \\
        نلاحظ أنّ:
        \begin{itemize}
        \item $u \in L_1$ عندما نعوّض $k = 1, n=1$ (هذه عمليا أقصر كلمة التي وجدناها في الفرع السابق.)
        \item $v \in R(L_1)$ عندما نعوّض $n = 2, k=1$
        \end{itemize}
        لذلك: $w = u \cdot d \cdot v \in L_1 \cdot d \cdot R(L_1) \rightarrow w \in L_2$

    \end{itemize}
\end{enumerate}
\end{boxSolution}
\clearpage
\fi
