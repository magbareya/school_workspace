\subsection*{سؤال 13 امتحان 899205 سنة 2017}

\insertFullImg{../../../bagrut_questions/computational_models/DFA_2017_899205_13.png}

\ifwithsols
\begin{boxSolution}[سؤال 13 امتحان 899205 سنة 2017]
\begin{enumerate}[itemsep=1.5em, label=\alph*.]
\item .
\begin{center}
    \includegraphics[width=0.7\textwidth]{../../../images/DFAs/DNA_startsAATcontainsATtwice.png}
\end{center}

\item
\begin{enumerate}[label=\roman*]
\item $L_6 \subset L_3$ \\
صحيح، لأنّ كل عدد يقسم على 6، فهو بالضرورة يقسم على 3. \\
لذلك، كل حد ينتمي إلى $L_6$ فإنّه ينتمي إلى $L_3$ بالضرورة.

\item $L_2 \cap L_3 = L_6$ \\
صحيح، لأنّ الأعداد التي تقسم على 6 هي الأعداد التي تقسم على 2 وأيضًا تقسم على 3.

\item $L_2 \cdot L_3 = L_6$ \\
غير صحيح، لأنّه ليس بالضرورة إذا ألصقنا عددًا يقسم على 3 مع عدد يقسم على 2 فإنّ النتيجة تقسم على 6. \\
مثلًا: العدد $123$ مكون من العدد $12$ الذي يقسم على 2، وألصقنا به العدد $3$ الذي يقسم على $3$. لذلك:
$123 \in L_2 \cdot L_3$ ولكن $123 \notin L_6$.
\end{enumerate}

\end{enumerate}
\end{boxSolution}
\fi
