\subsection*{سؤال 14 امتحان 899205 سنة 2016}

\insertFullImg{../../../bagrut_questions/computational_models/languages_2016_899205_14.png}

\ifwithsols
\begin{boxSolution}[سؤال 14 امتحان 899205 سنة 2016]
\begin{enumerate}[label=(\arabic*), itemsep=1em]
\item $\epsilon \in L_1 \cup L_3 \cup L_4$ - غير صحيح \\
 لكي نثبت ذلك علينا أن نثبت أنّ $\epsilon \notin L_1$ و $\epsilon \notin L_3$ و $\epsilon \notin L_4$. \\
بالنسبة لـ $L_1$ هذا واضح لأنّ $\epsilon$ لا يحتوي على أي حرف، فهو لا يحتوي على التسلسل $010$. \\
بالنسبة لـ $L_3$، عدد الأصفار في $\epsilon$ هو صفر، وكذلك عدد الآحاد (أي عدد مرات ظهور الرقم $1$) هو صفر، أي أنّ عدد الأصفار في $\epsilon$  مساوٍ لعدد الآحاد فيه. \\
بالنسبة لـ $L_4$ ، عدد الأصفار في $\epsilon$ هو صفر، أي عدد الأصفار فيه هو عدد زوجي.

\item $00100 \in L_1 \cap \overline{L_4} $ - صحيح. \\
في البداية نلاحظ أنّ $\overline{L_4} = \{ w \mid \#_0(w) \% 2 == 0 \}$ أي الكلمات التي فيها عدد الأصفار هو \textbf{زوجي}. \\
لكي نثبت أنّ $00100 \in L_1 \cap \overline{L_4}$ علينا أنّ نثبت أنّ $00100 \in L_1$ وأيضًا أنّ $00100 \in \overline{L_4}$. \\
بالنسبة لـ $00100 \in L_1$ هذا يتحقق لأنّ $00100$ تحتوي على التسلسل $010$. \\
بالنسبة لـ $00100 \in \overline{L_4}$ هذا يتحقق لأنّ عدد الأصفار في $00100$ هو 4 أي عدد زوجي.

\item $\overline{L_2} = \{w \mid 11 \text{تحوي التسلسل} w\}$ - غير صحيح. \\
 $\overline{L_2}$ هي مجموعة كل الكلمات التي ليست في اللغة $L_2$. \\
 حسب التعريف، فإنّ $L_2$ هي مجموعة كل الكلمات التي \underline{لا} تحتوي على التسلسل 00. \\
 لذلك، الكلمات التي ليست موجودة في $L_2$ هي كل الكلمات التي \textbf{تحتوي} على التسلسل 00.\\
 على سبيل المثال: الكلمة $11$ تحتوي على التسلسل 11، لذلك هي موجودة في اللغة المعطاة في هذا الفرع، وهي لا تحتوي على التسلسل 00، لذلك فهي ليست موجودة في $L_2$. \\
 فهذه الكلمة موجودة في المجموعتين، وهذا يدحص الادعاء أنّ اللغة المعطاة هي اللغة المكملة لـ $L_2$.

 \item $L_1 \subset L_3$ - غير صحيح. \\
لكي نثبت ذلك، يكفي أن نجد كلمة موجودة في $L_1$ ولكنها ليست موجودة في $L_3$. \\
على سبيل المثال الكلمة $0101$ موجودة في اللغة $L_1$ لأنّها تحتوي على السلسلة $010$، لكنها ليست موجودة في $L_3$ لأنّ عدد الأصفار والآحاد فيها متساوٍ.
\end{enumerate}
\end{boxSolution}
\begin{boxSolution}[سؤال 14 امتحان 899205 سنة 2016 - تكملة]
\begin{enumerate}[label=(\arabic*), itemsep=1em, start=5]

\item $L_4 = R(L_4)$ - صحيح. \\
    علينا أن نثبت أنّه لكل كلمة $w \in L_4$ فإنّ $R(w) \in L_4$ أيضًا. \\
    هذا يتحقق لأنّ قلب الكلمة لا يؤثر على عدد الأصفار التي فيها. فإذا كان عدد الأصفار فيها فردًّا، فمقلوبها أيضًا فيه عدد فردي من الأصفار وبالتالي سيكون منتميًا إلى $L_4$.

    \item $L_3 \cap L_4 = L_2$ - غير صحيح. \\
    لكي نثبت ذلك علينا أن نجد كلمة موجودة في $L_3$ وأيضًا في $L_4$ لكنها ليست موجودة في $L_2$. \textbf{أو} أن نجد كلمة موجودة في $L_2$ لكنها ليست موجودة في $L_3$ أو ليست موجودة في $L_4$ (أو ليست موجودة في كليهما.) \\

مثلًا: الكلمة $000$ موجودة في $L_3$ لأنّ عدد الأصفار فيها لا يساوي عدد الآحاد، وموجودة أيضًا في $L_4$ لأنّ عدد الأصفار فيها فردي. لكنّها غير موجودة في $L_2$ لأنّها تحتوي على التسلسل $00$. \\

مثال آخر: الكلمة $0101$ ليست موجودة في $L_3$ لأنّ عدد الأصفار فيها يساوي عدد الآحاد، وليست موجودة في $L_4$ لأنّ عدد الأصفار فيها زوجي. لكنّها موجودة في $L_2$ لأنّها لا تحتوي على التسلسل $00$. \\

\end{enumerate}
\end{boxSolution}
\clearpage
\fi
