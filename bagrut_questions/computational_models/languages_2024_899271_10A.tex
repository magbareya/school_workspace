\subsection*{سؤال 10A امتحان 899271 سنة 2024}
\addcontentsline{toc}{subsection}{سؤال 10A امتحان 899271 سنة 2024}

\insertFullImg{../../../bagrut_questions/computational_models/languages_2024_899271_10A.png}

\ifwithsols
\begin{boxSolution}[سؤال 10A امتحان 899271 سنة 2024]
\begin{enumerate}[itemsep=2em, label=\textbf{(\arabic*)}]
    \item $L_5 \cap L_6 = \{110\}$ \\
    \textbf{الشرح:} التقاطع يختار العناصر المشتركة فقط. الكلمة الوحيدة التي تظهر في $L_5$ وتظهر أيضًا في $L_6$ هي "$110$".

    \item $L_2^R = \Sigma^*$ \\
    \textbf{الشرح:} $L_2$ هي $\Sigma^*$ (كل الكلمات الممكنة). بما أن هذه المجموعة شاملة، فإن عكس حروف أي كلمة بداخلها يُنتج كلمة موجودة أيضًا ضمن المجموعة.

    \item $L_1 \cdot L_6 = \emptyset$ \\
    \textbf{الشرح:} $L_1$ هي المجموعة الخالية $\emptyset$. عند دمج "لا شيء" مع أي مجموعة أخرى، تكون النتيجة دائمًا مجموعة خالية.

    \item $L_3 \cdot L_4 = \{0110\}$ \\
    \textbf{الشرح:} $L_3 = \{\varepsilon\}$. الكلمة الفارغة $\varepsilon$ هي العنصر الحيادي في عملية الدمج، لذا عند دمجها مع $L_4$ تبقى $L_4$ كما هي.

    \item $L_4 \cdot L_5 = \{0110, 0110110, 011000, 0110001\}$ \\
    \textbf{الشرح:} نأخذ الكلمة الوحيدة في $L_4$ وهي ($0110$) وندمجها (نلصقها) ببداية كل كلمة موجودة في $L_5$ بالترتيب.

\end{enumerate}
\end{boxSolution}
\clearpage
\fi
