\subsection*{سؤال 13 امتحان 899381B سنة 2021}
\addcontentsline{toc}{subsection}{سؤال 13 امتحان 899381B سنة 2021}

\noindent
\makebox[\textwidth][c]{\includegraphics[width=0.9\paperwidth,keepaspectratio]{ ../../../bagrut_questions/computational_models/PDA_2021_899381B_13.png }%
}%

\ifwithsols
\begin{boxSolution}[سؤال 13 امتحان 899381B سنة 2021]
\begin{enumerate}[itemsep=1.5em, label=\alph*.]
    \item
    \begin{enumerate}[label=(\arabic*)]
        \item نحسب قيمة $k$ بناءً على الدالة $Foo(n)$:
\begin{align*}
    k &= (2 / 2) + (2 \% 2) \\
    k &= 1 + 0 \\
    k &= 1
\end{align*}
بما أن $n=2$ و $k=1$، فإن الكلمة هي $a^2 b^1 = aab$.

\item نحسب قيمة $k$ بنفس الطريقة:
\begin{align*}
    k &= (5 / 2) + (5 \% 2) \\
    k &= 2 + 1 \\
    k &= 3
\end{align*}
بما أن $n=5$ و $k=3$، فإن الكلمة هي $a^5 b^3 = aaaaabbb$.
    \end{enumerate}

    \item . \\
        \includegraphics[width=0.7\paperwidth,keepaspectratio]{../../../images/PDAs/an_bk_k_is_nDiv2_plus_nMod2.png}%
\end{enumerate}
\end{boxSolution}
\clearpage
\fi
