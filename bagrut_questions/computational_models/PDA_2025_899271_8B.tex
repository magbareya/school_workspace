\subsection*{سؤال 8B امتحان 899271 سنة 2025}
\addcontentsline{toc}{subsection}{سؤال 8B امتحان 899271 سنة 2025}

\noindent
\makebox[\textwidth][c]{\includegraphics[width=0.9\paperwidth,keepaspectratio]{ ../../../bagrut_questions/computational_models/PDA_2025_899271_8B.png }%
}%

\ifwithsols
\begin{boxSolution}[سؤال 8B امتحان 899271 سنة 2025]
\textbf{ب.}
\begin{center}
    \includegraphics[width=0.7\paperwidth,keepaspectratio]{../../../images/PDAs/ai_bj_ci-j_i_geq4_i-j_geq3.png}%
\end{center}
\end{boxSolution}
\clearpage
\begin{boxSolution}[سؤال 8B امتحان 899271 سنة 2025 - شرح الأوتومات]
\begin{itemize}
    \item \textbf{الحالات من $q_0$ إلى $q_3$ (العد الإجباري):}
    \begin{itemize}
        \item \textbf{الوظيفة:} هذه الحالات تضمن تحقيق الشرط الأول للغة ($i \ge 4$).
        \item \textbf{الانتقالات:} الأوتومات مجبر على قراءة 4 حروف $a$ متتالية للانتقال من $q_0$ وصولاً إلى $q_4$. في كل خطوة، نقوم بعملية \textbf{دفع (Push)} للرمز $A$ داخل الراصة لنحفظ عدد الحروف التي قرأناها.
    \end{itemize}

    \item \textbf{الحالة $q_4$ (تراكم الـ $a$):}
    \begin{itemize}
        \item \textbf{الوظيفة:} استقبال أي عدد إضافي من حروف $a$ (أكثر من 4).
        \item \textbf{الانتقال:} إذا جاءت $a$ إضافية، نبقى في نفس الحالة ونضيف $A$ للمكدس. إذا جاءت $b$، ننتقل للمرحلة التالية ($q_5$).
    \end{itemize}

    \item \textbf{الحالة $q_5$ (حساب الفرق):}
    \begin{itemize}
        \item \textbf{الوظيفة:} هذه الحالة مسؤولة عن "إلغاء" حروف $a$ مقابل حروف $b$.
        \item \textbf{الانتقال:} مع كل حرف $b$ نقرؤه، نقوم بعملية \textbf{حذف (Pop)} لرمز $A$ من الراصة. هذا يمثل عملية الطرح حسابياً ($i - j$).
    \end{itemize}

    \item \textbf{الحالتان $q_6$ و $q_7$ (ضمان الحد الأدنى للـ $c$):}
    \begin{itemize}
        \item \textbf{الوظيفة:} التأكد من أن المتبقي في الراصة يكفي لـ 3 حروف $c$ على الأقل.
        \item \textbf{انتقال $q_6$:} يقرأ الحرف الأول من $c$ ويحذف $A$ واحدة (بقي 2).
        \item \textbf{انتقال $q_7$:} يقرأ الحرف الثاني من $c$ ويحذف $A$ ثانية (بقي 1). إذا توقف الإدخال هنا، ترفض الكلمة لأننا لم نصل لحالة قبول، مما يمنع الكلمات التي يكون فرقها 2 فقط.
    \end{itemize}
\end{itemize}
\end{boxSolution}
\clearpage
\begin{boxSolution}[سؤال 8B امتحان 899271 سنة 2025 - تكملة]
\begin{itemize}


    \item \textbf{الحالة $q_8$ (معالجة الزيادة):}
    \begin{itemize}
        \item \textbf{الوظيفة:} التعامل مع الحالة التي يكون فيها الفرق أكبر من 3 ($(i-j) > 3$).
        \item \textbf{الانتقال:} نصل إليها إذا قرأنا ثالث حرف $c$ ووجدنا أن الراصة لا تزال ممتلئة بـ $A$. نستمر هنا في استهلاك باقي حروف $c$ وحذف ما يقابلها من $A$ حتى تفرغ الراصة.
    \end{itemize}

    \item \textbf{الحالة $q_9$ (القبول النهائي):}
    \begin{itemize}
        \item \textbf{الوظيفة:} هي المحطة الأخيرة التي نعلن فيها أن الكلمة صحيحة.
        \item \textbf{الانتقال:} نصل إليها عندما نقرأ آخر حرف $c$ ونجد في قاع الراصة الرمز $S$ (علامة الفراغ)، مما يعني أن عدد حروف $c$ طابق تماماً الفرق المتبقي.
    \end{itemize}
\end{itemize}
\textbf{ملخص:} الأوتومات يقبل الكلمة فقط إذا نجح في دفع 4 رموز $A$ في البداية، وبقي منها 3 رموز على الأقل بعد خصم الـ $b$ لتغطية تكلفة عبور الـ $c$.

\end{boxSolution}
\clearpage
\fi
