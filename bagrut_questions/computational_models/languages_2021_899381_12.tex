\subsection*{سؤال 12 امتحان 899381 سنة 2021}

\insertFullImg{../../../bagrut_questions/computational_models/languages_2021_899381_12.png}

\ifwithsols
\begin{boxSolution}[سؤال 12 امتحان 899381 سنة 2021]
    \begin{enumerate}[label=\alph*.]
    \item
    \begin{enumerate}[label=(\arabic*)]
        \item $bcbcac$
        \item $a$
    \end{enumerate}

    \item
    نبني أوتوماتًا نهائيًّا لكل واحد من المتطلّبات:
\begin{itemize}[label=-, itemsep=2em]
\item $L_1$ هي لغة كل الكلمات التي فيها الحرف قبل الأخير هو $a$، هذه لغة نظامية لأنّ الأوتومات التالي يقبلها: \\
\includegraphics[width=0.5\textwidth]{../../../images/DFAs/before_last_is_a.png}

\item $L_2$ هي لغة كل الكلمات التي تحتوي على الأقلّ مرّتين التسلسل $bc$، هذه لغة نظامية لأنّ الأوتومات التالي يقبلها: \\
\includegraphics[width=\textwidth]{../../../images/DFAs/contains_bc_twice.png}
\end{itemize}
\end{enumerate}
\end{boxSolution}

\begin{boxSolution}[سؤال 12 امتحان 899381 سنة 2021 - تكملة]
\begin{itemize}[label=-, itemsep=2em]

\item $L_3$ هي لغة كل الكلمات التي عدد مرات ظهور الحرف $b$ فيها هو زوجيّ. هذه لغة نظامية لأنّ الأوتومات التالي يقبلها: \\
\includegraphics[width=0.4\textwidth]{../../../images/DFAs/even_b.png}

\item $L_4$ هي لغة كل الكلمات التي لا تحوي التسلسل $bb$. هذه لغة نظامية لأنّ الأوتومات التالي يقبلها: \\
\includegraphics[width=0.5\textwidth]{../../../images/DFAs/no_bb.png}
\end{itemize}

اللغة $L$ الموصوفة في السؤال هي اللغة: $L = L_1 \cap L_2 \cap L_3 \cap L_4$. \\
رأينا أعلاه أنّ كل واحدة من اللغات $L_1, L_2, L_3, L_4$ هي لغة نظامية، لذلك $L$ هي لغة نظامية حسب صفة الانغلاق للغات النظامية تحت عملية التقاطع.
\end{boxSolution}

\clearpage
\fi
