\subsection*{سؤال 11 امتحان 899381 سنة 2016}
\addcontentsline{toc}{subsection}{سؤال 11 امتحان 899381 سنة 2016}

\insertFullImg{../../../bagrut_questions/computational_models/NFA_2016_899381_11.png}

\ifwithsols
\begin{boxSolution}[سؤال 11 امتحان 899381 سنة 2016]
\begin{enumerate}[itemsep=1.5em, label=\alph*.]
\item
نلاحظ أنّ:
$$\overline{L_1} = \{a^{2n+1} \mid n \geq 0 \}$$
و
$$\overline{L_2} = \{b^{2n} \mid n \geq 0 \}$$

يمكننا التعبير عن اللغة $L$ كما يلي:
$$L = (L1 \cap L2) \cup (\overline{L_1} \cap \overline{L_2})$$

\begin{itemize}
\item $L_1$ و $L_2$ نظاميّتان حسب المعطى $\leftarrow$ $L_1 \cap L_2$ نظامية لأنّ اللغات النظامية منغلقة تحت عملية التقاطع.
\item $\overline{L_1}$ و $\overline{L_2}$ نظاميّتان لأن كلّ واحدة منهما مكمّلة للغة نظامية، واللغات النّظامية منغلقة تحت عملية المكمّل
\item $\overline{L_1} \cap \overline{L_2}$ نظامية لأنّ اللغات النظامية منغلقة تحت عملية التقاطع.
\item $L = (L1 \cap L2) \cup (\overline{L_1} \cap \overline{L_2})$ نظامية لأنّها اتّحاد لغتين نظاميّتين.
\end{itemize}

\clearpage
\item .
\begin{center}
    \includegraphics[width=0.3\paperwidth,keepaspectratio]{../../../images/DFAs/an_bm_one_even_other_odd.png}%
\end{center}

\end{enumerate}
\end{boxSolution}
\clearpage
\fi
