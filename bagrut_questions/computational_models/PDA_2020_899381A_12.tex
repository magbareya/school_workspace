\subsection*{سؤال 12 امتحان 899381A سنة 2020}
\addcontentsline{toc}{subsection}{سؤال 12 امتحان 899381A سنة 2020}

\noindent
\makebox[\textwidth][c]{\includegraphics[width=0.9\paperwidth,keepaspectratio]{ ../../../bagrut_questions/computational_models/PDA_2020_899381A_12.png }%
}%

\ifwithsols
\begin{boxSolution}[سؤال 12 امتحان 899381A سنة 2020]
    \textbf{أ.} \\
    \includegraphics[width=0.7\paperwidth,keepaspectratio]{../../../images/PDAs/an_bk_cn_b_nOdd_kMod3Is1_repeated.png}%
\end{boxSolution}

\clearpage

\begin{boxSolution}[سؤال 12 امتحان 899381A سنة 2020 - تكملة]
    \textbf{ب.}
لتحديد كلمات اللغة $L_3$، نجد القيم التي تحقق شروط اللغتين معاً:
\begin{itemize}
    \item في اللغة $L_1$، الرمز $a$ يجب أن يظهر عددًا فرديًّا من المرّات. \\
    في اللغة $L_2$ الرمز $a$ يجب أن يظهر أقل من 5 مرات. \\
    لذلك، في الكلمات المشتركة بين اللغتين، الرمز $a$ يمكن أن يظهر إمّا مرّة واحدة، أو 3 مرات.
    أي: $k = 1 \text{ or } 3$
    \item في اللغة $L_1$ عدد مرات ظهور الرمز $a$ مساو لعدد مرات ظهور الرمز $c$. \\
    اعتمادًا على ما استنتجناه أعلاه، فإنّه في الكلمات المشتركة بين اللغتين، أيضًا $c$ يمكن أن يظهر مرة أو 3 مرات.
    أي: $x = k = 1 \text{ or } 3$
    \item في اللغة $L_1$، الرمز $b$ يجب أن يكون باقي قسمة عدد مرات ظهوره على 3 هو 1. \\
    في اللغة $L_2$ الرمز $b$ يجب أن يظهر أقل من 5 مرات. \\
    لذلك، في الكلمات المشتركة بين اللغتين، الرمز $b$ يمكن أن يظهر إمّا مرّة واحدة، أو 4 مرات.
    أي: $m = 1 \text{ or } 4$
\end{itemize}

لذلك، الكلمات الناتجة هي:
نعوض القيم في الصيغة $a^k b^m c^x$:
\begin{align*}
    1. \quad x=k=1, m=1 & \implies abc \\
    2. \quad x=k=1, m=4 & \implies abbbbc \\
    3. \quad x=k=3, m=1 & \implies aaabccc \\
    4. \quad x=k=3, m=4 & \implies aaabbbbccc
\end{align*}
\textbf{النتيجة النهائية:}
\[
L_3 = \{ abc, \ abbbbc, \ aaabccc, \ aaabbbbccc \}
\]
\end{boxSolution}
\clearpage
\fi
