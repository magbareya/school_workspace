\subsection*{سؤال 10B امتحان 899271 سنة 2024}
\addcontentsline{toc}{subsection}{سؤال 10B امتحان 899271 سنة 2024}

\noindent
\makebox[\textwidth][c]{\includegraphics[width=0.9\paperwidth,keepaspectratio]{ ../../../bagrut_questions/computational_models/PDA_2024_899271_10B.png }%
}%

\ifwithsols
\begin{boxSolution}[سؤال 10B امتحان 899271 سنة 2024]
\textbf{ب.}
\begin{center}
    \includegraphics[width=0.7\paperwidth,keepaspectratio]{../../../images/PDAs/abk_cm_bmPlus3k_mk_geq0.png}%
\end{center}
\end{boxSolution}
\clearpage
\begin{boxSolution}[سؤال 10B امتحان 899271 سنة 2024 - شرح الأوتومات]
يقوم الأوتومات بحساب المعادلة $(3k + m)$ باستخدام الراصة، حيث يضيف 3 رموز لكل زوج $ab$، ورمزاً واحداً لكل $c$، ثم يطابق المجموع مع عدد حروف $b$.

\begin{itemize}
    \item \textbf{الحالات $q_0, q_1, q_2$ (مرحلة التخزين المضاعف):}
    \begin{itemize}
        \item يتم هنا قراءة المقطع $(ab)^k$.
        \item عند الانتقال من $q_0 \to q_1$ ومن $q_2 \to q_1$ (عند قراءة $a$)، نقوم بإدخال 3 رموز في \textbf{الراصة}.
        \item الهدف: ضمان وجود 3 عناصر في الراصة مقابل كل تكرار لـ $ab$.
        \item الحالة $q_2$ محورية: تتيح العودة للتكرار، أو الانتقال لقراءة $c$، أو البدء فوراً بقراءة $b$ (إذا لم توجد $c$).
    \end{itemize}

    \item \textbf{الحالة $q_3$ (مرحلة التخزين الفردي):}
    \begin{itemize}
        \item تختص بقراءة حروف $c^m$.
        \item مقابل كل حرف $c$، يتم إدخال رمز واحد فقط في \textbf{الراصة}.
    \end{itemize}

    \item \textbf{الحالات $q_4, q_5$ (مرحلة التفريغ والمطابقة):}
    \begin{itemize}
        \item يتم هنا قراءة الجزء الأخير $b^{m+3k}$.
        \item كل حرف $b$ يتم قراءته يقابله حذف رمز واحد من \textbf{الراصة}.
        \item الوصول للحالة النهائية $q_5$ مع راصة فارغة يعني أن عدد الـ $b$ مساوٍ تماماً لمجموع ما تم تخزينه.
    \end{itemize}
\end{itemize}


\end{boxSolution}
\clearpage
\fi
