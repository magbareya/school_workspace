\subsection*{سؤال 9-أ امتحان 899271 سنة 2024}

\insertFullImg{../../../bagrut_questions/computational_models/regularity_2024_899271_9A.png}

\ifwithsols
\begin{boxSolution}[سؤال 9-أ امتحان 899271 سنة 2024]
\begin{enumerate}[itemsep=1.5em, label=\alph*.]
    \item غير صحيح. \\
    مثلًا: $L_1 = \{a^n b^n \mid n \geq 1\}$ و $L_2 = \{a^n b^k \mid n > k\}$ هما لغتان غير نظاميّتين، ولكن تقاطعهما هو اللغة الفارغة $\emptyset$، وهي لغة نظاميّة.

    \item غير صحيح. \\
    مثلًا: $L_1 = \{a\}$ و $L_2 = \{b\}$. عندها: \\
    $L_1^n = \{a^n\}$ و $L_2^n = \{b^n\}$. \\
    $(L_1 \cdot L_2)^n = \{(ab)^n\}$ لكن: $L_1^n \cdot L_2^n = \{a^n b^n\}$. \\
    وهما ليستا متساويتين، مثلًا: \\
    $abab \in (L_1 \cdot L_2)^n$ ولكن $abab \notin L_1^n \cdot L_2^n$. \\
    $aabb \notin (L_1 \cdot L_2)^n$ ولكن $aabb \in L_1^n \cdot L_2^n$.

\begin{boxNote}
    عملية التسلسل (الإلصاق) بين اللغات، عادة لا تحافظ على العمليات، بمعنى تطبيق عملية معينة على لغتين قبل الإلصاق يعطينا نتيجة مختلفة عن تطبيقها بعد الإلصاق. \\
    مثلًا:
    \begin{itemize}
        \item $\overline{L_1 \cdot L_2} \neq \overline{L_1} \cdot \overline{L_2}$
        \item $R(L_1 \cdot L_2) \neq R(L_1) \cdot R(L_2)$
    \end{itemize}
\end{boxNote}

\end{enumerate}
\end{boxSolution}
\clearpage
\fi
