\documentclass[14pt]{extarticle}
% Full article preamble (duplicated, no common file)
\usepackage{fontspec}
\usepackage[a4paper,top=2.4cm,bottom=2.4cm,left=2.3cm,right=2.3cm]{geometry}
\usepackage{polyglossia}
\usepackage{amsmath}
\usepackage{amssymb}
\usepackage{xcolor}
\usepackage{fancyhdr}
\usepackage{graphicx}
\usepackage{listings}
\usepackage[most]{tcolorbox}
\usepackage{pifont}
\usepackage{enumitem}
\usepackage{titlesec}
\usepackage[bottom]{footmisc}
\usepackage{titling}
\usepackage{minted}
\usepackage{etoolbox}
\usepackage{array}
\usepackage{extsizes}

\newfontfamily\emoji{Segoe UI Emoji}

\pagestyle{fancy}

\setmainlanguage[numerals=western]{arabic}
\setotherlanguage{english}
\newfontfamily\arabicfont[Script=Arabic]{Amiri}
\newfontfamily\arabicfonttt[Script=Arabic]{Courier New}

\lstset{
  language=[Sharp]C,
  numbers=left,
  stepnumber=1,
  numbersep=8pt,
  frame=single,
  basicstyle=\ttfamily\small,
  keywordstyle=\color{blue},
  stringstyle=\color{red},
  commentstyle=\color{green!50!black}
}

\newif\ifdetailed
\ifdefined\setdetailed
  \setdetailed
\fi

\newif\ifwithsols
\ifdefined\setwithsols
  \setwithsols
\fi

% unified tcolorboxes for articles
\tcbset{colback=white, colframe=black, fonttitle=\bfseries, boxrule=0.8pt}
\newtcolorbox{boxDef}[1][]{colback=blue!5!white,colframe=blue!75!black,
  title={{\emoji📘} تعريف\ifx\\#1\\\else ~#1\fi :}}
\newtcolorbox{boxExercise}[1][]{colback=cyan!5!white,colframe=cyan!70!black,
  title={{\emoji🧩} تمرين\ifx\\#1\\\else ~#1\fi :}}
\newtcolorbox{boxExample}[1][]{colback=yellow!5!white,colframe=orange!90!black,
  title={{\emoji📝} مثال\ifx\\#1\\\else ~#1\fi :}}
\newtcolorbox{boxNote}[1][]{colback=gray!10!white,colframe=black,
  title={{\emoji✨} ملاحظة\ifx\\#1\\\else ~#1\fi :}}
\newtcolorbox{boxAttention}[1][]{colback=magenta!10!white,colframe=magenta!80!black,
  title={{\emoji🔔} تنبيه\ifx\\#1\\\else ~#1\fi :}}
\newtcolorbox{boxWarning}[1][]{colback=red!5!white,colframe=red!75!black,
  title={{\emoji⚡} ملاحظة هامة\ifx\\#1\\\else ~#1\fi :}}
\newtcolorbox{boxSolution}[1][]{colback=green!5!white,colframe=green!60!black,
  title={{\emoji✅} حل\ifx\\#1\\\else ~#1\fi :}}
\newtcolorbox{boxSymbol}[1][]{colback=purple!5!white,colframe=purple!70!black,
  title={{\emoji🔣} رمز\ifx\\#1\\\else ~#1\fi :}}
\newtcolorbox{boxHint}[1][]{colback=teal!5!white,colframe=teal!60!black,
  title={{\emoji💡} تلميح\ifx\\#1\\\else ~#1\fi :}}


\tcbset{simplecode/.style={ colback=gray!5, colframe=black!50, boxrule=0.4pt, arc=2pt, left=4pt,right=4pt,top=4pt,bottom=4pt}}
\newenvironment{boxCode}{\begin{tcolorbox}[simplecode]}{\end{tcolorbox}}

\newcolumntype{C}[1]{>{\centering\arraybackslash}p{#1}}

% redefine spaces after titles
\makeatletter
\renewcommand{\@maketitle}{%
  \begin{center}
    {\huge \bfseries \@title \par}%
    \vskip 0.2em % space between title and author
    {\large \@author \par}%
    % \vskip 0.2em % space between author and date
    % {\normalsize \@date \par}%
  \end{center}
}
\makeatother

\fancyhf{} % clear default
\fancypagestyle{plain}{
  \fancyhf{}
  \fancyhead[L]{مدرسة التسامح الشاملة}
  % \fancyhead[L]{\includegraphics[height=1cm]{../../../images/logoTasamoh.png}}
  \fancyhead[R]{الأستاذ محمود اغبارية}
  \fancyfoot[C]{\thepage}
}

\fancyhead[L]{مدرسة التسامح الشاملة}
\fancyhead[R]{الأستاذ محمود اغبارية}
\fancyfoot[C]{\thepage}
% \date{\today}

\setcounter{tocdepth}{3} % only section subsection and subsubsection in TOC


% ----------------------


% \begin{document}

% \maketitle

% % \clearpage  % start TOC on a new page
% % \renewcommand{\contentsname}{جدول المحتويات}
% % \tableofcontents
% % \clearpage

% \part*{part 1} % the * prevents numbering
% \section*{مقدمة}
% \subsection*{مثال رياضي}
% \subsubsection*{مثال فرعي}
% \paragraph*{ paragraph 1}
% \subparagraph*{sub paragraph 1}

% \ifdetailed
% \begin{english}
% \begin{minted}{csharp}
% // C# Example
% \end{minted}
% \end{english}
% \fi

% OLD WAY
% \ifdetailed
% \begin{english}
% \begin{lstlisting}
% // C# Example
% \end{lstlisting}
% \end{english}
% \fi

% % \includegraphics[width=0.2\textwidth]{../../../images/DFAs/ex1_q1.png}



% \vspace{3cm}
% \begin{flushleft}
% أرجو لكم وقتًا ممتعًا.

% الأستاذ محمود اغبارية.
% \end{flushleft}


% \end{document}


\title{المصفوفات الأحادية البعد \textenglish{(1D Arrays)} في لغة \textenglish{C\#}}

\begin{document}
\maketitle
\thispagestyle{fancy}

\tableofcontents
\clearpage

\section{مقدمة}

\subsection{ما هي المصفوفة؟}
المصفوفة \textenglish{(Array)} هي بنية معطيات تمكننا من تخزين مجموعة من القيم من \textbf{نفس النوع} تحت اسم واحد.

\subsection{لماذا نستخدم المصفوفات؟}
\begin{itemize}
    \item تخزين مجموعة كبيرة من البيانات بطريقة منظمة
    \item سهولة الوصول إلى العناصر عن طريق الفهرس
    \item تسهيل معالجة البيانات باستخدام الحلقات
    \item توفير في الكود وتجنب التكرار
\end{itemize}

\begin{boxExample}[مثال توضيحي]
تخيّل أنك تريد تخزين درجات 30 طالباً في امتحان. بدون المصفوفات، ستحتاج إلى:
\begin{english}
\begin{minted}{csharp}
int grade1, grade2, grade3, ..., grade30;
\end{minted}
\end{english}
أما مع المصفوفات:
\begin{english}
\begin{minted}{csharp}
int[] grades = new int[30];
\end{minted}
\end{english}
\end{boxExample}

\subsection{خصائص المصفوفة}
\begin{itemize}
    \item \textbf{الحجم الثابت:} حجم المصفوفة يُحدّد عند إنشائها ولا يمكن تغييره لاحقاً
    \item \textbf{نوع واحد:} جميع العناصر يجب أن تكون من نفس النوع
    \item \textbf{الفهرسة:} يبدأ الفهرس من 0 وينتهي عند \textenglish{Length - 1}
    \item \textbf{التخزين المتجاور:} العناصر مخزنة في مواقع متجاورة في الذاكرة
\end{itemize}

\clearpage

\section{التصريح والإنشاء}

\subsection{التصريح عن المصفوفة}

التصريح عن المصفوفة يتم بتحديد نوع العناصر ووضع الأقواس المربعة \textenglish{[]}.

\begin{boxSymbol}
\begin{english}
\begin{minted}{csharp}
type[] arrayName;
\end{minted}
\end{english}
حيث:
\begin{itemize}
    \item \textenglish{type}: نوع العناصر (\textenglish{int, double, string, bool, char})
    \item \textenglish{[]}: تدل على أن المتغير هو مصفوفة
    \item \textenglish{arrayName}: اسم المصفوفة
\end{itemize}
\end{boxSymbol}

\begin{boxExample}[أمثلة على التصريح]
\begin{english}
\begin{minted}{csharp}
int[] numbers;
double[] prices;
string[] names;
bool[] flags;
char[] letters;
\end{minted}
\end{english}
\end{boxExample}

\begin{boxWarning}
التصريح وحده لا يكفي! يجب إنشاء المصفوفة باستخدام \textenglish{new}.
\end{boxWarning}

\subsection{إنشاء المصفوفة}

بعد التصريح، يجب إنشاء المصفوفة وتحديد حجمها باستخدام الكلمة المفتاحية \textenglish{new}.

\begin{boxExample}[1 - إنشاء مصفوفة]
\begin{english}
\begin{minted}{csharp}
int[] numbers = new int[5];
\end{minted}
\end{english}
تُنشئ مصفوفة من 5 أعداد صحيحة، جميع القيم تبدأ بـ 0.
\end{boxExample}

\begin{boxExample}[2 - التصريح والإنشاء بخطوتين]
\begin{english}
\begin{minted}{csharp}
double[] scores;
scores = new double[10];
\end{minted}
\end{english}
\end{boxExample}

\begin{boxNote}
القيم الافتراضية للمصفوفات:
\begin{itemize}
    \item الأعداد (\textenglish{int, double}): 0
    \item \textenglish{bool}: \textenglish{false}
    \item \textenglish{string, object}: \textenglish{null}
    \item \textenglish{char}: \textenglish{'\textbackslash 0'}
\end{itemize}
\end{boxNote}

\subsection{التهيئة المباشرة}

يمكن تهيئة المصفوفة بقيم مباشرة عند إنشائها:

\begin{boxExample}[3 - طرق التهيئة]
هناك ثلاث طرق للتهيئة، كلها تعطي نفس النتيجة:
\begin{english}
\begin{minted}{csharp}
// الطريقة الأولى: نحدد الحجم والقيم معاً
int[] numbers1 = new int[5] { 10, 20, 30, 40, 50 };

// الطريقة الثانية: دون تحديد الحجم
int[] numbers2 = new int[] { 10, 20, 30, 40, 50 };

// الطريقة الثالثة: الأقصر والأسهل
int[] numbers3 = { 10, 20, 30, 40, 50 };
\end{minted}
\end{english}
\end{boxExample}

\begin{boxExample}[أمثلة على التهيئة بأنوع مختلفة]
\begin{english}
\begin{minted}{csharp}
string[] colors = { "Red", "Green", "Blue" };
double[] grades = { 85.5, 92.0, 78.3, 95.7 };
\end{minted}
\end{english}
\end{boxExample}

\clearpage

\section{الوصول إلى العناصر}

\subsection{الفهرس \textenglish{(Index)}}

كل عنصر في المصفوفة له رقم فهرس \textenglish{(Index)} يبدأ من 0.

\begin{boxDef}
\textbf{الفهرس} هو رقم يحدد موقع العنصر في المصفوفة:
\begin{itemize}
    \item الفهرس الأول: 0
    \item الفهرس الأخير: \textenglish{Length - 1}
\end{itemize}
\end{boxDef}

\begin{boxExample}[4 - الوصول للعناصر]
\begin{english}
\begin{minted}{csharp}
int[] numbers = { 10, 20, 30, 40, 50 };

int first = numbers[0];
int second = numbers[1];
int third = numbers[2];
int fourth = numbers[3];
int fifth = numbers[4];

Console.WriteLine(first);
Console.WriteLine(numbers[2]);
\end{minted}
\end{english}
الناتج:
\begin{english}
\begin{boxCode}
10 \\
30
\end{boxCode}
\end{english}
\end{boxExample}

\subsection{تعديل قيم العناصر}

\begin{boxExample}[5 - التعديل]
\begin{english}
\begin{minted}{csharp}
int[] numbers = { 10, 20, 30, 40, 50 };

numbers[2] = 35;
numbers[4] = 55;
\end{minted}
\end{english}
الآن المصفوفة: \textenglish{\{10, 20, 35, 40, 55\}}
\end{boxExample}

\begin{boxWarning}
إذا حاولت الوصول إلى فهرس خارج نطاق المصفوفة، ستحصل على خطأ:
\begin{english}
\begin{minted}{csharp}
int[] arr = new int[5];
int x = arr[10];
\end{minted}
\end{english}
خطأ: \textenglish{IndexOutOfRangeException}
\\
الفهارس الصحيحة هي من 0 إلى 4 فقط.
\end{boxWarning}

\subsection{خاصية الطول \textenglish{Length}}

كل مصفوفة لها خاصية \textenglish{Length} تُعطينا عدد العناصر.

\begin{boxExample}[6 - استخدام خاصية Length]
\begin{english}
\begin{minted}{csharp}
int[] numbers = { 10, 20, 30, 40, 50 };
int size = numbers.Length;

Console.WriteLine(size);

int lastElement = numbers[numbers.Length - 1];
Console.WriteLine(lastElement);
\end{minted}
\end{english}
الناتج:
\begin{english}
\begin{boxCode}
5 \\
50
\end{boxCode}
\end{english}
\end{boxExample}

\clearpage

\section{المصفوفات والحلقات}

\subsection{حلقة \textenglish{for}}

حلقة \textenglish{for} هي الأكثر استخداماً مع المصفوفات لأنها توفر الفهرس.

\begin{boxExample}[7 - طباعة جميع العناصر]
\begin{english}
\begin{minted}{csharp}
int[] numbers = { 10, 20, 30, 40, 50 };

for (int i = 0; i < numbers.Length; i++)
{
    Console.WriteLine(numbers[i]);
}
\end{minted}
\end{english}
الناتج:
\begin{english}
\begin{boxCode}
10 \\
20 \\
30 \\
40 \\
50
\end{boxCode}
\end{english}
\end{boxExample}

\begin{boxExample}[8 - طباعة مع الفهارس]
\begin{english}
\begin{minted}{csharp}
string[] names = { "Ali", "Sara", "Omar" };

for (int i = 0; i < names.Length; i++)
{
    Console.WriteLine($"[{i}]: {names[i]}");
}
\end{minted}
\end{english}
الناتج:
\begin{english}
\begin{boxCode}
{[}0{]}: Ali \\
{[}1{]}: Sara \\
{[}2{]}: Omar
\end{boxCode}
\end{english}
\end{boxExample}

\subsection{قراءة قيم المصفوفة من المستخدم}

\begin{boxExample}[9 - استقبال من المستخدم]
\begin{english}
\begin{minted}{csharp}
int[] grades = new int[5];

Console.WriteLine("Enter 5 grades:");
for (int i = 0; i < grades.Length; i++)
{
    Console.Write($"Grade {i + 1}: ");
    grades[i] = int.Parse(Console.ReadLine());
}

Console.WriteLine("Your grades:");
for (int i = 0; i < grades.Length; i++)
{
    Console.WriteLine(grades[i]);
}
\end{minted}
\end{english}
\end{boxExample}

\begin{boxExample}[10 - طباعة المصفوفة بسطر واحد]
\begin{english}
\begin{minted}{csharp}
int[] numbers = { 10, 20, 30, 40, 50 };

for (int i = 0; i < numbers.Length; i++)
{
    if (i > 0)
        Console.Write(", ");
    Console.Write(numbers[i]);
}
Console.WriteLine();
\end{minted}
\end{english}
الناتج:
\begin{english}
\begin{boxCode}
10, 20, 30, 40, 50
\end{boxCode}
\end{english}
\end{boxExample}

\clearpage

\section{عمليات أساسية}

\subsection{جمع عناصر المصفوفة}

\begin{boxExample}[11 - حساب المجموع]
\begin{english}
\begin{minted}{csharp}
int[] numbers = { 10, 20, 30, 40, 50 };
int sum = 0;

for (int i = 0; i < numbers.Length; i++)
{
    sum += numbers[i];
}

Console.WriteLine("Sum = " + sum);
\end{minted}
\end{english}
الناتج: \textenglish{Sum = 150}
\end{boxExample}

\begin{boxExample}[12 - حساب المعدل]
\begin{english}
\begin{minted}{csharp}
double[] grades = { 85.5, 92.0, 78.5, 88.0, 95.0 };
double sum = 0;

for (int i = 0; i < grades.Length; i++)
{
    sum += grades[i];
}

double average = sum / grades.Length;
Console.WriteLine("Average = " + average);
\end{minted}
\end{english}
الناتج: \textenglish{Average = 87.8}
\end{boxExample}

\subsection{عد العناصر بشرط}

\begin{boxExample}[13 - عد الأعداد الزوجية]
\begin{english}
\begin{minted}{csharp}
int[] numbers = { 5, 12, 8, 3, 15, 20, 7 };
int evenCount = 0;

for (int i = 0; i < numbers.Length; i++)
{
    if (numbers[i] % 2 == 0)
        evenCount++;
}

Console.WriteLine("Even numbers: " + evenCount);
\end{minted}
\end{english}
الناتج: \textenglish{Even numbers: 3}
\end{boxExample}

\begin{boxExample}[14 - عد العناصر الموجبة]
\begin{english}
\begin{minted}{csharp}
int[] numbers = { -5, 12, -8, 3, -15, 20, 7 };
int positiveCount = 0;

for (int i = 0; i < numbers.Length; i++)
{
    if (numbers[i] > 0)
        positiveCount++;
}

Console.WriteLine("Positive numbers: " + positiveCount);
\end{minted}
\end{english}
الناتج: \textenglish{Positive numbers: 4}
\end{boxExample}

\subsection{إيجاد الأكبر والأصغر}

\begin{boxExample}[15 - إيجاد أكبر عنصر]
\begin{english}
\begin{minted}{csharp}
int[] numbers = { 5, 12, 8, 3, 15, 20, 7 };
int max = numbers[0];

for (int i = 1; i < numbers.Length; i++)
{
    if (numbers[i] > max)
        max = numbers[i];
}

Console.WriteLine("Maximum = " + max);
\end{minted}
\end{english}
الناتج: \textenglish{Maximum = 20}
\end{boxExample}

\begin{boxExample}[16 - إيجاد أصغر عنصر]
\begin{english}
\begin{minted}{csharp}
int[] numbers = { 5, 12, 8, 3, 15, 20, 7 };
int min = numbers[0];

for (int i = 1; i < numbers.Length; i++)
{
    if (numbers[i] < min)
        min = numbers[i];
}

Console.WriteLine("Minimum = " + min);
\end{minted}
\end{english}
الناتج: \textenglish{Minimum = 3}
\end{boxExample}

\begin{boxExample}[17 - إيجاد الأكبر مع موقعه]
\begin{english}
\begin{minted}{csharp}
int[] numbers = { 5, 12, 8, 20, 15, 3, 7 };
int max = numbers[0];
int maxIndex = 0;

for (int i = 1; i < numbers.Length; i++)
{
    if (numbers[i] > max)
    {
        max = numbers[i];
        maxIndex = i;
    }
}

Console.WriteLine($"Max = {max} at index {maxIndex}");
\end{minted}
\end{english}
الناتج: \textenglish{Max = 20 at index 3}
\end{boxExample}

\clearpage

\section{البحث في المصفوفات}

\subsection{البحث الخطي \textenglish{(Linear Search)}}

\begin{boxExample}[18 - البحث عن عنصر]
\begin{english}
\begin{minted}{csharp}
int[] numbers = { 5, 12, 8, 3, 15, 20, 7 };
int searchFor = 15;
bool found = false;
int position = -1;

for (int i = 0; i < numbers.Length; i++)
{
    if (numbers[i] == searchFor)
    {
        found = true;
        position = i;
        break;
    }
}

if (found)
    Console.WriteLine($"Found at index {position}");
else
    Console.WriteLine("Not found");
\end{minted}
\end{english}
الناتج: \textenglish{Found at index 4}
\end{boxExample}

\begin{boxExample}[19 - عد تكرار العنصر]
\begin{english}
\begin{minted}{csharp}
int[] numbers = { 5, 12, 8, 12, 15, 12, 7 };
int searchFor = 12;
int count = 0;

for (int i = 0; i < numbers.Length; i++)
{
    if (numbers[i] == searchFor)
        count++;
}

Console.WriteLine($"Element {searchFor} appears {count} times");
\end{minted}
\end{english}
الناتج: \textenglish{Element 12 appears 3 times}
\end{boxExample}

\begin{boxExample}[20 - إيجاد جميع المواقع]
\begin{english}
\begin{minted}{csharp}
int[] numbers = { 5, 12, 8, 12, 15, 12, 7 };
int searchFor = 12;

Console.WriteLine($"Positions of {searchFor}:");
for (int i = 0; i < numbers.Length; i++)
{
    if (numbers[i] == searchFor)
        Console.WriteLine($"Index: {i}");
}
\end{minted}
\end{english}
الناتج:
\begin{english}
\begin{boxCode}
Positions of 12: \\
Index: 1 \\
Index: 3 \\
Index: 5
\end{boxCode}
\end{english}
\end{boxExample}

\clearpage

\section{نسخ ومقارنة المصفوفات}

\subsection{نسخ المصفوفات}

\begin{boxWarning}
\textbf{خطأ شائع!} هذا الكود لا ينسخ المصفوفة:
\begin{english}
\begin{minted}{csharp}
int[] arr1 = { 1, 2, 3 };
int[] arr2 = arr1;

arr2[0] = 100;
Console.WriteLine(arr1[0]);
\end{minted}
\end{english}
الناتج: \textenglish{100} (تغيّر أيضاً!)
\\
\textenglish{arr2 = arr1} لا ينسخ المصفوفة، بل يجعل \textenglish{arr2} يشير لنفس المصفوفة.
\end{boxWarning}

\begin{boxExample}[21 - النسخ الصحيح يدوياً]
\begin{english}
\begin{minted}{csharp}
int[] arr1 = { 1, 2, 3, 4, 5 };
int[] arr2 = new int[arr1.Length];

for (int i = 0; i < arr1.Length; i++)
{
    arr2[i] = arr1[i];
}

arr2[0] = 100;
Console.WriteLine(arr1[0]);
Console.WriteLine(arr2[0]);
\end{minted}
\end{english}
الناتج:
\begin{english}
\begin{boxCode}
1 \\
100
\end{boxCode}
\end{english}
\end{boxExample}

\begin{boxExample}[22 - النسخ باستخدام \textenglish{Array.Copy}]
بدلاً من الحلقة يمكن استخدام عملية جاهزة:
\begin{english}
\begin{minted}{csharp}
int[] arr1 = { 1, 2, 3, 4, 5 };
int[] arr2 = new int[arr1.Length];

// ينسخ من arr1 إلى arr2 عدد arr1.Length عنصر
Array.Copy(arr1, arr2, arr1.Length);
\end{minted}
\end{english}
\end{boxExample}

\subsection{مقارنة المصفوفات}

\begin{boxExample}[23 - فحص التساوي]
\begin{english}
\begin{minted}{csharp}
int[] arr1 = { 1, 2, 3, 4, 5 };
int[] arr2 = { 1, 2, 3, 4, 5 };
bool areEqual = true;

if (arr1.Length != arr2.Length)
{
    areEqual = false;
}
else
{
    for (int i = 0; i < arr1.Length; i++)
    {
        if (arr1[i] != arr2[i])
        {
            areEqual = false;
            break;
        }
    }
}

Console.WriteLine(areEqual ? "Equal" : "Not Equal");
\end{minted}
\end{english}
الناتج: \textenglish{Equal}
\end{boxExample}

\clearpage

\section{عمليات متقدمة}

\subsection{عكس المصفوفة}

\begin{boxExample}[24 - عكس ترتيب العناصر بالتبديل]
الفكرة: نبدّل الأول مع الأخير، والثاني مع قبل الأخير، وهكذا حتى نصف المصفوفة.
\begin{english}
\begin{minted}{csharp}
int[] numbers = { 1, 2, 3, 4, 5 };

for (int i = 0; i < numbers.Length / 2; i++)
{
    int temp = numbers[i];
    numbers[i] = numbers[numbers.Length - 1 - i];
    numbers[numbers.Length - 1 - i] = temp;
}
\end{minted}
\end{english}
النتيجة: \textenglish{\{5, 4, 3, 2, 1\}}
\end{boxExample}

\begin{boxExample}[25 - نسخ بترتيب عكسي]
\begin{english}
\begin{minted}{csharp}
int[] original = { 1, 2, 3, 4, 5 };
int[] reversed = new int[original.Length];

for (int i = 0; i < original.Length; i++)
{
    reversed[i] = original[original.Length - 1 - i];
}
\end{minted}
\end{english}
\textenglish{reversed} يحتوي على: \textenglish{\{5, 4, 3, 2, 1\}}
\end{boxExample}

\subsection{إزاحة العناصر}

\begin{boxExample}[26 - إزاحة لليمين]
\begin{english}
\begin{minted}{csharp}
int[] arr = { 1, 2, 3, 4, 5 };

int last = arr[arr.Length - 1];

for (int i = arr.Length - 1; i > 0; i--)
{
    arr[i] = arr[i - 1];
}

arr[0] = last;
\end{minted}
\end{english}
النتيجة: \textenglish{\{5, 1, 2, 3, 4\}}
\end{boxExample}

\begin{boxExample}[27 - إزاحة لليسار]
\begin{english}
\begin{minted}{csharp}
int[] arr = { 1, 2, 3, 4, 5 };

int first = arr[0];

for (int i = 0; i < arr.Length - 1; i++)
{
    arr[i] = arr[i + 1];
}

arr[arr.Length - 1] = first;
\end{minted}
\end{english}
النتيجة: \textenglish{\{2, 3, 4, 5, 1\}}
\end{boxExample}

\subsection{تبديل عنصرين}

\begin{boxExample}[28 - تبديل الأول والأخير]
\begin{english}
\begin{minted}{csharp}
int[] arr = { 1, 2, 3, 4, 5 };

int temp = arr[0];
arr[0] = arr[arr.Length - 1];
arr[arr.Length - 1] = temp;
\end{minted}
\end{english}
النتيجة: \textenglish{\{5, 2, 3, 4, 1\}}
\end{boxExample}

\begin{boxExample}[29 - تبديل عنصرين متجاورين]
\begin{english}
\begin{minted}{csharp}
int[] arr = { 1, 2, 3, 4, 5, 6 };

for (int i = 0; i < arr.Length - 1; i += 2)
{
    int temp = arr[i];
    arr[i] = arr[i + 1];
    arr[i + 1] = temp;
}
\end{minted}
\end{english}
النتيجة: \textenglish{\{2, 1, 4, 3, 6, 5\}}
\end{boxExample}

\subsection{فلترة وإنشاء مصفوفة جديدة}

\begin{boxExample}[30 - نسخ الأعداد الزوجية فقط]
\begin{english}
\begin{minted}{csharp}
int[] numbers = { 5, 12, 8, 3, 15, 20, 7, 2 };

int evenCount = 0;
for (int i = 0; i < numbers.Length; i++)
{
    if (numbers[i] % 2 == 0)
        evenCount++;
}

int[] evenNumbers = new int[evenCount];
int index = 0;

for (int i = 0; i < numbers.Length; i++)
{
    if (numbers[i] % 2 == 0)
    {
        evenNumbers[index] = numbers[i];
        index++;
    }
}
\end{minted}
\end{english}
\begin{english}
\begin{minted}{csharp}
// طباعة النتيجة للتحقق
for (int i = 0; i < evenNumbers.Length; i++)
    Console.Write(evenNumbers[i] + " ");
\end{minted}
\end{english}
الناتج: \textenglish{12 8 20 2}
\end{boxExample}

\clearpage

\section{المصفوفات والعمليات الخارجية}

\subsection{تمرير المصفوفة كبارامتر}

\begin{boxExample}[31 - عملية لطباعة المصفوفة]
\begin{english}
\begin{minted}{csharp}
public static void PrintArray(int[] arr)
{
    Console.Write("[ ");
    for (int i = 0; i < arr.Length; i++)
    {
        Console.Write(arr[i]);
        if (i < arr.Length - 1)
            Console.Write(", ");
    }
    Console.WriteLine(" ]");
}

public static void Main()
{
    int[] numbers = { 1, 2, 3, 4, 5 };
    PrintArray(numbers);
}
\end{minted}
\end{english}
الناتج: \textenglish{[ 1, 2, 3, 4, 5 ]}
\end{boxExample}

\begin{boxExample}[32 - عملية لحساب المجموع]
\begin{english}
\begin{minted}{csharp}
public static int SumArray(int[] arr)
{
    int sum = 0;
    for (int i = 0; i < arr.Length; i++)
    {
        sum += arr[i];
    }
    return sum;
}

public static void Main()
{
    int[] numbers = { 10, 20, 30, 40, 50 };
    int total = SumArray(numbers);
    Console.WriteLine("Total = " + total);
}
\end{minted}
\end{english}
الناتج: \textenglish{Total = 150}
\end{boxExample}

\begin{boxExample}[33 - عملية لحساب المعدل]
\begin{english}
\begin{minted}{csharp}
public static double GetAverage(double[] arr)
{
    double sum = 0;
    for (int i = 0; i < arr.Length; i++)
    {
        sum += arr[i];
    }
    return sum / arr.Length;
}

public static void Main()
{
    double[] grades = { 85.5, 92.0, 78.5, 88.0 };
    double avg = GetAverage(grades);
    Console.WriteLine("Average = " + avg);
}
\end{minted}
\end{english}
الناتج: \textenglish{Average = 86.0}
\end{boxExample}

\begin{boxExample}[34 - عملية للبحث]
\begin{english}
\begin{minted}{csharp}
public static int FindElement(int[] arr, int value)
{
    for (int i = 0; i < arr.Length; i++)
    {
        if (arr[i] == value)
            return i;
    }
    return -1;
}

public static void Main()
{
    int[] numbers = { 5, 12, 8, 3, 15 };
    int pos = FindElement(numbers, 8);

    if (pos != -1)
        Console.WriteLine($"Found at index {pos}");
    else
        Console.WriteLine("Not found");
}
\end{minted}
\end{english}
الناتج: \textenglish{Found at index 2}
\end{boxExample}

\subsection{إرجاع مصفوفة من عملية}

\begin{boxExample}[35 - عملية تُرجع مصفوفة]
\begin{english}
\begin{minted}{csharp}
public static int[] CreateArray(int size, int value)
{
    int[] arr = new int[size];
    for (int i = 0; i < arr.Length; i++)
    {
        arr[i] = value;
    }
    return arr;
}

public static void Main()
{
    int[] numbers = CreateArray(5, 10);
    PrintArray(numbers);
}
\end{minted}
\end{english}
الناتج: \textenglish{[ 10, 10, 10, 10, 10 ]}
\end{boxExample}

\begin{boxExample}[36 - عملية تُنشئ مصفوفة معكوسة]
\begin{english}
\begin{minted}{csharp}
public static int[] ReverseArray(int[] arr)
{
    int[] reversed = new int[arr.Length];
    for (int i = 0; i < arr.Length; i++)
    {
        reversed[i] = arr[arr.Length - 1 - i];
    }
    return reversed;
}

public static void Main()
{
    int[] original = { 1, 2, 3, 4, 5 };
    int[] reversed = ReverseArray(original);

    Console.Write("Original: ");
    PrintArray(original);

    Console.Write("Reversed: ");
    PrintArray(reversed);
}
\end{minted}
\end{english}
الناتج:
\begin{english}
\begin{boxCode}
Original: [ 1, 2, 3, 4, 5 ] \\
Reversed: [ 5, 4, 3, 2, 1 ]
\end{boxCode}
\end{english}
\end{boxExample}

\begin{boxNote}
\textbf{ملاحظة مهمة:} عند تمرير مصفوفة لعملية خارجية، التعديلات داخل العملية تؤثر على المصفوفة الأصلية.
\end{boxNote}

\begin{boxExample}[37 - التعديل داخل العملية]
\begin{english}
\begin{minted}{csharp}
public static void DoubleValues(int[] arr)
{
    for (int i = 0; i < arr.Length; i++)
    {
        arr[i] *= 2;
    }
}

public static void Main()
{
    int[] numbers = { 1, 2, 3, 4, 5 };

    Console.Write("Before: ");
    PrintArray(numbers);

    DoubleValues(numbers);

    Console.Write("After: ");
    PrintArray(numbers);
}
\end{minted}
\end{english}
الناتج:
\begin{english}
\begin{boxCode}
Before: [ 1, 2, 3, 4, 5 ] \\
After: [ 2, 4, 6, 8, 10 ]
\end{boxCode}
\end{english}
\end{boxExample}

\clearpage

\section{عمليات إضافية متقدمة}

\subsection{فحص التماثل \textenglish{(Palindrome)}}

\begin{boxExample}[38 - فحص المصفوفة المتماثلة]
\begin{english}
\begin{minted}{csharp}
public static bool IsPalindrome(int[] arr)
{
    for (int i = 0; i < arr.Length / 2; i++)
    {
        if (arr[i] != arr[arr.Length - 1 - i])
            return false;
    }
    return true;
}

public static void Main()
{
    int[] arr1 = { 1, 2, 3, 2, 1 };
    int[] arr2 = { 1, 2, 3, 4, 5 };

    Console.WriteLine(IsPalindrome(arr1));
    Console.WriteLine(IsPalindrome(arr2));
}
\end{minted}
\end{english}
الناتج:
\begin{english}
\begin{boxCode}
True \\
False
\end{boxCode}
\end{english}
\end{boxExample}

\subsection{دمج مصفوفتين}

\begin{boxExample}[39 - دمج مصفوفتين متتاليتين]
\begin{english}
\begin{minted}{csharp}
public static int[] MergeArrays(int[] arr1, int[] arr2)
{
    int[] result = new int[arr1.Length + arr2.Length];

    for (int i = 0; i < arr1.Length; i++)
    {
        result[i] = arr1[i];
    }

    for (int i = 0; i < arr2.Length; i++)
    {
        result[arr1.Length + i] = arr2[i];
    }

    return result;
}

public static void Main()
{
    int[] A = { 1, 2, 3 };
    int[] B = { 4, 5, 6 };
    int[] merged = MergeArrays(A, B);

    PrintArray(merged);
}
\end{minted}
\end{english}
الناتج: \textenglish{[ 1, 2, 3, 4, 5, 6 ]}
\end{boxExample}

\begin{boxExample}[40 - دمج بالتناوب]
عنصر من الأولى، عنصر من الثانية، وهكذا:
\begin{english}
\begin{minted}{csharp}
public static int[] AlternateMerge(int[] arr1, int[] arr2)
{
    int[] result = new int[arr1.Length + arr2.Length];

    for (int i = 0; i < arr1.Length; i++)
    {
        result[2 * i] = arr1[i];
        result[2 * i + 1] = arr2[i];
    }

    return result;
}

public static void Main()
{
    int[] A = { 10, 30, 50 };
    int[] B = { 20, 40, 60 };
    int[] merged = AlternateMerge(A, B);

    PrintArray(merged);
}
\end{minted}
\end{english}
الناتج: \textenglish{[ 10, 20, 30, 40, 50, 60 ]}
\end{boxExample}

\subsection{إيجاد ثاني أكبر عنصر}

\begin{boxExample}[41 - إيجاد ثاني أكبر عنصر]
\begin{english}
\begin{minted}{csharp}
public static int FindSecondMax(int[] arr)
{
    int max1 = int.MinValue;
    int max2 = int.MinValue;

    for (int i = 0; i < arr.Length; i++)
    {
        if (arr[i] > max1)
        {
            max2 = max1;
            max1 = arr[i];
        }
        else if (arr[i] > max2 && arr[i] != max1)
        {
            max2 = arr[i];
        }
    }

    return max2;
}

public static void Main()
{
    int[] numbers = { 5, 12, 8, 20, 15, 3, 7 };
    int secondMax = FindSecondMax(numbers);
    Console.WriteLine("Second Maximum = " + secondMax);
}
\end{minted}
\end{english}
الناتج: \textenglish{Second Maximum = 15}
\end{boxExample}

\clearpage

\section{مصفوفات من أنواع أخرى}

\subsection{مصفوفة نصوص}

\begin{boxExample}[42 - مصفوفة أسماء]
\begin{english}
\begin{minted}{csharp}
string[] names = { "Ali", "Sara", "Omar", "Mona" };

for (int i = 0; i < names.Length; i++)
{
    Console.WriteLine($"Student {i + 1}: {names[i]}");
}
\end{minted}
\end{english}
الناتج:
\begin{english}
\begin{boxCode}
Student 1: Ali \\
Student 2: Sara \\
Student 3: Omar \\
Student 4: Mona
\end{boxCode}
\end{english}
\end{boxExample}

\begin{boxExample}[43 - البحث في مصفوفة نصوص]
\begin{english}
\begin{minted}{csharp}
string[] colors = { "Red", "Green", "Blue", "Yellow" };
string search = "Blue";
int position = -1;

for (int i = 0; i < colors.Length; i++)
{
    if (colors[i] == search)
    {
        position = i;
        break;
    }
}

if (position != -1)
    Console.WriteLine($"{search} found at index {position}");
else
    Console.WriteLine($"{search} not found");
\end{minted}
\end{english}
الناتج: \textenglish{Blue found at index 2}
\end{boxExample}

\subsection{مصفوفة حروف}

\begin{boxExample}[44 - مصفوفة أحرف]
\begin{english}
\begin{minted}{csharp}
char[] letters = { 'A', 'B', 'C', 'D', 'E' };

for (int i = 0; i < letters.Length; i++)
{
    Console.WriteLine(letters[i]);
}
\end{minted}
\end{english}
الناتج:
\begin{english}
\begin{boxCode}
A \\
B \\
C \\
D \\
E
\end{boxCode}
\end{english}
\end{boxExample}

\begin{boxExample}[45 - عد حرف معين]
\begin{english}
\begin{minted}{csharp}
char[] text = { 'H', 'e', 'l', 'l', 'o' };
char search = 'l';
int count = 0;

for (int i = 0; i < text.Length; i++)
{
    if (text[i] == search)
        count++;
}

Console.WriteLine($"Letter '{search}' appears {count} times");
\end{minted}
\end{english}
الناتج: \textenglish{Letter 'l' appears 2 times}
\end{boxExample}

\subsection{مصفوفة قيم منطقية}

\begin{boxExample}[46 - عد القيم المنطقية الصحيحة]
\begin{english}
\begin{minted}{csharp}
bool[] answers = { true, false, true, true, false };
int correctCount = 0;

for (int i = 0; i < answers.Length; i++)
{
    if (answers[i])
        correctCount++;
}

Console.WriteLine($"Correct answers: {correctCount}");
\end{minted}
\end{english}
الناتج: \textenglish{Correct answers: 3}
\end{boxExample}

\clearpage

\section{أخطاء شائعة}

\begin{boxAttention}[1 - نسيان أن الفهارس تبدأ من 0]
\begin{english}
\begin{minted}{csharp}
int[] arr = new int[5];
arr[5] = 10;
\end{minted}
\end{english}
\textbf{خطأ!} الفهارس من 0 إلى 4 فقط.
\\[0.5em]
\textbf{الصحيح:}
\begin{english}
\begin{minted}{csharp}
arr[4] = 10;
\end{minted}
\end{english}
\end{boxAttention}

\begin{boxAttention}[2 - استخدام = بدلاً من == في الشروط]
\begin{english}
\begin{minted}{csharp}
if (arr[i] = 10)
\end{minted}
\end{english}
\textbf{خطأ!} هذا تعيين وليس مقارنة.
\\[0.5em]
\textbf{الصحيح:}
\begin{english}
\begin{minted}{csharp}
if (arr[i] == 10)
\end{minted}
\end{english}
\end{boxAttention}

\begin{boxAttention}[3 - نسيان تهيئة المصفوفة]
\begin{english}
\begin{minted}{csharp}
int[] arr;
arr[0] = 10;
\end{minted}
\end{english}
\textbf{خطأ!} المصفوفة غير مُنشأة.
\\[0.5em]
\textbf{الصحيح:}
\begin{english}
\begin{minted}{csharp}
int[] arr = new int[5];
arr[0] = 10;
\end{minted}
\end{english}
\end{boxAttention}

\begin{boxAttention}[4 - استخدام \textenglish{Length} بدون أقواس صحيحة]
\begin{english}
\begin{minted}{csharp}
for (int i = 0; i <= arr.Length; i++)
\end{minted}
\end{english}
\textbf{خطأ!} سيتجاوز حدود المصفوفة.
\\[0.5em]
\textbf{الصحيح:}
\begin{english}
\begin{minted}{csharp}
for (int i = 0; i < arr.Length; i++)
\end{minted}
\end{english}
\end{boxAttention}

\begin{boxAttention}[5 - محاولة تغيير الحجم]
\begin{english}
\begin{minted}{csharp}
int[] arr = new int[5];
arr.Length = 10;
\end{minted}
\end{english}
\textbf{خطأ!} حجم المصفوفة ثابت ولا يمكن تغييره.
\\[0.5em]
\textbf{الحل:} إنشاء مصفوفة جديدة بالحجم المطلوب.
\end{boxAttention}

\end{document}
