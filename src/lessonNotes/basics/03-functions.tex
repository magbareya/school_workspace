\documentclass[12pt]{article}
% Full article preamble (duplicated, no common file)
\usepackage{fontspec}
\usepackage[a4paper,top=2.4cm,bottom=2.4cm,left=2.3cm,right=2.3cm]{geometry}
\usepackage{polyglossia}
\usepackage{amsmath}
\usepackage{amssymb}
\usepackage{xcolor}
\usepackage{fancyhdr}
\usepackage{graphicx}
\usepackage{listings}
\usepackage[most]{tcolorbox}
\usepackage{pifont}
\usepackage{enumitem}
\usepackage{titlesec}
\usepackage[bottom]{footmisc}
\usepackage{titling}
\usepackage{minted}
\usepackage{etoolbox}
\usepackage{array}
\usepackage{extsizes}

\newfontfamily\emoji{Segoe UI Emoji}

\pagestyle{fancy}

\setmainlanguage[numerals=western]{arabic}
\setotherlanguage{english}
\newfontfamily\arabicfont[Script=Arabic]{Amiri}
\newfontfamily\arabicfonttt[Script=Arabic]{Courier New}

\lstset{
  language=[Sharp]C,
  numbers=left,
  stepnumber=1,
  numbersep=8pt,
  frame=single,
  basicstyle=\ttfamily\small,
  keywordstyle=\color{blue},
  stringstyle=\color{red},
  commentstyle=\color{green!50!black}
}

\newif\ifdetailed
\ifdefined\setdetailed
  \setdetailed
\fi

\newif\ifwithsols
\ifdefined\setwithsols
  \setwithsols
\fi

% unified tcolorboxes for articles
\tcbset{colback=white, colframe=black, fonttitle=\bfseries, boxrule=0.8pt}
\newtcolorbox{boxDef}[1][]{colback=blue!5!white,colframe=blue!75!black,
  title={{\emoji📘} تعريف\ifx\\#1\\\else ~#1\fi :}}
\newtcolorbox{boxExercise}[1][]{colback=cyan!5!white,colframe=cyan!70!black,
  title={{\emoji🧩} تمرين\ifx\\#1\\\else ~#1\fi :}}
\newtcolorbox{boxExample}[1][]{colback=yellow!5!white,colframe=orange!90!black,
  title={{\emoji📝} مثال\ifx\\#1\\\else ~#1\fi :}}
\newtcolorbox{boxNote}[1][]{colback=gray!10!white,colframe=black,
  title={{\emoji✨} ملاحظة\ifx\\#1\\\else ~#1\fi :}}
\newtcolorbox{boxAttention}[1][]{colback=magenta!10!white,colframe=magenta!80!black,
  title={{\emoji🔔} تنبيه\ifx\\#1\\\else ~#1\fi :}}
\newtcolorbox{boxWarning}[1][]{colback=red!5!white,colframe=red!75!black,
  title={{\emoji⚡} ملاحظة هامة\ifx\\#1\\\else ~#1\fi :}}
\newtcolorbox{boxSolution}[1][]{colback=green!5!white,colframe=green!60!black,
  title={{\emoji✅} حل\ifx\\#1\\\else ~#1\fi :}}
\newtcolorbox{boxSymbol}[1][]{colback=purple!5!white,colframe=purple!70!black,
  title={{\emoji🔣} رمز\ifx\\#1\\\else ~#1\fi :}}
\newtcolorbox{boxHint}[1][]{colback=teal!5!white,colframe=teal!60!black,
  title={{\emoji💡} تلميح\ifx\\#1\\\else ~#1\fi :}}


\tcbset{simplecode/.style={ colback=gray!5, colframe=black!50, boxrule=0.4pt, arc=2pt, left=4pt,right=4pt,top=4pt,bottom=4pt}}
\newenvironment{boxCode}{\begin{tcolorbox}[simplecode]}{\end{tcolorbox}}

\newcolumntype{C}[1]{>{\centering\arraybackslash}p{#1}}

% redefine spaces after titles
\makeatletter
\renewcommand{\@maketitle}{%
  \begin{center}
    {\huge \bfseries \@title \par}%
    \vskip 0.2em % space between title and author
    {\large \@author \par}%
    % \vskip 0.2em % space between author and date
    % {\normalsize \@date \par}%
  \end{center}
}
\makeatother

\fancyhf{} % clear default
\fancypagestyle{plain}{
  \fancyhf{}
  \fancyhead[L]{مدرسة التسامح الشاملة}
  % \fancyhead[L]{\includegraphics[height=1cm]{../../../images/logoTasamoh.png}}
  \fancyhead[R]{الأستاذ محمود اغبارية}
  \fancyfoot[C]{\thepage}
}

\fancyhead[L]{مدرسة التسامح الشاملة}
\fancyhead[R]{الأستاذ محمود اغبارية}
\fancyfoot[C]{\thepage}
% \date{\today}

\setcounter{tocdepth}{3} % only section subsection and subsubsection in TOC


% ----------------------


% \begin{document}

% \maketitle

% % \clearpage  % start TOC on a new page
% % \renewcommand{\contentsname}{جدول المحتويات}
% % \tableofcontents
% % \clearpage

% \part*{part 1} % the * prevents numbering
% \section*{مقدمة}
% \subsection*{مثال رياضي}
% \subsubsection*{مثال فرعي}
% \paragraph*{ paragraph 1}
% \subparagraph*{sub paragraph 1}

% \ifdetailed
% \begin{english}
% \begin{minted}{csharp}
% // C# Example
% \end{minted}
% \end{english}
% \fi

% OLD WAY
% \ifdetailed
% \begin{english}
% \begin{lstlisting}
% // C# Example
% \end{lstlisting}
% \end{english}
% \fi

% % \includegraphics[width=0.2\textwidth]{../../../images/DFAs/ex1_q1.png}



% \vspace{3cm}
% \begin{flushleft}
% أرجو لكم وقتًا ممتعًا.

% الأستاذ محمود اغبارية.
% \end{flushleft}


% \end{document}


\title{العمليات الخارجية في لغة \textenglish{C\#} }

\begin{document}
\maketitle

\section*{مقدمة}
العمليات هي وسيلة لتقسيم البرنامج إلى مقاطع صغيرة تؤدي مهام محددة.
تساعد العمليات على:
\begin{itemize}
    \item تنظيم الكود وجعله أسهل للقراءة والصيانة.
    \item إعادة استخدام نفس الكود أكثر من مرة.
\end{itemize}

في هذا الدرس سنتعرّف على أنواع العمليات من حيث استقبال المعطيات (البارامترات) وإرجاع القيم، وكيف نعرّفها داخل الصف.

\section{مبنى تعريف العملية}
كل عملية في لغة C\# تتكوّن من قسمين رئيسيين:

\begin{enumerate}
    \item \textbf{تعريف العملية} \textenglish{(Method Signature)}
    وهي السطر الأول الذي يحتوي على:
    \begin{itemize}
        \item نوع العملية، في هذه المرحلة دائمًا سوف نستعمل: \textenglish{public static}
        \item نوع القيمة التي تعيدها (\textenglish{void, int, double, bool, string, char})
        \item اسم العملية. ومن المتبع أنّ اسم العملية يبدأ دائمًا بحرف كبير
        \item البارامترات بين الأقواس (مثل \textenglish{(int x, int y)})
    \end{itemize}

    \item \textbf{جسم العملية} \textenglish{(Function Body)}
    وهو الكود الموجود بين القوسين \textenglish{\{ \}}، ويحتوي على التعليمات التي تُنفّذ عند استدعاء العملية.
\end{enumerate}

\begin{boxExample}[1]
\begin{english}
\begin{minted}{csharp}
public static void PrintHello()
{
    Console.WriteLine("Hello");
}
\end{minted}
\end{english}
\hrule
\begin{itemize}
    \item \textenglish{public static} $\leftarrow$ جزء ثابت.
    \item \textenglish{void} $\leftarrow$ العملية لا ترجع شيئًا.
    \item \textenglish{PrintHello} $\leftarrow$ اسم العملية.
    \item \textenglish{()} $\leftarrow$ الأقواس الفارغة بعد اسم العملية معناها أنّ هذه العملية لا تتلقى بارامترات.
    \item ما بين الأقواس \textenglish{\{ \}} هو جسم العملية، وفي هذه الحالة يطبع رسالة \textenglish{Hello}.
\end{itemize}
\end{boxExample}

\begin{boxExample}[2]
\begin{english}
\begin{minted}{csharp}
public static int AddNumbers(int a, int b)
{
    int sum = a + b;
    return sum;
}
\end{minted}
\end{english}
\hrule
\begin{itemize}
    \item \textenglish{public static} $\leftarrow$ جزء ثابت.
    \item \textenglish{int} $\leftarrow$ نوع القيمة المرجعة.
    \item \textenglish{AddNumbers} $\leftarrow$ اسم العملية.
    \item \textenglish{(int a, int b)} $\leftarrow$ البارامترات التي تتلقاها العملية، في هذه الحالة تتلقى متغيرين كلاهما من نمط \textenglish{int}.
    \item ما بين الأقواس \textenglish{\{ \}} هو جسم العملية، الذي ينفّذ عملية الجمع ويعيد النتيجة.
    \item نستخدم كلمة \textenglish{return} لننهي العملية ونعيد القيمة المطلوبة. \\
    \textbf{ملاحظة:} نوع القيمة التي ترجعها العملية يجب أن يطابق نوع القيمة بعد كلمة \textenglish{return}.
\end{itemize}
\end{boxExample}

\section{استدعاء العملية}
بشكل عام، لكي نستدعي عملية خارجية نكتب اسم الفئة، ثم نقطة ثم اسم العملية.

في البداية سوف نتعامل مع عمليات معرفة داخل الفئة التي نعمل فيها \textenglish{Program}. \\
في هذه الحالة لا حاجة لكتابة اسم الفئة، نكتب فقط اسم العملية.

مثلًا، لكي نستدعي العمليتين اللتين عرفناهما أعلاه:

\begin{boxExample}
\begin{english}
\begin{minted}{csharp}
public static void Main(string[] args)
{
    PrintHello();
    int sum1 = AddNumbers(5, 10);
    int a = 5;
    int b = 10;
    int sum2 = AddNumbers(a, b);
}
\end{minted}
\end{english}
\end{boxExample}


\begin{itemize}
    \item العملية \textenglish{PrintHello()} لا ترجع أي قيمة، لذلك لم نحتج أن نحفظها في متغير.
    \item العملية \textenglish{AddNumbers(int a, int b)} ترجع قيمة، لذلك نحفظها في متغير \textenglish{sum}.
    \item المتغير الذي نحفظ فيه قيمة العملية يجب أن يكون من نفس النوع الذي ترجعه العملية.
    \item البارامترات التي نعطيها للعملية يجب أن تكون من نفس النوع الموجود في تعريف العملية وبنفس الترتيب.
\end{itemize}

\subsection*{استدعاء عملية من فئة أخرى}
كما ذكرنا أعلاه، في هذه الحالة علينا أن نكتب اسم الفئة، ثم نقطة ثم اسم العملية. \\
نحن نعرف الكثير من هذه العمليات:
\begin{itemize}
    \item \textenglish{Console.WriteLine("Hello World!");} \\
    هذا استدعاء للعملية \textenglish{WriteLine} من الفئة \textenglish{Console}
    \item \textenglish{Console.ReadLine();} \\
    هذا استدعاء للعملية \textenglish{ReadLine} من الفئة \textenglish{Console}
    \item \textenglish{Math.Max(5, 10);} \\
    هذا استدعاء للعملية \textenglish{Max} من الفئة \textenglish{Math}
    \item \textenglish{Math.Min(5, 10);} \\
    هذا استدعاء للعملية \textenglish{Min} من الفئة \textenglish{Math}
    \item \textenglish{Math.Abs(-5); } \\
    هذا استدعاء للعملية \textenglish{Abs} من الفئة \textenglish{Math}
    \item \textenglish{Math.Pow(2, 3); } \\
    هذا استدعاء للعملية \textenglish{Pow} من الفئة \textenglish{Math}
    \item \textenglish{Math.Sqrt(25); } \\
    هذا استدعاء للعملية \textenglish{Sqrt} من الفئة \textenglish{Math}
    \item \textenglish{Math.Round(2.5); } \\
    هذا استدعاء للعملية \textenglish{Round} من الفئة \textenglish{Math}
\end{itemize}

\clearpage
\section{أنواع العمليات}

\subsection{لا ترجع ولا تتلقى بارامترات}
نستخدمها عندما نريد تنفيذ أمر محدد دون الحاجة لإدخال أو إخراج بيانات.

\begin{boxExample}
\begin{english}
\begin{minted}{csharp}
public static void PrintWelcome()
{
    Console.WriteLine("Welcome to the program!");
}

public static void Main()
{
    PrintWelcome();
}
\end{minted}
\end{english}
\end{boxExample}

عملية بسيطة تطبع رسالة فقط. لا تحتاج لأي معطيات ولا ترجع أي نتيجة.

\subsection{تتلقى ولا ترجع}
تتلقى بارامترات لكنها لا تعيد قيمة.

\begin{boxExample}[1]
\begin{english}
\begin{minted}{csharp}
public static void PrintSum(int a, int b)
{
    int sum = a + b;
    Console.WriteLine("Sum = " + sum);
}

public static void Main()
{
    PrintSum(5, 3);
}
\end{minted}
\end{english}
\end{boxExample}

العملية تتلقى عددين وتطبع ناتج جمعهما فقط، دون إرجاع النتيجة. \\

يمكننا أيضًا استخدام متغيرات كبرامترات عند استدعاء العملية:
\begin{boxExample}[2]
\begin{english}
\begin{minted}{csharp}
public static void PrintSum(int a, int b)
{
    int sum = a + b;
    Console.WriteLine("Sum = " + sum);
}

public static void Main()
{
    int x = 5;
    int y = 3;
    PrintSum(x, y);
}
\end{minted}
\end{english}
\end{boxExample}

\textbf{مثال لعملية كهذه نعرفها:} \\
\textenglish{Console.WriteLine()} \\
هي عملية تتلقى بارامتر واحد من نمط \textenglish{string} ولا تعيد أي قيمة. \\
بما أنّ هذه العملية معرّفة في فئة أخرى غير، فعلينا كتابة اسم الفئة في هذه الحالة \textenglish{Console}.

\subsection{ترجع ولا تتلقى}
تقوم بإرجاع نتيجة معينة دون الحاجة لمعطيات.

\begin{boxExample}
\begin{english}
\begin{minted}{csharp}
public static int ReadNumber()
{
    int num = int.Parse(Console.ReadLine());
    return num;
}

public static void Main()
{
    int x = ReadNumber();
    Console.WriteLine("The number: " + x);
}
\end{minted}
\end{english}
\end{boxExample}

العملية لا تحتاج مدخلات، لكنها ترجع عددًا يمكننا استخدامه في البرنامج.

\textbf{مثال لعملية كهذه نعرفها:} \\
\textenglish{Console.ReadLine()} \\
هي عملية لا تتلقى أي بارامتر، لكنها تعيد قيمة من نمط \textenglish{string}. \\
بما أنّ هذه العملية معرّفة في فئة أخرى غير، فعلينا كتابة اسم الفئة في هذه الحالة \textenglish{Console}.

\subsection{تتلقى وترجع}
تتلقى معطيات وتعيد ناتجًا جديدًا بعد معالجتها.

\begin{boxExample}
\begin{english}
\begin{minted}{csharp}
public static double GetAverage(double a, double b)
{
    double avg = (a + b) / 2.0;
    return avg;
}

public static void Main()
{
    double result = GetAverage(7.5, 9.0);
    Console.WriteLine("Average = " + result);
}
\end{minted}
\end{english}
\end{boxExample}

العملية تتلقى عددين وتحسب معدلهما ثم تعيد النتيجة.

\subsection*{أمثلة نعرفها}
\begin{itemize}
    \item \textenglish{Math.Max} و \textenglish{Math.Min} هما عمليتان تتلقى كل واحدة منهما عددين من نمط \textenglish{int} وترجع قيمة من نمط \textenglish{int}.
    \item \textenglish{Math.Abs} و \textenglish{Math.Pow} هما عمليتان تتلقى كل واحدة منهما عددين من نمط \textenglish{double} وترجع قيمة من نمط \textenglish{double}.
    \item \textenglish{Math.Round} و \textenglish{Math.Sqrt} هما عمليتان تتلقى كل واحدة منهما عددًا من نمط \textenglish{double} وترجع قيمة من نمط \textenglish{double}.
    \item \textenglish{int.Parse} هي عملية تتلقى بارامترًا واحدًا من نمط \textenglish{string} وترجع قيمة من نمط \textenglish{int}.
    \item \textenglish{double.Parse} هي عملية تتلقى بارامترًا واحدًا من نمط \textenglish{string} وترجع قيمة من نمط \textenglish{double}.
\end{itemize}

\section{ملاحظات مهمة}

\begin{itemize}
\item
    يُفضّل أن يبدأ اسم العملية بحرف كبير وأن يصف وظيفتها بوضوح. \\
وإذا كان اسمها مكونًا من أكثر من كلمة، فكل كلمة تبدأ بحرف كبير
\begin{itemize}
    \item \textenglish{CalculateSum()}
    \item \textenglish{GetMaxValue()}
    \item \textenglish{PrintResult()}
\end{itemize}

\item اجعل كل عملية تقوم بمهمة واحدة فقط.
\item لا تكرر الكود - استخدم العمليات بدل النسخ.
\item تأكد من اختبار العملية جيدًا قبل دمجها مع الباقي.
\item داخل كل عملية يمكنك إضافة كل الأوامر التي تعرفها، تماما كما تكتب في عملية الـ \textenglish{Main}. \\
مثلًا يمكن استدعاء عملية من داخل عملية أخرى. \\
يمكنك أيضًا إضافة جمل شرطية \textenglish{if..else} في العملية.
\item يمكن تمرير القيمة التي ترجعها عملية كبرامتر لعملية أخرى.

\end{itemize}

\clearpage
\section{أمثلة متنوعة على العمليات}

في هذا القسم سنعرض مجموعة من الأمثلة القصيرة المتنوعة على تعريف العمليات واستدعائها.

\subsection{عملية لا تتلقى ولا ترجع (\textenglish{void})}
\begin{boxExample}
\begin{english}
\begin{minted}{csharp}
public static void SayHi()
{
    Console.WriteLine("Hi!");
}

public static void Main()
{
    SayHi();
}
\end{minted}
\end{english}
\end{boxExample}

تطبع رسالة بسيطة عند استدعائها. \\
- لا تتلقى $\leftarrow$ نستدعيها بدون بارامترات. \\
- لا ترجع (\textenglish{void}) $\leftarrow$ لا يوجد فيها \textenglish{return}، ولا نحفظ القيمة في متغير عند استدعائها.

\subsection{عملية تتلقى ولا ترجع (\textenglish{void})}
\begin{boxExample}
\begin{english}
\begin{minted}{csharp}
public static void PrintSquare(int n)
{
    Console.WriteLine("Square = " + (n * n));
}

public static void Main()
{
    PrintSquare(5);
}
\end{minted}
\end{english}
\end{boxExample}

تتلقى عددًا وتطبع مربعه فقط دون إرجاع أي قيمة. \\
- تتلقى $\leftarrow$ نستدعيها مع بارامترات بنفس النوع المطلوب في تعريفها. \\
- لا ترجع (\textenglish{void}) $\leftarrow$ لا يوجد فيها \textenglish{return}، ولا نحفظ القيمة في متغير عند استدعائها.



\subsection{عملية لا تتلقى ولكن ترجع عددًا صحيحًا (\textenglish{int})}
\begin{boxExample}
\begin{english}
\begin{minted}{csharp}
public static int GetNumber()
{
    return 42;
}

public static void Main()
{
    int x = GetNumber();
    Console.WriteLine("The number is " + x);
}
\end{minted}
\end{english}
\end{boxExample}

ترجع عددًا ثابتًا بدون تلقي أي بارامتر. \\
- لا تتلقى $\leftarrow$ نستدعيها بدون بارامترات. \\
- ترجع $\leftarrow$ يوجد فيها \textenglish{return}، وبعده تعبير من نفس نوع الترجيع (\textenglish{int})،
  وعند استدعائها نحفظ القيمة المرجعة في متغير من نفس النوع.



\subsection{عملية تتلقى وترجع عددًا صحيحًا (\textenglish{int})}
\begin{boxExample}
\begin{english}
\begin{minted}{csharp}
public static int Multiply(int a, int b)
{
    return a * b;
}

public static void Main()
{
    int result = Multiply(6, 7);
    Console.WriteLine("Result = " + result);
}
\end{minted}
\end{english}
\end{boxExample}

تتلقى عددين وتعيد ناتج ضربهما. \\
- تتلقى $\leftarrow$ نستدعيها مع بارامترات بنفس النوع المطلوب في تعريفها. \\
- ترجع $\leftarrow$ يوجد فيها \textenglish{return}، وبعده تعبير من نوع \textenglish{int}،
  وعند استدعائها نحفظ القيمة المرجعة في متغير من نفس النوع.



\subsection{عملية تتلقى وترجع عددًا عشريًا (\textenglish{double})}
\begin{boxExample}
\begin{english}
\begin{minted}{csharp}
public static double CircleArea(double radius)
{
    return Math.PI * radius * radius;
}

public static void Main()
{
    double area = CircleArea(2.5);
    Console.WriteLine("Area = " + area);
}
\end{minted}
\end{english}
\end{boxExample}

تحسب مساحة دائرة من نصف قطر معطى وتعيدها. \\
- تتلقى $\leftarrow$ نستدعيها مع بارامتر من نفس النوع (\textenglish{double}). \\
- ترجع $\leftarrow$ يوجد فيها \textenglish{return}، وبعده تعبير من نوع \textenglish{double}،
  وعند استدعائها نحفظ القيمة المرجعة في متغير من نفس النوع.



\subsection{عملية تتلقى وترجع قيمة منطقية (\textenglish{bool})}
\begin{boxExample}
\begin{english}
\begin{minted}{csharp}
public static bool IsEven(int num)
{
    return num % 2 == 0;
}

public static void Main()
{
    bool even = IsEven(8);
    Console.WriteLine("Even? " + even);
}
\end{minted}
\end{english}
\end{boxExample}

تتحقق مما إذا كان العدد زوجيًا وتعيد \textenglish{true} أو \textenglish{false}. \\
- تتلقى $\leftarrow$ نستدعيها مع بارامتر من نفس النوع (\textenglish{int}). \\
- ترجع $\leftarrow$ يوجد فيها \textenglish{return}، وبعده تعبير من نوع \textenglish{bool}،
  وعند استدعائها نحفظ القيمة المرجعة في متغير من نفس النوع.



\subsection{عملية تتلقى وترجع نصًا (\textenglish{string})}
\begin{boxExample}
\begin{english}
\begin{minted}{csharp}
public static string Greet(string name)
{
    return "Hello " + name + "!";
}

public static void Main()
{
    string msg = Greet("Ali");
    Console.WriteLine(msg);
}
\end{minted}
\end{english}
\end{boxExample}

تتلقى اسمًا وتعيد تحية نصية. \\
- تتلقى $\leftarrow$ نستدعيها مع بارامتر من نفس النوع (\textenglish{string}). \\
- ترجع $\leftarrow$ يوجد فيها \textenglish{return}، وبعده تعبير من نوع \textenglish{string}،
  وعند استدعائها نحفظ القيمة المرجعة في متغير من نفس النوع.



\subsection{عملية لا تتلقى وترجع نصًا (\textenglish{string})}
\begin{boxExample}
\begin{english}
\begin{minted}{csharp}
public static string GetDay()
{
    return "Friday";
}

public static void Main()
{
    Console.WriteLine("Today is " + GetDay());
}
\end{minted}
\end{english}
\end{boxExample}

ترجع نصًا ثابتًا دون تلقي أي بارامتر. \\
- لا تتلقى $\leftarrow$ نستدعيها بدون بارامترات. \\
- ترجع $\leftarrow$ يوجد فيها \textenglish{return}، وبعده تعبير من نفس النوع (\textenglish{string}).



\subsection{عملية تتلقى عددًا وترجع قيمة منطقية (\textenglish{bool})}
\begin{boxExample}
\begin{english}
\begin{minted}{csharp}
public static bool IsPositive(int n)
{
    return n > 0;
}

public static void Main()
{
    Console.WriteLine(IsPositive(10));
    Console.WriteLine(IsPositive(-3));
}
\end{minted}
\end{english}
\end{boxExample}

تتحقق إن كان العدد موجبًا أو سالبًا. \\
- تتلقى $\leftarrow$ نستدعيها مع بارامتر من نفس النوع (\textenglish{int}). \\
- ترجع $\leftarrow$ يوجد فيها \textenglish{return}، وبعده تعبير من نفس النوع (\textenglish{bool}).



\subsection*{10. عملية لا ترجع وتتلقى أكثر من بارامتر من أنواع مختلفة}
\begin{boxExample}
\begin{english}
\begin{minted}{csharp}
public static void PrintPerson(string name, int age)
{
    Console.WriteLine(name + " is " + age + " years old.");
}

public static void Main()
{
    PrintPerson("Mona", 25);
}
\end{minted}
\end{english}
\end{boxExample}

تطبع جملة تحتوي على الاسم والعمر دون إرجاع أي قيمة. \\
- تتلقى $\leftarrow$ نستدعيها مع بارامترات بنفس الأنواع المذكورة في تعريفها. \\
- لا ترجع (\textenglish{void}) $\leftarrow$ لا يوجد فيها \textenglish{return}، ولا نحفظ القيمة في متغير عند استدعائها.



\subsection*{11. عملية تستخدم شرطًا بسيطًا داخلها وترجع نتيجة (\textenglish{string})}
\begin{boxExample}
\begin{english}
\begin{minted}{csharp}
public static string CheckGrade(int grade)
{
    if (grade >= 60)
        return "Pass";
    else
        return "Fail";
}

public static void Main()
{
    Console.WriteLine(CheckGrade(85));
    Console.WriteLine(CheckGrade(40));
}
\end{minted}
\end{english}
\end{boxExample}

تتحقق إن كان الطالب قد نجح أو رسب. \\
- تتلقى $\leftarrow$ نستدعيها مع بارامتر من نفس النوع (\textenglish{int}). \\
- ترجع $\leftarrow$ يوجد فيها \textenglish{return} في كل حالة، وبعده تعبير من نوع \textenglish{string}.



\subsection*{12. عملية تتلقى عددين عشريين وترجع عددًا عشريًا (قسمة آمنة)}
\begin{boxExample}
\begin{english}
\begin{minted}{csharp}
public static double SafeDivide(double a, double b)
{
    if (b == 0)
        return 0;
    return a / b;
}

public static void Main()
{
    Console.WriteLine(SafeDivide(10, 2));
    Console.WriteLine(SafeDivide(5, 0));
}
\end{minted}
\end{english}
\end{boxExample}

تنفّذ قسمة وتتحقق من القسمة على صفر قبل الإرجاع. \\
- تتلقى $\leftarrow$ نستدعيها مع بارامترين من نفس النوع (\textenglish{double}). \\
- ترجع $\leftarrow$ يوجد فيها \textenglish{return}، وبعده تعبير من نفس النوع (\textenglish{double})،
  وعند استدعائها يمكننا حفظ القيمة في متغير من نفس النوع.


\end{document}
