\documentclass[13pt]{beamer}
% Fonts and languages
\usepackage{fontspec}
\usepackage{polyglossia}
\usepackage{bidi}
\usepackage{listings}
\usepackage{xcolor}

\setmainlanguage[numerals=western]{arabic}
\setotherlanguage{english}
\newfontfamily\arabicfont[Script=Arabic]{Amiri}
\newfontfamily\arabicfontsf[Script=Arabic]{Amiri}
\newfontfamily\arabicfonttt[Script=Arabic]{Courier New}

% RTL support and nicer bullets for Arabic beamer
\setbeamertemplate{itemize items}[circle]
\setbeamertemplate{itemize subitem}[triangle]

% Listings for C# code blocks

\lstdefinelanguage{CSharp}{
  language={[Sharp]C},
  morekeywords={string,bool,void}
}

\lstset{
  language=CSharp,
  basicstyle=\ttfamily\small,
  keywordstyle=\color{blue},
  stringstyle=\color{red},
  commentstyle=\color{green!50!black},
  showstringspaces=false,
  numbers=none,
  frame=single,
  breaklines=true
}



\title{أساسيات في \textenglish{C\#}: الطباعة، النصوص، العمليات الرياضية، العشوائية، والمحارف}
\author{}
\date{}

\begin{document}

% -------------------------------------------------------------
 \begin{frame}[fragile]
  \titlepage
\end{frame}

% -------------------------------------------------------------
 \begin{frame}[fragile]{مقدمة}
يتناول هذا العرض بعض الأدوات الأساسية في \textenglish{C\#}:
\begin{itemize}
    \item الفرق بين \textenglish{WriteLine} و \textenglish{Write}
    \item أهم دوال فئة \textenglish{string}
    \item دوال \textenglish{Math}
    \item استخدام الفئة \textenglish{Random}
    \item التعامل مع النوع \textenglish{char}
\end{itemize}
\end{frame}

% =============================================================
% WriteLine vs Write
% =============================================================
 \begin{frame}[fragile]{\textenglish{WriteLine} مقابل \textenglish{Write}}
\textenglish{WriteLine}: تطبع النص وتنتقل لسطر جديد.\\
\textenglish{Write}: تطبع النص بدون سطر جديد.

\begin{english}
\begin{lstlisting}
Console.WriteLine("Hello");
Console.Write("World");
Console.WriteLine("!");
\end{lstlisting}
\end{english}

الناتج:
\begin{verbatim}
Hello
World!
\end{verbatim}
\end{frame}

% =============================================================
% STRING CLASS
% =============================================================

% -------------------------------------------------------------
 \begin{frame}[fragile]{الوصول إلى حرف داخل نص \textenglish{(Indexing)}}
يمكن الوصول لأي حرف باستخدام الفهرس الذي يبدأ من 0.

\begin{english}
\begin{lstlisting}
string s = "Hello";
char c = s[1];
Console.WriteLine(c);
\end{lstlisting}
\end{english}

الناتج:
\begin{verbatim}
e
\end{verbatim}
\end{frame}

% -------------------------------------------------------------
 \begin{frame}[fragile]{\textenglish{Contains}}
تفحص وجود نص داخل نص آخر.

\begin{english}
\begin{lstlisting}
string s = "programming";
bool b = s.Contains("gram");
Console.WriteLine(b);
\end{lstlisting}
\end{english}

الناتج:
\begin{verbatim}
True
\end{verbatim}
\end{frame}

% -------------------------------------------------------------
 \begin{frame}[fragile]{\textenglish{IndexOf}}
تعيد موقع ظهور نص داخل نص آخر.

\begin{english}
\begin{lstlisting}
string s = "banana";
int index = s.IndexOf("na");
Console.WriteLine(index);
\end{lstlisting}
\end{english}

الناتج:
\begin{verbatim}
2
\end{verbatim}
\end{frame}

% -------------------------------------------------------------
 \begin{frame}[fragile]{\textenglish{Length}}
تمثل طول النص بعدد الحروف.

\begin{english}
\begin{lstlisting}
string s = "Hello";
Console.WriteLine(s.Length);
\end{lstlisting}
\end{english}

الناتج:
\begin{verbatim}
5
\end{verbatim}
\end{frame}

% -------------------------------------------------------------
 \begin{frame}[fragile]{\textenglish{Substring}}
تستخرج جزءًا من النص.

\begin{english}
\begin{lstlisting}
string s = "HelloWorld";
Console.WriteLine( s.Substring(5) );
Console.WriteLine( s.Substring(0, 4) );
\end{lstlisting}
\end{english}

الناتج:
\begin{verbatim}
World
Hell
\end{verbatim}
\end{frame}

% -------------------------------------------------------------
 \begin{frame}[fragile]{\textenglish{ToLower / ToUpper}}
تحويل الحروف إلى صغيرة أو كبيرة.

\begin{english}
\begin{lstlisting}
string s = "HeLLo";
Console.WriteLine( s.ToLower() );
Console.WriteLine( s.ToUpper() );
\end{lstlisting}
\end{english}

الناتج:
\begin{verbatim}
hello
HELLO
\end{verbatim}
\end{frame}

% =============================================================
% MATH CLASS
% =============================================================

% -------------------------------------------------------------
 \begin{frame}[fragile]{\textenglish{Math.Max} و \textenglish{Math.Min}}
\begin{english}
\begin{lstlisting}
Console.WriteLine( Math.Max(10, 25) );
Console.WriteLine( Math.Min(10, 25) );
\end{lstlisting}
\end{english}

الناتج:
\begin{verbatim}
25
10
\end{verbatim}
\end{frame}

% -------------------------------------------------------------
 \begin{frame}[fragile]{\textenglish{Math.Pow}}
\begin{english}
\begin{lstlisting}
Console.WriteLine( Math.Pow(2, 3) );
\end{lstlisting}
\end{english}

الناتج:
\begin{verbatim}
8
\end{verbatim}
\end{frame}

% -------------------------------------------------------------
 \begin{frame}[fragile]{\textenglish{Math.Sqrt}}
\begin{english}
\begin{lstlisting}
Console.WriteLine( Math.Sqrt(81) );
\end{lstlisting}
\end{english}

الناتج:
\begin{verbatim}
9
\end{verbatim}
\end{frame}

% -------------------------------------------------------------
 \begin{frame}[fragile]{\textenglish{Math.Abs}}
\begin{english}
\begin{lstlisting}
Console.WriteLine( Math.Abs(-12) );
\end{lstlisting}
\end{english}

الناتج:
\begin{verbatim}
12
\end{verbatim}
\end{frame}

% -------------------------------------------------------------
 \begin{frame}[fragile]{\textenglish{Math.Round}}
\begin{english}
\begin{lstlisting}
Console.WriteLine( Math.Round(3.67) );
\end{lstlisting}
\end{english}

الناتج:
\begin{verbatim}
4
\end{verbatim}
\end{frame}

% =============================================================
% RANDOM CLASS
% =============================================================

% -------------------------------------------------------------
 \begin{frame}[fragile]{إنشاء كائن \textenglish{Random}}
يُستخدم لتوليد أعداد عشوائية.

\begin{english}
\begin{lstlisting}
Random rnd = new Random();
int x = rnd.Next(1, 6);
Console.WriteLine(x);
\end{lstlisting}
\end{english}

الناتج: رقم بين 1 و 5
\end{frame}

% -------------------------------------------------------------
 \begin{frame}[fragile]{\textenglish{Next()}}
\begin{english}
\begin{lstlisting}
Random r = new Random();
Console.WriteLine( r.Next() );
\end{lstlisting}
\end{english}

الناتج: رقم كبير عشوائي
\end{frame}

% -------------------------------------------------------------
 \begin{frame}[fragile]{\textenglish{Next(max)}}
\begin{english}
\begin{lstlisting}
Random r = new Random();
Console.WriteLine( r.Next(10) );
\end{lstlisting}
\end{english}

الناتج: رقم بين 0 و 9
\end{frame}

% -------------------------------------------------------------
 \begin{frame}[fragile]{\textenglish{Next(min, max)}}
\begin{english}
\begin{lstlisting}
Random r = new Random();
Console.WriteLine( r.Next(5, 11) );
\end{lstlisting}
\end{english}

الناتج: رقم بين 5 و 10
\end{frame}

% -------------------------------------------------------------
 \begin{frame}[fragile]{\textenglish{NextDouble}}
\begin{english}
\begin{lstlisting}
Random r = new Random();
Console.WriteLine( r.NextDouble() );
\end{lstlisting}
\end{english}

الناتج: عدد كسري بين 0 و 1
\end{frame}

% =============================================================
% CHAR
% =============================================================

% -------------------------------------------------------------
 \begin{frame}[fragile]{تعريف \textenglish{char}}
\begin{english}
\begin{lstlisting}
char c = 'A';
Console.WriteLine(c);
\end{lstlisting}
\end{english}

الناتج:
\begin{verbatim}
A
\end{verbatim}
\end{frame}

% -------------------------------------------------------------
 \begin{frame}[fragile]{قراءة محرف من المستخدم}
\begin{english}
\begin{lstlisting}
char c = char.Parse(Console.ReadLine());
Console.WriteLine(c);
\end{lstlisting}
\end{english}

الناتج: حسب إدخال المستخدم
\end{frame}

% -------------------------------------------------------------
 \begin{frame}[fragile]{مقارنة محارف}
\begin{english}
\begin{lstlisting}
char a = 'A';
char b = 'B';
Console.WriteLine(a < b);
\end{lstlisting}
\end{english}

الناتج:
\begin{verbatim}
True
\end{verbatim}
\end{frame}

% -------------------------------------------------------------
 \begin{frame}[fragile]{التحويل بين \textenglish{char} و \textenglish{int}}
\begin{english}
\begin{lstlisting}
char c = 'A';
int code = (int)c;
Console.WriteLine(code);

int x = 66;
char d = (char)x;
Console.WriteLine(d);
\end{lstlisting}
\end{english}

الناتج:
\begin{verbatim}
65
B
\end{verbatim}
\end{frame}

% -------------------------------------------------------------
 \begin{frame}[fragile]{خاتمة}
\begin{itemize}
    \item فئة \textenglish{string} توفر أدوات قوية لمعالجة النصوص.
    \item فئة \textenglish{Math} توفر عمليات رياضية جاهزة.
    \item الفئة \textenglish{Random} تُستخدم لتوليد الأعداد العشوائية.
    \item النوع \textenglish{char} يمثل محرفًا واحدًا ويمكن تحويله إلى \textenglish{ASCII}.
\end{itemize}
\end{frame}

\end{document}
