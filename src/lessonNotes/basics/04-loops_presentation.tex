\documentclass[13pt]{beamer}
% Fonts and languages
\usepackage{fontspec}
\usepackage{polyglossia}
\usepackage{bidi}
\usepackage{listings}
\usepackage{xcolor}

\setmainlanguage[numerals=western]{arabic}
\setotherlanguage{english}
\newfontfamily\arabicfont[Script=Arabic]{Amiri}
\newfontfamily\arabicfontsf[Script=Arabic]{Amiri}
\newfontfamily\arabicfonttt[Script=Arabic]{Courier New}

% RTL support and nicer bullets for Arabic beamer
\setbeamertemplate{itemize items}[circle]
\setbeamertemplate{itemize subitem}[triangle]

% Listings for C# code blocks

\lstdefinelanguage{CSharp}{
  language={[Sharp]C},
  morekeywords={string,bool,void}
}

\lstset{
  language=CSharp,
  basicstyle=\ttfamily\small,
  keywordstyle=\color{blue},
  stringstyle=\color{red},
  commentstyle=\color{green!50!black},
  showstringspaces=false,
  numbers=none,
  frame=single,
  breaklines=true
}



\title{الحلقات \textenglish{(Loops)} في لغة \textenglish{C\#}}
\author{}
\date{}

\begin{document}

% -----------------------
\begin{frame}
  \titlepage
\end{frame}

% ------------------------------------------------------------------
\begin{frame}{مقدمة}
الحلقة هي مقطع كود يتكرر عدة مرات لتحقيق مهمة متكررة دون إعادة كتابة الأوامر.
\begin{itemize}
  \item التكرار يجعل البرنامج أقصر وأكثر مرونة.
  \item يُمكن للحاسوب تنفيذ نفس التعليمات آلاف المرات بسرعة.
  \item توجد أنواع مختلفة من الحلقات، أهمها: \textenglish{while} و \textenglish{for}.
\end{itemize}
\end{frame}

% =======================
% WHILE section start
% =======================
\begin{frame}[fragile]{مفهوم حلقة \textenglish{while}}
    فكرة الحلقة: \textbf{طالما أن شرطًا ما صحيح، استمر في تنفيذ الأوامر.}
    \begin{itemize}
        \item عندما لا نعرف عدد التكرارات مسبقًا.
        \item نستمر بالتكرار حتى يحصل "شيء ما" (يتحقق شرط معين)
        \item عند كل تكرار يُفحص الشرط.
        \item إذا كان الشرط \textenglish{true} يُنفَّذ الجسم، وإذا أصبح \textenglish{false} تتوقف الحلقة.
    \end{itemize}
\end{frame}

\begin{frame}[fragile]{مبنى حلقة \textenglish{while}}

\begin{english}
\begin{lstlisting}[language=CSharp]
while (condition)
{
    // body
}
\end{lstlisting}
\end{english}
\begin{itemize}
  \item يختبر الشرط قبل تنفيذ أي تكرار.
  \item إذا كان الشرط \textenglish{true} يُنفَّذ جسم الحلقة، وإذا كان \textenglish{false} تتوقف الحلقة،
  \item إذا كان الشرط \textenglish{false} من البداية، فلا ينفذ الجسم ولا مرة.
\end{itemize}
\end{frame}

\begin{frame}[fragile]{جمع حتى إدخال 0}
يحسب مجموع الأعداد حتى إدخال 0.

\begin{english}
\begin{lstlisting}
int sum = 0;
int num = int.Parse(Console.ReadLine());
while (num != 0)
{
    sum += num;
    num = int.Parse(Console.ReadLine());
}
Console.WriteLine("Sum = " + sum);
\end{lstlisting}
\end{english}
\end{frame}

% ------------------------------------------------------------------
\begin{frame}[fragile]{جمع حتى يصبح المجموع أكبر من 10}
يستمر حتى يصبح مجموع الأعداد أكبر من 10.

\begin{english}
\begin{lstlisting}
int sum = 0;
while (sum <= 10)
{
    int x = int.Parse(Console.ReadLine());
    sum += x;
}
Console.WriteLine("Total = " + sum);
\end{lstlisting}
\end{english}
\end{frame}

% ------------------------------------------------------------------
\begin{frame}[fragile]{الضرب حتى إدخال 1}
يضرب الأعداد حتى إدخال 1.

\begin{english}
\begin{lstlisting}
int prod = 1;
int num = int.Parse(Console.ReadLine());
while (num != 1)
{
    prod *= num;
    num = int.Parse(Console.ReadLine());
}
Console.WriteLine("Product = " + prod);
\end{lstlisting}
\end{english}
\end{frame}

% ------------------------------------------------------------------
\begin{frame}[fragile]{عدد الأعداد الموجبة حتى إدخال عدد سالب}
يعد الأعداد الموجبة فقط.

\begin{english}
\begin{lstlisting}
int count = 0;
int num = int.Parse(Console.ReadLine());
while (num >= 0)
{
    count++;
    num = int.Parse(Console.ReadLine());
}
Console.WriteLine("Count = " + count);
\end{lstlisting}
\end{english}
\end{frame}

% ------------------------------------------------------------------
\begin{frame}[fragile]{عدد الأعداد الزوجية حتى إدخال -1}
يحسب عدد الأعداد الزوجية فقط.

\begin{english}
\begin{lstlisting}
int evenCount = 0;
int num = int.Parse(Console.ReadLine());
while (num != -1)
{
    if (num % 2 == 0)
        evenCount++;
    num = int.Parse(Console.ReadLine());
}
Console.WriteLine("Even count = " + evenCount);
\end{lstlisting}
\end{english}
\end{frame}

% ------------------------------------------------------------------
\begin{frame}[fragile]{مثال: قراءة 3 أعداد زوجية بالضبط }

\begin{english}
\begin{lstlisting}[language=CSharp]
int number, count = 0;
while (count < 3)
{
    Console.WriteLine("enter a number");
    number = int.Parse(Console.ReadLine());
    if (number % 2 == 0)
        count++;
}
Console.WriteLine("End loop");
Console.WriteLine("count=" + count);
\end{lstlisting}
\end{english}
\end{frame}

% ------------------------------------------------------------------
\begin{frame}[fragile]{إيجاد أكبر عدد حتى إدخال 0}
يجد أكبر رقم من المدخلات.

\begin{english}
\begin{lstlisting}
int max = int.MinValue;
int num = int.Parse(Console.ReadLine());
while (num != 0)
{
    if (num > max)
        max = num;
    num = int.Parse(Console.ReadLine());
}
Console.WriteLine("Max = " + max);
\end{lstlisting}
\end{english}
\end{frame}

% ------------------------------------------------------------------
\begin{frame}[fragile]{إيجاد أصغر عدد حتى إدخال 0}
يجد أصغر رقم من المدخلات.

\begin{english}
\begin{lstlisting}
int min = int.MaxValue;
int num = int.Parse(Console.ReadLine());
while (num != 0)
{
    if (num < min)
        min = num;
    num = int.Parse(Console.ReadLine());
}
Console.WriteLine("Min = " + min);
\end{lstlisting}
\end{english}
\end{frame}

% ------------------------------------------------------------------
\begin{frame}[fragile]{عدد خانات عدد معين}
يحسب عدد خانات الرقم.

\begin{english}
\begin{lstlisting}
int n = int.Parse(Console.ReadLine());
int count = 0;
while (n > 0)
{
    n /= 10;
    count++;
}
Console.WriteLine("Digits = " + count);
\end{lstlisting}
\end{english}
\end{frame}

\begin{frame}[fragile]{التحقق الصارم من المدخلات }

\begin{english}
\begin{lstlisting}[language=CSharp]
int n = int.Parse(Console.ReadLine());
while (n < 1 || n > 10)
{
    Console.WriteLine("Invalid! Enter 1-10:");
    n = int.Parse(Console.ReadLine());
}
\end{lstlisting}
\end{english}
\end{frame}

\begin{frame}[fragile]{ شرط توقّف مركب في \textenglish{while}}

\begin{english}
\begin{lstlisting}[language=CSharp]
int sum = 0, i = 0;
while (i < 10 && sum < 200)
{
    sum += int.Parse(Console.ReadLine());
    i++;
}
\end{lstlisting}
\end{english}
\begin{itemize}
  \item هنا شرط التوقف يعتمد على عداد وشرط آخر (هدف مالي).
  \item مثال مأخوذ من المواد: التوقف إذا وصلنا إلى هدف.
\end{itemize}
\end{frame}

% =======================
% FOR section start (after while)
% =======================
\begin{frame}{متى نستخدم \textenglish{for}؟}
\begin{itemize}
  \item عندما نعرف عدد التكرارات أو يمكن حسابه بسهولة.
  \item بنية مركّبة في سطر واحد: \textenglish{initialization; condition; update;}
\end{itemize}
\end{frame}

\begin{frame}[fragile]{مبنى حلقة \textenglish{for}}

\begin{english}
\begin{lstlisting}[language=CSharp]
for (initialization; condition; update)
{
    // body
}
\end{lstlisting}
\end{english}
\begin{itemize}
  \item ترتيب التنفيذ:
    \begin{enumerate}
        \item \textenglish{initialization} - التهئية، مرة واحدة
        \item كرر التالي:
        \begin{enumerate}
            \item \textenglish{condition} - شرط التوقف، يُفحَص كل مرة \textbf{قبل} تنفيذ جسم الحلقة. إذا كان الشرط لا يتحقق فأوقف التكرار.
            \item تنفيذ جسم الحلقة
            \item \textenglish{update} - التحديث، يحصل كل مرة \textbf{بعد} تنفيذ جسم الحلقة
        \end{enumerate}
    \end{enumerate}
\end{itemize}
\end{frame}

\begin{frame}[fragile]{مثال: طباعة الأعداد 1..3 }
\begin{english}
\begin{lstlisting}[language=CSharp]
for (int i = 1; i <= 3; i++)
{
    Console.WriteLine(i);
}
\end{lstlisting}
\end{english}

الكود أعلاه يفعل نفس ما يفعله هذا الكود:

\begin{english}
\begin{lstlisting}[language=CSharp]
int i = 1;
// check that i <= 3  --> true
Console.WriteLine(i);
i++; // i <-- 2
// check that i <= 3  --> true
Console.WriteLine(i);
i++; // i <-- 3
// check that i <= 3  --> true
Console.WriteLine(i);
i++; // i <-- 4
// check that i <= 3  --> false
\end{lstlisting}
\end{english}
\end{frame}


\begin{frame}[fragile]{متغير عداد الحلقة موجود دائمًا حتى لو لم نستخدمه في جسم الحلقة. }
\begin{english}
\begin{lstlisting}[language=CSharp]
for (int i = 0; i < 5; i++)
{
    Console.WriteLine("Hello");
}
\end{lstlisting}
\end{english}
\end{frame}


\begin{frame}[fragile]{مثال: طباعة الأعداد الزوجية بين 1 و 100 }
\begin{english}
\begin{lstlisting}[language=CSharp]
for (int i = 2; i <= 100; i += 2)
    Console.WriteLine(i);
\end{lstlisting}
\end{english}
\end{frame}

\begin{frame}[fragile]{مثال: طباعة الأعداد الزوجية بين 1 و 100  - طريقة 2}

\begin{english}
\begin{lstlisting}[language=CSharp]
for (int i = 1; i <= 100; i++)
{
    if (i % 2 == 0)
        Console.WriteLine(i);
}
\end{lstlisting}
\end{english}
\end{frame}


\begin{frame}[fragile]{مثال: طباعة الأعداد التي تقبل القسمة على 5 ضمن 0..N }
\begin{english}
\begin{lstlisting}[language=CSharp]
int N = int.Parse(Console.ReadLine());
for (int i = 0; i <= N; i++)
{
    if (i % 5 == 0)
        Console.WriteLine(i);
}
\end{lstlisting}
\end{english}
\end{frame}

\begin{frame}[fragile]{مثال: طباعة الأعداد التي تقبل القسمة على 5 ضمن 0..N  - حل 2}
\begin{english}
\begin{lstlisting}[language=CSharp]
int N = int.Parse(Console.ReadLine());
for (int i = 0; i <= N; i+=5)
{
    Console.WriteLine(i);
}
\end{lstlisting}
\end{english}
\end{frame}


\begin{frame}[fragile]{مثال: حساب المجموع }
\begin{english}
\begin{lstlisting}[language=CSharp]
int sum = 0;
for (int i = 1; i <= 100; i++)
    sum += i;
Console.WriteLine(sum);
\end{lstlisting}
\end{english}
\end{frame}

\begin{frame}[fragile]{مثال: إيجاد القيمة العظمى (max) أثناء القراءة }
\begin{english}
\begin{lstlisting}[language=CSharp]
int max = int.MinValue;
for (int i = 0; i < 10; i++)
{
    int x = int.Parse(Console.ReadLine());
    if (x > max) max = x;
}
Console.WriteLine("Max = " + max);
\end{lstlisting}
\end{english}
\end{frame}

\begin{frame}[fragile]{مثال: متوسط مجموعة أعداد }

\begin{english}
\begin{lstlisting}[language=CSharp]
int sum = 0, n = 5;
for (int i = 0; i < n; i++)
{
    int x = int.Parse(Console.ReadLine());
    sum += x;
}
double avg = sum / (double)n;
Console.WriteLine("Average = " + avg);
\end{lstlisting}
\end{english}
\end{frame}

\begin{frame}[fragile]{أيّهما أقوى: \textenglish{for} أم \textenglish{while}؟}
    هل يمكنك التفكير في مشكلة يمكن حلّها فقط بواسطة \textenglish{for} ولا يمكن حلّها بواسطة \textenglish{while}?

    أو العكس:

    هل يمكنك التفكير في مشكلة يمكن حلّها فقط بواسطة \textenglish{while} ولا يمكن حلّها بواسطة \textenglish{for}?
\end{frame}

\begin{frame}[fragile]{مثال: تحويل \textenglish{for} إلى \textenglish{while} }

\begin{english}
\begin{lstlisting}[language=CSharp]
// for-style
for (int i = 1; i <= 5; i++)
    Console.WriteLine(i);

// while-style equivalent
int i = 1;
while (i <= 5)
{
    Console.WriteLine(i);
    i++;
}
\end{lstlisting}
\end{english}
\end{frame}

% =======================
% Infinite loops and showing examples where stop condition never met
% =======================
\begin{frame}{أمثلة على حلقات شرط التوقف لا يتحقق — حالات حقيقية}
\begin{itemize}
  \item ننسى تحديث العداد.
  \item استخدام شرط خاطئ (مثلاً: `i > 0` مع `i` يبدأ بصفر ويتناقص إلى سالب).
  \item مدخلات خارجة عن السيطرة (انتظار إدخال لن يحدث).
\end{itemize}
\end{frame}

\begin{frame}[fragile]{مبدأ الحلقات اللانهائية — مثال توضيحي }
\begin{english}
\begin{lstlisting}[language=CSharp]
int x = 5;
while (x > 0)
{
    Console.WriteLine(x);
    // x is not updated anywhere --> infinite loop
}
\end{lstlisting}
\end{english}
\begin{itemize}
  \item هذه الحلقة خاطئة منطقيًا، لأنّ شرط التوقف في الحلقة يجب أن يتعلق بمتغيرات تتغير قيمتها في جسم الحلقة.
  \item الحل: تحديث المتغير أو تغيير الشرط داخل الجسم.
\end{itemize}
\end{frame}

\begin{frame}[fragile]{مثال: for بلا تحديث }
\begin{english}
\begin{lstlisting}[language=CSharp]
for (int i = 1; i > 0; ) //i is not updated
{
    Console.WriteLine("Infinite?");
}
\end{lstlisting}
\end{english}

أو

\begin{english}
\begin{lstlisting}[language=CSharp]
for (int i = 1; i > 0; i++) //always true
{
    Console.WriteLine("Infinite?");
}
\end{lstlisting}
\end{english}
\end{frame}

% % =======================
% % Validity and program correctness with follow tables
% % =======================
% \begin{frame}{تحليل صحة البرنامج باستخدام جدول متابعة}
% \begin{itemize}
%   \item أضف عمودًا لشرط الحلقة، وعمودًا لمتغير عداد الحلقة حتى لو لم يُستخدم في الجسم.
%   \item الجدول يوضح القيم قبل وبعد كل تكرار — يساعد على إثبات صحة الخروج.
% \end{itemize}
% \end{frame}

% \begin{frame}[fragile]{مثال جدول متابعة — برنامج يحسب مجموع أول n أعداد }
% \begin{tabular}{|c|c|c|c|c|}
% \hline
% تكرار & i (قبل) & شرط & sum (قبل) & sum (بعد) \\
% \hline
% 0 & 1 & true & 0 & 1 \\
% 1 & 2 & true & 1 & 3 \\
% 2 & 3 & true & 3 & 6 \\
% ... & ... & ... & ... & ... \\
% n & n & true/false & ... & ... \\
% \hline
% \end{tabular}
% \end{frame}

% % =======================
% % Complexity / number of iterations
% % =======================
% \begin{frame}{تحليل النجاعة: عدد التكرارات}
% \begin{itemize}
%   \item نعدّ كم مرة ينفّذ جسم الحلقة — يساعد في تقدير الزمن (تقدير تقريبِي).
%   \item مثال: من 1 إلى 100 خطوة 1 $\Rightarrow$ 100 تكرار.
%   \item مع التداخل: حلقة داخل حلقة، إذا كانت الأولى m والثانية n => $m\times n$ تكرار.
% \end{itemize}
% \end{frame}

% % =======================
% % Nested loops
% % =======================
% \begin{frame}[fragile]{مثال: تداخل حلقات }

% \begin{english}
% \begin{lstlisting}[language=CSharp]
% for (int i = 1; i <= 3; i++)
% {
%     for (int j = 1; j <= 4; j++)
%     {
%         Console.Write(i + "," + j + " ");
%     }
%     Console.WriteLine();
% }
% \end{lstlisting}
% \end{english}
% \end{frame}

% % =======================
% % Using loops for sequences (arith/geom)
% % =======================
% \begin{frame}{الحلقات والمتواليات الحسابية والهندسية}
% \begin{itemize}
%   \item متتالية حسابية: \(a, a+d, a+2d, \dots\)
%   \item متتالية هندسية: \(a, ar, ar^2, \dots\)
%   \item الحلقات تنفّذ بسهولة توليد الحدود الأولى.
% \end{itemize}
% \end{frame}

% \begin{frame}[fragile]{مثال: أول 10 حدود من متوالية هندسية }

% \begin{english}
% \begin{lstlisting}[language=CSharp]
% double a = 2, r = 3;
% for (int k = 0; k < 10; k++)
% {
%     Console.WriteLine(a * Math.Pow(r, k));
% }
% \end{lstlisting}
% \end{english}
% \end{frame}

% % =======================
% % External methods with loops
% % =======================
% \begin{frame}[fragile]{عمليات خارجية تحتوي على حلقات - مثال }

% \begin{english}
% \begin{lstlisting}[language=CSharp]
% public static int SumToN(int n)
% {
%     int sum = 0;
%     for (int i = 1; i <= n; i++)
%         sum += i;
%     return sum;
% }

% public static void Main()
% {
%     Console.WriteLine(SumToN(10));
% }
% \end{lstlisting}
% \end{english}
% \end{frame}

% % =======================
% % Exercises — each in its own frame (several)
% % =======================
% \begin{frame}{سؤال 1 }
% اكتب حلقة \textenglish{while} تطلب من المستخدم أعدادًا حتى يدخل 0، ثم تطبع مجموعها.
% \end{frame}

% \begin{frame}[fragile]{حل سؤال 1 }

% \begin{english}
% \begin{lstlisting}[language=CSharp]
% int sum = 0;
% int x = int.Parse(Console.ReadLine());
% while (x != 0)
% {
%     sum += x;
%     x = int.Parse(Console.ReadLine());
% }
% Console.WriteLine("Sum = " + sum);
% \end{lstlisting}
% \end{english}
% \end{frame}

% \begin{frame}{سؤال 2 }
% اكتب حلقة \textenglish{for} تطبع الأعداد الزوجية من 2 إلى 20.
% \end{frame}

% \begin{frame}[fragile]{حل سؤال 2 }

% \begin{english}
% \begin{lstlisting}[language=CSharp]
% for (int i = 2; i <= 20; i += 2)
%     Console.WriteLine(i);
% \end{lstlisting}
% \end{english}
% \end{frame}

% \begin{frame}{سؤال 3 }
% اكتب برنامجًا يقرأ 10 أعداد ويطبع أكبر واحد منها.
% \end{frame}

% \begin{frame}[fragile]{حل سؤال 3 }

% \begin{english}
% \begin{lstlisting}[language=CSharp]
% int max = int.MinValue;
% for (int i = 0; i < 10; i++)
% {
%     int x = int.Parse(Console.ReadLine());
%     if (x > max) max = x;
% }
% Console.WriteLine("Max = " + max);
% \end{lstlisting}
% \end{english}
% \end{frame}

% \begin{frame}{سؤال 4 }
% اكتب حلقة تعرض أول 8 حدود من متتالية حسابية تبدأ من 5 وتزداد بمقدار 3.
% \end{frame}

% \begin{frame}[fragile]{حل سؤال 4 }

% \begin{english}
% \begin{lstlisting}[language=CSharp]
% int a = 5, d = 3;
% for (int k = 0; k < 8; k++)
% {
%     Console.WriteLine(a + k*d);
% }
% \end{lstlisting}
% \end{english}
% \end{frame}

\end{document}
