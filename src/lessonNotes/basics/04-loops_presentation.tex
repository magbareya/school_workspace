\documentclass[13pt]{beamer}
% Fonts and languages
\usepackage{fontspec}
\usepackage{polyglossia}
\usepackage{bidi}
\usepackage{listings}
\usepackage{xcolor}

\setmainlanguage[numerals=western]{arabic}
\setotherlanguage{english}
\newfontfamily\arabicfont[Script=Arabic]{Amiri}
\newfontfamily\arabicfontsf[Script=Arabic]{Amiri}
\newfontfamily\arabicfonttt[Script=Arabic]{Courier New}

% RTL support and nicer bullets for Arabic beamer
\setbeamertemplate{itemize items}[circle]
\setbeamertemplate{itemize subitem}[triangle]

% Listings for C# code blocks

\lstdefinelanguage{CSharp}{
  language={[Sharp]C},
  morekeywords={string,bool,void}
}

\lstset{
  language=CSharp,
  basicstyle=\ttfamily\small,
  keywordstyle=\color{blue},
  stringstyle=\color{red},
  commentstyle=\color{green!50!black},
  showstringspaces=false,
  numbers=none,
  frame=single,
  breaklines=true
}



%% ------------------------------------------------------------------
\title{الحلقات في لغة \textenglish{C\#}}
\author{}
\date{}

\begin{document}

%% Title page
\begin{frame}
  \titlepage
\end{frame}


% ------------------------------------------------------------------
\begin{frame}{مقدمة}
الحلقة هي مقطع كود يتكرر عدة مرات لتحقيق مهمة متكررة دون إعادة كتابة الأوامر.
\begin{itemize}
  \item التكرار يجعل البرنامج أقصر وأكثر مرونة.
  \item يُمكن للحاسوب تنفيذ نفس التعليمات آلاف المرات بسرعة.
  \item توجد أنواع مختلفة من الحلقات، أهمها: \textenglish{while} و \textenglish{for}.
\end{itemize}
\end{frame}

% ------------------------------------------------------------------
\begin{frame}[fragile]{مفهوم الحلقة بشكل عام}
فكرة الحلقة: \textbf{طالما أن شرطًا ما صحيح، استمر في تنفيذ الأوامر.}
\begin{itemize}
  \item عند كل تكرار يُفحص الشرط.
  \item إذا كان الشرط \textenglish{true} يُنفَّذ الجسم، وإذا أصبح \textenglish{false} تتوقف الحلقة.
\end{itemize}
\end{frame}

% ------------------------------------------------------------------
\begin{frame}[fragile]{حلقة \textenglish{while}}
تُستخدم عندما لا نعرف عدد التكرارات مسبقًا.
\begin{english}
\begin{lstlisting}
while (condition)
{
    // الأوامر
}
\end{lstlisting}
\end{english}
\end{frame}

% ------------------------------------------------------------------
\begin{frame}[fragile]{جمع حتى إدخال 0}
يحسب مجموع الأعداد حتى إدخال 0.
\begin{english}
\begin{lstlisting}
int sum = 0;
int num = int.Parse(Console.ReadLine());
while (num != 0)
{
    sum += num;
    num = int.Parse(Console.ReadLine());
}
Console.WriteLine("Sum = " + sum);
\end{lstlisting}
\end{english}
\end{frame}

% ------------------------------------------------------------------
\begin{frame}[fragile]{جمع حتى يصبح المجموع أكبر من 10}
يستمر حتى يصبح مجموع الأعداد أكبر من 10.
\begin{english}
\begin{lstlisting}
int sum = 0;
while (sum <= 10)
{
    int x = int.Parse(Console.ReadLine());
    sum += x;
}
Console.WriteLine("Total = " + sum);
\end{lstlisting}
\end{english}
\end{frame}

% ------------------------------------------------------------------
\begin{frame}[fragile]{الضرب حتى إدخال 1}
يضرب الأعداد حتى إدخال 1.
\begin{english}
\begin{lstlisting}
int prod = 1;
int num = int.Parse(Console.ReadLine());
while (num != 1)
{
    prod *= num;
    num = int.Parse(Console.ReadLine());
}
Console.WriteLine("Product = " + prod);
\end{lstlisting}
\end{english}
\end{frame}

% ------------------------------------------------------------------
\begin{frame}[fragile]{عدد الأعداد الموجبة حتى إدخال عدد سالب}
يعد الأعداد الموجبة فقط.
\begin{english}
\begin{lstlisting}
int count = 0;
int num = int.Parse(Console.ReadLine());
while (num >= 0)
{
    count++;
    num = int.Parse(Console.ReadLine());
}
Console.WriteLine("Count = " + count);
\end{lstlisting}
\end{english}
\end{frame}

% ------------------------------------------------------------------
\begin{frame}[fragile]{عدد الأعداد الزوجية حتى إدخال -1}
يحسب عدد الأعداد الزوجية فقط.
\begin{english}
\begin{lstlisting}
int evenCount = 0;
int num = int.Parse(Console.ReadLine());
while (num != -1)
{
    if (num % 2 == 0)
        evenCount++;
    num = int.Parse(Console.ReadLine());
}
Console.WriteLine("Even count = " + evenCount);
\end{lstlisting}
\end{english}
\end{frame}

% ------------------------------------------------------------------
\begin{frame}[fragile]{إيجاد أكبر عدد حتى إدخال 0}
يجد أكبر رقم من المدخلات.
\begin{english}
\begin{lstlisting}
int max = int.MinValue;
int num = int.Parse(Console.ReadLine());
while (num != 0)
{
    if (num > max)
        max = num;
    num = int.Parse(Console.ReadLine());
}
Console.WriteLine("Max = " + max);
\end{lstlisting}
\end{english}
\end{frame}

% ------------------------------------------------------------------
\begin{frame}[fragile]{إيجاد أصغر عدد حتى إدخال 0}
يجد أصغر رقم من المدخلات.
\begin{english}
\begin{lstlisting}
int min = int.MaxValue;
int num = int.Parse(Console.ReadLine());
while (num != 0)
{
    if (num < min)
        min = num;
    num = int.Parse(Console.ReadLine());
}
Console.WriteLine("Min = " + min);
\end{lstlisting}
\end{english}
\end{frame}

% ------------------------------------------------------------------
\begin{frame}[fragile]{عدد خانات عدد معين}
يحسب عدد خانات الرقم.
\begin{english}
\begin{lstlisting}
int n = int.Parse(Console.ReadLine());
int count = 0;
while (n > 0)
{
    n /= 10;
    count++;
}
Console.WriteLine("Digits = " + count);
\end{lstlisting}
\end{english}
\end{frame}

% ------------------------------------------------------------------
\begin{frame}[fragile]{من \textenglish{while} إلى \textenglish{for}}
توضيح أن \textenglish{for} هي شكل مختصر من \textenglish{while}.
\begin{english}
\begin{lstlisting}
// باستخدام while
int i = 0;
while (i < 10)
{
    Console.WriteLine(i);
    i++;
}

// باستخدام for
for (int i = 0; i < 10; i++)
{
    Console.WriteLine(i);
}
\end{lstlisting}
\end{english}
\end{frame}

% ------------------------------------------------------------------
\begin{frame}[fragile]{حلقة \textenglish{for}}
تُستخدم عندما نعرف عدد التكرارات مسبقًا.
\begin{english}
\begin{lstlisting}
for (initialization; condition; update)
{
    // الأوامر
}
\end{lstlisting}
\end{english}
\end{frame}

% ------------------------------------------------------------------
\begin{frame}[fragile]{طباعة الأعداد من 1 إلى 10}
يطبع الأعداد من 1 إلى 10.
\begin{english}
\begin{lstlisting}
for (int i = 1; i <= 10; i++)
    Console.WriteLine(i);
\end{lstlisting}
\end{english}
\end{frame}

% ------------------------------------------------------------------
\begin{frame}[fragile]{طباعة الأعداد الزوجية من 1 إلى 100}
يطبع الأعداد الزوجية فقط.
\begin{english}
\begin{lstlisting}
for (int i = 2; i <= 100; i += 2)
    Console.WriteLine(i);
\end{lstlisting}
\end{english}
\end{frame}

% ------------------------------------------------------------------
\begin{frame}[fragile]{مجموع الأعداد من 1 إلى n}
يحسب مجموع الأعداد من 1 إلى n.
\begin{english}
\begin{lstlisting}
int n = int.Parse(Console.ReadLine());
int sum = 0;
for (int i = 1; i <= n; i++)
    sum += i;
Console.WriteLine("Sum = " + sum);
\end{lstlisting}
\end{english}
\end{frame}

% ------------------------------------------------------------------
\begin{frame}[fragile]{مجموع الأعداد من m إلى n}
يحسب مجموع الأعداد بين رقمين.
\begin{english}
\begin{lstlisting}
int m = int.Parse(Console.ReadLine());
int n = int.Parse(Console.ReadLine());
int sum = 0;
for (int i = m; i <= n; i++)
    sum += i;
Console.WriteLine("Sum = " + sum);
\end{lstlisting}
\end{english}
\end{frame}

% ------------------------------------------------------------------
\begin{frame}[fragile]{مجموع الأعداد الزوجية من 1 إلى n}
يحسب مجموع الأعداد الزوجية فقط.
\begin{english}
\begin{lstlisting}
int n = int.Parse(Console.ReadLine());
int sum = 0;
for (int i = 2; i <= n; i += 2)
    sum += i;
Console.WriteLine("Even sum = " + sum);
\end{lstlisting}
\end{english}
\end{frame}

% ------------------------------------------------------------------
\begin{frame}[fragile]{استقبال 10 أرقام وطباعتها}
يطبع كل الأرقام المدخلة.
\begin{english}
\begin{lstlisting}
for (int i = 1; i <= 10; i++)
{
    int num = int.Parse(Console.ReadLine());
    Console.WriteLine(num);
}
\end{lstlisting}
\end{english}
\end{frame}

% ------------------------------------------------------------------
\begin{frame}[fragile]{استقبال 10 أرقام وحساب معدلها}
يحسب معدل 10 أعداد.
\begin{english}
\begin{lstlisting}
int sum = 0;
for (int i = 1; i <= 10; i++)
{
    int num = int.Parse(Console.ReadLine());
    sum += num;
}
double avg = sum / 10.0;
Console.WriteLine("Average = " + avg);
\end{lstlisting}
\end{english}
\end{frame}

% ------------------------------------------------------------------
\begin{frame}[fragile]{معدل حتى إدخال -1}
يحسب معدل الأعداد حتى إدخال -1.
\begin{english}
\begin{lstlisting}
int sum = 0, count = 0;
int num = int.Parse(Console.ReadLine());
while (num != -1)
{
    sum += num;
    count++;
    num = int.Parse(Console.ReadLine());
}
double avg = (double)sum / count;
Console.WriteLine("Average = " + avg);
\end{lstlisting}
\end{english}
\end{frame}

% ------------------------------------------------------------------
\begin{frame}[fragile]{حساب الباقي والقسمة}
يحسب خارج القسمة والباقي.
\begin{english}
\begin{lstlisting}
int a = int.Parse(Console.ReadLine());
int b = int.Parse(Console.ReadLine());
int quotient = a / b;
int remainder = a % b;
Console.WriteLine("Quotient = " + quotient);
Console.WriteLine("Remainder = " + remainder);
\end{lstlisting}
\end{english}
\end{frame}

% ------------------------------------------------------------------
\begin{frame}[fragile]{طباعة جدول ضرب عدد معين}
يطبع جدول الضرب لعدد محدد.
\begin{english}
\begin{lstlisting}
int n = int.Parse(Console.ReadLine());
for (int i = 1; i <= 10; i++)
{
    Console.WriteLine(n + " x " + i + " = " + (n * i));
}
\end{lstlisting}
\end{english}
\end{frame}

% ------------------------------------------------------------------
\begin{frame}[fragile]{مجموع الأرقام داخل عدد واحد}
يجمع خانات العدد.
\begin{english}
\begin{lstlisting}
int n = int.Parse(Console.ReadLine());
int sum = 0;
while (n > 0)
{
    sum += n % 10;
    n /= 10;
}
Console.WriteLine("Digits sum = " + sum);
\end{lstlisting}
\end{english}
\end{frame}

% ------------------------------------------------------------------
\begin{frame}[fragile]{حساب المضروب}
يحسب مضروب عدد n.
\begin{english}
\begin{lstlisting}
int n = int.Parse(Console.ReadLine());
int factorial = 1;
for (int i = 1; i <= n; i++)
    factorial *= i;
Console.WriteLine(n + "! = " + factorial);
\end{lstlisting}
\end{english}
\end{frame}

% ------------------------------------------------------------------
\begin{frame}{خاتمة}
\begin{itemize}
  \item حلقة \textenglish{while} تكرّر الأوامر طالما الشرط صحيح.
  \item حلقة \textenglish{for} تستخدم عند معرفة عدد التكرارات.
  \item احذر من نسيان تحديث العداد أو تغيير الشرط.
  \item استخدم المتغيرات الجامعة والعدادات لتبسيط الحسابات.
\end{itemize}
\end{frame}

\end{document}
