\documentclass[14pt]{extarticle}
% Full article preamble (duplicated, no common file)
\usepackage{fontspec}
\usepackage[a4paper,top=2.4cm,bottom=2.4cm,left=2.3cm,right=2.3cm]{geometry}
\usepackage{polyglossia}
\usepackage{amsmath}
\usepackage{amssymb}
\usepackage{xcolor}
\usepackage{fancyhdr}
\usepackage{graphicx}
\usepackage{listings}
\usepackage[most]{tcolorbox}
\usepackage{pifont}
\usepackage{enumitem}
\usepackage{titlesec}
\usepackage[bottom]{footmisc}
\usepackage{titling}
\usepackage{minted}
\usepackage{etoolbox}
\usepackage{array}
\usepackage{extsizes}

\newfontfamily\emoji{Segoe UI Emoji}

\pagestyle{fancy}

\setmainlanguage[numerals=western]{arabic}
\setotherlanguage{english}
\newfontfamily\arabicfont[Script=Arabic]{Amiri}
\newfontfamily\arabicfonttt[Script=Arabic]{Courier New}

\lstset{
  language=[Sharp]C,
  numbers=left,
  stepnumber=1,
  numbersep=8pt,
  frame=single,
  basicstyle=\ttfamily\small,
  keywordstyle=\color{blue},
  stringstyle=\color{red},
  commentstyle=\color{green!50!black}
}

\newif\ifdetailed
\ifdefined\setdetailed
  \setdetailed
\fi

\newif\ifwithsols
\ifdefined\setwithsols
  \setwithsols
\fi

% unified tcolorboxes for articles
\tcbset{colback=white, colframe=black, fonttitle=\bfseries, boxrule=0.8pt}
\newtcolorbox{boxDef}[1][]{colback=blue!5!white,colframe=blue!75!black,
  title={{\emoji📘} تعريف\ifx\\#1\\\else ~#1\fi :}}
\newtcolorbox{boxExercise}[1][]{colback=cyan!5!white,colframe=cyan!70!black,
  title={{\emoji🧩} تمرين\ifx\\#1\\\else ~#1\fi :}}
\newtcolorbox{boxExample}[1][]{colback=yellow!5!white,colframe=orange!90!black,
  title={{\emoji📝} مثال\ifx\\#1\\\else ~#1\fi :}}
\newtcolorbox{boxNote}[1][]{colback=gray!10!white,colframe=black,
  title={{\emoji✨} ملاحظة\ifx\\#1\\\else ~#1\fi :}}
\newtcolorbox{boxAttention}[1][]{colback=magenta!10!white,colframe=magenta!80!black,
  title={{\emoji🔔} تنبيه\ifx\\#1\\\else ~#1\fi :}}
\newtcolorbox{boxWarning}[1][]{colback=red!5!white,colframe=red!75!black,
  title={{\emoji⚡} ملاحظة هامة\ifx\\#1\\\else ~#1\fi :}}
\newtcolorbox{boxSolution}[1][]{colback=green!5!white,colframe=green!60!black,
  title={{\emoji✅} حل\ifx\\#1\\\else ~#1\fi :}}
\newtcolorbox{boxSymbol}[1][]{colback=purple!5!white,colframe=purple!70!black,
  title={{\emoji🔣} رمز\ifx\\#1\\\else ~#1\fi :}}
\newtcolorbox{boxHint}[1][]{colback=teal!5!white,colframe=teal!60!black,
  title={{\emoji💡} تلميح\ifx\\#1\\\else ~#1\fi :}}


\tcbset{simplecode/.style={ colback=gray!5, colframe=black!50, boxrule=0.4pt, arc=2pt, left=4pt,right=4pt,top=4pt,bottom=4pt}}
\newenvironment{boxCode}{\begin{tcolorbox}[simplecode]}{\end{tcolorbox}}

\newcolumntype{C}[1]{>{\centering\arraybackslash}p{#1}}

% redefine spaces after titles
\makeatletter
\renewcommand{\@maketitle}{%
  \begin{center}
    {\huge \bfseries \@title \par}%
    \vskip 0.2em % space between title and author
    {\large \@author \par}%
    % \vskip 0.2em % space between author and date
    % {\normalsize \@date \par}%
  \end{center}
}
\makeatother

\fancyhf{} % clear default
\fancypagestyle{plain}{
  \fancyhf{}
  \fancyhead[L]{مدرسة التسامح الشاملة}
  % \fancyhead[L]{\includegraphics[height=1cm]{../../../images/logoTasamoh.png}}
  \fancyhead[R]{الأستاذ محمود اغبارية}
  \fancyfoot[C]{\thepage}
}

\fancyhead[L]{مدرسة التسامح الشاملة}
\fancyhead[R]{الأستاذ محمود اغبارية}
\fancyfoot[C]{\thepage}
% \date{\today}

\setcounter{tocdepth}{3} % only section subsection and subsubsection in TOC


% ----------------------


% \begin{document}

% \maketitle

% % \clearpage  % start TOC on a new page
% % \renewcommand{\contentsname}{جدول المحتويات}
% % \tableofcontents
% % \clearpage

% \part*{part 1} % the * prevents numbering
% \section*{مقدمة}
% \subsection*{مثال رياضي}
% \subsubsection*{مثال فرعي}
% \paragraph*{ paragraph 1}
% \subparagraph*{sub paragraph 1}

% \ifdetailed
% \begin{english}
% \begin{minted}{csharp}
% // C# Example
% \end{minted}
% \end{english}
% \fi

% OLD WAY
% \ifdetailed
% \begin{english}
% \begin{lstlisting}
% // C# Example
% \end{lstlisting}
% \end{english}
% \fi

% % \includegraphics[width=0.2\textwidth]{../../../images/DFAs/ex1_q1.png}



% \vspace{3cm}
% \begin{flushleft}
% أرجو لكم وقتًا ممتعًا.

% الأستاذ محمود اغبارية.
% \end{flushleft}


% \end{document}


\title{المصفوفات الأحادية البعد \textenglish{(1D Arrays)} في لغة \textenglish{C\#}}

\begin{document}
\maketitle

\section*{مقدمة}
المصفوفة \textenglish{(Array)} هي بنية معطيات تمكننا من تخزين مجموعة من القيم من نفس النوع تحت اسم واحد. بدلاً من إنشاء متغيرات منفصلة لكل قيمة، نستخدم المصفوفة لتنظيم البيانات بشكل أفضل.

\subsection*{لماذا نستخدم المصفوفات؟}
\begin{itemize}
    \item تخزين مجموعة كبيرة من البيانات بطريقة منظمة
    \item سهولة الوصول إلى العناصر عن طريق الفهرس
    \item تسهيل معالجة البيانات باستخدام الحلقات
    \item توفير في الكود وتجنب التكرار
\end{itemize}

\begin{boxExample}[مثال حياتي]
تخيّل أنك تريد تخزين درجات 30 طالباً في امتحان. بدون المصفوفات، ستحتاج إلى:
\begin{english}
\begin{minted}{csharp}
int grade1, grade2, grade3, ..., grade30;
\end{minted}
\end{english}
أما مع المصفوفات:
\begin{english}
\begin{minted}{csharp}
int[] grades = new int[30];
\end{minted}
\end{english}
\end{boxExample}

\section{التصريح عن المصفوفة وإنشاؤها}

\subsection{التصريح عن المصفوفة}
التصريح عن المصفوفة يتم بتحديد نوع العناصر ووضع الأقواس المربعة \textenglish{[]}.

\begin{boxSymbol}
\begin{english}
\begin{minted}{csharp}
type[] arrayName;
\end{minted}
\end{english}
حيث:
\begin{itemize}
    \item \textenglish{type}: نوع العناصر (\textenglish{int, double, string, bool, char}, إلخ)
    \item \textenglish{[]}: تدل على أن المتغير هو مصفوفة
    \item \textenglish{arrayName}: اسم المصفوفة
\end{itemize}
\end{boxSymbol}

\subsection{إنشاء المصفوفة}
بعد التصريح، يجب إنشاء المصفوفة وتحديد حجمها باستخدام الكلمة المفتاحية \textenglish{new}.

\begin{boxExample}[1]
\begin{english}
\begin{minted}{csharp}
// التصريح والإنشاء بخطوة واحدة
int[] numbers = new int[5];
\end{minted}
\end{english}
تُنشئ مصفوفة من 5 أعداد صحيحة، جميع القيم تبدأ بـ 0.
\end{boxExample}

\begin{boxExample}[2]
\begin{english}
\begin{minted}{csharp}
// التصريح والإنشاء بخطوتين
double[] scores;
scores = new double[10];
\end{minted}
\end{english}
\end{boxExample}

\begin{boxNote}
القيم الافتراضية للمصفوفات:
\begin{itemize}
    \item الأعداد (\textenglish{int, double}): 0
    \item \textenglish{bool}: \textenglish{false}
    \item \textenglish{string, object}: \textenglish{null}
    \item \textenglish{char}: \textenglish{'\\0'}
\end{itemize}
\end{boxNote}

\subsection{التهيئة المباشرة للمصفوفة}
يمكن تهيئة المصفوفة بقيم مباشرة عند إنشائها:

\begin{boxExample}[3]
\begin{english}
\begin{minted}{csharp}
// طريقة 1: مع تحديد الحجم
int[] numbers = new int[5] { 10, 20, 30, 40, 50 };

// طريقة 2: بدون تحديد الحجم (يُحدد تلقائياً)
int[] numbers = new int[] { 10, 20, 30, 40, 50 };

// طريقة 3: الأقصر (مفضلة)
int[] numbers = { 10, 20, 30, 40, 50 };
\end{minted}
\end{english}
\end{boxExample}

\section{الوصول إلى عناصر المصفوفة}

\subsection{الفهرس \textenglish{(Index)}}
كل عنصر في المصفوفة له رقم فهرس \textenglish{(Index)} يبدأ من 0.

\begin{boxDef}
\textbf{الفهرس} هو رقم يحدد موقع العنصر في المصفوفة:
\begin{itemize}
    \item الفهرس الأول: 0
    \item الفهرس الأخير: \textenglish{length - 1}
\end{itemize}
\end{boxDef}

\begin{boxExample}[4]
\begin{english}
\begin{minted}{csharp}
int[] numbers = { 10, 20, 30, 40, 50 };

// الوصول إلى العناصر
int first = numbers[0];    // 10
int second = numbers[1];   // 20
int last = numbers[4];     // 50

// تعديل قيمة عنصر
numbers[2] = 35;  // الآن numbers = { 10, 20, 35, 40, 50 }
\end{minted}
\end{english}
\end{boxExample}

\begin{boxWarning}
إذا حاولت الوصول إلى فهرس خارج نطاق المصفوفة، ستحصل على خطأ:
\begin{english}
\begin{minted}{csharp}
int[] arr = new int[5];
int x = arr[10];  // خطأ! IndexOutOfRangeException
\end{minted}
\end{english}
الفهارس الصحيحة هي من 0 إلى 4 فقط.
\end{boxWarning}

\subsection{خاصية الطول \textenglish{Length}}
كل مصفوفة لها خاصية \textenglish{Length} تُعطينا عدد العناصر.

\begin{boxExample}[5]
\begin{english}
\begin{minted}{csharp}
int[] numbers = { 10, 20, 30, 40, 50 };
int size = numbers.Length;  // 5

// الوصول للعنصر الأخير
int lastElement = numbers[numbers.Length - 1];  // 50
\end{minted}
\end{english}
\end{boxExample}

\clearpage

\section{التعامل مع المصفوفات باستخدام الحلقات}

\subsection{حلقة \textenglish{for}}
الحلقة \textenglish{for} هي الأكثر استخداماً مع المصفوفات.

\begin{boxExample}[6 - طباعة جميع العناصر]
\begin{english}
\begin{minted}{csharp}
int[] numbers = { 10, 20, 30, 40, 50 };

for (int i = 0; i < numbers.Length; i++)
{
    Console.WriteLine(numbers[i]);
}
\end{minted}
\end{english}
\end{boxExample}

\begin{boxExample}[7 - قراءة قيم المصفوفة من المستخدم]
\begin{english}
\begin{minted}{csharp}
int[] grades = new int[5];

Console.WriteLine("أدخل 5 درجات:");
for (int i = 0; i < grades.Length; i++)
{
    Console.Write($"الدرجة {i + 1}: ");
    grades[i] = int.Parse(Console.ReadLine());
}
\end{minted}
\end{english}
\end{boxExample}

\clearpage

\section{عمليات أساسية على المصفوفات}

\subsection{العد والجمع}

\subsubsection{جمع عناصر المصفوفة}
\begin{boxExample}[9]
\begin{english}
\begin{minted}{csharp}
int[] numbers = { 10, 20, 30, 40, 50 };
int sum = 0;

for (int i = 0; i < numbers.Length; i++)
{
    sum += numbers[i];
}

Console.WriteLine("المجموع: " + sum);  // 150
\end{minted}
\end{english}
\end{boxExample}

\subsubsection{عد عناصر بشرط معين}
\begin{boxExample}[10 - عد الأعداد الزوجية]
\begin{english}
\begin{minted}{csharp}
int[] numbers = { 5, 12, 8, 3, 15, 20, 7 };
int evenCount = 0;

for (int i = 0; i < numbers.Length; i++)
{
    if (numbers[i] % 2 == 0)
        evenCount++;
}

Console.WriteLine("عدد الأعداد الزوجية: " + evenCount);  // 3
\end{minted}
\end{english}
\end{boxExample}

\subsection{إيجاد الأكبر والأصغر}

\subsubsection{إيجاد أكبر عنصر}
\begin{boxExample}[11]
\begin{english}
\begin{minted}{csharp}
int[] numbers = { 5, 12, 8, 3, 15, 20, 7 };
int max = numbers[0];  // نفترض أن الأول هو الأكبر

for (int i = 1; i < numbers.Length; i++)
{
    if (numbers[i] > max)
        max = numbers[i];
}

Console.WriteLine("أكبر عنصر: " + max);  // 20
\end{minted}
\end{english}
\end{boxExample}

\subsubsection{إيجاد أكبر عنصر مع موقعه}
\begin{boxExample}[12]
\begin{english}
\begin{minted}{csharp}
int[] numbers = { 5, 12, 8, 3, 15, 20, 7 };
int max = numbers[0];
int maxIndex = 0;

for (int i = 1; i < numbers.Length; i++)
{
    if (numbers[i] > max)
    {
        max = numbers[i];
        maxIndex = i;
    }
}

Console.WriteLine($"أكبر عنصر: {max} في الموقع: {maxIndex}");
\end{minted}
\end{english}
\end{boxExample}

\clearpage

\subsection{البحث في المصفوفة}

\subsubsection{البحث الخطي \textenglish{(Linear Search)}}
\begin{boxExample}[13]
\begin{english}
\begin{minted}{csharp}
int[] numbers = { 5, 12, 8, 3, 15, 20, 7 };
int searchFor = 15;
bool found = false;

for (int i = 0; i < numbers.Length; i++)
{
    if (numbers[i] == searchFor)
    {
        found = true;
        Console.WriteLine($"العنصر موجود في الموقع: {i}");
        break;
    }
}

if (!found)
    Console.WriteLine("العنصر غير موجود");
\end{minted}
\end{english}
\end{boxExample}

\subsubsection{إيجاد جميع مواقع العنصر}
\begin{boxExample}[14]
\begin{english}
\begin{minted}{csharp}
int[] numbers = { 5, 12, 8, 12, 15, 12, 7 };
int searchFor = 12;
int count = 0;

Console.WriteLine($"مواقع العنصر {searchFor}:");
for (int i = 0; i < numbers.Length; i++)
{
    if (numbers[i] == searchFor)
    {
        Console.WriteLine($"الموقع: {i}");
        count++;
    }
}

Console.WriteLine($"العنصر موجود {count} مرات");
\end{minted}
\end{english}
\end{boxExample}

\subsection{نسخ المصفوفات}

\begin{boxWarning}
\textbf{انتبه!} هذا الكود لا ينسخ المصفوفة:
\begin{english}
\begin{minted}{csharp}
int[] arr1 = { 1, 2, 3 };
int[] arr2 = arr1;  // لا ينسخ! كلاهما يشير لنفس المصفوفة

arr2[0] = 100;
Console.WriteLine(arr1[0]);  // 100 (تغيّر أيضاً!)
\end{minted}
\end{english}
\end{boxWarning}

\subsubsection{النسخ الصحيح}
\begin{boxExample}[15 - نسخ يدوي]
\begin{english}
\begin{minted}{csharp}
int[] arr1 = { 1, 2, 3, 4, 5 };
int[] arr2 = new int[arr1.Length];

for (int i = 0; i < arr1.Length; i++)
{
    arr2[i] = arr1[i];
}

arr2[0] = 100;
Console.WriteLine(arr1[0]);  // 1 (لم يتغير)
Console.WriteLine(arr2[0]);  // 100
\end{minted}
\end{english}
\end{boxExample}

\begin{boxExample}[16 - استخدام \textenglish{Array.Copy}]
\begin{english}
\begin{minted}{csharp}
int[] arr1 = { 1, 2, 3, 4, 5 };
int[] arr2 = new int[arr1.Length];

Array.Copy(arr1, arr2, arr1.Length);
\end{minted}
\end{english}
\end{boxExample}

\clearpage

\section{عمليات متقدمة}

\subsection{عكس المصفوفة}
\begin{boxExample}[17]
\begin{english}
\begin{minted}{csharp}
int[] numbers = { 1, 2, 3, 4, 5 };

// عكس المصفوفة
for (int i = 0; i < numbers.Length / 2; i++)
{
    int temp = numbers[i];
    numbers[i] = numbers[numbers.Length - 1 - i];
    numbers[numbers.Length - 1 - i] = temp;
}

// الآن: numbers = { 5, 4, 3, 2, 1 }
\end{minted}
\end{english}
\end{boxExample}

\subsection{إزاحة العناصر}

\subsubsection{إزاحة لليمين}
\begin{boxExample}[18]
\begin{english}
\begin{minted}{csharp}
int[] arr = { 1, 2, 3, 4, 5 };

// حفظ العنصر الأخير
int last = arr[arr.Length - 1];

// إزاحة جميع العناصر لليمين
for (int i = arr.Length - 1; i > 0; i--)
{
    arr[i] = arr[i - 1];
}

// وضع العنصر الأخير في البداية
arr[0] = last;

// الآن: arr = { 5, 1, 2, 3, 4 }
\end{minted}
\end{english}
\end{boxExample}

\subsection{فلترة العناصر}
\begin{boxExample}[19 - نسخ الأعداد الزوجية فقط]
\begin{english}
\begin{minted}{csharp}
int[] numbers = { 5, 12, 8, 3, 15, 20, 7, 2 };

// عد الأعداد الزوجية أولاً
int evenCount = 0;
for (int i = 0; i < numbers.Length; i++)
{
    if (numbers[i] % 2 == 0)
        evenCount++;
}

// إنشاء مصفوفة جديدة للأعداد الزوجية
int[] evenNumbers = new int[evenCount];
int index = 0;

for (int i = 0; i < numbers.Length; i++)
{
    if (numbers[i] % 2 == 0)
    {
        evenNumbers[index] = numbers[i];
        index++;
    }
}

// evenNumbers = { 12, 8, 20, 2 }
\end{minted}
\end{english}
\end{boxExample}

\clearpage

\section{دوال مساعدة للمصفوفات}

\subsection{دالة لطباعة المصفوفة}
\begin{boxExample}[20]
\begin{english}
\begin{minted}{csharp}
public static void PrintArray(int[] arr)
{
    Console.Write("[ ");
    for (int i = 0; i < arr.Length; i++)
    {
        Console.Write(arr[i]);
        if (i < arr.Length - 1)
            Console.Write(", ");
    }
    Console.WriteLine(" ]");
}

// الاستخدام:
int[] numbers = { 1, 2, 3, 4, 5 };
PrintArray(numbers);  // [ 1, 2, 3, 4, 5 ]
\end{minted}
\end{english}
\end{boxExample}

\subsection{دالة لحساب المجموع}
\begin{boxExample}[21]
\begin{english}
\begin{minted}{csharp}
public static int SumArray(int[] arr)
{
    int sum = 0;
    for (int i = 0; i < arr.Length; i++)
    {
        sum += arr[i];
    }
    return sum;
}

// الاستخدام:
int[] numbers = { 10, 20, 30 };
int total = SumArray(numbers);  // 60
\end{minted}
\end{english}
\end{boxExample}

\subsection{دالة للبحث}
\begin{boxExample}[22]
\begin{english}
\begin{minted}{csharp}
public static int FindElement(int[] arr, int value)
{
    for (int i = 0; i < arr.Length; i++)
    {
        if (arr[i] == value)
            return i;  // إرجاع الموقع
    }
    return -1;  // لم يُعثر على العنصر
}

// الاستخدام:
int[] numbers = { 5, 12, 8, 3, 15 };
int position = FindElement(numbers, 8);  // 2
int notFound = FindElement(numbers, 100);  // -1
\end{minted}
\end{english}
\end{boxExample}

\clearpage

\section{أخطاء شائعة وكيفية تجنبها}

\begin{boxAttention}[1 - نسيان أن الفهارس تبدأ من 0]
\begin{english}
\begin{minted}{csharp}
int[] arr = new int[5];
arr[5] = 10;  // خطأ! الفهارس من 0 إلى 4
\end{minted}
\end{english}
\textbf{الصحيح:}
\begin{english}
\begin{minted}{csharp}
arr[4] = 10;  // الفهرس الأخير هو Length - 1
\end{minted}
\end{english}
\end{boxAttention}

\begin{boxAttention}[2 - استخدام = بدلاً من == في الشروط]
\begin{english}
\begin{minted}{csharp}
if (arr[i] = 10)  // خطأ! هذا يقوم بالتعيين
\end{minted}
\end{english}
\textbf{الصحيح:}
\begin{english}
\begin{minted}{csharp}
if (arr[i] == 10)  // صحيح - مقارنة
\end{minted}
\end{english}
\end{boxAttention}

\begin{boxAttention}[3 - نسيان تهيئة المصفوفة]
\begin{english}
\begin{minted}{csharp}
int[] arr;
arr[0] = 10;  // خطأ! المصفوفة غير مُنشأة
\end{minted}
\end{english}
\textbf{الصحيح:}
\begin{english}
\begin{minted}{csharp}
int[] arr = new int[5];
arr[0] = 10;  // صحيح
\end{minted}
\end{english}
\end{boxAttention}

\section{تمارين للتدريب}

\begin{boxExercise}[1]
اكتب برنامجاً يقرأ 10 أعداد من المستخدم ويطبع:
\begin{itemize}
    \item مجموعها
    \item معدلها
    \item أكبر عدد
    \item أصغر عدد
\end{itemize}
\end{boxExercise}

\begin{boxExercise}[2]
اكتب برنامجاً يقرأ مصفوفة من 8 أعداد ويطبع:
\begin{itemize}
    \item جميع الأعداد الزوجية
    \item جميع الأعداد الفردية
    \item عدد الأعداد الموجبة
    \item عدد الأعداد السالبة
\end{itemize}
\end{boxExercise}

\begin{boxExercise}[3]
اكتب دالة تأخذ مصفوفة من الأعداد وتُرجع \textenglish{true} إذا كانت المصفوفة متناظرة (مثل \{1, 2, 3, 2, 1\})، و\textenglish{false} بخلاف ذلك.
\end{boxExercise}

\begin{boxExercise}[4]
اكتب برنامجاً يقرأ مصفوفتين بنفس الحجم ويُنشئ مصفوفة ثالثة تحتوي على مجموع كل عنصرين متقابلين.
\end{boxExercise}

\begin{boxExercise}[5]
اكتب دالة تأخذ مصفوفة وعدداً \textenglish{n}، وتُزيح عناصر المصفوفة \textenglish{n} مرات لليمين.
\end{boxExercise}

\section*{ملخص}

\begin{itemize}
    \item المصفوفة تخزن مجموعة من القيم من نفس النوع
    \item الفهارس تبدأ من 0 وتنتهي عند \textenglish{Length - 1}
    \item نستخدم حلقة \textenglish{for} للتعامل مع المصفوفات
    \item العمليات الأساسية: الجمع، العد، البحث، إيجاد الأكبر/الأصغر
    \item النسخ بـ = لا ينسخ المصفوفة، بل يُنشئ مرجعاً لنفس المصفوفة
    \item يمكن كتابة دوال لتسهيل العمليات المتكررة على المصفوفات
\end{itemize}

\vspace{1cm}
\begin{flushleft}
حظاً موفقاً في التعلم!

الأستاذ محمود اغبارية.
\end{flushleft}

\end{document}
