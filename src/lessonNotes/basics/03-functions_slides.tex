\documentclass[13pt]{beamer}

\usepackage{fontspec}
\usepackage{polyglossia}
\setmainlanguage[numerals=western]{arabic}
\setotherlanguage{english}
\newfontfamily\arabicfont[Script=Arabic]{Amiri}
\newfontfamily\arabicfontsf[Script=Arabic]{Amiri}
\newfontfamily\arabicfonttt[Script=Arabic]{Courier New}

% ============================================================
% Fix: visible bullets + proper RTL alignment
% ============================================================
\usepackage{bidi} % <-- enables proper RTL lists and bullets
\setbeamertemplate{itemize items}[circle] % visible bullets
\setbeamertemplate{itemize subitem}[triangle] % optional, for nested lists

\usepackage{listings}
\usepackage{xcolor}

\lstdefinelanguage{CSharp}{
  language={[Sharp]C},
  morekeywords={string,bool,void}
}

\lstset{
  language=CSharp,
  basicstyle=\ttfamily\small,
  keywordstyle=\color{blue},
  stringstyle=\color{red},
  commentstyle=\color{green!50!black},
  showstringspaces=false,
  numbers=none,
  frame=single,
  breaklines=true
}

\title{العمليات الخارجية في لغة \textenglish{C\#}}
\author{}
\date{}

\begin{document}

\begin{frame}
  \titlepage
\end{frame}

\begin{frame}[fragile]{مقدمة}
العمليات تقسّم البرنامج إلى مقاطع تؤدي مهام محددة.
\begin{itemize}
  \item تنظيم الكود وجعله أسهل للقراءة والصيانة.
  \item إعادة استخدام نفس الكود أكثر من مرة.
  \item سنعرّف الأنواع بحسب الاستقبال والإرجاع وكيفية التعريف والاستدعاء.
\end{itemize}
\end{frame}

% مبنى تعريف العملية
\begin{frame}[fragile]{مبنى تعريف العملية}
كل عملية تتكوّن من:
\begin{itemize}
  \item \textbf{تعريف العملية} \textenglish{(Method Signature)}: \textenglish{public static}، نوع القيمة، اسم العملية، البرامترات \textenglish{(int x, int y)}.
  \item \textbf{جسم العملية} \textenglish{(Function Body)}: الأوامر بين \textenglish{\{\}} وتُنفّذ عند الاستدعاء.
\end{itemize}
\end{frame}

% مثال: PrintHello
\begin{frame}[fragile]{مثال: عملية لا تُعيد ولا تستقبل}
\begin{english}
\begin{lstlisting}[language=CSharp]
public static void PrintHello()
{
    Console.WriteLine("Hello");
}
\end{lstlisting}
\end{english}
\vspace{0.5em}
\begin{itemize}
  \item \textenglish{public static} جزء ثابت في هذه المرحلة.
  \item \textenglish{void} لا تُعيد شيئًا.
  \item \textenglish{()} لا تستقبل بارامترات.
\end{itemize}
\end{frame}

% مثال: AddNumbers
\begin{frame}[fragile]{مثال: عملية تُعيد وتستقبل}
\begin{english}
\begin{lstlisting}[language=CSharp]
public static int AddNumbers(int a, int b)
{
    int sum = a + b;
    return sum;
}
\end{lstlisting}
\end{english}
\vspace{0.5em}
\begin{itemize}
  \item تُعيد \textenglish{int} ويجب أن يطابق نوع \textenglish{return}.
  \item تستقبل \textenglish{(int a, int b)}.
\end{itemize}
\end{frame}

% استدعاء العملية
\begin{frame}[fragile]{استدعاء العملية}
\begin{english}
\begin{lstlisting}[language=CSharp]
public static void Main(string[] args)
{
    PrintHello();
    int sum1 = AddNumbers(5, 10);
    int a = 5;
    int b = 10;
    int sum2 = AddNumbers(a, b);
}
\end{lstlisting}
\end{english}
\vspace{0.5em}
\begin{itemize}
  \item \textenglish{PrintHello()} لا تُعيد قيمة.
  \item \textenglish{AddNumbers} تُعيد قيمة تُحفظ في متغير من نفس النوع.
  \item ترتيب وأنواع البرامترات يجب أن يطابق التعريف.
\end{itemize}
\end{frame}

% استدعاء من فئة أخرى
\begin{frame}[fragile]{استدعاء عملية من فئة أخرى}
نكتب اسم الفئة ثم نقطة ثم اسم العملية:
\begin{itemize}
  \item \textenglish{Console.WriteLine("Hello World!");}
  \item \textenglish{Console.ReadLine();}
  \item \textenglish{Math.Max(5, 10); ~ Math.Min(5, 10);}
  \item \textenglish{Math.Abs(-5); ~ Math.Pow(2, 3);}
  \item \textenglish{Math.Sqrt(25); ~ Math.Round(2.5);}
\end{itemize}
\end{frame}

% أنواع العمليات (概览)
\begin{frame}[fragile]{أنواع العمليات}
\begin{itemize}
  \item لا تُرجع ولا تستقبل \textenglish{(void, no params)}
  \item تستقبل ولا تُرجع \textenglish{(void, with params)}
  \item تُرجع ولا تستقبل
  \item تستقبل وتُرجع
\end{itemize}
\end{frame}

% لا ترجع ولا تتلقى
\begin{frame}[fragile]{لا ترجع ولا تتلقى}
\begin{english}
\begin{lstlisting}[language=CSharp]
public static void PrintWelcome()
{
    Console.WriteLine("Welcome to the program!");
}

public static void Main()
{
    PrintWelcome();
}
\end{lstlisting}
\end{english}
\end{frame}

% تتلقى ولا ترجع - مثال 1
\begin{frame}[fragile]{تتلقى ولا ترجع}
\begin{english}
\begin{lstlisting}[language=CSharp]
public static void PrintSum(int a, int b)
{
    int sum = a + b;
    Console.WriteLine("Sum = " + sum);
}

public static void Main()
{
    PrintSum(5, 3);
}
\end{lstlisting}
\end{english}
\end{frame}

% تتلقى ولا ترجع - مثال 2
\begin{frame}[fragile]{تتلقى ولا ترجع (بمتغيّرات)}
\begin{english}
\begin{lstlisting}[language=CSharp]
public static void Main()
{
    int x = 5;
    int y = 3;
    PrintSum(x, y);
}
\end{lstlisting}
\end{english}
\end{frame}

% ترجع ولا تتلقى
\begin{frame}[fragile]{ترجع ولا تتلقى}
\begin{english}
\begin{lstlisting}[language=CSharp]
public static int ReadNumber()
{
    int num = int.Parse(Console.ReadLine());
    return num;
}

public static void Main()
{
    int x = ReadNumber();
    Console.WriteLine("The number: " + x);
}
\end{lstlisting}
\end{english}
\end{frame}

% تتلقى وترجع
\begin{frame}[fragile]{تتلقى وترجع}
\begin{english}
\begin{lstlisting}[language=CSharp]
public static double GetAverage(double a, double b)
{
    double avg = (a + b) / 2.0;
    return avg;
}

public static void Main()
{
    double result = GetAverage(7.5, 9.0);
    Console.WriteLine("Average = " + result);
}
\end{lstlisting}
\end{english}
\end{frame}

% أمثلة نعرفها
\begin{frame}[fragile]{أمثلة نعرفها}
\begin{english}
\begin{itemize}
  \item \textenglish{Math.Max/Min(int,int) -> int}
  \item \textenglish{Math.Abs/Pow(double, double) -> double}
  \item \textenglish{Math.Round/Sqrt(double) -> double}
  \item \textenglish{int.Parse(string) -> int}
  \item \textenglish{double.Parse(string) -> double}
\end{itemize}
\end{english}
\end{frame}

% ملاحظات مهمة
\begin{frame}[fragile]{ملاحظات مهمة}
\begin{itemize}
  \item أسماء العمليات تبدأ بحرف كبير وتصف الوظيفة: \textenglish{CalculateSum(), GetMaxValue(), PrintResult()}
  \item اجعل كل عملية تقوم بمهمة واحدة، وتجنّب تكرار الكود.
  \item اختبر العملية جيدًا، ويمكن استدعاء عملية من داخل أخرى.
  \item ممكن استخدام \textenglish{if..else} داخل العملية وتمرير ناتج عملية كبرامتر لأخرى.
\end{itemize}
\end{frame}

% أمثلة متنوعة
\begin{frame}[fragile]{مثال: void بدون برامترات}
\begin{english}
\begin{lstlisting}[language=CSharp]
public static void SayHi()
{
    Console.WriteLine("Hi!");
}

public static void Main()
{
    SayHi();
}
\end{lstlisting}
\end{english}
\end{frame}

\begin{frame}[fragile]{مثال: void مع برامتر}
\begin{english}
\begin{lstlisting}[language=CSharp]
public static void PrintSquare(int n)
{
    Console.WriteLine("Square = " + (n * n));
}

public static void Main()
{
    PrintSquare(5);
}
\end{lstlisting}
\end{english}
\end{frame}

\begin{frame}[fragile]{مثال: int مع برامترات}
\begin{english}
\begin{lstlisting}[language=CSharp]
public static int Multiply(int x, int y)
{
    return x * y;
}

public static void Main()
{
    int p = Multiply(4, 6);
    Console.WriteLine(p);
}
\end{lstlisting}
\end{english}
\end{frame}

\begin{frame}[fragile]{مثال: double مع برامتر}
\begin{english}
\begin{lstlisting}[language=CSharp]
public static double CircleArea(double radius)
{
    return Math.PI * radius * radius;
}

public static void Main()
{
    double area = CircleArea(2.5);
    Console.WriteLine("Area = " + area);
}
\end{lstlisting}
\end{english}
\end{frame}

\begin{frame}[fragile]{مثال: bool يتحقق من الزوجية}
\begin{english}
\begin{lstlisting}[language=CSharp]
public static bool IsEven(int num)
{
    return num % 2 == 0;
}

public static void Main()
{
    bool even = IsEven(8);
    Console.WriteLine("Even? " + even);
}
\end{lstlisting}
\end{english}
\end{frame}

\begin{frame}[fragile]{مثال: string يُعيد تحية}
\begin{english}
\begin{lstlisting}[language=CSharp]
public static string Greet(string name)
{
    return "Hello " + name + "!";
}

public static void Main()
{
    string msg = Greet("Ali");
    Console.WriteLine(msg);
}
\end{lstlisting}
\end{english}
\end{frame}

\begin{frame}[fragile]{مثال: string بدون برامترات}
\begin{english}
\begin{lstlisting}[language=CSharp]
public static string GetDay()
{
    return "Friday";
}

public static void Main()
{
    Console.WriteLine("Today is " + GetDay());
}
\end{lstlisting}
\end{english}
\end{frame}

\begin{frame}[fragile]{مثال: bool يتحقق من الإيجابية}
\begin{english}
\begin{lstlisting}[language=CSharp]
public static bool IsPositive(int n)
{
    return n > 0;
}

public static void Main()
{
    Console.WriteLine(IsPositive(10));
    Console.WriteLine(IsPositive(-3));
}
\end{lstlisting}
\end{english}
\end{frame}

\begin{frame}[fragile]{مثال: void مع أنواع مختلفة}
\begin{english}
\begin{lstlisting}[language=CSharp]
public static void PrintPerson(string name, int age)
{
    Console.WriteLine(name + " is " + age + " years old.");
}

public static void Main()
{
    PrintPerson("Mona", 25);
}
\end{lstlisting}
\end{english}
\end{frame}

\begin{frame}[fragile]{مثال: شرط وإرجاع string}
\begin{english}
\begin{lstlisting}[language=CSharp]
public static string CheckGrade(int grade)
{
    if (grade >= 60)
        return "Pass";
    else
        return "Fail";
}

public static void Main()
{
    Console.WriteLine(CheckGrade(85));
    Console.WriteLine(CheckGrade(40));
}
\end{lstlisting}
\end{english}
\end{frame}

\begin{frame}[fragile]{مثال: قسمة آمنة}
\begin{english}
\begin{lstlisting}[language=CSharp]
public static double SafeDivide(double a, double b)
{
    if (b == 0)
        return 0;
    return a / b;
}

public static void Main()
{
    Console.WriteLine(SafeDivide(10, 2));
    Console.WriteLine(SafeDivide(5, 0));
}
\end{lstlisting}
\end{english}
\end{frame}

\end{document}
