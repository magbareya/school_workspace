\documentclass[14pt]{extarticle}
% Full article preamble (duplicated, no common file)
\usepackage{fontspec}
\usepackage[a4paper,top=2.4cm,bottom=2.4cm,left=2.3cm,right=2.3cm]{geometry}
\usepackage{polyglossia}
\usepackage{amsmath}
\usepackage{amssymb}
\usepackage{xcolor}
\usepackage{fancyhdr}
\usepackage{graphicx}
\usepackage{listings}
\usepackage[most]{tcolorbox}
\usepackage{pifont}
\usepackage{enumitem}
\usepackage{titlesec}
\usepackage[bottom]{footmisc}
\usepackage{titling}
\usepackage{minted}
\usepackage{etoolbox}
\usepackage{array}
\usepackage{extsizes}

\newfontfamily\emoji{Segoe UI Emoji}

\pagestyle{fancy}

\setmainlanguage[numerals=western]{arabic}
\setotherlanguage{english}
\newfontfamily\arabicfont[Script=Arabic]{Amiri}
\newfontfamily\arabicfonttt[Script=Arabic]{Courier New}

\lstset{
  language=[Sharp]C,
  numbers=left,
  stepnumber=1,
  numbersep=8pt,
  frame=single,
  basicstyle=\ttfamily\small,
  keywordstyle=\color{blue},
  stringstyle=\color{red},
  commentstyle=\color{green!50!black}
}

\newif\ifdetailed
\ifdefined\setdetailed
  \setdetailed
\fi

\newif\ifwithsols
\ifdefined\setwithsols
  \setwithsols
\fi

% unified tcolorboxes for articles
\tcbset{colback=white, colframe=black, fonttitle=\bfseries, boxrule=0.8pt}
\newtcolorbox{boxDef}[1][]{colback=blue!5!white,colframe=blue!75!black,
  title={{\emoji📘} تعريف\ifx\\#1\\\else ~#1\fi :}}
\newtcolorbox{boxExercise}[1][]{colback=cyan!5!white,colframe=cyan!70!black,
  title={{\emoji🧩} تمرين\ifx\\#1\\\else ~#1\fi :}}
\newtcolorbox{boxExample}[1][]{colback=yellow!5!white,colframe=orange!90!black,
  title={{\emoji📝} مثال\ifx\\#1\\\else ~#1\fi :}}
\newtcolorbox{boxNote}[1][]{colback=gray!10!white,colframe=black,
  title={{\emoji✨} ملاحظة\ifx\\#1\\\else ~#1\fi :}}
\newtcolorbox{boxAttention}[1][]{colback=magenta!10!white,colframe=magenta!80!black,
  title={{\emoji🔔} تنبيه\ifx\\#1\\\else ~#1\fi :}}
\newtcolorbox{boxWarning}[1][]{colback=red!5!white,colframe=red!75!black,
  title={{\emoji⚡} ملاحظة هامة\ifx\\#1\\\else ~#1\fi :}}
\newtcolorbox{boxSolution}[1][]{colback=green!5!white,colframe=green!60!black,
  title={{\emoji✅} حل\ifx\\#1\\\else ~#1\fi :}}
\newtcolorbox{boxSymbol}[1][]{colback=purple!5!white,colframe=purple!70!black,
  title={{\emoji🔣} رمز\ifx\\#1\\\else ~#1\fi :}}
\newtcolorbox{boxHint}[1][]{colback=teal!5!white,colframe=teal!60!black,
  title={{\emoji💡} تلميح\ifx\\#1\\\else ~#1\fi :}}


\tcbset{simplecode/.style={ colback=gray!5, colframe=black!50, boxrule=0.4pt, arc=2pt, left=4pt,right=4pt,top=4pt,bottom=4pt}}
\newenvironment{boxCode}{\begin{tcolorbox}[simplecode]}{\end{tcolorbox}}

\newcolumntype{C}[1]{>{\centering\arraybackslash}p{#1}}

% redefine spaces after titles
\makeatletter
\renewcommand{\@maketitle}{%
  \begin{center}
    {\huge \bfseries \@title \par}%
    \vskip 0.2em % space between title and author
    {\large \@author \par}%
    % \vskip 0.2em % space between author and date
    % {\normalsize \@date \par}%
  \end{center}
}
\makeatother

\fancyhf{} % clear default
\fancypagestyle{plain}{
  \fancyhf{}
  \fancyhead[L]{مدرسة التسامح الشاملة}
  % \fancyhead[L]{\includegraphics[height=1cm]{../../../images/logoTasamoh.png}}
  \fancyhead[R]{الأستاذ محمود اغبارية}
  \fancyfoot[C]{\thepage}
}

\fancyhead[L]{مدرسة التسامح الشاملة}
\fancyhead[R]{الأستاذ محمود اغبارية}
\fancyfoot[C]{\thepage}
% \date{\today}

\setcounter{tocdepth}{3} % only section subsection and subsubsection in TOC


% ----------------------


% \begin{document}

% \maketitle

% % \clearpage  % start TOC on a new page
% % \renewcommand{\contentsname}{جدول المحتويات}
% % \tableofcontents
% % \clearpage

% \part*{part 1} % the * prevents numbering
% \section*{مقدمة}
% \subsection*{مثال رياضي}
% \subsubsection*{مثال فرعي}
% \paragraph*{ paragraph 1}
% \subparagraph*{sub paragraph 1}

% \ifdetailed
% \begin{english}
% \begin{minted}{csharp}
% // C# Example
% \end{minted}
% \end{english}
% \fi

% OLD WAY
% \ifdetailed
% \begin{english}
% \begin{lstlisting}
% // C# Example
% \end{lstlisting}
% \end{english}
% \fi

% % \includegraphics[width=0.2\textwidth]{../../../images/DFAs/ex1_q1.png}



% \vspace{3cm}
% \begin{flushleft}
% أرجو لكم وقتًا ممتعًا.

% الأستاذ محمود اغبارية.
% \end{flushleft}


% \end{document}


\title{الحلقات}
\date{}
\author{}

\begin{document}
\maketitle

\section*{مقدمة}
الحلقة هي مقطع كود يتكرر عدة مرات لتحقيق مهمة متكررة دون إعادة كتابة الأوامر:
\begin{itemize}
  \item التكرار يجعل البرنامج أقصر وأكثر مرونة.
  \item يمكن للحاسوب تنفيذ نفس التعليمات آلاف المرات بسرعة.
  \item توجد أنواع مختلفة، أهمها: \textenglish{while} و \textenglish{for}.
\end{itemize}

% ========================
\section{حلقة while}

\subsection{مفهوم حلقة while}
فكرة الحلقة: \textbf{طالما أن شرطًا ما صحيح، تستمر الأوامر بالتنفيذ}.

\begin{itemize}
    \item تستخدم عندما لا نعرف عدد التكرارات مسبقًا.
    \item يفحص الشرط قبل كل تكرار.
    \item إذا أصبح الشرط \textenglish{false} تتوقف الحلقة.
\end{itemize}

\subsection{مبنى حلقة while}
\begin{english}
\begin{lstlisting}
while (condition)
{
    // body
}
\end{lstlisting}
\end{english}

\subsection{جمع الأعداد حتى إدخال 0}
\begin{english}
\begin{lstlisting}
int sum = 0;
int num = int.Parse(Console.ReadLine());
while (num != 0)
{
    sum += num;
    num = int.Parse(Console.ReadLine());
}
Console.WriteLine("Sum = " + sum);
\end{lstlisting}
\end{english}

\subsection{جمع حتى يصبح المجموع أكبر من 10}
\begin{english}
\begin{lstlisting}
int sum = 0;
while (sum <= 10)
{
    int x = int.Parse(Console.ReadLine());
    sum += x;
}
Console.WriteLine("Total = " + sum);
\end{lstlisting}
\end{english}

\subsection{الضرب حتى إدخال 1}
\begin{english}
\begin{lstlisting}
int prod = 1;
int num = int.Parse(Console.ReadLine());
while (num != 1)
{
    prod *= num;
    num = int.Parse(Console.ReadLine());
}
Console.WriteLine("Product = " + prod);
\end{lstlisting}
\end{english}

\subsection{عدد الأعداد الموجبة حتى إدخال عدد سالب}
\begin{english}
\begin{lstlisting}
int count = 0;
int num = int.Parse(Console.ReadLine());
while (num >= 0)
{
    count++;
    num = int.Parse(Console.ReadLine());
}
Console.WriteLine("Count = " + count);
\end{lstlisting}
\end{english}

\subsection{عدد الأعداد الزوجية حتى إدخال -1}
\begin{english}
\begin{lstlisting}
int evenCount = 0;
int num = int.Parse(Console.ReadLine());
while (num != -1)
{
    if (num % 2 == 0)
        evenCount++;
    num = int.Parse(Console.ReadLine());
}
Console.WriteLine("Even count = " + evenCount);
\end{lstlisting}
\end{english}

\subsection{قراءة 3 أعداد زوجية بالضبط}
\begin{english}
\begin{lstlisting}
int number, count = 0;
while (count < 3)
{
    Console.WriteLine("enter a number");
    number = int.Parse(Console.ReadLine());
    if (number % 2 == 0)
        count++;
}
Console.WriteLine("End loop");
Console.WriteLine("count=" + count);
\end{lstlisting}
\end{english}

\subsection{إيجاد أكبر عدد حتى إدخال 0}
\begin{english}
\begin{lstlisting}
int max = int.MinValue;
int num = int.Parse(Console.ReadLine());
while (num != 0)
{
    if (num > max)
        max = num;
    num = int.Parse(Console.ReadLine());
}
Console.WriteLine("Max = " + max);
\end{lstlisting}
\end{english}

\subsection{إيجاد أصغر عدد حتى إدخال 0}
\begin{english}
\begin{lstlisting}
int min = int.MaxValue;
int num = int.Parse(Console.ReadLine());
while (num != 0)
{
    if (num < min)
        min = num;
    num = int.Parse(Console.ReadLine());
}
Console.WriteLine("Min = " + min);
\end{lstlisting}
\end{english}

\clearpage
\subsection{عدد خانات عدد معين}
\begin{english}
\begin{lstlisting}
int n = int.Parse(Console.ReadLine());
int count = 0;
while (n > 0)
{
    n /= 10;
    count++;
}
Console.WriteLine("Digits = " + count);
\end{lstlisting}
\end{english}

\subsection{التحقق الصارم من المدخلات}
\begin{english}
\begin{lstlisting}
int n = int.Parse(Console.ReadLine());
while (n < 1 || n > 10)
{
    Console.WriteLine("Invalid! Enter 1-10:");
    n = int.Parse(Console.ReadLine());
}
\end{lstlisting}
\end{english}

\subsection{شرط توقف مركب}
\begin{english}
\begin{lstlisting}
int sum = 0, i = 0;
while (i < 10 && sum < 200)
{
    sum += int.Parse(Console.ReadLine());
    i++;
}
\end{lstlisting}
\end{english}

% ========================
\clearpage
\section{حلقة for}

\subsection{متى نستخدم for؟}
\begin{itemize}
  \item عند معرفة عدد التكرارات.
  \item بنية الحلقة تحتوي على: تهئية - شرط - تحديث.
\end{itemize}

\subsection{مبنى حلقة for}
\begin{english}
\begin{lstlisting}
for (initialization; condition; update)
{
    // body
}
\end{lstlisting}
\end{english}

\subsection{مثال: طباعة الأعداد 1..3}
\begin{english}
\begin{lstlisting}
for (int i = 1; i <= 3; i++)
{
    Console.WriteLine(i);
}
\end{lstlisting}
\end{english}

\subsection{طباعة "Hello" 5 مرات}
\begin{english}
\begin{lstlisting}
for (int i = 0; i < 5; i++)
{
    Console.WriteLine("Hello");
}
\end{lstlisting}
\end{english}

\subsection{الأعداد الزوجية 1..100}
\begin{english}
\begin{lstlisting}
for (int i = 2; i <= 100; i += 2)
    Console.WriteLine(i);
\end{lstlisting}
\end{english}

\clearpage
\subsection{القسمة على 5 حتى N}
\begin{english}
\begin{lstlisting}
int N = int.Parse(Console.ReadLine());
for (int i = 0; i <= N; i++)
{
    if (i % 5 == 0)
        Console.WriteLine(i);
}
\end{lstlisting}
\end{english}

\subsection{حساب المجموع}
\begin{english}
\begin{lstlisting}
int sum = 0;
for (int i = 1; i <= 100; i++)
    sum += i;
Console.WriteLine(sum);
\end{lstlisting}
\end{english}

\subsection{إيجاد القيمة العظمى}
\begin{english}
\begin{lstlisting}
int max = int.MinValue;
for (int i = 0; i < 10; i++)
{
    int x = int.Parse(Console.ReadLine());
    if (x > max) max = x;
}
Console.WriteLine("Max = " + max);
\end{lstlisting}
\end{english}

\subsection{المتوسط}
\begin{english}
\begin{lstlisting}
int sum = 0, n = 5;
for (int i = 0; i < n; i++)
{
    int x = int.Parse(Console.ReadLine());
    sum += x;
}
double avg = sum / (double)n;
Console.WriteLine("Average = " + avg);
\end{lstlisting}
\end{english}
\end{document}
