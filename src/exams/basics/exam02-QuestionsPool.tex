\documentclass[14pt]{extarticle}
% Full article preamble (duplicated, no common file)
\usepackage{fontspec}
\usepackage[a4paper,top=2.4cm,bottom=2.4cm,left=2.3cm,right=2.3cm]{geometry}
\usepackage{polyglossia}
\usepackage{amsmath}
\usepackage{amssymb}
\usepackage{xcolor}
\usepackage{fancyhdr}
\usepackage{graphicx}
\usepackage{listings}
\usepackage[most]{tcolorbox}
\usepackage{pifont}
\usepackage{enumitem}
\usepackage{titlesec}
\usepackage[bottom]{footmisc}
\usepackage{titling}
\usepackage{minted}
\usepackage{etoolbox}
\usepackage{array}
\usepackage{extsizes}

\newfontfamily\emoji{Segoe UI Emoji}

\pagestyle{fancy}

\setmainlanguage[numerals=western]{arabic}
\setotherlanguage{english}
\newfontfamily\arabicfont[Script=Arabic]{Amiri}
\newfontfamily\arabicfonttt[Script=Arabic]{Courier New}

\lstset{
  language=[Sharp]C,
  numbers=left,
  stepnumber=1,
  numbersep=8pt,
  frame=single,
  basicstyle=\ttfamily\small,
  keywordstyle=\color{blue},
  stringstyle=\color{red},
  commentstyle=\color{green!50!black}
}

\newif\ifdetailed
\ifdefined\setdetailed
  \setdetailed
\fi

\newif\ifwithsols
\ifdefined\setwithsols
  \setwithsols
\fi

% unified tcolorboxes for articles
\tcbset{colback=white, colframe=black, fonttitle=\bfseries, boxrule=0.8pt}
\newtcolorbox{boxDef}[1][]{colback=blue!5!white,colframe=blue!75!black,
  title={{\emoji📘} تعريف\ifx\\#1\\\else ~#1\fi :}}
\newtcolorbox{boxExercise}[1][]{colback=cyan!5!white,colframe=cyan!70!black,
  title={{\emoji🧩} تمرين\ifx\\#1\\\else ~#1\fi :}}
\newtcolorbox{boxExample}[1][]{colback=yellow!5!white,colframe=orange!90!black,
  title={{\emoji📝} مثال\ifx\\#1\\\else ~#1\fi :}}
\newtcolorbox{boxNote}[1][]{colback=gray!10!white,colframe=black,
  title={{\emoji✨} ملاحظة\ifx\\#1\\\else ~#1\fi :}}
\newtcolorbox{boxAttention}[1][]{colback=magenta!10!white,colframe=magenta!80!black,
  title={{\emoji🔔} تنبيه\ifx\\#1\\\else ~#1\fi :}}
\newtcolorbox{boxWarning}[1][]{colback=red!5!white,colframe=red!75!black,
  title={{\emoji⚡} ملاحظة هامة\ifx\\#1\\\else ~#1\fi :}}
\newtcolorbox{boxSolution}[1][]{colback=green!5!white,colframe=green!60!black,
  title={{\emoji✅} حل\ifx\\#1\\\else ~#1\fi :}}
\newtcolorbox{boxSymbol}[1][]{colback=purple!5!white,colframe=purple!70!black,
  title={{\emoji🔣} رمز\ifx\\#1\\\else ~#1\fi :}}
\newtcolorbox{boxHint}[1][]{colback=teal!5!white,colframe=teal!60!black,
  title={{\emoji💡} تلميح\ifx\\#1\\\else ~#1\fi :}}


\tcbset{simplecode/.style={ colback=gray!5, colframe=black!50, boxrule=0.4pt, arc=2pt, left=4pt,right=4pt,top=4pt,bottom=4pt}}
\newenvironment{boxCode}{\begin{tcolorbox}[simplecode]}{\end{tcolorbox}}

\newcolumntype{C}[1]{>{\centering\arraybackslash}p{#1}}

% redefine spaces after titles
\makeatletter
\renewcommand{\@maketitle}{%
  \begin{center}
    {\huge \bfseries \@title \par}%
    \vskip 0.2em % space between title and author
    {\large \@author \par}%
    % \vskip 0.2em % space between author and date
    % {\normalsize \@date \par}%
  \end{center}
}
\makeatother

\fancyhf{} % clear default
\fancypagestyle{plain}{
  \fancyhf{}
  \fancyhead[L]{مدرسة التسامح الشاملة}
  % \fancyhead[L]{\includegraphics[height=1cm]{../../../images/logoTasamoh.png}}
  \fancyhead[R]{الأستاذ محمود اغبارية}
  \fancyfoot[C]{\thepage}
}

\fancyhead[L]{مدرسة التسامح الشاملة}
\fancyhead[R]{الأستاذ محمود اغبارية}
\fancyfoot[C]{\thepage}
% \date{\today}

\setcounter{tocdepth}{3} % only section subsection and subsubsection in TOC


% ----------------------


% \begin{document}

% \maketitle

% % \clearpage  % start TOC on a new page
% % \renewcommand{\contentsname}{جدول المحتويات}
% % \tableofcontents
% % \clearpage

% \part*{part 1} % the * prevents numbering
% \section*{مقدمة}
% \subsection*{مثال رياضي}
% \subsubsection*{مثال فرعي}
% \paragraph*{ paragraph 1}
% \subparagraph*{sub paragraph 1}

% \ifdetailed
% \begin{english}
% \begin{minted}{csharp}
% // C# Example
% \end{minted}
% \end{english}
% \fi

% OLD WAY
% \ifdetailed
% \begin{english}
% \begin{lstlisting}
% // C# Example
% \end{lstlisting}
% \end{english}
% \fi

% % \includegraphics[width=0.2\textwidth]{../../../images/DFAs/ex1_q1.png}



% \vspace{3cm}
% \begin{flushleft}
% أرجو لكم وقتًا ممتعًا.

% الأستاذ محمود اغبارية.
% \end{flushleft}


% \end{document}


\title{تمارين للاختيار للامتحان الفصلي}

\begin{document}
\maketitle
\thispagestyle{fancy}

\begin{enumerate}[itemsep=2em]
    \item
    اكتب عملية تتلقى عددًا صحيحا وتعيد أكبر فرق بالقيمة المطلقة بين خانتين متجاورتين فيه.

    \item
اكتب عملية خارجية تتلقى درجة حرارة ونوع القياس (1 للسيلسيوس، 2 للفهرنهايت)، وتطبع تصنيف الحرارة:

"Cold" إذا كانت أقل من 15 سيلسيوس (أو 59 فهرنهايت)
"Moderate" إذا كانت بين 15 و 25 سيلسيوس (أو 59 و 77 فهرنهايت)
"Hot" إذا كانت أكبر من 25 سيلسيوس (أو 77 فهرنهايت)


\item
اكتب برنامجًا يستقبل من المستخدم 10 أعداد صحيحة، ويحسب مجموع الأعداد التي تقع بين 10 و 50 (شاملة الحدود) من بين الأعداد التي استقبلها.
مثال: إذا كانت الأعداد 5، 25، 60، 30 → يطبع 55 (لأن 25 + 30 = 55)

\item
اكتب برنامجًا يستقبل من المستخدم 20 رقمًا، ويطبع عدد الأرقام التي كانت زوجيتها مخالفة للعد السابق لها.

\item
اكتب برنامجًا يستقبل عددًا صحيحًا، ويعيد عدد مرات ظهور الخانة 1 فيه.


\item
 Collatz
اختر أي عدد صحيح موجب:

إذا كان زوجيًا: اقسمه على 2
إذا كان فرديًا: اضربه في 3 وأضف 1
كرر حتى تصل إلى 1

مثال: إذا بدأنا من 6:

6 → 3 → 10 → 5 → 16 → 8 → 4 → 2 → 1

اكتب عملية تتلقى عددًا موجبًا وتطبع السلسلة كاملة حتى الوصول إلى 1، ثم تعيد عدد الخطوات.
مثال: إذا تلقت 6، تطبع: 6 3 10 5 16 8 4 2 1 وتعيد 8 (عدد الخطوات).

\item
السؤال 18: مضاعفات رقمية
اكتب برنامجًا يستقبل عددين صحيحين n و k، ويطبع أول k عددًا مجموع أرقامه يساوي n.
مثال: إذا كان n = 5 و k = 10:

5 → مجموع الأرقام = 5 ✓
14 → مجموع الأرقام = 5 ✓
23 → مجموع الأرقام = 5 ✓
32 → مجموع الأرقام = 5 ✓
...

يطبع: 5, 14, 23, 32, 41, 50, 104, 113, 122, 131



\item \textbf{السؤال الرابع}  (20 علامة + 5)  \\
نعرف المتوالية التالية:
$$a_1 = 1 , \qquad a_n = (a_{n-1} \times 2) + 1$$
مثلًا:
\begin{align*}
    a_1 &= 1 \\
    a_2 &= (1 \times 2) + 1 = 3 \\
    a_3 &= (3 \times 2) + 1 = 7 \\
    a_4 &= (7 \times 2) + 1 = 15 \\
    a_5 &= (15 \times 2) + 1 = 31
\end{align*}

اكتب عملية خارجية باسم \textenglish{PrintSeries} تتلقى عددًا $n$ وتطبع أوّل $n$ حدود من السلسلة الموصوفة أعلاه.

\textbf{مثال:} إذا تلقت العملية العدد 6، فإنّها ستطبع:
$$1,\; 3,\; 7,\; 15,\; 31,\; 63$$

\textbf{سؤال بونوس} (5 علامات): بعد طباعة الحدود، اطبع أيضًا معدّلها.

\ifwithsols
\begin{boxSolution}
\begin{english}
\begin{minted}{csharp}
public static void PrintSeries(int n)
{
    int val = 1;
    int sum = 0;
    for (int i = 1; i <= n; i++)
    {
        Console.WriteLine(val);
        sum += val;
        val = 2 * val + 1;
    }

    // Bonus
    double average = (double)sum / n;
    Console.WriteLine("Average = " + average);
}
\end{minted}
\end{english}
\end{boxSolution}
\fi

\clearpage
\item \textbf{السؤال الخامس}  (20 علامة + 5) \\
معطاة العملية \textenglish{AppendDigit} تتلقى عددًا صحيحًا ورقمًا مكوّنًا من منزلة واحدة، وتعيد الرقم الناتج عن إلصاق الخانة بالعدد الصحيح من جهة الآحاد. \\
\textbf{مثلا:} $341 \leftarrow \text{AppendDigit}(34, 1)$ و $465 \leftarrow \text{AppendDigit}(46, 5)$. \\
\textbf{العملية معطاة لكم ويمكنكم استخدامها بدون تطبيقها.}

اكتب عملية خارجية باسم \textenglish{ReverseNumber} تتلقى عددًا صحيحًا وتعيد معكوس العدد. \\
\textbf{مثال:} إذا تلقّت العدد 123 فإنّها تعيد 321. وإذا تلقّت العدد 10 فإنّها تعيد 1.

\textbf{ملاحظة:} \underline{يجب} عليكم استخدام العملية \textenglish{AppendDigit}. \\

\textbf{سؤال بونوس} (5 علامات): في نهاية العملية \textenglish{ReverseNumber}، اطبع كلمة \textenglish{Palindrome} إذا كان الرقم الأصلي يُقرأ من اليمين ومن اليسار بنفس الطريقة. \\
مثلًا: الرقم 1221 هو \textenglish{Palindrome} لأنّك إذا قرأته من اليمين لليسار أو من اليسار لليمين تحصل على نفس الرقم.

\ifwithsols
\begin{boxSolution}
\begin{english}
\begin{minted}{csharp}
public static int AppendDigit(int num, int digit)
{
    return num * 10 + digit;
}

public static int ReverseNumber(int num)
{
    int original = num;
    int reversed = 0;
    while (num > 0)
    {
        int digit = num % 10;
        reversed = AppendDigit(reversed, digit);
        num /= 10;
    }

    // Bonus
    if (original == reversed)
        Console.WriteLine("Palindrome");

    return reversed;
}
\end{minted}
\end{english}
\end{boxSolution}
\fi

\clearpage
\item \textbf{السؤال السادس}  (20 علامة + 5) \\
في دولة معينة، يتم احتساب ضريبة الدخل الشهرية بطريقة تراكمية كالتالي:
\begin{itemize}
    \item \textbf{أوّل 5000 شيكل} من الراتب: معفية من الضريبة (0\%).
    \item \textbf{المبلغ بين 5001 و 10000 شيكل}: ضريبة بنسبة 10\%.
    \item \textbf{أي مبلغ فوق الـ 10000 شيكل}: ضريبة بنسبة 20\%.
\end{itemize}

\textbf{مثال لتوضيح حساب الضريبة:} \\
إذا كان الراتب $12000$ شيكل، فإن الضريبة تحسب كالتالي:
\begin{itemize}
    \item الـ 5000 الأولى $\leftarrow$ ضريبة 0.
    \item الـ 5000 الثانية (من 5001 لـ 10000) $\leftarrow$ $5000 \times 0.10 = 500$.
    \item المبلغ المتبقي ($12000 - 10000 = 2000$) $\leftarrow$ $2000 \times 0.20 = 400$.
    \item إجمالي الضريبة: $0 + 500 + 400 = 900$ شيكل.
\end{itemize}

\begin{enumerate}
\item
اكتب عملية خارجية باسم \textenglish{EmployeeTax} تتلقى قيمة الراتب كعدد صحيح، وتعيد مبلغ الضريبة الذي يجب دفعه لصاحب هذا الراتب حسب الجدول أعلاه.

\item
اكتب عملية خارجية باسم \textenglish{CompanyTax} تقوم بما يلي:
\begin{itemize}
    \item تتلقى عدد الموظفين في شركة ($n$).
    \item تستقبل الراتب الشهري لكل موظف.
    \item تطبع قيمة الضريبة التي يجب أن يدفعها كل موظف.
    \item \underline{تعيد} \textbf{مجموع} الضرائب الكلي الذي دفعه كل الموظفين.
\end{itemize}
\textbf{ملاحظة:} \underline{يجب} عليكم استخدام العملية \textenglish{EmployeeTax} التي كتبتها في البند السابق.
\end{enumerate}

\textbf{سؤال بونوس} (5 علامات): في نهاية العملية \textenglish{CompanyTax}، اطبع \textbf{أعلى} مبلغ ضريبة دفعه موظف واحد في الشركة.

\ifwithsols
\begin{boxSolution}
\begin{english}
\begin{minted}{csharp}
public static double EmployeeTax(int salary)
{
    double tax = 0;
    if (salary <= 5000)
    {
        tax = 0;
    }
    else if (salary <= 10000)
    {
        tax = (salary - 5000) * 0.10;
    }
    else
    {
        tax = 5000 * 0.10 + (salary - 10000) * 0.20;
    }
    return tax;
}

public static double CompanyTax(int n)
{
    double totalTax = 0, maxTax = 0;
    for (int i = 1; i <= n; i++)
    {
        int salary = int.Parse(Console.ReadLine());
        double empTax = EmployeeTax(salary);
        Console.WriteLine(empTax);
        totalTax += empTax;

        if (empTax > maxTax) // Bonus
            maxTax = empTax;
    }

    Console.WriteLine("Maximum tax = " + maxTax); // Bonus

    return totalTax;
}
\end{minted}
\end{english}
\end{boxSolution}
\fi


\item \textbf{السؤال الرابع} (20 علامة) \\
اكتب عملية خارجية باسم \textenglish{BiggestDivisor} تتلقى عددًا صحيحًا موجبًا $num$. \\
على العملية أن تعيد أكبر قاسم للعدد، أي أكبر عدد يقسم عليه العدد $num$ (لا يشمل $num$ نفسه).

\textbf{أمثلة للتوضيح:}
\begin{itemize}
    \item إذا تلقت العملية العدد \textbf{10}: القواسم هي 1، 2، 5. القاسم الأكبر هو \textbf{5}.
    \item إذا تلقت العملية العدد \textbf{24}: القواسم (غير 24) هي 1، 2، 3، 4، 6، 8، 12. القاسم الأكبر هو \textbf{12}.
    \item إذا تلقت العملية العدد \textbf{7}: القواسم (غير 7) هي 1 فقط. القاسم الأكبر هو \textbf{1}.
\end{itemize}

\ifwithsols
\begin{boxSolution}
\begin{english}
\begin{minted}{csharp}
public static int BiggestDivisor(int num)
{
    for (int i = num - 1; i >= 1; i--)
    {
        if (num % i == 0)
            return i;
    }
    return 1;
}
\end{minted}
\end{english}
\end{boxSolution}
\fi


\end{enumerate}

\end{document}
