\documentclass[14pt]{extarticle}
% Full article preamble (duplicated, no common file)
\usepackage{fontspec}
\usepackage[a4paper,top=2.4cm,bottom=2.4cm,left=2.3cm,right=2.3cm]{geometry}
\usepackage{polyglossia}
\usepackage{amsmath}
\usepackage{amssymb}
\usepackage{xcolor}
\usepackage{fancyhdr}
\usepackage{graphicx}
\usepackage{listings}
\usepackage[most]{tcolorbox}
\usepackage{pifont}
\usepackage{enumitem}
\usepackage{titlesec}
\usepackage[bottom]{footmisc}
\usepackage{titling}
\usepackage{minted}
\usepackage{etoolbox}
\usepackage{array}
\usepackage{extsizes}

\newfontfamily\emoji{Segoe UI Emoji}

\pagestyle{fancy}

\setmainlanguage[numerals=western]{arabic}
\setotherlanguage{english}
\newfontfamily\arabicfont[Script=Arabic]{Amiri}
\newfontfamily\arabicfonttt[Script=Arabic]{Courier New}

\lstset{
  language=[Sharp]C,
  numbers=left,
  stepnumber=1,
  numbersep=8pt,
  frame=single,
  basicstyle=\ttfamily\small,
  keywordstyle=\color{blue},
  stringstyle=\color{red},
  commentstyle=\color{green!50!black}
}

\newif\ifdetailed
\ifdefined\setdetailed
  \setdetailed
\fi

\newif\ifwithsols
\ifdefined\setwithsols
  \setwithsols
\fi

% unified tcolorboxes for articles
\tcbset{colback=white, colframe=black, fonttitle=\bfseries, boxrule=0.8pt}
\newtcolorbox{boxDef}[1][]{colback=blue!5!white,colframe=blue!75!black,
  title={{\emoji📘} تعريف\ifx\\#1\\\else ~#1\fi :}}
\newtcolorbox{boxExercise}[1][]{colback=cyan!5!white,colframe=cyan!70!black,
  title={{\emoji🧩} تمرين\ifx\\#1\\\else ~#1\fi :}}
\newtcolorbox{boxExample}[1][]{colback=yellow!5!white,colframe=orange!90!black,
  title={{\emoji📝} مثال\ifx\\#1\\\else ~#1\fi :}}
\newtcolorbox{boxNote}[1][]{colback=gray!10!white,colframe=black,
  title={{\emoji✨} ملاحظة\ifx\\#1\\\else ~#1\fi :}}
\newtcolorbox{boxAttention}[1][]{colback=magenta!10!white,colframe=magenta!80!black,
  title={{\emoji🔔} تنبيه\ifx\\#1\\\else ~#1\fi :}}
\newtcolorbox{boxWarning}[1][]{colback=red!5!white,colframe=red!75!black,
  title={{\emoji⚡} ملاحظة هامة\ifx\\#1\\\else ~#1\fi :}}
\newtcolorbox{boxSolution}[1][]{colback=green!5!white,colframe=green!60!black,
  title={{\emoji✅} حل\ifx\\#1\\\else ~#1\fi :}}
\newtcolorbox{boxSymbol}[1][]{colback=purple!5!white,colframe=purple!70!black,
  title={{\emoji🔣} رمز\ifx\\#1\\\else ~#1\fi :}}
\newtcolorbox{boxHint}[1][]{colback=teal!5!white,colframe=teal!60!black,
  title={{\emoji💡} تلميح\ifx\\#1\\\else ~#1\fi :}}


\tcbset{simplecode/.style={ colback=gray!5, colframe=black!50, boxrule=0.4pt, arc=2pt, left=4pt,right=4pt,top=4pt,bottom=4pt}}
\newenvironment{boxCode}{\begin{tcolorbox}[simplecode]}{\end{tcolorbox}}

\newcolumntype{C}[1]{>{\centering\arraybackslash}p{#1}}

% redefine spaces after titles
\makeatletter
\renewcommand{\@maketitle}{%
  \begin{center}
    {\huge \bfseries \@title \par}%
    \vskip 0.2em % space between title and author
    {\large \@author \par}%
    % \vskip 0.2em % space between author and date
    % {\normalsize \@date \par}%
  \end{center}
}
\makeatother

\fancyhf{} % clear default
\fancypagestyle{plain}{
  \fancyhf{}
  \fancyhead[L]{مدرسة التسامح الشاملة}
  % \fancyhead[L]{\includegraphics[height=1cm]{../../../images/logoTasamoh.png}}
  \fancyhead[R]{الأستاذ محمود اغبارية}
  \fancyfoot[C]{\thepage}
}

\fancyhead[L]{مدرسة التسامح الشاملة}
\fancyhead[R]{الأستاذ محمود اغبارية}
\fancyfoot[C]{\thepage}
% \date{\today}

\setcounter{tocdepth}{3} % only section subsection and subsubsection in TOC


% ----------------------


% \begin{document}

% \maketitle

% % \clearpage  % start TOC on a new page
% % \renewcommand{\contentsname}{جدول المحتويات}
% % \tableofcontents
% % \clearpage

% \part*{part 1} % the * prevents numbering
% \section*{مقدمة}
% \subsection*{مثال رياضي}
% \subsubsection*{مثال فرعي}
% \paragraph*{ paragraph 1}
% \subparagraph*{sub paragraph 1}

% \ifdetailed
% \begin{english}
% \begin{minted}{csharp}
% // C# Example
% \end{minted}
% \end{english}
% \fi

% OLD WAY
% \ifdetailed
% \begin{english}
% \begin{lstlisting}
% // C# Example
% \end{lstlisting}
% \end{english}
% \fi

% % \includegraphics[width=0.2\textwidth]{../../../images/DFAs/ex1_q1.png}



% \vspace{3cm}
% \begin{flushleft}
% أرجو لكم وقتًا ممتعًا.

% الأستاذ محمود اغبارية.
% \end{flushleft}


% \end{document}


\ifwithsols
\title{حل اختبار مرحلي - المصفوفات الأحادية \\ الصف العاشر 10}
\else
\title{اختبار مرحلي - المصفوفات الأحادية \\ الصف العاشر 10}
\fi

\begin{document}

\maketitle
\thispagestyle{fancy}

\begin{enumerate}[itemsep=3em]

    \item
    معطى مقطع الكود الذي أمامك:
    \begin{boxCode}
    \begin{english}
    \begin{minted}{csharp}
int middle = arr.Length / 2;
int count = 0;
for(int i = 0; i < middle; i++)
{
    if (arr[i] == arr[middle + i])
        count++;
}
    \end{minted}
    \end{english}
    \end{boxCode}

    \begin{enumerate}[label=(\alph*)]
        \item اكتب جدول متابعة لمقطع البرنامج أعلاه إذا كانت المصفوفة \\
        \textenglish{\{1, 2, 5, 4, 1, 2, 3, 4\}}. \\
        يجب أن يحتوي الجدول على عمود لكل واحد من المتغيرات: \\
         \textenglish{i}، \textenglish{arr[i]}، \textenglish{arr[middle + i]}، و\textenglish{count} \\
         وقيمة الشرط \textenglish{(arr[i] == arr[middle + i])}.
        \item ماذا ستكون قيمة المتغير \textenglish{count} إذا شغلنا الكود أعلاه على المصفوفة \\
        \textenglish{\{5, 6, 7, 5, 8, 9, 10\}}.
        \item أعط مثالًا لمصفوفة طولها 6 (أي فيها 6 عناصر)، إذا شغلنا الكود أعلاه عليها، ستكون قيمة المتغير \textenglish{count} تساوي 1.
    \end{enumerate}

    \clearpage



    \item
    اكتب عملية خارجية \textenglish{CountMax} تتلقى مصفوفة أعداد صحيحة، وتعيد عدد مرات ظهور القيمة الكبرى في هذه المصفوفة. \\
    \textbf{مثال:} إذا تلقت المصفوفة \textenglish{\{1, 5, 3, 5, 2\}}، تعيد 2 (لأن أكبر قيمة هي 5 وظهرت مرتين).
    \ifwithsols
    \begin{boxSolution}
    \begin{english}
    \begin{minted}{csharp}
public static int CountMax(int[] arr)
{
    int maxVal = arr[0];
    for (int i = 1; i < arr.Length; i++)
    {
        if (arr[i] > maxVal)
            maxVal = arr[i];
    }

    int count = 0;
    for (int i = 0; i < arr.Length; i++)
    {
        if (arr[i] == maxVal)
            count++;
    }
    return count;
}
    \end{minted}
    \end{english}
    \end{boxSolution}
    \fi

    \item
    اكتب عملية خارجية \textenglish{ExpandWithSquares} تتلقى مصفوفة أعداد صحيحة وتعيد مصفوفة جديدة بحجم ضعف المصفوفة الأصلية، بحيث تضع بعد كل عنصر تربيعه.\\
    \textbf{مثال:} إذا تلقت المصفوفة \textenglish{\{1, 3, 4\}}، تعيد \textenglish{\{1, 1, 3, 9, 4, 16\}}.
    \ifwithsols
    \begin{boxSolution}
    \begin{english}
    \begin{minted}{csharp}
public static int[] ExpandWithSquares(int[] arr)
{
    int[] result = new int[arr.Length * 2];
    int index = 0;

    for (int i = 0; i < arr.Length; i++)
    {
        result[index] = arr[i];
        index++;
        result[index] = arr[i] * arr[i];
        index++;
    }

    return result;
}
    \end{minted}
    \end{english}
    \end{boxSolution}
    \fi

    \item
    لديك قائمتان بأسعار نفس المنتجات في متجرين مختلفين. لكل منتج من المنتجات نريد إيجاد المتجر الذي يبيعه بسعر أرخص.\\
    اكتب عملية خارجية \textenglish{FindCheaperStore} تتلقى مصفوفتين من الأسعار (الأعداد الصحيحة): \textenglish{store1} و \textenglish{store2}،
    بحيث أنّ \textenglish{store1[i]} هو سعر المنتج \textenglish{i} في المتجر الأول، و\textenglish{store2[i]} هو سعر نفس المنتج في المتجر الثاني.
    العملية تتلقى المصفوفتين وتعيد مصفوفة جديدة بحيث لكل منتج تحتوي على:
\begin{itemize}
    \item 1 إذا كان المتجر الأول أرخص
    \item 2 إذا كان المتجر الثاني أرخص
    \item 0 إذا كان السعر نفسه في المتجرين
\end{itemize}
    \textbf{ملاحظة:} افترض أن المصفوفتين بنفس الطول، وكل موقع فيهما يمثل نفس المنتج.\\
    \textbf{مثال:} إذا تلقت:
    \begin{itemize}
        \item \textenglish{store1 = \{10, 25, 30, 10, 45\}}
        \item \textenglish{store2 = \{15, 30, 20, 10, 42\}}
        \item النتيجة: \textenglish{\{1, 1, 2, 0, 2\}}
    \end{itemize}
    \ifwithsols
    \begin{boxSolution}
    \begin{english}
    \begin{minted}{csharp}
public static int[] FindCheaperStore(int[] store1, int[] store2)
{
    int[] result = new int[store1.Length];
    for (int i = 0; i < store1.Length; i++)
    {
        if (store1[i] < store2[i])
            result[i] = 1;
        else if (store1[i] > store2[i])
            result[i] = 2;
        else
            result[i] = 0;
    }
    return result;
}
    \end{minted}
    \end{english}
    \end{boxSolution}
    \fi

\end{enumerate}

\vspace{2cm}
\begin{flushleft}
    أرجو لكم التفوّق.

    الأستاذ محمود اغبارية.
\end{flushleft}

\end{document}
