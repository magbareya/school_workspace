\documentclass[14pt]{extarticle}
% Full article preamble (duplicated, no common file)
\usepackage{fontspec}
\usepackage[a4paper,top=2.4cm,bottom=2.4cm,left=2.3cm,right=2.3cm]{geometry}
\usepackage{polyglossia}
\usepackage{amsmath}
\usepackage{amssymb}
\usepackage{xcolor}
\usepackage{fancyhdr}
\usepackage{graphicx}
\usepackage{listings}
\usepackage[most]{tcolorbox}
\usepackage{pifont}
\usepackage{enumitem}
\usepackage{titlesec}
\usepackage[bottom]{footmisc}
\usepackage{titling}
\usepackage{minted}
\usepackage{etoolbox}
\usepackage{array}
\usepackage{extsizes}

\newfontfamily\emoji{Segoe UI Emoji}

\pagestyle{fancy}

\setmainlanguage[numerals=western]{arabic}
\setotherlanguage{english}
\newfontfamily\arabicfont[Script=Arabic]{Amiri}
\newfontfamily\arabicfonttt[Script=Arabic]{Courier New}

\lstset{
  language=[Sharp]C,
  numbers=left,
  stepnumber=1,
  numbersep=8pt,
  frame=single,
  basicstyle=\ttfamily\small,
  keywordstyle=\color{blue},
  stringstyle=\color{red},
  commentstyle=\color{green!50!black}
}

\newif\ifdetailed
\ifdefined\setdetailed
  \setdetailed
\fi

\newif\ifwithsols
\ifdefined\setwithsols
  \setwithsols
\fi

% unified tcolorboxes for articles
\tcbset{colback=white, colframe=black, fonttitle=\bfseries, boxrule=0.8pt}
\newtcolorbox{boxDef}[1][]{colback=blue!5!white,colframe=blue!75!black,
  title={{\emoji📘} تعريف\ifx\\#1\\\else ~#1\fi :}}
\newtcolorbox{boxExercise}[1][]{colback=cyan!5!white,colframe=cyan!70!black,
  title={{\emoji🧩} تمرين\ifx\\#1\\\else ~#1\fi :}}
\newtcolorbox{boxExample}[1][]{colback=yellow!5!white,colframe=orange!90!black,
  title={{\emoji📝} مثال\ifx\\#1\\\else ~#1\fi :}}
\newtcolorbox{boxNote}[1][]{colback=gray!10!white,colframe=black,
  title={{\emoji✨} ملاحظة\ifx\\#1\\\else ~#1\fi :}}
\newtcolorbox{boxAttention}[1][]{colback=magenta!10!white,colframe=magenta!80!black,
  title={{\emoji🔔} تنبيه\ifx\\#1\\\else ~#1\fi :}}
\newtcolorbox{boxWarning}[1][]{colback=red!5!white,colframe=red!75!black,
  title={{\emoji⚡} ملاحظة هامة\ifx\\#1\\\else ~#1\fi :}}
\newtcolorbox{boxSolution}[1][]{colback=green!5!white,colframe=green!60!black,
  title={{\emoji✅} حل\ifx\\#1\\\else ~#1\fi :}}
\newtcolorbox{boxSymbol}[1][]{colback=purple!5!white,colframe=purple!70!black,
  title={{\emoji🔣} رمز\ifx\\#1\\\else ~#1\fi :}}
\newtcolorbox{boxHint}[1][]{colback=teal!5!white,colframe=teal!60!black,
  title={{\emoji💡} تلميح\ifx\\#1\\\else ~#1\fi :}}


\tcbset{simplecode/.style={ colback=gray!5, colframe=black!50, boxrule=0.4pt, arc=2pt, left=4pt,right=4pt,top=4pt,bottom=4pt}}
\newenvironment{boxCode}{\begin{tcolorbox}[simplecode]}{\end{tcolorbox}}

\newcolumntype{C}[1]{>{\centering\arraybackslash}p{#1}}

% redefine spaces after titles
\makeatletter
\renewcommand{\@maketitle}{%
  \begin{center}
    {\huge \bfseries \@title \par}%
    \vskip 0.2em % space between title and author
    {\large \@author \par}%
    % \vskip 0.2em % space between author and date
    % {\normalsize \@date \par}%
  \end{center}
}
\makeatother

\fancyhf{} % clear default
\fancypagestyle{plain}{
  \fancyhf{}
  \fancyhead[L]{مدرسة التسامح الشاملة}
  % \fancyhead[L]{\includegraphics[height=1cm]{../../../images/logoTasamoh.png}}
  \fancyhead[R]{الأستاذ محمود اغبارية}
  \fancyfoot[C]{\thepage}
}

\fancyhead[L]{مدرسة التسامح الشاملة}
\fancyhead[R]{الأستاذ محمود اغبارية}
\fancyfoot[C]{\thepage}
% \date{\today}

\setcounter{tocdepth}{3} % only section subsection and subsubsection in TOC


% ----------------------


% \begin{document}

% \maketitle

% % \clearpage  % start TOC on a new page
% % \renewcommand{\contentsname}{جدول المحتويات}
% % \tableofcontents
% % \clearpage

% \part*{part 1} % the * prevents numbering
% \section*{مقدمة}
% \subsection*{مثال رياضي}
% \subsubsection*{مثال فرعي}
% \paragraph*{ paragraph 1}
% \subparagraph*{sub paragraph 1}

% \ifdetailed
% \begin{english}
% \begin{minted}{csharp}
% // C# Example
% \end{minted}
% \end{english}
% \fi

% OLD WAY
% \ifdetailed
% \begin{english}
% \begin{lstlisting}
% // C# Example
% \end{lstlisting}
% \end{english}
% \fi

% % \includegraphics[width=0.2\textwidth]{../../../images/DFAs/ex1_q1.png}



% \vspace{3cm}
% \begin{flushleft}
% أرجو لكم وقتًا ممتعًا.

% الأستاذ محمود اغبارية.
% \end{flushleft}


% \end{document}


\title{اختبار فصلي للصف 10-10}

\begin{document}
\baselineskip=1.3\baselineskip
\maketitle

\ifwithsols
\else
\begin{boxCode}
    \begin{itemize}
        \item \textbf{الاختبار مكوّن من فصلين:}
        \begin{itemize}
            \item \textbf{الفصل الأول: أجب عن كل الأسئلة}
            \item \textbf{الفصل الثاني: أجب عن سؤالين}
        \end{itemize}
        \item الوقت المخصص: ساعتان.
        \item يسمح باستخدام كل مادة مساعدة، عدا الآلة التي يمكن برمجتها.
        \item أجب بلغة \textenglish{C\#} فقط.
        \item اكتب بقلم حبر فقط، وبخط واضح.
        \item الحل على ورقة خارجية، لا تنسَ كتابة اسمك
        \item مرفقة ورقة للتسليم فيها جدول متابعة للسؤال الأول، يمكنك ملؤها وتسليمها.
    \end{itemize}
\end{boxCode}
\vspace{2cm}
\begin{flushleft}
    أرجو لكم التفوّق.

    الأستاذ محمود اغبارية.
\end{flushleft}
\clearpage
\fi

\section{الفصل الأوّل}
\textbf{أجب عن \underline{جميع} الأسئلة 1 - 3}.

\begin{enumerate}[itemsep=3em]
\item \textbf{السؤال الأوّل} (25 علامة) \\
أمامك البرنامج التالي:
\begin{boxCode}
\begin{english}
\begin{minted}{csharp}
static void Main()
{
    int count = 0;
    int n = int.Parse(Console.ReadLine());
    for (int i = 1; i <= n; i++)
    {
        int x = int.Parse(Console.ReadLine());
        if (x < 50)
        {
            if (x % 2 == 0)
                Console.WriteLine("hello");
            else
                count++;
        }
    }
    Console.WriteLine(count);
}
\end{minted}
\end{english}
\end{boxCode}

\begin{enumerate}
\item اكتب جدول متابعة للبرنامج عندما تكون المدخلات كالتالي: \\
\textenglish{n = 6} \\
الأعداد: \textenglish{30, 45, 60, 15, 22, 50} \\
على الجدول أن يشمل عمودًا لكل واحد من المتغيرات: \textenglish{count, n, i, x} والشرطين \textenglish{if(x<50)} و \textenglish{if(x\%2==0)} وعمودا للطباعة.

\item أعط مثالاً لمدخلات (\textenglish{n=4}) ينتج عنها طباعة \textenglish{"hello"} مرتين و \textenglish{count = 2}.

\item أعط مثالاً لمدخلات (\textenglish{n=4}) ينتج عنها طباعة \textenglish{"hello"} مرة واحدة و \textenglish{count = 1}.

\end{enumerate}

\ifwithsols
\begin{boxSolution}
\textbf{أ.} جدول المتابعة:

\begin{center}
\renewcommand{\arraystretch}{1.5}
\begin{tabular}{|c|c|c|c|c|c|c|}
\hline
\textbf{n} & \textbf{i} & \textbf{x} & \textbf{x < 50} & \textbf{x \% 2 == 0} & \textbf{count} & \textbf{الطباعة} \\
\hline
6 &  &  &  &  & 0 &  \\
\hline
 & 1 & 30 & \textenglish{true} & \textenglish{true} & 0 & \textenglish{hello} \\
\hline
 & 2 & 45 & \textenglish{true} & \textenglish{false} & 1 &  \\
\hline
 & 3 & 60 & \textenglish{false} & — & 1 &  \\
\hline
 & 4 & 15 & \textenglish{true} & \textenglish{false} & 2 &  \\
\hline
 & 5 & 22 & \textenglish{true} & \textenglish{true} & 2 & \textenglish{hello} \\
\hline
 & 6 & 50 & \textenglish{false} & — & 2 &  \\
\hline
 &  &  &  &  & 2 & \textenglish{2} \\
\hline
\end{tabular}
\end{center}

\vspace{0.5cm}
\textbf{ب.} مثال: \textenglish{n = 4}، المدخلات: \textenglish{10, 20, 5, 15} \\
- العدد 10: أقل من 50 وزوجي $\leftarrow$ يطبع \textenglish{"hello"} \\
- العدد 20: أقل من 50 وزوجي $\leftarrow$ يطبع \textenglish{"hello"} \\
- العدد 5: أقل من 50 وفردي $\leftarrow$ \textenglish{count++} \\
- العدد 15: أقل من 50 وفردي $\leftarrow$ \textenglish{count++} \\
النتيجة: طباعة \textenglish{"hello"} مرتين و \textenglish{count = 2}

\vspace{0.5cm}
\textbf{ج.} مثال: \textenglish{n = 4}، المدخلات: \textenglish{8, 60, 70, 13} \\
- العدد 8: أقل من 50 وزوجي $\leftarrow$ يطبع \textenglish{"hello"} \\
- العدد 60: أكبر من أو يساوي 50 $\leftarrow$ لا يدخل الشرط \\
- العدد 70: أكبر من أو يساوي 50 $\leftarrow$ لا يدخل الشرط \\
- العدد 13: أقل من 50 وفردي $\leftarrow$ \textenglish{count++} \\
النتيجة: طباعة \textenglish{"hello"} مرة واحدة و \textenglish{count = 1}
\end{boxSolution}
\fi

\clearpage
\item \textbf{السؤال الثاني}  (20 علامة) \\
اكتب مقطع برنامج يستقبل من المستخدم 10 أعداد صحيحة، ويطبع العدد الذي له أكبر تربيع من بين الأعداد التي أدخلها المستخدم. \\
\textbf{مثال:} إذا كانت سلسلة الاستقبال: $1, 5, -10, 8, 0, -3, -40 , 22, 30, -12$ على البرنامج أن يطبع $-40$، لأنّ $(-40)^2$ أكبر من تربيع أي رقم آخر أدخله المستخدم.

\ifwithsols
\begin{boxSolution}[1]
\begin{english}
\begin{minted}{csharp}
int max = int.Parse(Console.ReadLine());
for (int i = 2; i <= 10; i++)
{
    int num = int.Parse(Console.ReadLine());
    if (Math.Pow(num, 2) > Math.Pow(max, 2))
        max = num;
}
Console.WriteLine(max);
\end{minted}
\end{english}
\end{boxSolution}
\begin{boxSolution}[2]
\begin{english}
\begin{minted}{csharp}
int max = int.Parse(Console.ReadLine());
for (int i = 2; i <= 10; i++)
{
    int num = int.Parse(Console.ReadLine());
    if (Math.Abs(num) > Math.Abs(maxNum))
        max = num;
}
Console.WriteLine(max);
\end{minted}
\end{english}
\end{boxSolution}
\clearpage
\fi

\item \textbf{السؤال الثالث} (20 علامة)  \\
اكتب عملية خارجية باسم \textenglish{SumBigDigits} تتلقى عددًا صحيحًا وتعيد مجموع الخانات التي قيمتها أكبر من 5 فيه. \\
\textbf{مثال:}  إذا استقبلت العدد $837815$ فإنّها تعيد: $8+7+8=23$ \\
\textbf{مثال آخر:}  إذا استقبلت العدد $12345$ فإنّها تعيد: $0$ لأنّه لا يوجد خانات أكبر من 5.

\ifwithsols
\begin{boxSolution}
\begin{english}
\begin{minted}{csharp}
public static int SumBigDigits(int num)
{
    int sum = 0;
    while (num > 0)
    {
        int digit = num % 10;
        if (digit > 5)
            sum += digit;
        num /= 10;
    }
    return sum;
}
\end{minted}
\end{english}
\end{boxSolution}
\fi

\end{enumerate}

\clearpage
\section{الفصل الثاني}
\textbf{أجب عن \underline{سؤالين} من الأسئلة 4 - 6}.
\begin{enumerate}[itemsep=3em, start=4]

\item \textbf{السؤال الرابع}  (20 علامة + 5)  \\
نعرف المتوالية التالية:
$$a_1 = 1 , \qquad a_n = a_{n-1} + (n-1)$$
مثلًا:
\begin{align*}
    a_2 &= a_1 + 1 = 1 + 1 = 2 \\
    a_3 &= a_2 + 2 = 2 + 2 = 4 \\
    a_4 &= a_3 + 3 = 4 + 3 = 7 \\
    a_5 &= a_4 + 4 = 7 + 4 = 11
\end{align*}

اكتب عملية خارجية باسم \textenglish{PrintSeries} تتلقى عددًا $n$ وتطبع أوّل $n$ حدود من السلسلة الموصوفة أعلاه.

\textbf{مثال:}إذا تلقت العملية العدد 7، فإنّها ستطبع أول 7 حدود، أي:
$$1,\; 2,\; 4,\; 7,\; 11,\; 16,\; 22$$

\textbf{سؤال بونوس} (5 علامات): بعد طباعة الحدود، اطبع أيضًا معدّلها.

\ifwithsols
\begin{boxSolution}
\begin{english}
\begin{minted}{csharp}
public static void PrintSeries(int n)
{
    int current = 1;
    int sum = 0;

    for (int i = 1; i <= n; i++)
    {
        Console.WriteLine(current);
        sum += current;
        current = current + i;
    }

    // Bonus
    double average = (double)sum / n;
    Console.WriteLine("Average = " + average);
}
\end{minted}
\end{english}
\end{boxSolution}
\fi

\clearpage
\item \textbf{السؤال الخامس}  (20 علامة + 5) \\
معطاة العملية \textenglish{AppendDigit} تتلقى عددًا صحيحًا ورقمًا مكوّنًا من منزلة واحدة، وتعيد الرقم الناتج عن إلصاق الخانة بالعدد الصحيح من جهة الآحاد. \\
\textbf{مثلا:} $341 \leftarrow \text{AppendDigit}(34, 1)$ و $465 \leftarrow \text{AppendDigit}(46, 5)$. \\
\textbf{العملية معطاة لكم ويمكنكم استخدامها بدون تطبيقها.}

اكتب عملية خارجية باسم \textenglish{ReverseNumber} تتلقى عددًا صحيحًا وتعيد معكوس العدد. \\
\textbf{مثال:} إذا تلقّت العدد 123 فإنّها تعيد 321. وإذا تلقّت العدد 10 فإنّها تعيد 1.

\textbf{ملاحظة:} \underline{يجب} عليكم استخدام العملية \textenglish{AppendDigit}. \\

\textbf{سؤال بونوس} (5 علامات): في نهاية العملية \textenglish{ReverseNumber}، اطبع كلمة \textenglish{Palindrome} إذا كان الرقم الأصلي يُقرأ من اليمين ومن اليسار بنفس الطريقة. \\
مثلًا: الرقم 1221 هو \textenglish{Palindrome} لأنّك إذا قرأته من اليمين لليسار أو من اليسار لليمين تحصل على نفس الرقم.

\ifwithsols
\begin{boxSolution}
\begin{english}
\begin{minted}{csharp}
public static int AppendDigit(int num, int digit)
{
    return num * 10 + digit;
}

public static int ReverseNumber(int num)
{
    int original = num;
    int reversed = 0;
    while (num > 0)
    {
        int digit = num % 10;
        reversed = AppendDigit(reversed, digit);
        num /= 10;
    }

    // Bonus
    if (original == reversed)
        Console.WriteLine("Palindrome");

    return reversed;
}
\end{minted}
\end{english}
\end{boxSolution}
\fi

\clearpage
\item \textbf{السؤال السادس}  (20 علامة + 5) \\
في دولة معينة، يتم احتساب ضريبة الدخل الشهرية بطريقة تراكمية كالتالي:
\begin{itemize}
    \item \textbf{أوّل 5000 شيكل} من الراتب: معفية من الضريبة (0\%).
    \item \textbf{المبلغ بين 5001 و 10000 شيكل}: ضريبة بنسبة 10\%.
    \item \textbf{أي مبلغ فوق الـ 10000 شيكل}: ضريبة بنسبة 20\%.
\end{itemize}

\textbf{مثال لتوضيح حساب الضريبة:} \\
إذا كان الراتب $12000$ شيكل، فإن الضريبة تحسب كالتالي:
\begin{itemize}
    \item الـ 5000 الأولى $\leftarrow$ ضريبة 0.
    \item الـ 5000 الثانية (من 5001 لـ 10000) $\leftarrow$ $5000 \times 0.10 = 500$.
    \item المبلغ المتبقي ($12000 - 10000 = 2000$) $\leftarrow$ $2000 \times 0.20 = 400$.
    \item إجمالي الضريبة: $0 + 500 + 400 = 900$ شيكل.
\end{itemize}

\begin{enumerate}
\item
اكتب عملية خارجية باسم \textenglish{EmployeeTax} تتلقى قيمة الراتب كعدد صحيح، وتعيد مبلغ الضريبة الذي يجب دفعه لصاحب هذا الراتب حسب الجدول أعلاه.

\item
اكتب عملية خارجية باسم \textenglish{CompanyTax} تقوم بما يلي:
\begin{itemize}
    \item تتلقى عدد الموظفين في شركة ($n$).
    \item تستقبل الراتب الشهري لكل موظف.
    \item تطبع قيمة الضريبة التي يجب أن يدفعها كل موظف.
    \item \underline{تعيد} \textbf{مجموع} الضرائب الكلي الذي دفعه كل الموظفين.
\end{itemize}
\textbf{ملاحظة:} \underline{يجب} عليكم استخدام العملية \textenglish{EmployeeTax} التي كتبتها في البند السابق.
\end{enumerate}

\textbf{سؤال بونوس} (5 علامات): في نهاية العملية \textenglish{CompanyTax}، اطبع \textbf{أعلى} مبلغ ضريبة دفعه موظف واحد في الشركة.

\ifwithsols
\begin{boxSolution}
\begin{english}
\begin{minted}{csharp}
public static double EmployeeTax(int salary)
{
    double tax = 0;
    if (salary <= 5000)
    {
        tax = 0;
    }
    else if (salary <= 10000)
    {
        tax = (salary - 5000) * 0.10;
    }
    else
    {
        tax = 5000 * 0.10 + (salary - 10000) * 0.20;
    }
    return tax;
}

public static double CompanyTax(int n)
{
    double totalTax = 0, maxTax = 0;
    for (int i = 1; i <= n; i++)
    {
        int salary = int.Parse(Console.ReadLine());
        double empTax = EmployeeTax(salary);
        Console.WriteLine(empTax);
        totalTax += empTax;

        if (empTax > maxTax) // Bonus
            maxTax = empTax;
    }

    Console.WriteLine("Maximum tax = " + maxTax); // Bonus

    return totalTax;
}
\end{minted}
\end{english}
\end{boxSolution}
\fi

\end{enumerate}

\end{document}