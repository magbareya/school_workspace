\documentclass[14pt]{extarticle}
% Full article preamble (duplicated, no common file)
\usepackage{fontspec}
\usepackage[a4paper,top=2.4cm,bottom=2.4cm,left=2.3cm,right=2.3cm]{geometry}
\usepackage{polyglossia}
\usepackage{amsmath}
\usepackage{amssymb}
\usepackage{xcolor}
\usepackage{fancyhdr}
\usepackage{graphicx}
\usepackage{listings}
\usepackage[most]{tcolorbox}
\usepackage{pifont}
\usepackage{enumitem}
\usepackage{titlesec}
\usepackage[bottom]{footmisc}
\usepackage{titling}
\usepackage{minted}
\usepackage{etoolbox}
\usepackage{array}
\usepackage{extsizes}

\newfontfamily\emoji{Segoe UI Emoji}

\pagestyle{fancy}

\setmainlanguage[numerals=western]{arabic}
\setotherlanguage{english}
\newfontfamily\arabicfont[Script=Arabic]{Amiri}
\newfontfamily\arabicfonttt[Script=Arabic]{Courier New}

\lstset{
  language=[Sharp]C,
  numbers=left,
  stepnumber=1,
  numbersep=8pt,
  frame=single,
  basicstyle=\ttfamily\small,
  keywordstyle=\color{blue},
  stringstyle=\color{red},
  commentstyle=\color{green!50!black}
}

\newif\ifdetailed
\ifdefined\setdetailed
  \setdetailed
\fi

\newif\ifwithsols
\ifdefined\setwithsols
  \setwithsols
\fi

% unified tcolorboxes for articles
\tcbset{colback=white, colframe=black, fonttitle=\bfseries, boxrule=0.8pt}
\newtcolorbox{boxDef}[1][]{colback=blue!5!white,colframe=blue!75!black,
  title={{\emoji📘} تعريف\ifx\\#1\\\else ~#1\fi :}}
\newtcolorbox{boxExercise}[1][]{colback=cyan!5!white,colframe=cyan!70!black,
  title={{\emoji🧩} تمرين\ifx\\#1\\\else ~#1\fi :}}
\newtcolorbox{boxExample}[1][]{colback=yellow!5!white,colframe=orange!90!black,
  title={{\emoji📝} مثال\ifx\\#1\\\else ~#1\fi :}}
\newtcolorbox{boxNote}[1][]{colback=gray!10!white,colframe=black,
  title={{\emoji✨} ملاحظة\ifx\\#1\\\else ~#1\fi :}}
\newtcolorbox{boxAttention}[1][]{colback=magenta!10!white,colframe=magenta!80!black,
  title={{\emoji🔔} تنبيه\ifx\\#1\\\else ~#1\fi :}}
\newtcolorbox{boxWarning}[1][]{colback=red!5!white,colframe=red!75!black,
  title={{\emoji⚡} ملاحظة هامة\ifx\\#1\\\else ~#1\fi :}}
\newtcolorbox{boxSolution}[1][]{colback=green!5!white,colframe=green!60!black,
  title={{\emoji✅} حل\ifx\\#1\\\else ~#1\fi :}}
\newtcolorbox{boxSymbol}[1][]{colback=purple!5!white,colframe=purple!70!black,
  title={{\emoji🔣} رمز\ifx\\#1\\\else ~#1\fi :}}
\newtcolorbox{boxHint}[1][]{colback=teal!5!white,colframe=teal!60!black,
  title={{\emoji💡} تلميح\ifx\\#1\\\else ~#1\fi :}}


\tcbset{simplecode/.style={ colback=gray!5, colframe=black!50, boxrule=0.4pt, arc=2pt, left=4pt,right=4pt,top=4pt,bottom=4pt}}
\newenvironment{boxCode}{\begin{tcolorbox}[simplecode]}{\end{tcolorbox}}

\newcolumntype{C}[1]{>{\centering\arraybackslash}p{#1}}

% redefine spaces after titles
\makeatletter
\renewcommand{\@maketitle}{%
  \begin{center}
    {\huge \bfseries \@title \par}%
    \vskip 0.2em % space between title and author
    {\large \@author \par}%
    % \vskip 0.2em % space between author and date
    % {\normalsize \@date \par}%
  \end{center}
}
\makeatother

\fancyhf{} % clear default
\fancypagestyle{plain}{
  \fancyhf{}
  \fancyhead[L]{مدرسة التسامح الشاملة}
  % \fancyhead[L]{\includegraphics[height=1cm]{../../../images/logoTasamoh.png}}
  \fancyhead[R]{الأستاذ محمود اغبارية}
  \fancyfoot[C]{\thepage}
}

\fancyhead[L]{مدرسة التسامح الشاملة}
\fancyhead[R]{الأستاذ محمود اغبارية}
\fancyfoot[C]{\thepage}
% \date{\today}

\setcounter{tocdepth}{3} % only section subsection and subsubsection in TOC


% ----------------------


% \begin{document}

% \maketitle

% % \clearpage  % start TOC on a new page
% % \renewcommand{\contentsname}{جدول المحتويات}
% % \tableofcontents
% % \clearpage

% \part*{part 1} % the * prevents numbering
% \section*{مقدمة}
% \subsection*{مثال رياضي}
% \subsubsection*{مثال فرعي}
% \paragraph*{ paragraph 1}
% \subparagraph*{sub paragraph 1}

% \ifdetailed
% \begin{english}
% \begin{minted}{csharp}
% // C# Example
% \end{minted}
% \end{english}
% \fi

% OLD WAY
% \ifdetailed
% \begin{english}
% \begin{lstlisting}
% // C# Example
% \end{lstlisting}
% \end{english}
% \fi

% % \includegraphics[width=0.2\textwidth]{../../../images/DFAs/ex1_q1.png}



% \vspace{3cm}
% \begin{flushleft}
% أرجو لكم وقتًا ممتعًا.

% الأستاذ محمود اغبارية.
% \end{flushleft}


% \end{document}


\title{اختبار فصلي للصف 10-10}

\begin{document}
\maketitle

\ifwithsols
\else
\begin{boxCode}
    \begin{itemize}
        \item \textbf{الاختبار مكوّن من ثلاثة فصول:}
        \begin{itemize}
            \item \textbf{الفصل الأول:} السؤال الأول وهو \underline{إجباري}.
            \item \textbf{الفصل الثاني:} الأسئلة 2-3، أجب عن \underline{كلا السؤالين}.
            \item \textbf{الفصل الثالث:} الأسئلة 4-5، أجب عن \underline{سؤال واحد} فقط.
        \end{itemize}
        \item الوقت المخصص: ساعتان.
        \item يسمح باستخدام كل مادة مساعدة، عدا الآلة التي يمكن برمجتها.
        \item أجب بلغة \textenglish{C\#} فقط.
        \item اكتب بقلم حبر فقط، وبخط واضح.
        \item الحل على ورقة خارجية، لا تنسَ كتابة اسمك.
        \item مرفق ورقة خاصة لتسليم حل السؤال الأول (جدول المتابعة)
    \end{itemize}
\end{boxCode}

\begin{center}
  \includegraphics[width=0.4\textwidth]{../../../images/logoTasamoh.png}
\end{center}

\vspace{2cm}
\begin{flushleft}
    أرجو لكم التفوّق.

    الأستاذ محمود اغبارية.
\end{flushleft}
\clearpage
\fi

% ================= الفصل الأول =================
\section{الفصل الأوّل}
\textbf{أجب عن السؤال الأول (إجباري)}.

\begin{enumerate}[itemsep=3em]
\item \textbf{السؤال الأوّل} (30 علامة) \\
أمامك البرنامج التالي:
\begin{boxCode}
\begin{english}
\begin{minted}{csharp}
static void Main()
{
    int count = 0;
    int n = int.Parse(Console.ReadLine());
    for (int i = 1; i <= n; i++)
    {
        int x = int.Parse(Console.ReadLine());
        if (x < 50)
        {
            if (x % 2 == 0)
                Console.WriteLine("hello");
            else
                count++;
        }
    }
    Console.WriteLine(count);
}
\end{minted}
\end{english}
\end{boxCode}

\begin{enumerate}
\item اكتب جدول متابعة للبرنامج عندما تكون المدخلات كالتالي: \\
\textenglish{n = 6} \\
الأعداد: \textenglish{30, 45, 60, 15, 22, 50} \\
على الجدول أن يشمل عمودًا لكل واحد من المتغيرات: \textenglish{count, i, x} والشرطين \textenglish{if(x<50)} و \textenglish{if(x\%2==0)} وعمودا للطباعة.

\item أعط مثالاً لمدخلات (\textenglish{n=4}) ينتج عنها طباعة \textenglish{"hello"} مرتين و \textenglish{count = 1}.

\end{enumerate}

\ifwithsols
\begin{boxSolution}
\textbf{أ.} جدول المتابعة:
\begin{center}
\renewcommand{\arraystretch}{1.5}
\begin{tabular}{|c|c|c|c|c|c|c|}
\hline
\textbf{n} & \textbf{i} & \textbf{x} & \textbf{x < 50} & \textbf{x \% 2 == 0} & \textbf{count} & \textbf{الطباعة} \\
\hline
6 &  &  &  &  & 0 &  \\
\hline
 & 1 & 30 & T & T & 0 & \textenglish{hello} \\
\hline
 & 2 & 45 & T & F & 1 &  \\
\hline
 & 3 & 60 & F & — & 1 &  \\
\hline
 & 4 & 15 & T & F & 2 &  \\
\hline
 & 5 & 22 & T & T & 2 & \textenglish{hello} \\
\hline
 & 6 & 50 & F & — & 2 &  \\
\hline
 &  &  &  &  & 2 & \textenglish{2} \\
\hline
\end{tabular}
\end{center}
\end{boxSolution}
\fi
\end{enumerate}

\clearpage

% ================= الفصل الثاني =================
\section{الفصل الثاني}
\textbf{أجب عن \underline{كلا السؤالين} 2 - 3}.

\begin{enumerate}[itemsep=3em, start=2]

\item \textbf{السؤال الثاني}  (20 علامة) \\
اكتب مقطع برنامج يستقبل من المستخدم 10 أعداد صحيحة، ويطبع العدد الذي له أكبر تربيع من بين الأعداد التي أدخلها المستخدم. \\
إذا كان هناك أكثر من عدد لها نفس التربيع وهو أكبر تربيع، اطبع أحدها. \\
\textbf{مثال:} إذا كانت سلسلة الاستقبال: $1, 5, -10, 8, 0, -3, -40 , 22, 30, -12$ على البرنامج أن يطبع $-40$، لأنّ $(-40)^2$ أكبر من تربيع أي رقم آخر أدخله المستخدم.

\ifwithsols
\begin{boxSolution}
\begin{english}
\begin{minted}{csharp}
int max = int.Parse(Console.ReadLine());
for (int i = 2; i <= 10; i++)
{
    int num = int.Parse(Console.ReadLine());
    if (Math.Pow(num, 2) > Math.Pow(max, 2))
        max = num;
}
Console.WriteLine(max);
\end{minted}
\end{english}
\end{boxSolution}
\clearpage
\fi

\item \textbf{السؤال الثالث} (20 علامة)  \\
اكتب عملية خارجية باسم \textenglish{SumBigDigits} تتلقى عددًا صحيحًا موجبًا وتعيد مجموع الخانات التي قيمتها أكبر من 5 فيه. \\
\textbf{مثال:}  إذا استقبلت العدد $837815$ فإنّها تعيد: $8+7+8=23$ \\
\textbf{مثال آخر:}  إذا استقبلت العدد $12345$ فإنّها تعيد: $0$ لأنّه لا يوجد خانات أكبر من 5.

\ifwithsols
\begin{boxSolution}
\begin{english}
\begin{minted}{csharp}
public static int SumBigDigits(int num)
{
    int sum = 0;
    while (num > 0)
    {
        int digit = num % 10;
        if (digit > 5)
            sum += digit;
        num /= 10;
    }
    return sum;
}
\end{minted}
\end{english}
\end{boxSolution}
\clearpage
\fi

\end{enumerate}

\clearpage

% ================= الفصل الثالث =================
\section{الفصل الثالث}
\textbf{أجب عن \underline{أحد السؤالين} 4 - 5}.

\begin{enumerate}[itemsep=3em, start=4]

\item \textbf{السؤال الرابع} (30 علامة) \\
يوجد في مصنع معيّن موقف سيّارات لعاملي المصنع. \\
تستطيع أن تدخل إلى موقف السيّارات كلّ سيّارة وزنها أقلّ من 10 أطنان، وارتفاعها أقلّ من 5 أمتار. \\
العاملون الذين يستعملون موقف السيّارات يدخلون مع سيّاراتهم في الصباح ويغادرون بعد انتهاء يوم العمل فقط.

\begin{enumerate}
    \item اكتب عملية تتلقى عددين صحيحين يمثّلان وزن سيارة (بالأطنان) وارتفاعها (بالأمتار). \\
    تُعيد العملية \textenglish{true} إذا كانت السيارة تستطيع الدخول إلى الموقف، وتُعيد \textenglish{false} إذا لم تكن تستطيع.
    \\ \textbf{مثال:} وزن 8 وارتفاع 4 $\leftarrow$ \textenglish{true}. وزن 12 وارتفاع 3 $\leftarrow$ \textenglish{false}.

    \item اكتب مقطع برنامج يستقبل في البداية عدد أماكن الوقوف الكليّة في الموقف. \\
    بعد ذلك، يستقبل البرنامج وزن ثم ارتفاع كل سيارة ترغب في الدخول. \\
    لكل سيارة، يجب فحص إمكانية دخولها باستخدام العملية التي كتبتها في البند "أ".
    \begin{itemize}
        \item إذا كانت السيارة مناسبة: يطبع البرنامج \textenglish{"IN"}، ويتم حجز مكان واحد في الموقف.
        \item إذا لم تكن مناسبة: يطبع البرنامج \textenglish{"OUT"}.
    \end{itemize}
    ينتهي البرنامج عندما يمتلئ الموقف تمامًا (لا تتبقى أماكن خالية). \\

    \textbf{(بونوس +5 علامات)} في النهاية، يطبع البرنامج عدد السيارات التي \underline{لم} تستطع الدخول.
\end{enumerate}

\ifwithsols
\begin{boxSolution}
\begin{english}
\begin{minted}{csharp}
public static bool CanPark(int weight, int height)
{
    return (weight < 10) && (height < 5);
}

static void Main(string[] args)
{
    int available_parks = int.Parse(Console.ReadLine());
    int weight, height, count = 0;
    while (available_parks > 0)
    {
        Console.WriteLine("Enter the car's weight (tons): ");
        weight = int.Parse(Console.ReadLine());
        Console.WriteLine("Enter the car's height (meters): ");
        height = int.Parse(Console.ReadLine());
        if(CanPark(weight, height))
        {
            Console.WriteLine("IN");
            available_parks--;
        }
        else
        {
            count++;
            Console.WriteLine("OUT");
        }
    }
    Console.WriteLine(count);
}
\end{minted}
\end{english}
\end{boxSolution}
\fi

\clearpage

\item \textbf{السؤال الخامس} (30 علامة) \\
في مطعم "الشبعي" يبيعون نوعين من الشطائر: شطيرة جبنة وشطيرة أفوكادو. \\
من أجل طلب شطيرة جبنة، يجب اختيار نوع 0، وسعرها 10 شواكل. \\
من أجل طلب شطيرة أفوكادو، يجب اختيار نوع 1، وسعرها 12 شيكلاً. \\
الطلبيّة التي تشمل 10 شطائر أو أكثر تمنح تخفيضًا قدره 20 شيكلاً لكلّ الطلبيّة.

\begin{enumerate}
    \item اكتب عملية تتلقى نوع الشطيرة (0 أو 1) وعدد الشطائر في الطلبية. \\
    تحسب العملية وتعيد \textbf{السعر النهائي} للطلبية (مع احتساب التخفيض إذا استحق ذلك).
    \\ \textbf{أمثلة:}
    \begin{itemize}
        \item 5 شطائر جبنة (نوع 0): السعر $5 \times 10 = 50$.
        \item 10 شطائر جبنة (نوع 0): السعر $(10 \times 10) - 20 = 80$.
        \item 15 شطيرة أبوكادو (نوع 1): السعر $(15 \times 12) - 20 = 160$.
    \end{itemize}

    \item اكتب برنامجًا يستقبل الطلبيّات التي تلقاها المطعم في يوم معيّن. \\
    بالنسبة لكلّ طلبيّة، يستقبل البرنامج نوع الشطيرة ثم عدد الشطائر. \\
    لكل طلبية يستقبلها البرنامج، عليه أن يطبع التكلفة النهائية لهذه الطلبية. \\
    يجب أن تستخدموا العملية من البند "أ" لحساب تكلفة كل طلبية. \\
    ينتهي الاستقبال عندما يتم إدخال 0 في عدد الشطائر. \\

    \textbf{(بونوس +5 علامات)} بعد نهاية الاستقبال، على البرنامج أن يطبع عدد الطلبيّات الكلي التي تلقاها المطعم في ذلك اليوم.
\end{enumerate}

\ifwithsols
\begin{boxSolution}
\begin{english}
\begin{minted}{csharp}
public static int CalcPrice(int type, int count)
{
    int total;
    if (type == 0)
        total = 10 * count;
    else
        total = 12 * count;

    if (count >= 10)
        total -= 20;
    return total;
}

static void Main(string[] args)
{
    int type = int.Parse(Console.ReadLine());
    int count = int.Parse(Console.ReadLine());
    int sum = 0;
    while(count > 0)
    {
        Console.WriteLine(CalcPrice(type, count));
        sum += count;

        type = int.Parse(Console.ReadLine());
        count = int.Parse(Console.ReadLine());
    }
    Console.WriteLine(sum);
}
\end{minted}
\end{english}
\end{boxSolution}
\fi

\end{enumerate}


\clearpage              % Push all content to the next page
\thispagestyle{empty}   % Remove page numbers and headers for this specific page
\mbox{}                 % Insert an invisible box so the page isn't "empty" to LaTeX
\clearpage

\begin{center}
    \textbf{\Large السؤال الأول - جدول متابعة}
\end{center}
\vspace{0.5cm}

\begin{flushright}
    \textbf{\Large الاسم: \underline{\hspace{6cm}}}
\end{flushright}

\vspace{1cm}

\textbf{\large أ.}
\begin{center}
    \begin{tabular}{|c|c|c|c|c|c|}
        \hline
        \textbf{\textenglish{i}} & \textbf{\textenglish{x}} & \textbf{\textenglish{count}} & \textbf{\textenglish{x < 50}} & \textbf{\textenglish{x \% 2 == 0}} & \textbf{الطباعة} \\
        \hline
        \hspace{1cm} & \hspace{1.5cm} & \hspace{1.5cm} & \hspace{2cm} & \hspace{2.5cm} & \hspace{3cm} \\ \hline
         & & & & & \\ \hline
         & & & & & \\ \hline
         & & & & & \\ \hline
         & & & & & \\ \hline
         & & & & & \\ \hline
         & & & & & \\ \hline
         & & & & & \\ \hline
         & & & & & \\ \hline
         & & & & & \\ \hline
    \end{tabular}
\end{center}

\vspace{1.5cm}

\textbf{\large ب.}
\textbf{\textenglish{n = 4}}، الأعداد هي: \\

\begin{center}
    \Large
    \textenglish{\underline{\hspace{1.5cm}} , \underline{\hspace{1.5cm}} , \underline{\hspace{1.5cm}} , \underline{\hspace{1.5cm}}}
\end{center}
\end{document}