\documentclass[14pt]{extarticle}
% Full article preamble (duplicated, no common file)
\usepackage{fontspec}
\usepackage[a4paper,top=2.4cm,bottom=2.4cm,left=2.3cm,right=2.3cm]{geometry}
\usepackage{polyglossia}
\usepackage{amsmath}
\usepackage{amssymb}
\usepackage{xcolor}
\usepackage{fancyhdr}
\usepackage{graphicx}
\usepackage{listings}
\usepackage[most]{tcolorbox}
\usepackage{pifont}
\usepackage{enumitem}
\usepackage{titlesec}
\usepackage[bottom]{footmisc}
\usepackage{titling}
\usepackage{minted}
\usepackage{etoolbox}
\usepackage{array}
\usepackage{extsizes}

\newfontfamily\emoji{Segoe UI Emoji}

\pagestyle{fancy}

\setmainlanguage[numerals=western]{arabic}
\setotherlanguage{english}
\newfontfamily\arabicfont[Script=Arabic]{Amiri}
\newfontfamily\arabicfonttt[Script=Arabic]{Courier New}

\lstset{
  language=[Sharp]C,
  numbers=left,
  stepnumber=1,
  numbersep=8pt,
  frame=single,
  basicstyle=\ttfamily\small,
  keywordstyle=\color{blue},
  stringstyle=\color{red},
  commentstyle=\color{green!50!black}
}

\newif\ifdetailed
\ifdefined\setdetailed
  \setdetailed
\fi

\newif\ifwithsols
\ifdefined\setwithsols
  \setwithsols
\fi

% unified tcolorboxes for articles
\tcbset{colback=white, colframe=black, fonttitle=\bfseries, boxrule=0.8pt}
\newtcolorbox{boxDef}[1][]{colback=blue!5!white,colframe=blue!75!black,
  title={{\emoji📘} تعريف\ifx\\#1\\\else ~#1\fi :}}
\newtcolorbox{boxExercise}[1][]{colback=cyan!5!white,colframe=cyan!70!black,
  title={{\emoji🧩} تمرين\ifx\\#1\\\else ~#1\fi :}}
\newtcolorbox{boxExample}[1][]{colback=yellow!5!white,colframe=orange!90!black,
  title={{\emoji📝} مثال\ifx\\#1\\\else ~#1\fi :}}
\newtcolorbox{boxNote}[1][]{colback=gray!10!white,colframe=black,
  title={{\emoji✨} ملاحظة\ifx\\#1\\\else ~#1\fi :}}
\newtcolorbox{boxAttention}[1][]{colback=magenta!10!white,colframe=magenta!80!black,
  title={{\emoji🔔} تنبيه\ifx\\#1\\\else ~#1\fi :}}
\newtcolorbox{boxWarning}[1][]{colback=red!5!white,colframe=red!75!black,
  title={{\emoji⚡} ملاحظة هامة\ifx\\#1\\\else ~#1\fi :}}
\newtcolorbox{boxSolution}[1][]{colback=green!5!white,colframe=green!60!black,
  title={{\emoji✅} حل\ifx\\#1\\\else ~#1\fi :}}
\newtcolorbox{boxSymbol}[1][]{colback=purple!5!white,colframe=purple!70!black,
  title={{\emoji🔣} رمز\ifx\\#1\\\else ~#1\fi :}}
\newtcolorbox{boxHint}[1][]{colback=teal!5!white,colframe=teal!60!black,
  title={{\emoji💡} تلميح\ifx\\#1\\\else ~#1\fi :}}


\tcbset{simplecode/.style={ colback=gray!5, colframe=black!50, boxrule=0.4pt, arc=2pt, left=4pt,right=4pt,top=4pt,bottom=4pt}}
\newenvironment{boxCode}{\begin{tcolorbox}[simplecode]}{\end{tcolorbox}}

\newcolumntype{C}[1]{>{\centering\arraybackslash}p{#1}}

% redefine spaces after titles
\makeatletter
\renewcommand{\@maketitle}{%
  \begin{center}
    {\huge \bfseries \@title \par}%
    \vskip 0.2em % space between title and author
    {\large \@author \par}%
    % \vskip 0.2em % space between author and date
    % {\normalsize \@date \par}%
  \end{center}
}
\makeatother

\fancyhf{} % clear default
\fancypagestyle{plain}{
  \fancyhf{}
  \fancyhead[L]{مدرسة التسامح الشاملة}
  % \fancyhead[L]{\includegraphics[height=1cm]{../../../images/logoTasamoh.png}}
  \fancyhead[R]{الأستاذ محمود اغبارية}
  \fancyfoot[C]{\thepage}
}

\fancyhead[L]{مدرسة التسامح الشاملة}
\fancyhead[R]{الأستاذ محمود اغبارية}
\fancyfoot[C]{\thepage}
% \date{\today}

\setcounter{tocdepth}{3} % only section subsection and subsubsection in TOC


% ----------------------


% \begin{document}

% \maketitle

% % \clearpage  % start TOC on a new page
% % \renewcommand{\contentsname}{جدول المحتويات}
% % \tableofcontents
% % \clearpage

% \part*{part 1} % the * prevents numbering
% \section*{مقدمة}
% \subsection*{مثال رياضي}
% \subsubsection*{مثال فرعي}
% \paragraph*{ paragraph 1}
% \subparagraph*{sub paragraph 1}

% \ifdetailed
% \begin{english}
% \begin{minted}{csharp}
% // C# Example
% \end{minted}
% \end{english}
% \fi

% OLD WAY
% \ifdetailed
% \begin{english}
% \begin{lstlisting}
% // C# Example
% \end{lstlisting}
% \end{english}
% \fi

% % \includegraphics[width=0.2\textwidth]{../../../images/DFAs/ex1_q1.png}



% \vspace{3cm}
% \begin{flushleft}
% أرجو لكم وقتًا ممتعًا.

% الأستاذ محمود اغبارية.
% \end{flushleft}


% \end{document}


\title{اختبار شهري للصف 10-10 - موعد ب}

\begin{document}
\maketitle
% \thispagestyle{fancy}

\ifwithsols
\else
\begin{boxCode}
    \begin{itemize}[nosep]
        \item \textbf{أجب عن جميع الأسئلة}
        \item الوقت المخصص: ساعة ونصف.
        \item يسمح باستخدام كل مادة مساعة، عدا الآلة التي يمكن برمجتها.
        \item أجب بلغة \textenglish{C\#} فقط.
        \item اكتب بقلم حبر فقط، وبخط واضح.
        \item الحل على ورقة خارجية، لا تنسَ كتابة اسمك
    \end{itemize}
\end{boxCode}
\fi

\vspace{0.5cm}

\begin{enumerate}[itemsep=2em]

\item
اكتب مقطع برنامج يستقبل عددين صحيحين ويطبع \textenglish{"Double"} إذا كان أحدهما ضعف الآخر، و \textenglish{"Not Double"} خلاف ذلك.

\item
اكتب مقطع برنامج يستقبل عددًا صحيحا موجبًا، ويطبع الرقم نفسه إذا كان فرديًّا، وإلا فإنّه يطبع الرقم الفردي السّابق.\\
\textbf{مثال}: إذا استقبل العدد 5، يطبع البرنامج 5. وإذا استقبل 6 يطبع البرنامج 5.

\item
اكتب مقطع برنامج يستقبل عددًا صحيحًا وموجبًا مكوّنًا من ثلاث منازل. \\
على البرنامج أن يطبع \textenglish{Yes} إذا كانت منزلة الآحاد في العدد مساوية لمنزلة العشرات أو منزلة المئات. ويطبع \textenglish{No} في باقي الحالات. \\
\textbf{مثال}: إذا كان العدد المدخل هو 343، أو 433، يطبع البرنامج \textenglish{Yes}. \\
إذا كان العدد المدخل 443، أو 123 فعلى البرنامج أن يطبع \textenglish{No}. \\

\item
معطاة الدالة التالية:
\[ g(x) = \frac{x^3 + 2}{\sqrt{x^2 - 16}} \]

اكتب مقطع برنامج يستقبل من المستخدم عددًا عشريًّا يمثّل قيمة $x$ ويطبع قيمة $g(x)$. \\
لكن علينا الانتباه لمجال تعريف الدالة، لذلك:
\begin{itemize}
    \item إذا كان $x = 4$ أو $x = -4$ على البرنامج أن يطبع "خطأ: قسمة على صفر"، \\ \textenglish{("ERROR: division by 0")}.
    \item إذا كان $x$ بين $4$ و $-4$ على البرنامج أن يطبع: "خطأ: جذر عدد سالب"، \textenglish{("ERROR: Negative square root")}.
    \item في باقي الحالات (أي إذا كان $x$ أكبر من $4$ أو أصغر من $-4$) ، على البرنامج أن يطبع قيمة $g(x)$.
\end{itemize}

\item
نفترض أنّ $(x_1,y_1)$ و $(x_2,y_2)$ هما إحداثيات نقطتين في المستوى تمثلان زاويتين متقابلتين لمستطيل. \\
يمكننا فحص هل النقطة $(x,y)$ تقع داخل المستطيل (بما في ذلك الحواف) أم لا عن طريق فحص هل تتحقق \textbf{كل} الشروط التالية: \\
$x_1 \leq x \leq x_2$ و $y_1 \leq y \leq y_2$.

اكتب مقطع برنامج يستقبل عددين عشريّين يمثلان إحداثيات النقطة $(x, y)$ ويفحص إذا كانت موجودة داخل المستطيل أم لا. \\
لا حاجة لاستقبال قيم $x_1, y_1, x_2, y_2$ من المستخدم - افترض أنّها معرفة في البرنامج. \\
إذا كانت النقطة تقع داخل المستطيل، على البرنامج أن يطبع \textenglish{"Inside"}، وإلا فإنّه يطبع \textenglish{"Outside"} في باقي الحالات.

\textbf{مثال}: إذا كانت النقطة $(x, y) = (15, 8)$ والمستطيل محدد بـ $(x_1, y_1) = (10, 5)$ و $(x_2, y_2) = (20, 10)$، فعلى البرنامج أن يطبع \textenglish{"Inside"} لأنّ $10 \leq 15 \leq 20$ و $5 \leq 8 \leq 10$.

\end{enumerate}

\clearpage

\noindent\textbf{السؤال 4 — مخطط الحل}

\begin{boxCode}
\begin{english}
\begin{minted}{csharp}
double x = double.Parse(Console.ReadLine());

if (____________________________________)
{
    Console.WriteLine("ERROR: division by 0");
}
else if (_________________________________________)
{
    Console.WriteLine("ERROR: Negative square root");
}
else
{
    _________________________________________________;

    _________________________________________________;

    _________________________________________________;

    _________________________________________________;

    _________________________________________________;


    Console.WriteLine("f(x) = " +        );
}
\end{minted}
\end{english}
\end{boxCode}
\end{document}
