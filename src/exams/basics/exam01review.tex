\documentclass[12pt]{beamer}
% Fonts and languages
\usepackage{fontspec}
\usepackage{polyglossia}
\usepackage{bidi}
\usepackage{listings}
\usepackage{xcolor}

\setmainlanguage[numerals=western]{arabic}
\setotherlanguage{english}
\newfontfamily\arabicfont[Script=Arabic]{Amiri}
\newfontfamily\arabicfontsf[Script=Arabic]{Amiri}
\newfontfamily\arabicfonttt[Script=Arabic]{Courier New}

% RTL support and nicer bullets for Arabic beamer
\setbeamertemplate{itemize items}[circle]
\setbeamertemplate{itemize subitem}[triangle]

% Listings for C# code blocks

\lstdefinelanguage{CSharp}{
  language={[Sharp]C},
  morekeywords={string,bool,void}
}

\lstset{
  language=CSharp,
  basicstyle=\ttfamily\small,
  keywordstyle=\color{blue},
  stringstyle=\color{red},
  commentstyle=\color{green!50!black},
  showstringspaces=false,
  numbers=none,
  frame=single,
  breaklines=true
}


\usetheme{Madrid}

\title{مراجعة قبل الامتحان – لغة C\#}
\author{}
\date{}

\begin{document}

\begin{frame}
  \titlepage
\end{frame}

%-----------------------------------
\begin{frame}[fragile]{الطباعة والإدخال في الكونسول}
\textbf{سؤال:}
اكتب برنامجًا يطلب من المستخدم اسمه ثم يطبع رسالة ترحيب.

\pause
\textbf{الحل:}
\begin{lstlisting}[language={[Sharp]C}]
Console.Write("اكتب اسمك: ");
string name = Console.ReadLine();
Console.WriteLine("أهلاً بك يا " + name + "!");
\end{lstlisting}
\end{frame}

%-----------------------------------
\begin{frame}{الطباعة والإدخال في الكونسول}
\textbf{سؤال:}
ما الفرق بين \texttt{Write()} و \texttt{WriteLine()}؟

\pause
\textbf{الحل:}
\texttt{Write()} لا تنتقل إلى سطر جديد بعد الطباعة،
أما \texttt{WriteLine()} فتنتقل إلى سطر جديد تلقائيًا.
\end{frame}

%-----------------------------------
\begin{frame}{القسمة وباقي القسمة}
\textbf{سؤال:}
احسب ناتج كل من:
\begin{itemize}
  \item \texttt{7 / 2}
  \item \texttt{7 / 2.0}
  \item \texttt{7 \% 2}
\end{itemize}

\pause
\textbf{الحل:}
\begin{itemize}
  \item \texttt{7 / 2} $\rightarrow$ الناتج 3
  \item \texttt{7 / 2.0} $\rightarrow$ الناتج 3.5
  \item \texttt{7 \% 2} $\rightarrow$ الناتج 1
\end{itemize}
\end{frame}

%-----------------------------------
\begin{frame}[fragile]{القسمة وباقي القسمة}
\textbf{سؤال:}
حوّل السطر التالي ليعطي نتيجة عشريّة:
\begin{lstlisting}[language={[Sharp]C}]
int result = 5 / 2;
\end{lstlisting}

\pause
\textbf{الحل:}
\begin{lstlisting}[language={[Sharp]C}]
double result = 5 / 2.0;  // أو (double)5 / 2
\end{lstlisting}
\end{frame}

%-----------------------------------
\begin{frame}[fragile]{مكتبة Math}
\textbf{سؤال:}
اكتب برنامجًا يطلب رقمين ويطبع الأكبر بينهما.

\pause
\textbf{الحل:}
\begin{lstlisting}[language={[Sharp]C}]
Console.Write("أدخل العدد الأول: ");
int a = int.Parse(Console.ReadLine());
Console.Write("أدخل العدد الثاني: ");
int b = int.Parse(Console.ReadLine());
Console.WriteLine("الأكبر هو: " + Math.Max(a, b));
\end{lstlisting}
\end{frame}

%-----------------------------------
\begin{frame}[fragile]{مكتبة Math}
\textbf{سؤال:}
احسب الجذر التربيعي والتقريب لأقرب عدد صحيح للعدد 7.6.

\pause
\textbf{الحل:}
\begin{lstlisting}[language={[Sharp]C}]
Console.WriteLine(Math.Sqrt(7.6));   // 2.757...
Console.WriteLine(Math.Round(7.6));  // 8
\end{lstlisting}
\end{frame}

%-----------------------------------
\begin{frame}[fragile]{جملة if-else}
\textbf{سؤال:}
اكتب برنامجًا يطلب عمر المستخدم، وإذا كان أقل من 18 يطبع "قاصر"، وإلا يطبع "بالغ".

\pause
\textbf{الحل:}
\begin{lstlisting}[language={[Sharp]C}]
Console.Write("أدخل عمرك: ");
int age = int.Parse(Console.ReadLine());

if (age < 18)
    Console.WriteLine("قاصر");
else
    Console.WriteLine("بالغ");
\end{lstlisting}
\end{frame}

%-----------------------------------
\begin{frame}{جملة if-else}
\textbf{سؤال:}
ما الفرق بين كتابة الأقواس \{\} وعدم كتابتها في جملة \texttt{if}؟

\pause
\textbf{الحل:}
إذا كانت الجملة تحتوي على تعليمة واحدة فقط يمكن حذف الأقواس،
لكن يُفضّل دائمًا كتابتها لتجنّب الأخطاء المستقبلية.
\end{frame}

%-----------------------------------
\begin{frame}[fragile]{جملة if-else مع عمليات منطقية}
\textbf{سؤال:}
اكتب برنامجًا يطلب رقمًا ويطبع "موجب وصغير" إذا كان أكبر من 0 وأصغر من 10.

\pause
\textbf{الحل:}
\begin{lstlisting}[language={[Sharp]C}]
Console.Write("أدخل رقمًا: ");
int x = int.Parse(Console.ReadLine());

if (x > 0 && x < 10)
    Console.WriteLine("موجب وصغير");
else
    Console.WriteLine("الشرط غير محقق");
\end{lstlisting}
\end{frame}

%-----------------------------------
\begin{frame}[fragile]{عمليات منطقية}
\textbf{سؤال:}
اكتب شرطًا يتحقق إذا كان الرقم سالبًا أو أكبر من 100.

\pause
\textbf{الحل:}
\begin{lstlisting}[language={[Sharp]C}]
if (x < 0 || x > 100)
    Console.WriteLine("خارج النطاق");
\end{lstlisting}
\end{frame}

%-----------------------------------
\begin{frame}[fragile]{عمليات منطقية}
\textbf{سؤال:}
اكتب شرطًا يطبع "صفر" فقط إذا كان الرقم ليس موجبًا وليس سالبًا.

\pause
\textbf{الحل:}
\begin{lstlisting}[language={[Sharp]C}]
if (!(x > 0) && !(x < 0))
    Console.WriteLine("صفر");
\end{lstlisting}
\end{frame}

%-----------------------------------
\begin{frame}
\centering
بالتوفيق في الامتحان 🌟\\
راجِع الأمثلة وجرّب تنفيذها بنفسك!
\end{frame}

\end{document}
