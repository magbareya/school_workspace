\documentclass[14pt]{extarticle}
% Full article preamble (duplicated, no common file)
\usepackage{fontspec}
\usepackage[a4paper,top=2.4cm,bottom=2.4cm,left=2.3cm,right=2.3cm]{geometry}
\usepackage{polyglossia}
\usepackage{amsmath}
\usepackage{amssymb}
\usepackage{xcolor}
\usepackage{fancyhdr}
\usepackage{graphicx}
\usepackage{listings}
\usepackage[most]{tcolorbox}
\usepackage{pifont}
\usepackage{enumitem}
\usepackage{titlesec}
\usepackage[bottom]{footmisc}
\usepackage{titling}
\usepackage{minted}
\usepackage{etoolbox}
\usepackage{array}
\usepackage{extsizes}

\newfontfamily\emoji{Segoe UI Emoji}

\pagestyle{fancy}

\setmainlanguage[numerals=western]{arabic}
\setotherlanguage{english}
\newfontfamily\arabicfont[Script=Arabic]{Amiri}
\newfontfamily\arabicfonttt[Script=Arabic]{Courier New}

\lstset{
  language=[Sharp]C,
  numbers=left,
  stepnumber=1,
  numbersep=8pt,
  frame=single,
  basicstyle=\ttfamily\small,
  keywordstyle=\color{blue},
  stringstyle=\color{red},
  commentstyle=\color{green!50!black}
}

\newif\ifdetailed
\ifdefined\setdetailed
  \setdetailed
\fi

\newif\ifwithsols
\ifdefined\setwithsols
  \setwithsols
\fi

% unified tcolorboxes for articles
\tcbset{colback=white, colframe=black, fonttitle=\bfseries, boxrule=0.8pt}
\newtcolorbox{boxDef}[1][]{colback=blue!5!white,colframe=blue!75!black,
  title={{\emoji📘} تعريف\ifx\\#1\\\else ~#1\fi :}}
\newtcolorbox{boxExercise}[1][]{colback=cyan!5!white,colframe=cyan!70!black,
  title={{\emoji🧩} تمرين\ifx\\#1\\\else ~#1\fi :}}
\newtcolorbox{boxExample}[1][]{colback=yellow!5!white,colframe=orange!90!black,
  title={{\emoji📝} مثال\ifx\\#1\\\else ~#1\fi :}}
\newtcolorbox{boxNote}[1][]{colback=gray!10!white,colframe=black,
  title={{\emoji✨} ملاحظة\ifx\\#1\\\else ~#1\fi :}}
\newtcolorbox{boxAttention}[1][]{colback=magenta!10!white,colframe=magenta!80!black,
  title={{\emoji🔔} تنبيه\ifx\\#1\\\else ~#1\fi :}}
\newtcolorbox{boxWarning}[1][]{colback=red!5!white,colframe=red!75!black,
  title={{\emoji⚡} ملاحظة هامة\ifx\\#1\\\else ~#1\fi :}}
\newtcolorbox{boxSolution}[1][]{colback=green!5!white,colframe=green!60!black,
  title={{\emoji✅} حل\ifx\\#1\\\else ~#1\fi :}}
\newtcolorbox{boxSymbol}[1][]{colback=purple!5!white,colframe=purple!70!black,
  title={{\emoji🔣} رمز\ifx\\#1\\\else ~#1\fi :}}
\newtcolorbox{boxHint}[1][]{colback=teal!5!white,colframe=teal!60!black,
  title={{\emoji💡} تلميح\ifx\\#1\\\else ~#1\fi :}}


\tcbset{simplecode/.style={ colback=gray!5, colframe=black!50, boxrule=0.4pt, arc=2pt, left=4pt,right=4pt,top=4pt,bottom=4pt}}
\newenvironment{boxCode}{\begin{tcolorbox}[simplecode]}{\end{tcolorbox}}

\newcolumntype{C}[1]{>{\centering\arraybackslash}p{#1}}

% redefine spaces after titles
\makeatletter
\renewcommand{\@maketitle}{%
  \begin{center}
    {\huge \bfseries \@title \par}%
    \vskip 0.2em % space between title and author
    {\large \@author \par}%
    % \vskip 0.2em % space between author and date
    % {\normalsize \@date \par}%
  \end{center}
}
\makeatother

\fancyhf{} % clear default
\fancypagestyle{plain}{
  \fancyhf{}
  \fancyhead[L]{مدرسة التسامح الشاملة}
  % \fancyhead[L]{\includegraphics[height=1cm]{../../../images/logoTasamoh.png}}
  \fancyhead[R]{الأستاذ محمود اغبارية}
  \fancyfoot[C]{\thepage}
}

\fancyhead[L]{مدرسة التسامح الشاملة}
\fancyhead[R]{الأستاذ محمود اغبارية}
\fancyfoot[C]{\thepage}
% \date{\today}

\setcounter{tocdepth}{3} % only section subsection and subsubsection in TOC


% ----------------------


% \begin{document}

% \maketitle

% % \clearpage  % start TOC on a new page
% % \renewcommand{\contentsname}{جدول المحتويات}
% % \tableofcontents
% % \clearpage

% \part*{part 1} % the * prevents numbering
% \section*{مقدمة}
% \subsection*{مثال رياضي}
% \subsubsection*{مثال فرعي}
% \paragraph*{ paragraph 1}
% \subparagraph*{sub paragraph 1}

% \ifdetailed
% \begin{english}
% \begin{minted}{csharp}
% // C# Example
% \end{minted}
% \end{english}
% \fi

% OLD WAY
% \ifdetailed
% \begin{english}
% \begin{lstlisting}
% // C# Example
% \end{lstlisting}
% \end{english}
% \fi

% % \includegraphics[width=0.2\textwidth]{../../../images/DFAs/ex1_q1.png}



% \vspace{3cm}
% \begin{flushleft}
% أرجو لكم وقتًا ممتعًا.

% الأستاذ محمود اغبارية.
% \end{flushleft}


% \end{document}


\begin{document}
\thispagestyle{fancy}

\subsection*{سؤال 13 امتحان 899205 سنة 2017}

\insertFullImg{../../../bagrut_questions/computational_models/DFA_2017_899205_13.png}

\ifwithsols
\begin{boxSolution}[سؤال 13 امتحان 899205 سنة 2017]
\begin{enumerate}[itemsep=1.5em, label=\alph*.]
\item .
\begin{center}
    \includegraphics[width=0.7\textwidth]{../../../images/DFAs/DNA_startsAATcontainsATtwice.png}
\end{center}

\item
\begin{enumerate}[label=\roman*]
\item $L_6 \subset L_3$ \\
صحيح، لأنّ كل عدد يقسم على 6، فهو بالضرورة يقسم على 3. \\
لذلك، كل حد ينتمي إلى $L_6$ فإنّه ينتمي إلى $L_3$ بالضرورة.

\item $L_2 \cap L_3 = L_6$ \\
صحيح، لأنّ الأعداد التي تقسم على 6 هي الأعداد التي تقسم على 2 وأيضًا تقسم على 3.

\item $L_2 \cdot L_3 = L_6$ \\
غير صحيح، لأنّه ليس بالضرورة إذا ألصقنا عددًا يقسم على 3 مع عدد يقسم على 2 فإنّ النتيجة تقسم على 6. \\
مثلًا: العدد $123$ مكون من العدد $12$ الذي يقسم على 2، وألصقنا به العدد $3$ الذي يقسم على $3$. لذلك:
$123 \in L_2 \cdot L_3$ ولكن $123 \notin L_6$.
\end{enumerate}

\end{enumerate}
\end{boxSolution}
\fi


\clearpage

\begin{boxSolution}
\begin{center}
    \includegraphics[width=0.7\textwidth]{../../../images/DFAs/different_parity_for_a_and_b.png}
\end{center}
\end{boxSolution}

\clearpage

\begin{boxSolution}
\begin{enumerate}[label=(\arabic*), itemsep=1em]
\item $\epsilon \in L_1 \cup L_3 \cup L_4$ - غير صحيح \\
 لكي نثبت ذلك علينا أن نثبت أنّ $\epsilon \notin L_1$ و $\epsilon \notin L_3$ و $\epsilon \notin L_4$. \\
بالنسبة لـ $L_1$ هذا واضح لأنّ $\epsilon$ لا يحتوي على أي حرف، فهو لا يحتوي على التسلسل $010$. \\
بالنسبة لـ $L_3$، عدد الأصفار في $\epsilon$ هو صفر، وكذلك عدد الآحاد (أي عدد مرات ظهور الرقم $1$) هو صفر، أي أنّ عدد الأصفار في $\epsilon$  مساوٍ لعدد الآحاد فيه. \\
بالنسبة لـ $L_4$ ، عدد الأصفار في $\epsilon$ هو صفر، أي عدد الأصفار فيه هو عدد زوجي.

\item $00100 \in L_1 \cap \overline{L_4} $ - صحيح. \\
في البداية نلاحظ أنّ $\overline{L_4} = \{ w \mid \#_0(w) \% 2 == 0 \}$ أي الكلمات التي فيها عدد الأصفار هو \textbf{زوجي}. \\
لكي نثبت أنّ $00100 \in L_1 \cap \overline{L_4}$ علينا أنّ نثبت أنّ $00100 \in L_1$ وأيضًا أنّ $00100 \in \overline{L_4}$. \\
بالنسبة لـ $00100 \in L_1$ هذا يتحقق لأنّ $00100$ تحتوي على التسلسل $010$. \\
بالنسبة لـ $00100 \in \overline{L_4}$ هذا يتحقق لأنّ عدد الأصفار في $00100$ هو 4 أي عدد زوجي.

\item $\overline{L_2} = \{w \mid 11 \text{تحوي التسلسل} w\}$ - غير صحيح. \\
 $\overline{L_2}$ هي مجموعة كل الكلمات التي ليست في اللغة $L_2$. \\
 حسب التعريف، فإنّ $L_2$ هي مجموعة كل الكلمات التي \underline{لا} تحتوي على التسلسل 00. \\
 لذلك، الكلمات التي ليست موجودة في $L_2$ هي كل الكلمات التي \textbf{تحتوي} على التسلسل 00.\\
 على سبيل المثال: الكلمة $11$ تحتوي على التسلسل 11، لذلك هي موجودة في اللغة المعطاة في هذا الفرع، وهي لا تحتوي على التسلسل 00، لذلك فهي ليست موجودة في $L_2$. \\
 فهذه الكلمة موجودة في المجموعتين، وهذا يدحص الادعاء أنّ اللغة المعطاة هي اللغة المكملة لـ $L_2$.
\end{enumerate}
\end{boxSolution}

\begin{boxSolution}
\begin{enumerate}[label=(\arabic*), itemsep=1em, start=4]

\item $L_1 \subset L_3$ - غير صحيح. \\
لكي نثبت ذلك، يكفي أن نجد كلمة موجودة في $L_1$ ولكنها ليست موجودة في $L_3$. \\
على سبيل المثال الكلمة $0101$ موجودة في اللغة $L_1$ لأنّها تحتوي على السلسلة $010$، لكنها ليست موجودة في $L_3$ لأنّ عدد الأصفار والآحاد فيها متساوٍ.
    \item $L_4 = R(L_4)$ - صحيح. \\
    علينا أن نثبت أنّه لكل كلمة $w \in L_4$ فإنّ $R(w) \in L_4$ أيضًا. \\
    هذا يتحقق لأنّ قلب الكلمة لا يؤثر على عدد الأصفار التي فيها. فإذا كان عدد الأصفار فيها فردًّا، فمقلوبها أيضًا فيه عدد فردي من الأصفار وبالتالي سيكون منتميًا إلى $L_4$.

    \item $L_3 \cap L_4 = L_2$ - غير صحيح. \\
    لكي نثبت ذلك علينا أن نجد كلمة موجودة في $L_3$ وأيضًا في $L_4$ لكنها ليست موجودة في $L_2$. \textbf{أو} أن نجد كلمة موجودة في $L_2$ لكنها ليست موجودة في $L_3$ أو ليست موجودة في $L_4$ (أو ليست موجودة في كليهما.) \\

مثلًا: الكلمة $000$ موجودة في $L_3$ لأنّ عدد الأصفار فيها لا يساوي عدد الآحاد، وموجودة أيضًا في $L_4$ لأنّ عدد الأصفار فيها فردي. لكنّها غير موجودة في $L_2$ لأنّها تحتوي على التسلسل $00$. \\

مثال آخر: الكلمة $0101$ ليست موجودة في $L_3$ لأنّ عدد الأصفار فيها يساوي عدد الآحاد، وليست موجودة في $L_4$ لأنّ عدد الأصفار فيها زوجي. لكنّها موجودة في $L_2$ لأنّها لا تحتوي على التسلسل $00$. \\

\end{enumerate}
\end{boxSolution}

\clearpage

\begin{boxSolution}
    لكي نجد أقصر كلمة في اللغة، نعوّض أصغر قيمة ممكنة للبارامترات، أي:
    $n=1 , k = 1$ \\
    لذلك، أقصر كلمة هي:
    $$ a^1b^{3+1}c^1  \rightarrow abbbbc $$
\end{boxSolution}

\clearpage

\importpdfpage{../../../bagrut_questions/computational_models/DFA_2011_899205_14.pdf}{1}
\importpdfpage{../../../bagrut_questions/computational_models/DFA_2011_899205_14.pdf}{2}

\ifwithsols

\paragraph*{حل سؤال 14 امتحان 899205 سنة 2011}:
\bigskip

\begin{enumerate}[label=\alph*., itemsep=2em]
    \item .
\begin{center}
    \includegraphics[width=0.7\textwidth]{../../../images/DFAs/last_letter_appears_even_times.png}
\end{center}

\item
\begin{enumerate}[label=(\arabic*)]
\item

\begin{enumerate}[label=(\arabic*)]
\item $aaba$  لا يتلقاها الأوتومات.
\item $bbaabb$ يتلقاها الأوتومات: $$ q0 \xrightarrow{\text{b}} q3 \xrightarrow{\text{b}} q4 \xrightarrow{\text{a}} q1 \xrightarrow{\text{a}} q2 \xrightarrow{\text{b}} q3 \xrightarrow{\text{b}} q4$$
\item $abaa$ يتلقاها الأوتومات: $$ q0 \xrightarrow{\text{a}} q1 \xrightarrow{\text{b}} q3 \xrightarrow{\text{a}} q1 \xrightarrow{\text{a}} q2$$
\item $bb$ يتلقاها الأوتومات: $$ q0 \xrightarrow{\text{b}} q3 \xrightarrow{\text{b}} q4$$

\end{enumerate}

\item لغة الأوتومات هي كل الكلمات التي يكون الحرف قبل الأخير فيها هو نفس الحرف الأخير. \\
بكلمات أخرى، هي كل الكلمات التي تنتهي بـ $aa$ أو $bb$.
\end{enumerate}

\end{enumerate}

\fi


\clearpage


\begin{boxSolution}
\begin{enumerate}[label=(\arabic*)]
\item $L_1 \cap L2 = \emptyset$ \\
لأنّ كل الكلمات في $L$ تبدأ بالحرف $0$، بينما في $L_1$ فإنّه لا توجد كلمات تحتوي على $0$ أصلًا.

\item .
\begin{center}
    \includegraphics[width=0.4\textwidth]{../../../images/DFAs/1power_n_2.png}
\end{center}
\end{enumerate}
\end{boxSolution}

\end{document}