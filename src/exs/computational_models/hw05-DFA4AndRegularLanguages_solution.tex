\documentclass[14pt]{extarticle}
% Full article preamble (duplicated, no common file)
\usepackage{fontspec}
\usepackage[a4paper,top=2.4cm,bottom=2.4cm,left=2.3cm,right=2.3cm]{geometry}
\usepackage{polyglossia}
\usepackage{amsmath}
\usepackage{amssymb}
\usepackage{xcolor}
\usepackage{fancyhdr}
\usepackage{graphicx}
\usepackage{listings}
\usepackage[most]{tcolorbox}
\usepackage{pifont}
\usepackage{enumitem}
\usepackage{titlesec}
\usepackage[bottom]{footmisc}
\usepackage{titling}
\usepackage{minted}
\usepackage{etoolbox}
\usepackage{array}
\usepackage{extsizes}

\newfontfamily\emoji{Segoe UI Emoji}

\pagestyle{fancy}

\setmainlanguage[numerals=western]{arabic}
\setotherlanguage{english}
\newfontfamily\arabicfont[Script=Arabic]{Amiri}
\newfontfamily\arabicfonttt[Script=Arabic]{Courier New}

\lstset{
  language=[Sharp]C,
  numbers=left,
  stepnumber=1,
  numbersep=8pt,
  frame=single,
  basicstyle=\ttfamily\small,
  keywordstyle=\color{blue},
  stringstyle=\color{red},
  commentstyle=\color{green!50!black}
}

\newif\ifdetailed
\ifdefined\setdetailed
  \setdetailed
\fi

\newif\ifwithsols
\ifdefined\setwithsols
  \setwithsols
\fi

% unified tcolorboxes for articles
\tcbset{colback=white, colframe=black, fonttitle=\bfseries, boxrule=0.8pt}
\newtcolorbox{boxDef}[1][]{colback=blue!5!white,colframe=blue!75!black,
  title={{\emoji📘} تعريف\ifx\\#1\\\else ~#1\fi :}}
\newtcolorbox{boxExercise}[1][]{colback=cyan!5!white,colframe=cyan!70!black,
  title={{\emoji🧩} تمرين\ifx\\#1\\\else ~#1\fi :}}
\newtcolorbox{boxExample}[1][]{colback=yellow!5!white,colframe=orange!90!black,
  title={{\emoji📝} مثال\ifx\\#1\\\else ~#1\fi :}}
\newtcolorbox{boxNote}[1][]{colback=gray!10!white,colframe=black,
  title={{\emoji✨} ملاحظة\ifx\\#1\\\else ~#1\fi :}}
\newtcolorbox{boxAttention}[1][]{colback=magenta!10!white,colframe=magenta!80!black,
  title={{\emoji🔔} تنبيه\ifx\\#1\\\else ~#1\fi :}}
\newtcolorbox{boxWarning}[1][]{colback=red!5!white,colframe=red!75!black,
  title={{\emoji⚡} ملاحظة هامة\ifx\\#1\\\else ~#1\fi :}}
\newtcolorbox{boxSolution}[1][]{colback=green!5!white,colframe=green!60!black,
  title={{\emoji✅} حل\ifx\\#1\\\else ~#1\fi :}}
\newtcolorbox{boxSymbol}[1][]{colback=purple!5!white,colframe=purple!70!black,
  title={{\emoji🔣} رمز\ifx\\#1\\\else ~#1\fi :}}
\newtcolorbox{boxHint}[1][]{colback=teal!5!white,colframe=teal!60!black,
  title={{\emoji💡} تلميح\ifx\\#1\\\else ~#1\fi :}}


\tcbset{simplecode/.style={ colback=gray!5, colframe=black!50, boxrule=0.4pt, arc=2pt, left=4pt,right=4pt,top=4pt,bottom=4pt}}
\newenvironment{boxCode}{\begin{tcolorbox}[simplecode]}{\end{tcolorbox}}

\newcolumntype{C}[1]{>{\centering\arraybackslash}p{#1}}

% redefine spaces after titles
\makeatletter
\renewcommand{\@maketitle}{%
  \begin{center}
    {\huge \bfseries \@title \par}%
    \vskip 0.2em % space between title and author
    {\large \@author \par}%
    % \vskip 0.2em % space between author and date
    % {\normalsize \@date \par}%
  \end{center}
}
\makeatother

\fancyhf{} % clear default
\fancypagestyle{plain}{
  \fancyhf{}
  \fancyhead[L]{مدرسة التسامح الشاملة}
  % \fancyhead[L]{\includegraphics[height=1cm]{../../../images/logoTasamoh.png}}
  \fancyhead[R]{الأستاذ محمود اغبارية}
  \fancyfoot[C]{\thepage}
}

\fancyhead[L]{مدرسة التسامح الشاملة}
\fancyhead[R]{الأستاذ محمود اغبارية}
\fancyfoot[C]{\thepage}
% \date{\today}

\setcounter{tocdepth}{3} % only section subsection and subsubsection in TOC


% ----------------------


% \begin{document}

% \maketitle

% % \clearpage  % start TOC on a new page
% % \renewcommand{\contentsname}{جدول المحتويات}
% % \tableofcontents
% % \clearpage

% \part*{part 1} % the * prevents numbering
% \section*{مقدمة}
% \subsection*{مثال رياضي}
% \subsubsection*{مثال فرعي}
% \paragraph*{ paragraph 1}
% \subparagraph*{sub paragraph 1}

% \ifdetailed
% \begin{english}
% \begin{minted}{csharp}
% // C# Example
% \end{minted}
% \end{english}
% \fi

% OLD WAY
% \ifdetailed
% \begin{english}
% \begin{lstlisting}
% // C# Example
% \end{lstlisting}
% \end{english}
% \fi

% % \includegraphics[width=0.2\textwidth]{../../../images/DFAs/ex1_q1.png}



% \vspace{3cm}
% \begin{flushleft}
% أرجو لكم وقتًا ممتعًا.

% الأستاذ محمود اغبارية.
% \end{flushleft}


% \end{document}



\begin{document}
\thispagestyle{fancy}

\begin{boxSolution}
    \begin{enumerate}[label=\alph*.]
    \item
    \begin{enumerate}[label=(\arabic*)]
        \item $bcbcac$
        \item $a$
    \end{enumerate}

    \item
    نبني أوتوماتًا نهائيًّا لكل واحد من المتطلّبات:
\begin{itemize}[label=-, itemsep=2em]
\item $L_1$ هي لغة كل الكلمات التي فيها الحرف قبل الأخير هو $a$، هذه لغة نظامية لأنّ الأوتومات التالي يقبلها: \\
\includegraphics[width=0.5\textwidth]{../../../images/DFAs/before_last_is_a.png}

\item $L_2$ هي لغة كل الكلمات التي تحتوي على الأقلّ مرّتين التسلسل $bc$، هذه لغة نظامية لأنّ الأوتومات التالي يقبلها: \\
\includegraphics[width=\textwidth]{../../../images/DFAs/contains_bc_twice.png}
\end{itemize}
\end{enumerate}
\end{boxSolution}

\begin{boxSolution}[تكملة]
\begin{itemize}[label=-, itemsep=2em]

\item $L_3$ هي لغة كل الكلمات التي عدد مرات ظهور الحرف $b$ فيها هو زوجيّ. هذه لغة نظامية لأنّ الأوتومات التالي يقبلها: \\
\includegraphics[width=0.4\textwidth]{../../../images/DFAs/even_b.png}

\item $L_4$ هي لغة كل الكلمات التي لا تحوي التسلسل $bb$. هذه لغة نظامية لأنّ الأوتومات التالي يقبلها: \\
\includegraphics[width=0.5\textwidth]{../../../images/DFAs/no_bb.png}
\end{itemize}

اللغة $L$ الموصوفة في السؤال هي اللغة: $L = L_1 \cap L_2 \cap L_3 \cap L_4$. \\
رأينا أعلاه أنّ كل واحدة من اللغات $L_1, L_2, L_3, L_4$ هي لغة نظامية، لذلك $L$ هي لغة نظامية حسب صفة الانغلاق للغات النظامية تحت عملية التقاطع.
\end{boxSolution}

\clearpage

\begin{boxSolution}
\begin{enumerate}[label=(\arabic*)]
\item $\overline{L_1} = \emptyset$ .

\item $L_3 = L_2 \cap \overline{L_1} = L_2 \cap \emptyset = \emptyset$ .\\
إذن، اللغة $L_3$ هي اللغة الفارغة، وهي لغة نظامية، لأنّه يوجد أوتومات يقبل هذه اللغة (أي لا يقبل أي كلمة. ) \\
وهو الأوتومات الذي فيه حالة واحدة هي حالة فخ:
\begin{center} \includegraphics[width=0.2\textwidth]{../../../images/DFAs/emptyset.png} \end{center}
\end{enumerate}
\end{boxSolution}

\clearpage

\begin{boxSolution}
    \begin{enumerate}[label=\alph*., itemsep=2em]
        \item . \\
\begin{tabular}{|r|c|c|c|}
\hline
\textbf{هل مقبولة؟} & $m \% 4$ & $n\%2$ & \textbf{الكلمة} \\
\hline
غير مقبولة & 0 & 1 & $a = a^{1}b^{0}$ \\
\hline
مقبولة & 1 & 1 & $ab = a^{1}b^{1}$ \\
\hline
 مقبولة & 3 & 3 & $aaabbb = a^{3}b^{3}$ \\
\hline
مقبولة & 1 & 1 & $aaabbbbb = a^{3}b^{5}$ \\
\hline
غير مقبولة & 2 & 0 & $aabb = a^{2}b^{2}$ \\
\hline
مقبولة & 1 & 1 & $aaab = a^{3}b^{1}$ \\
\hline
مقبولة & 0 & 0 & $aa = a^{2}b^{0}$ \\
\hline
غير مقبولة (لأن شرط اللغة يطلب $n>0$) & 0 & 0 & $bbbb = a^{0}b^{4}$ \\
\hline
مقبولة & 0 & 0 & $aaaa = a^{4}b^{0}$ \\
\hline
غير مقبولة & 3 & 1 & $abbb = a^{1}b^{3}$ \\
\hline
\end{tabular}

\item . \\
    \includegraphics[width=0.9\textwidth]{../../../images/DFAs/an_bm_mmod4_equals_nmod2.png}
\end{enumerate}
\end{boxSolution}

\clearpage


\begin{boxSolution}
\begin{enumerate}[label=\alph*., itemsep=2em]
    \item نحصل على أقصر كلمة عندما نعوّض أصغر قيمة لكل بارامتر، أي: $k = 1, n=1$. \\
    لذلك أقصر كلمة هي:
    $$ c^{1+1+1} b^1 a^{2 \cdot 1} \rightarrow c^3 b^1 a^2 \rightarrow cccbaa $$ \\
    كي نحدد إذا كانت كلمة ما موجودة في $L_2$ يجب أن يظهر الحرف $d$ مرة واحدة في الوسط بالضبط. \\
    وأيضًا، القسم الذي قبل الحرف $d$ يجب أن يكون كلمة في $L_1$. \\
    والقسم الذي بعد الحرف $d$ يجب ان يكون مقلوب كلمة في $L_1$ (ليس بالضرورة نفس الكلمة التي كانت قبل الحرف $d$).

    \item $L2 = L_1 \cdot d \cdot R(L_1) = \{ c^{1+k+n} b^k a^{2n} d a^{2m} b^q c^{1+q+m} \mid, m,n,k,q \geq 1 \}$
    \begin{itemize}
        \item \textenglish{cccbaaddaabccc} \\
        ليست موجودة في اللغة، لأنّ الحرف $d$  يظهر مرتين في الكلمة.

        \item \textenglish{cccbaadcccbaa} \\
        ليست موجودة في الكلمة لأنّ القسم الموجود بعد الحرف $d$ ليس مقلوب كلمة من اللغة $L_1$، بل هو نفسه موجود في اللغة $L_1$.

        \item $w = cccbaadaaaabcccc$ \\
        موجودة في اللغة $L_2$. لنوضح ذلك نعرّف كلمتين: \\
        $u = cccbaa = c^3 b^1 a^2$ و $v = aaaabcccc = a^4 b^1 c^4$. \\
        يمكننا كتابة:  $w = u \cdot d \cdot v$. \\
        نلاحظ أنّ:
        \begin{itemize}
        \item $u \in L_1$ عندما نعوّض $k = 1, n=1$ (هذه عمليا أقصر كلمة التي وجدناها في الفرع السابق.)
        \item $v \in R(L_1)$ عندما نعوّض $n = 2, k=1$
        \end{itemize}
        لذلك: $w = u \cdot d \cdot v \in L_1 \cdot d \cdot R(L_1) \rightarrow w \in L_2$

    \end{itemize}
\end{enumerate}
\end{boxSolution}

\end{document}