\documentclass[14pt]{extarticle}
% Full article preamble (duplicated, no common file)
\usepackage{fontspec}
\usepackage[a4paper,top=2.4cm,bottom=2.4cm,left=2.3cm,right=2.3cm]{geometry}
\usepackage{polyglossia}
\usepackage{amsmath}
\usepackage{amssymb}
\usepackage{xcolor}
\usepackage{fancyhdr}
\usepackage{graphicx}
\usepackage{listings}
\usepackage[most]{tcolorbox}
\usepackage{pifont}
\usepackage{enumitem}
\usepackage{titlesec}
\usepackage[bottom]{footmisc}
\usepackage{titling}
\usepackage{minted}
\usepackage{etoolbox}
\usepackage{array}
\usepackage{extsizes}

\newfontfamily\emoji{Segoe UI Emoji}

\pagestyle{fancy}

\setmainlanguage[numerals=western]{arabic}
\setotherlanguage{english}
\newfontfamily\arabicfont[Script=Arabic]{Amiri}
\newfontfamily\arabicfonttt[Script=Arabic]{Courier New}

\lstset{
  language=[Sharp]C,
  numbers=left,
  stepnumber=1,
  numbersep=8pt,
  frame=single,
  basicstyle=\ttfamily\small,
  keywordstyle=\color{blue},
  stringstyle=\color{red},
  commentstyle=\color{green!50!black}
}

\newif\ifdetailed
\ifdefined\setdetailed
  \setdetailed
\fi

\newif\ifwithsols
\ifdefined\setwithsols
  \setwithsols
\fi

% unified tcolorboxes for articles
\tcbset{colback=white, colframe=black, fonttitle=\bfseries, boxrule=0.8pt}
\newtcolorbox{boxDef}[1][]{colback=blue!5!white,colframe=blue!75!black,
  title={{\emoji📘} تعريف\ifx\\#1\\\else ~#1\fi :}}
\newtcolorbox{boxExercise}[1][]{colback=cyan!5!white,colframe=cyan!70!black,
  title={{\emoji🧩} تمرين\ifx\\#1\\\else ~#1\fi :}}
\newtcolorbox{boxExample}[1][]{colback=yellow!5!white,colframe=orange!90!black,
  title={{\emoji📝} مثال\ifx\\#1\\\else ~#1\fi :}}
\newtcolorbox{boxNote}[1][]{colback=gray!10!white,colframe=black,
  title={{\emoji✨} ملاحظة\ifx\\#1\\\else ~#1\fi :}}
\newtcolorbox{boxAttention}[1][]{colback=magenta!10!white,colframe=magenta!80!black,
  title={{\emoji🔔} تنبيه\ifx\\#1\\\else ~#1\fi :}}
\newtcolorbox{boxWarning}[1][]{colback=red!5!white,colframe=red!75!black,
  title={{\emoji⚡} ملاحظة هامة\ifx\\#1\\\else ~#1\fi :}}
\newtcolorbox{boxSolution}[1][]{colback=green!5!white,colframe=green!60!black,
  title={{\emoji✅} حل\ifx\\#1\\\else ~#1\fi :}}
\newtcolorbox{boxSymbol}[1][]{colback=purple!5!white,colframe=purple!70!black,
  title={{\emoji🔣} رمز\ifx\\#1\\\else ~#1\fi :}}
\newtcolorbox{boxHint}[1][]{colback=teal!5!white,colframe=teal!60!black,
  title={{\emoji💡} تلميح\ifx\\#1\\\else ~#1\fi :}}


\tcbset{simplecode/.style={ colback=gray!5, colframe=black!50, boxrule=0.4pt, arc=2pt, left=4pt,right=4pt,top=4pt,bottom=4pt}}
\newenvironment{boxCode}{\begin{tcolorbox}[simplecode]}{\end{tcolorbox}}

\newcolumntype{C}[1]{>{\centering\arraybackslash}p{#1}}

% redefine spaces after titles
\makeatletter
\renewcommand{\@maketitle}{%
  \begin{center}
    {\huge \bfseries \@title \par}%
    \vskip 0.2em % space between title and author
    {\large \@author \par}%
    % \vskip 0.2em % space between author and date
    % {\normalsize \@date \par}%
  \end{center}
}
\makeatother

\fancyhf{} % clear default
\fancypagestyle{plain}{
  \fancyhf{}
  \fancyhead[L]{مدرسة التسامح الشاملة}
  % \fancyhead[L]{\includegraphics[height=1cm]{../../../images/logoTasamoh.png}}
  \fancyhead[R]{الأستاذ محمود اغبارية}
  \fancyfoot[C]{\thepage}
}

\fancyhead[L]{مدرسة التسامح الشاملة}
\fancyhead[R]{الأستاذ محمود اغبارية}
\fancyfoot[C]{\thepage}
% \date{\today}

\setcounter{tocdepth}{3} % only section subsection and subsubsection in TOC


% ----------------------


% \begin{document}

% \maketitle

% % \clearpage  % start TOC on a new page
% % \renewcommand{\contentsname}{جدول المحتويات}
% % \tableofcontents
% % \clearpage

% \part*{part 1} % the * prevents numbering
% \section*{مقدمة}
% \subsection*{مثال رياضي}
% \subsubsection*{مثال فرعي}
% \paragraph*{ paragraph 1}
% \subparagraph*{sub paragraph 1}

% \ifdetailed
% \begin{english}
% \begin{minted}{csharp}
% // C# Example
% \end{minted}
% \end{english}
% \fi

% OLD WAY
% \ifdetailed
% \begin{english}
% \begin{lstlisting}
% // C# Example
% \end{lstlisting}
% \end{english}
% \fi

% % \includegraphics[width=0.2\textwidth]{../../../images/DFAs/ex1_q1.png}



% \vspace{3cm}
% \begin{flushleft}
% أرجو لكم وقتًا ممتعًا.

% الأستاذ محمود اغبارية.
% \end{flushleft}


% \end{document}



\ifwithsols
\title{حل وظيفة بيتية 11 - الحلقات}
\else
\title{وظيفة بيتية 11 - الحلقات}
\fi

\begin{document}

\maketitle
\thispagestyle{fancy}

\begin{enumerate}[itemsep=1.5em]


\item
اكتب عملية تتلقى عددين صحيحَين. على العملية أن تعيد مجموع كل الأعداد الفردية الواقعة داخل هذا المجال (يشمل الطرفين).
\ifwithsols
\begin{boxSolution}
\begin{english}
\begin{minted}{csharp}
public static int SumOddRange(int start, int end)
{
    int sum = 0;
    int mn = Math.Min(start, end);
    int mx = Math.Max(start, end);
    for (int i = mn; i <= mx; i++)
    {
        if (i % 2 != 0)
            sum += i;
    }
    return sum;
}
\end{minted}
\end{english}
\end{boxSolution}
\fi


\item
 اكتب عملية خارجية تتلقى عددًا صحيحًا موجبًا $n$. على العملية أن تعيد قيمة "مضروب" العدد، أي $n!$\\
مضروب العدد $n$ هو حاصل ضرب كل الأعداد من 1 إلى $n$. \\
\textbf{مثلًا}: مضروب العدد 5 هو: $1 \times 2 \times 3 \times 4 \times 5 = 120$.
\ifwithsols
\begin{boxSolution}
\begin{english}
\begin{minted}{csharp}
public static long Factorial(int n)
{
    long f = 1;
    for (int i = 1; i <= n; i++)
    {
        f = f * i;
    }
    return f;
}
\end{minted}
\end{english}
\end{boxSolution}
\clearpage
\fi

\item
في هذا السؤال سنكتب عملية تحسب عددًا مرفوعًا لقوة عدد آخر: $b^e$\\
اكتب عملية تتلقى عددين صحيحين: $b$ (الأساس) و $e$ (القوة ). على العملية حساب وإرجاع قيمة $b^e$، دون استخدام الدالة الجاهزة \texttt{Math.Pow}. \\
\textbf{ملاحظة:} $b^e$ هي ضرب العدد $b$ بنفسه $e$ مرّات.

\ifwithsols
\begin{boxSolution}
\begin{english}
\begin{minted}{csharp}
public static int Power(int baseNum, int exp)
{
    int res = 1;
    for (int i = 1; i <= exp; i++)
    {
        res = res * baseNum;
    }
    return res;
}
\end{minted}
\end{english}
\end{boxSolution}
\clearpage
\fi

\item
اكتب عملية تتلقى عددًا صحيحًا $n$، وتقوم بحساب وإرجاع مجموع السلسلة التالية: \\
$Sum = 1 + \frac{1}{2} + \frac{1}{3} + \dots + \frac{1}{n}$
\begin{boxExample}
\begin{itemize}
\item
إذا تلقّت 1 فإنّها تعيد 1.
\item
إذا تلقّت 2 فإنّها تعيد: $1 + \frac{1}{2} = 1.5$
\item
إذا تلقّت 3 فإنّها تعيد: $1 + \frac{1}{2} +  \frac{1}{3} = 1.833$
\item
إذا تلقّت 10 فإنّها تعيد: $1 + \frac{1}{2} +  \frac{1}{3} + \ldots + \frac{1}{10} = 2.9289$
\end{itemize}
\end{boxExample}
\ifwithsols
\begin{boxSolution}
\begin{english}
\begin{minted}{csharp}
public static double SeriesSum(int n)
{
    double sum = 0;
    for (int i = 1; i <= n; i++)
    {
        sum += 1.0 / i;
    }
    return sum;
}
\end{minted}
\end{english}
\end{boxSolution}
\clearpage
\fi

\item
اكتب برنامجًا يستقبل 10 أعداد. على البرنامج أن يفحص هل الأعداد مدخلة بترتيب تصاعدي (كل عدد أكبر من سابقه). \\
يطبع البرنامج "Sorted" إذا كانت مرتبة، و "Not Sorted" إذا لم تكن كذلك.
\ifwithsols
\begin{boxSolution}[1]
\begin{english}
\begin{minted}{csharp}
int prev = int.Parse(Console.ReadLine());
int cntSorted = 0;
for (int i = 1; i < 10; i++)
{
    int curr = int.Parse(Console.ReadLine());
    if (curr > prev)
        cntSorted++;
    prev = curr;
}
if (cnt == 0)
    Console.WriteLine("Sorted");
else
    Console.WriteLine("Not Sorted");
\end{minted}
\end{english}
\end{boxSolution}
\begin{boxSolution}[2]
\begin{english}
\begin{minted}{csharp}
bool isSorted = true;
int prev = int.Parse(Console.ReadLine());
for (int i = 1; i < 10; i++)
{
    int curr = int.Parse(Console.ReadLine());
    if (curr <= prev)
        isSorted = false;
    prev = curr;
}
if (isSorted)
    Console.WriteLine("Sorted");
else
    Console.WriteLine("Not Sorted");
\end{minted}
\end{english}
\end{boxSolution}
\fi

\clearpage
\item
\textbf{سؤال بونوس} \\
اكتب عملية خارجية تتلقى عددين صحيحين $a$ و $b$ وتطبع حاصل قسمة $a$ على $b$ والباقي من القسمة، بدون استخدام العمليات $/$ و $\%$.

\begin{boxHint}
    يمكن حساب حاصل قسمة $a$ على $b$ والباقي بالطريقة التالية: نكرر طرح العدد $b$ من $a$  حتى يصبح أصغر من $b$ . \\
عدد المرات التي طرحنا فيها $b$ من $a$  هي حاصل القسمة. \\
والعدد الأصغر من $b$ الذي نتج بعد تكرار الطرح هو الباقي. \\
مثلًا: إذا كان $a=20$ و $b=6$، فإنّنا نكرّر طرح 6 من 20 حتى يكون الناتج أصغر من 6 $\leftarrow$ طرحنا 6 من 20 ثلاث مرات. بعد أن طرحنا من ال 20 العدد 6 ثلاث مرات، سيبقى 2. إذن حاصل قسمة 20 على 6 هو 3 والباقي 2.
\end{boxHint}
\ifwithsols
\begin{boxSolution}
\begin{english}
\begin{minted}{csharp}
public static void DivMod(int a, int b)
{
    int cnt = 0;
    while (a >= b)
    {
        cnt++;
        a -= b;
    }
    Console.WriteLine("a / b = " + cnt);
    Console.WriteLine("a % b = " + a);
}
\end{minted}
\end{english}
\end{boxSolution}
\fi

\end{enumerate}

\end{document}