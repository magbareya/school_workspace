\documentclass[14pt]{extarticle}
% Full article preamble (duplicated, no common file)
\usepackage{fontspec}
\usepackage[a4paper,top=2.4cm,bottom=2.4cm,left=2.3cm,right=2.3cm]{geometry}
\usepackage{polyglossia}
\usepackage{amsmath}
\usepackage{amssymb}
\usepackage{xcolor}
\usepackage{fancyhdr}
\usepackage{graphicx}
\usepackage{listings}
\usepackage[most]{tcolorbox}
\usepackage{pifont}
\usepackage{enumitem}
\usepackage{titlesec}
\usepackage[bottom]{footmisc}
\usepackage{titling}
\usepackage{minted}
\usepackage{etoolbox}
\usepackage{array}
\usepackage{extsizes}

\newfontfamily\emoji{Segoe UI Emoji}

\pagestyle{fancy}

\setmainlanguage[numerals=western]{arabic}
\setotherlanguage{english}
\newfontfamily\arabicfont[Script=Arabic]{Amiri}
\newfontfamily\arabicfonttt[Script=Arabic]{Courier New}

\lstset{
  language=[Sharp]C,
  numbers=left,
  stepnumber=1,
  numbersep=8pt,
  frame=single,
  basicstyle=\ttfamily\small,
  keywordstyle=\color{blue},
  stringstyle=\color{red},
  commentstyle=\color{green!50!black}
}

\newif\ifdetailed
\ifdefined\setdetailed
  \setdetailed
\fi

\newif\ifwithsols
\ifdefined\setwithsols
  \setwithsols
\fi

% unified tcolorboxes for articles
\tcbset{colback=white, colframe=black, fonttitle=\bfseries, boxrule=0.8pt}
\newtcolorbox{boxDef}[1][]{colback=blue!5!white,colframe=blue!75!black,
  title={{\emoji📘} تعريف\ifx\\#1\\\else ~#1\fi :}}
\newtcolorbox{boxExercise}[1][]{colback=cyan!5!white,colframe=cyan!70!black,
  title={{\emoji🧩} تمرين\ifx\\#1\\\else ~#1\fi :}}
\newtcolorbox{boxExample}[1][]{colback=yellow!5!white,colframe=orange!90!black,
  title={{\emoji📝} مثال\ifx\\#1\\\else ~#1\fi :}}
\newtcolorbox{boxNote}[1][]{colback=gray!10!white,colframe=black,
  title={{\emoji✨} ملاحظة\ifx\\#1\\\else ~#1\fi :}}
\newtcolorbox{boxAttention}[1][]{colback=magenta!10!white,colframe=magenta!80!black,
  title={{\emoji🔔} تنبيه\ifx\\#1\\\else ~#1\fi :}}
\newtcolorbox{boxWarning}[1][]{colback=red!5!white,colframe=red!75!black,
  title={{\emoji⚡} ملاحظة هامة\ifx\\#1\\\else ~#1\fi :}}
\newtcolorbox{boxSolution}[1][]{colback=green!5!white,colframe=green!60!black,
  title={{\emoji✅} حل\ifx\\#1\\\else ~#1\fi :}}
\newtcolorbox{boxSymbol}[1][]{colback=purple!5!white,colframe=purple!70!black,
  title={{\emoji🔣} رمز\ifx\\#1\\\else ~#1\fi :}}
\newtcolorbox{boxHint}[1][]{colback=teal!5!white,colframe=teal!60!black,
  title={{\emoji💡} تلميح\ifx\\#1\\\else ~#1\fi :}}


\tcbset{simplecode/.style={ colback=gray!5, colframe=black!50, boxrule=0.4pt, arc=2pt, left=4pt,right=4pt,top=4pt,bottom=4pt}}
\newenvironment{boxCode}{\begin{tcolorbox}[simplecode]}{\end{tcolorbox}}

\newcolumntype{C}[1]{>{\centering\arraybackslash}p{#1}}

% redefine spaces after titles
\makeatletter
\renewcommand{\@maketitle}{%
  \begin{center}
    {\huge \bfseries \@title \par}%
    \vskip 0.2em % space between title and author
    {\large \@author \par}%
    % \vskip 0.2em % space between author and date
    % {\normalsize \@date \par}%
  \end{center}
}
\makeatother

\fancyhf{} % clear default
\fancypagestyle{plain}{
  \fancyhf{}
  \fancyhead[L]{مدرسة التسامح الشاملة}
  % \fancyhead[L]{\includegraphics[height=1cm]{../../../images/logoTasamoh.png}}
  \fancyhead[R]{الأستاذ محمود اغبارية}
  \fancyfoot[C]{\thepage}
}

\fancyhead[L]{مدرسة التسامح الشاملة}
\fancyhead[R]{الأستاذ محمود اغبارية}
\fancyfoot[C]{\thepage}
% \date{\today}

\setcounter{tocdepth}{3} % only section subsection and subsubsection in TOC


% ----------------------


% \begin{document}

% \maketitle

% % \clearpage  % start TOC on a new page
% % \renewcommand{\contentsname}{جدول المحتويات}
% % \tableofcontents
% % \clearpage

% \part*{part 1} % the * prevents numbering
% \section*{مقدمة}
% \subsection*{مثال رياضي}
% \subsubsection*{مثال فرعي}
% \paragraph*{ paragraph 1}
% \subparagraph*{sub paragraph 1}

% \ifdetailed
% \begin{english}
% \begin{minted}{csharp}
% // C# Example
% \end{minted}
% \end{english}
% \fi

% OLD WAY
% \ifdetailed
% \begin{english}
% \begin{lstlisting}
% // C# Example
% \end{lstlisting}
% \end{english}
% \fi

% % \includegraphics[width=0.2\textwidth]{../../../images/DFAs/ex1_q1.png}



% \vspace{3cm}
% \begin{flushleft}
% أرجو لكم وقتًا ممتعًا.

% الأستاذ محمود اغبارية.
% \end{flushleft}


% \end{document}


\ifwithsols
\title{حل ورقة تمرن 6 للصف العاشر 10 - قسم 2 \\ أسئلة أساسية}
\else
\title{ورقة تمرن 6 للصف العاشر 10 - قسم 2 \\ أسئلة أساسية}
\fi

\begin{document}

\maketitle
\thispagestyle{fancy}

\begin{enumerate}[itemsep=1.5em]
    \item
    \begin{enumerate}
        \item اكتب عمليّة تتلقى عددين صحيحين، على العمليّة أن تعيد عدد الأعداد التي تقسم على 3 بينهما (افترض أن العدد الأول أصغر من الثاني)
        \item في البرنامج الرئيسي، قم باستدعاء العملية مع العددين 20 و70، واطبع النتيجة.
    \end{enumerate}
\ifwithsols
\begin{boxSolution}
\begin{english}
\begin{minted}{csharp}
public static int CountDivBy3(int a, int b)
{
    int c = 0;
    for (int i = a; i <= b; i++)
    {
        if (i % 3 == 0)
            c++;
    }
    return c;
}

private static void Main(string[] args)
{
    Console.WriteLine(CountDivBy3(20, 70));
}
\end{minted}
\end{english}
\end{boxSolution}
\clearpage
\fi

\item
\begin{enumerate}
    \item اكتب عمليّة تتلقى عددين صحيحين، على العمليّة أن تعيد مجموع الأعداد بينهما (يشمل العددين نفسيهما. ). \\
لا تفترض أنّ العدد الثاني أصغر من الأول. (قد يكون الأول أكبر وقد يكون الثاني أكبر).
    \item في البرنامج الرئيسي، قم باستدعاء العملية مع العددين 30 و20، واطبع النتيجة.
\end{enumerate}
\ifwithsols
\begin{boxSolution}
\begin{english}
\begin{minted}{csharp}
public static int SumRange(int a, int b)
{
    int max = Math.Max(a, b);
    int min = Math.Min(a, b);
    int sum = 0;
    for(int i = min; i <= max; i++)
    {
        sum += i;
    }
    return sum;
}
private static void Main(string[] args)
{
    Console.WriteLine(SumRange(30, 20));
}
\end{minted}
\end{english}
\end{boxSolution}
\clearpage
\fi

\item اكتب عملية خارجية تتلقى عددًا صحيحًا، على العملية أن تطبع الأعداد من 1 إلى الرقم الذي تلقته وبجانب كل رقم تربيعه. \\
    مثلًا، إذا  تلقت العملية الرقم 5، فعليها أن تطبع:
    \begin{english}
    \begin{boxCode}
      1 1 \\
      2 4 \\
      3 9 \\
      4 16 \\
      5 25
    \end{boxCode}
    \end{english}
\ifwithsols
\begin{boxSolution}
\begin{english}
\begin{minted}{csharp}
public static void PrintSquares(int n)
{
    for (int i = 1; i <= n; i++)
        Console.WriteLine(i + " " + (i * i));
}

private static void Main(string[] args)
{
    int n = int.Parse(Console.ReadLine());
    PrintSquares(n);
}
\end{minted}
\end{english}
\end{boxSolution}
\clearpage
\fi

\item اكتب برنامجًا يستقبل من المستخدم أعدادًا صحيحة، على البرنامج أن يطبع الجذر التربيعي لكل عدد. (بعد التأكد من أنّ العدد ليس سالبًا).
ينتهي الاستقبال عندما يصبح عدد الأعداد السالبة التي أدخلها المستخدم 3 أعداد.
\ifwithsols
\begin{boxSolution}
\begin{english}
\begin{minted}{csharp}
int x;
int negCnt = 0;
while(negCnt < 3)
{
    x = int.Parse(Console.ReadLine());
    if(x >= 0)
        Console.WriteLine(Math.Sqrt(x));
    else
        negCnt++;
}
\end{minted}
\end{english}
\end{boxSolution}
\fi

\clearpage
\item
تحسب العلامة النهائية في الحاسوب على الشكل التالي: \\
- $30\%$ للوظائف \\
- $30\%$ للامتحان الشهري \\
- $40\%$ للامتحان الفصلي. \\
اكتب عملية خارجية تتلقى عدد طلاب الصف، ثم تستقبل علامات كل واحد من الطلاب (لكل طالب تستقبل 3 علامات)، وتطبع لكل طالب علامته النهائية. \\
على العملية في النهاية أن تعيد عدد الطلاب الناجحين، أي الذي حصلوا على علامة أعلى من 70.

\ifwithsols
\begin{boxSolution}
\begin{english}
\begin{minted}{csharp}
public static int CountSuccessful(int pupilsCnt)
{
    int hw, mid, final, cnt = 0;
    double avg;
    for(int i=1 ; i <= pupilsCnt; i++)
    {
        hw = int.Parse(Console.ReadLine());
        mid = int.Parse(Console.ReadLine());
        final = int.Parse(Console.ReadLine());
        avg = hw * 0.3 + mid * 0.3 + final * 0.4;
        Console.WriteLine(avg);
        if(avg >= 70)
            cnt++;
    }
    return cnt;
}
\end{minted}
\end{english}
\end{boxSolution}
\clearpage
\fi

\item اكتب عمليّة خارجيّة تتلقى عددًا صحيحًا $x$ الذي يمثل الحد الأقصى للوزن الإجمالي الذي يمكن أن تحمله شاحنة ما بالكيلوغرام، على العملية أن تستقبل وزن كل صندوق يتم تحميله للشاحنة، على العملية أن تعيد عدد الصناديق التي تم تحميلها للشاحنة حتى امتلأت. (فكّر ما هو شرط الحلقة؟)
\ifwithsols
\begin{boxSolution}
\begin{english}
\begin{minted}{csharp}
public static int LoadBoxes(int x)
{
    int sum = 0, count = 0;
    int w = int.Parse(Console.ReadLine());
    while (sum + w < x)
    {
        sum += w;
        count++;
        w = int.Parse(Console.ReadLine());
    }
    return count;
}
\end{minted}
\end{english}
\end{boxSolution}
\clearpage
\fi

\item
    في إحدى مسابقات العاب القوى هناك مسابقة "العرض الشيق" بحيث يقوم كل متسابق بعرض مواهبه الرياضية في القفز والحركة. \\
     في هذه اللعبة هناك 7 حكام، كل حكم علامة للمتسابق بعد عرضه. \\
     العلامة النهائية التي يستحقها اللاعب تكون بحساب معدل 5 علامات فقط من بين 7 وذلك بعد شطب العلامة الكبرى والعلامة الصغرى. \\
      اكتب مقطع برنامج يستقبل علامات الحكام السبعة لأحد المتسابقين، على البرنامج أن يطبع معدله النهائي.
\ifwithsols
\begin{boxSolution}
\begin{english}
\begin{minted}{csharp}
int min = int.MaxValue, max = int.MinValue, sum = 0;
for (int i = 0; i < 7; i++)
{
    int s = int.Parse(Console.ReadLine());
    sum += s;
    if (s < min)
        min = s;
    if (s > max)
        max = s;
}
double avg = (sum - min - max) / 5.0;
Console.WriteLine(avg);
\end{minted}
\end{english}
\end{boxSolution}
\clearpage
\fi

\item
اكتب برنامجًا يستقبل من المستخدم 100 عدد صحيح، ويحسب أكبر عدد زوجي وأكبر عدد فرديّ من الأعداد التي أدخلها المستخدم.
\ifwithsols
\begin{boxSolution}
\begin{english}
\begin{minted}{csharp}
int maxEven = int.MinValue;
int maxOdd = int.MinValue;
int x;
for (int i = 1; i <= 100; i++)
{
    x = int.Parse(Console.ReadLine());
    if (x % 2 == 0)
        maxEven = Math.Max(maxEven, x);
    else
        maxOdd = Math.Max(maxOdd, x);
}
Console.WriteLine(maxEven);
Console.WriteLine(maxOdd);
\end{minted}
\end{english}
\end{boxSolution}
%\clearpage
\fi

\item اكتب برنامجًا يستقبل من المستخدم 100 عدد صحيح. \\
على البرنامج أن يعدّ الأعداد التي لها نفس زوجية ترتيب إدخالها ويطبعه. \\
أي: إذا كان أول عدد يدخله المستخدم فرديًّا فنعده وإلا فلا. والعدد الثاني إذا كان زوجيًّا نعده وإلا فلا، والعدد الثالث إذا كان فرديًّا نعدّه وإلا فلا \ldots وهكذا.
\ifwithsols
\begin{boxSolution}
\begin{english}
\begin{minted}{csharp}
int cnt = 0;
int x;
for (int i = 1; i <= 100; i++)
{
    x = int.Parse(Console.ReadLine());
    if (i % 2 == x % 2 )
        cnt++;
}
Console.WriteLine(cnt);
\end{minted}
\end{english}
\end{boxSolution}
\clearpage
\fi

\item
اكتب عملية تتلقى عددًا $n$ وتعيد عدد القيم $i$ من 1 حتى $n$ التي يكون فيها
$i^2 - i$ عددًا يقسم على 3 بدون باقٍ.
\ifwithsols
\begin{boxSolution}
\begin{english}
\begin{minted}{csharp}
public static int CountEvenFormula(int n)
{
    int cnt = 0;
    for(int i=1; i<=n; i++)
    {
        int v = i*i - i;
        if (v % 3 == 0)
            cnt++;
    }
    return cnt;
}
\end{minted}
\end{english}
\end{boxSolution}
\fi


\item
اكتب برنامجًا يستقبل 25 عددًا صحيحًا ويطبع عدد الأعداد التي كانت أصغر من المتوسط الجزئي حتى تلك اللحظة
(أي: في اللحظة i، نحسب متوسط أوّل i أعداد، ونفحص هل العدد i أصغر من هذا المتوسط).
\ifwithsols
\begin{boxSolution}
\begin{english}
\begin{minted}{csharp}
int sum = 0;
int cnt = 0;
double avg;

for(int i=1; i<=25; i++)
{
    int x = int.Parse(Console.ReadLine());
    sum += x;
    avg = sum / (double)i;
    if (x < avg)
        cnt++;
}
Console.WriteLine(cnt);
\end{minted}
\end{english}
\end{boxSolution}
\clearpage
\fi

\item اكتب عملية تتلقى عددًا صحيحًا $n$، وتحسب قيمة التعبير التالي: \\
$1 - 2 + 3 - 4 + 5 - 6 \dots n$ \\
(لاحظ أن الأعداد الفردية تضاف، والأعداد الزوجية تطرح).
\ifwithsols
\begin{boxSolution}
\begin{english}
\begin{minted}{csharp}
public static int AlternatingSum(int n)
{
    int sum = 0;
    for (int i = 1; i <= n; i++)
    {
        if (i % 2 == 0)
            sum -= i;
        else
            sum += i;
    }
    return sum;
}
\end{minted}
\end{english}
\end{boxSolution}
\clearpage
\fi


\item اكتب برنامجًا يستقبل من المستخدم 20 زوجًا من الأعداد (أي ما مجمله 40 عددًا.) \\
لكل زوج من الأعداد، يحسب البرنامج معدّل العددين ويطبعه. \\
وفي النهاية يطبع العددين اللذين كان لهما أكبر معدّل.
\ifwithsols
\begin{boxSolution}
\begin{english}
\begin{minted}{csharp}
double maxAvg = double.MinValue;
int num1 = 0, num2 = 0;
for (int i = 0; i < 20; i++)
{
    int a = int.Parse(Console.ReadLine());
    int b = int.Parse(Console.ReadLine());
    double avg = (a + b) / 2.0;
    Console.WriteLine(avg);
    if (avg > maxAvg)
    {
        maxAvg = avg;
        num1 = a;
        num2 = b;
    }
}
Console.WriteLine(num1);
Console.WriteLine(num2);
\end{minted}
\end{english}
\end{boxSolution}
\clearpage
\fi

\item اكتب برنامجًا يستقبل من المستخدم 20 عددًا صحيحًا، على البرنامج أن يطبع أكبر عددين أدخلهما المستخدم (أكبر عدد وثاني أكبر عدد)
\ifwithsols
\begin{boxSolution}
\begin{english}
\begin{minted}{csharp}
int max1 = int.MinValue, max2 = int.MinValue;
for (int i = 0; i < 20; i++)
{
    int x = int.Parse(Console.ReadLine());
    if (x > max1)
    {
        max2 = max1;
        max1 = x;
    }
    else if (x > max2)
    {
        max2 = x;
    }
}
Console.WriteLine(max1);
Console.WriteLine(max2);
\end{minted}
\end{english}
\end{boxSolution}
\clearpage
\fi

\item اكتب برنامجًا يستقبل من المستخدم 20 عددًا صحيحًا، على البرنامج أن يطبع أصغر عددين أدخلهما المستخدم (أصغر عدد وثاني أصغر عدد)
\ifwithsols
\begin{boxSolution}
\begin{english}
\begin{minted}{csharp}
int min1 = int.MaxValue, min2 = int.MaxValue;
for (int i = 0; i < 20; i++)
{
    int x = int.Parse(Console.ReadLine());
    if (x < min1)
    {
        min2 = min1;
        min1 = x;
    }
    else if (x < min2)
    {
        min2 = x;
    }
}
Console.WriteLine(min1);
Console.WriteLine(min2);
\end{minted}
\end{english}
\end{boxSolution}
\clearpage
\fi

\item اكتب برنامجًا يستقبل 20 عددًا صحيحًا من المستخدم. على البرنامج أن يطبع عدد المرات التي أدخل فيها المستخدم رقمين متتاليين متساويين. \\
مثال: 5, 3, \textbf{4, 4}, 7, \textbf{9, 9} $\leftarrow$ (هنا العداد يساوي 2).
\ifwithsols
\begin{boxSolution}
\begin{english}
\begin{minted}{csharp}
int count = 0;
Console.WriteLine("Enter 1st number:");
int prev = int.Parse(Console.ReadLine());

for (int i = 1; i < 20; i++)
{
    int curr = int.Parse(Console.ReadLine());
    if (curr == prev)
    {
        count++;
    }
    prev = curr;
}
Console.WriteLine(count);
\end{minted}
\end{english}
\end{boxSolution}
\fi


\item
اكتب برنامجًا يستقبل 30 عددًا صحيحًا ويحسب أكبر فرق بين عددين متتاليين (بالقيمة المطلقة).
\ifwithsols
\begin{boxSolution}
\begin{english}
\begin{minted}{csharp}
int prev = int.Parse(Console.ReadLine());
int maxDiff = 0;

for(int i=1; i<30; i++)
{
    int x = int.Parse(Console.ReadLine());
    int diff = Math.Abs(x - prev);
    if (diff > maxDiff)
        maxDiff = diff;
    prev = x;
}
Console.WriteLine(maxDiff);
\end{minted}
\end{english}
\end{boxSolution}
\fi

\clearpage
\item اكتب برنامجًا يستقبل من المستخدم 10 أعداد صحيحة، على البرنامج أن يطبع عدد الأعداد التي كانت أكبر من الرقم الذي أدخِلَ قبلها مباشرة. \\
\begin{boxExample}
    إذا كانت الأرقام المدخلة هي:
    \begin{boxCode}
    \begin{english}
    \begin{minted}{csharp}
3
4 --> counted because 4 > 3
7 --> counted because 7 > 4
2 --> not counted because 2 < 7
6 --> counted because 6 > 2
7 --> counted because 7 > 6
1 --> not counted because 1 < 7
9 --> counted because 9 > 1
5 --> not counted because 5 < 9
6 --> counted because 6 > 5
    \end{minted}
    \end{english}
    \end{boxCode}
    على البرنامج أن يطبع 6.
\end{boxExample}
\ifwithsols
\begin{boxSolution}
\begin{english}
\begin{minted}{csharp}
int prev = int.Parse(Console.ReadLine());
int count = 0;
for (int i = 1; i < 10; i++)
{
    int current = int.Parse(Console.ReadLine());
    if (current > prev)
        count++;
    prev = current;
}
Console.WriteLine(count);
\end{minted}
\end{english}
\end{boxSolution}
\clearpage
\fi



\end{enumerate}

%------------------------------------------------

\vspace{1cm}
\begin{flushleft}
أرجو لكم وقتًا ممتعًا.

الأستاذ محمود اغبارية.
\end{flushleft}

\end{document}
