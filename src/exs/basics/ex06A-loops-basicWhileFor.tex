\documentclass[14pt]{extarticle}
% Full article preamble (duplicated, no common file)
\usepackage{fontspec}
\usepackage[a4paper,top=2.4cm,bottom=2.4cm,left=2.3cm,right=2.3cm]{geometry}
\usepackage{polyglossia}
\usepackage{amsmath}
\usepackage{amssymb}
\usepackage{xcolor}
\usepackage{fancyhdr}
\usepackage{graphicx}
\usepackage{listings}
\usepackage[most]{tcolorbox}
\usepackage{pifont}
\usepackage{enumitem}
\usepackage{titlesec}
\usepackage[bottom]{footmisc}
\usepackage{titling}
\usepackage{minted}
\usepackage{etoolbox}
\usepackage{array}
\usepackage{extsizes}

\newfontfamily\emoji{Segoe UI Emoji}

\pagestyle{fancy}

\setmainlanguage[numerals=western]{arabic}
\setotherlanguage{english}
\newfontfamily\arabicfont[Script=Arabic]{Amiri}
\newfontfamily\arabicfonttt[Script=Arabic]{Courier New}

\lstset{
  language=[Sharp]C,
  numbers=left,
  stepnumber=1,
  numbersep=8pt,
  frame=single,
  basicstyle=\ttfamily\small,
  keywordstyle=\color{blue},
  stringstyle=\color{red},
  commentstyle=\color{green!50!black}
}

\newif\ifdetailed
\ifdefined\setdetailed
  \setdetailed
\fi

\newif\ifwithsols
\ifdefined\setwithsols
  \setwithsols
\fi

% unified tcolorboxes for articles
\tcbset{colback=white, colframe=black, fonttitle=\bfseries, boxrule=0.8pt}
\newtcolorbox{boxDef}[1][]{colback=blue!5!white,colframe=blue!75!black,
  title={{\emoji📘} تعريف\ifx\\#1\\\else ~#1\fi :}}
\newtcolorbox{boxExercise}[1][]{colback=cyan!5!white,colframe=cyan!70!black,
  title={{\emoji🧩} تمرين\ifx\\#1\\\else ~#1\fi :}}
\newtcolorbox{boxExample}[1][]{colback=yellow!5!white,colframe=orange!90!black,
  title={{\emoji📝} مثال\ifx\\#1\\\else ~#1\fi :}}
\newtcolorbox{boxNote}[1][]{colback=gray!10!white,colframe=black,
  title={{\emoji✨} ملاحظة\ifx\\#1\\\else ~#1\fi :}}
\newtcolorbox{boxAttention}[1][]{colback=magenta!10!white,colframe=magenta!80!black,
  title={{\emoji🔔} تنبيه\ifx\\#1\\\else ~#1\fi :}}
\newtcolorbox{boxWarning}[1][]{colback=red!5!white,colframe=red!75!black,
  title={{\emoji⚡} ملاحظة هامة\ifx\\#1\\\else ~#1\fi :}}
\newtcolorbox{boxSolution}[1][]{colback=green!5!white,colframe=green!60!black,
  title={{\emoji✅} حل\ifx\\#1\\\else ~#1\fi :}}
\newtcolorbox{boxSymbol}[1][]{colback=purple!5!white,colframe=purple!70!black,
  title={{\emoji🔣} رمز\ifx\\#1\\\else ~#1\fi :}}
\newtcolorbox{boxHint}[1][]{colback=teal!5!white,colframe=teal!60!black,
  title={{\emoji💡} تلميح\ifx\\#1\\\else ~#1\fi :}}


\tcbset{simplecode/.style={ colback=gray!5, colframe=black!50, boxrule=0.4pt, arc=2pt, left=4pt,right=4pt,top=4pt,bottom=4pt}}
\newenvironment{boxCode}{\begin{tcolorbox}[simplecode]}{\end{tcolorbox}}

\newcolumntype{C}[1]{>{\centering\arraybackslash}p{#1}}

% redefine spaces after titles
\makeatletter
\renewcommand{\@maketitle}{%
  \begin{center}
    {\huge \bfseries \@title \par}%
    \vskip 0.2em % space between title and author
    {\large \@author \par}%
    % \vskip 0.2em % space between author and date
    % {\normalsize \@date \par}%
  \end{center}
}
\makeatother

\fancyhf{} % clear default
\fancypagestyle{plain}{
  \fancyhf{}
  \fancyhead[L]{مدرسة التسامح الشاملة}
  % \fancyhead[L]{\includegraphics[height=1cm]{../../../images/logoTasamoh.png}}
  \fancyhead[R]{الأستاذ محمود اغبارية}
  \fancyfoot[C]{\thepage}
}

\fancyhead[L]{مدرسة التسامح الشاملة}
\fancyhead[R]{الأستاذ محمود اغبارية}
\fancyfoot[C]{\thepage}
% \date{\today}

\setcounter{tocdepth}{3} % only section subsection and subsubsection in TOC


% ----------------------


% \begin{document}

% \maketitle

% % \clearpage  % start TOC on a new page
% % \renewcommand{\contentsname}{جدول المحتويات}
% % \tableofcontents
% % \clearpage

% \part*{part 1} % the * prevents numbering
% \section*{مقدمة}
% \subsection*{مثال رياضي}
% \subsubsection*{مثال فرعي}
% \paragraph*{ paragraph 1}
% \subparagraph*{sub paragraph 1}

% \ifdetailed
% \begin{english}
% \begin{minted}{csharp}
% // C# Example
% \end{minted}
% \end{english}
% \fi

% OLD WAY
% \ifdetailed
% \begin{english}
% \begin{lstlisting}
% // C# Example
% \end{lstlisting}
% \end{english}
% \fi

% % \includegraphics[width=0.2\textwidth]{../../../images/DFAs/ex1_q1.png}



% \vspace{3cm}
% \begin{flushleft}
% أرجو لكم وقتًا ممتعًا.

% الأستاذ محمود اغبارية.
% \end{flushleft}


% \end{document}


\ifwithsols
\title{حل ورقة تمرن 6 للصف العاشر 10 - قسم 1 \\ أسئلة تأسيسية}
\else
\title{ورقة تمرن 6 للصف العاشر 10 - قسم 1 \\ أسئلة تأسيسية}
\fi

\begin{document}

\maketitle
\thispagestyle{fancy}

\section{أسئلة على حلقة \textenglish{while}}
\begin{enumerate}[itemsep=1.5em]
    \item اطلب من المستخدم إدخال أعداد صحيحة، على البرنامج أن يطبع كل عدد يدخله المستخدم، إذا أدخل المستخدم العدد 0، يتوقف البرنامج.
\ifwithsols
\begin{boxSolution}
\begin{english}
\begin{minted}{csharp}
int n = int.Parse(Console.ReadLine());
while (n != 0)
{
    Console.WriteLine(n);
    n = int.Parse(Console.ReadLine());
}
\end{minted}
\end{english}
\end{boxSolution}
\fi

\item اطلب من المستخدم إدخال رقم بين 1 و 10. إذا أدخل رقمًا غير ذلك، اطبع رسالة خطأ واطلب رقمًا جديدًا حتى يُدخل رقمًا صحيحًا. في النهاية اطبع له عدد المرات التي أدخل فيها عددًا غير صحيح.
\ifwithsols
\begin{boxSolution}
\begin{english}
\begin{minted}{csharp}
int invalid = 0;
int x = int.Parse(Console.ReadLine());
while (x < 1 || x > 10)
{
    Console.WriteLine("Error");
    invalid++;
    x = int.Parse(Console.ReadLine());
}
Console.WriteLine(invalid);
\end{minted}
\end{english}
\end{boxSolution}
\clearpage
\fi

\item اطلب من المستخدم رقمًا صحيحًا يقسم على 7، فإذا أدخل رقمًا لا يقسم على 7، كرر الطلب حتى يدخل رقمًا يقبل القسمة على 7.
\ifwithsols
\begin{boxSolution}
\begin{english}
\begin{minted}{csharp}
int n = int.Parse(Console.ReadLine());
while (n % 7 != 0)
{
    n = int.Parse(Console.ReadLine());
}
Console.WriteLine(n);
\end{minted}
\end{english}
\end{boxSolution}
\fi

\item اقرأ أعدادًا من المستخدم، واجمعها جميعًا، وتوقف فقط عندما تصبح قيمة المجموع أكبر من 100.
\ifwithsols
\begin{boxSolution}
\begin{english}
\begin{minted}{csharp}
int sum = 0;
while (sum <= 100)
{
    int v = int.Parse(Console.ReadLine());
    sum += v;
}
Console.WriteLine(sum);
\end{minted}
\end{english}
\end{boxSolution}
\clearpage
\fi

\item اطلب من المستخدم إدخال أعداد صحيحة حتى يدخل العدد 999. بعد التوقف، اطبع أكبر عدد قام بإدخاله قبل 999.
\ifwithsols
\begin{boxSolution}
\begin{english}
\begin{minted}{csharp}
int max = int.MinValue;
int v = int.Parse(Console.ReadLine());
while (v != 999)
{
    if (v > max)
        max = v;
    v = int.Parse(Console.ReadLine());
}
Console.WriteLine(max);
\end{minted}
\end{english}
\end{boxSolution}
\clearpage
\fi

\item
\begin{enumerate}
\item اكتب عملية خارجية تتلقى عددًا صحيحًا يمثّل كلمة السر. \\
على العملية أن تستقبل من المستخدم كلمة مرور، إذا كانت خاطئة تستمر بالاستقبال حتى يدخل كلمة السر الصحيحة. \\
على العملية أن تعيد عدد المحاولات الفاشلة التي حاولها المستخدم.
\item
في البرنامج الرئيسي: استدع العملية مع كلمة المرور 8264، ثم اطبع (في البرنامج الرئيسي) عدد المحاولات الفاشلة التي حاولها المستخدم.
\item في البرنامج الرئيسي استقبل عددًا صحيحًا، واستدع العملية من البند أ مع كلمة المرور هذه، ثم اطبع عدد المحاولات الفاشلة.
\end{enumerate}
\ifwithsols
\begin{boxSolution}
\begin{english}
\begin{minted}{csharp}
public static int PasswordTries(int pass)
{
    int fails = 0;
    int x = int.Parse(Console.ReadLine());
    while (x != pass)
    {
        fails++;
        x = int.Parse(Console.ReadLine());
    }
    return fails;
}
private static void Main(string[] args)
{
    // b
    Console.WriteLine(PasswordTries(8264));
    // c
    int p = int.Parse(Console.ReadLine());
    Console.WriteLine(PasswordTries(p));
}
\end{minted}
\end{english}
\end{boxSolution}
\fi
\clearpage

    \item أمامك مقطع البرنامج التالي:
    \begin{english}
    \begin{boxCode}
    \begin{minted}{csharp}
while(x != y)
{
     x += 1;
     y -= 2;
}
\end{minted}
    \end{boxCode}
    \end{english}

    \begin{enumerate}
        \item أعط مثالًا لعددين x و  y بحيث تتكرر الحلقة 3 مرات.
        \item أعط مثالًا لعددين x و  y بحيث تتكرر الحلقة مرة واحدة فقط.
        \item أعط مثالًا لعددين x و  y بحيث لا تتكرر الحلقة أبدا.
        \item أعط مثالًا لعددين x و  y بحيث تتكرر الحلقة الى ما لا نهاية من المرّات.
    \end{enumerate}
\ifwithsols
\begin{boxSolution}
    \begin{enumerate}
\item نأخذ: $x=0$, $y=9$. بعد 3 خطوات يصبحان متساويين $(3,3)$ ثم يتوقف.(يمكن أخذ أي عددين بحيث $y-x=9$)

\item نأخذ: $x=0$, $y=3$. بعد خطوة واحدة يصبحان متساويين $(1,1)$ ثم يتوقف.(يمكن أخذ أي عددين بحيث $y-x=3$)

\item أي رقم نأخذه بحيث $x=y$ فإن الحلقة لن تتكرر لأنّ شرطها غير متحقّق من البداية.

\item نأخذ مثلًا: $x=1$ و $y=0$ عندها لن يتساويا أبدًا. \\
هنا يمكننا أخذ أي قيمتين بحيث يكون $x>y$ أو يكون الفرق بينهما ليس من مضاعفات الـ 3.
\end{enumerate}
\end{boxSolution}
\fi

\end{enumerate}


\clearpage
\section{أسئلة على حلقة \textenglish{for}}
\begin{enumerate}[itemsep=1.5em]
    \item اكتب حلقة \textenglish{for} تطبع الأعداد من 1 إلى 20.
\ifwithsols
\begin{boxSolution}
\begin{english}
\begin{minted}{csharp}
for (int i = 1; i <= 20; i++)
    Console.WriteLine(i);
\end{minted}
\end{english}
\end{boxSolution}
\fi

\item اقرأ عددًا من المستخدم $N$، ثم اطبع $N$ أسطر، في كل سطر كلمة \textenglish{Hello}.
\ifwithsols
\begin{boxSolution}
\begin{english}
\begin{minted}{csharp}
int N = int.Parse(Console.ReadLine());
for (int i = 0; i < N; i++)
    Console.WriteLine("Hello");
\end{minted}
\end{english}
\end{boxSolution}
\fi

\item اكتب برنامجًا يطبع الأعداد من 10 إلى 1 تنازليًا.
\ifwithsols
\begin{boxSolution}
\begin{english}
\begin{minted}{csharp}
for (int i = 10; i >= 1; i--)
    Console.WriteLine(i);
\end{minted}
\end{english}
\end{boxSolution}
\clearpage
\fi

    \item اطبع جميع الأعداد الزوجية بين 20 و 50.
\ifwithsols
\begin{boxSolution}[1]
\begin{english}
\begin{minted}{csharp}
for (int i = 20; i <= 50; i++)
{
    if (i % 2 == 0)
        Console.WriteLine(i);
}
\end{minted}
\end{english}
\end{boxSolution}
\begin{boxSolution}[2]
\begin{english}
\begin{minted}{csharp}
for (int i = 20; i <= 50; i+=2)
    Console.WriteLine(i);
\end{minted}
\end{english}
\end{boxSolution}
\fi

\item احسب مجموع الأعداد من 70 إلى 100.
\ifwithsols
\begin{boxSolution}
\begin{english}
\begin{minted}{csharp}
int sum = 0;
for (int i = 70; i <= 100; i++)
    sum += i;

Console.WriteLine(sum);
\end{minted}
\end{english}
\end{boxSolution}
\fi

\item اطبع أول 12 مضاعفًا للعدد 7.
\ifwithsols
\begin{boxSolution}
\begin{english}
\begin{minted}{csharp}
for (int i = 1; i <= 12; i++)
    Console.WriteLine(7 * i);
\end{minted}
\end{english}
\end{boxSolution}
\clearpage
\fi

\item اطبع تربيع الأعداد من 1 إلى 15.
\ifwithsols
\begin{boxSolution}
\begin{english}
\begin{minted}{csharp}
for (int i = 1; i <= 15; i++)
    Console.WriteLine(i * i);
\end{minted}
\end{english}
\end{boxSolution}
\fi

\item اكتب مقطع برنامج يطبع عدد الاعداد ثلاثية المنزلة التي تقسم على 7.
\ifwithsols
\begin{boxSolution}[1]
\begin{english}
\begin{minted}{csharp}
int count = 0;
for (int i = 100; i <= 999; i++)
{
    if (i % 7 == 0)
        count++;
}
Console.WriteLine(count);
\end{minted}
\end{english}
\end{boxSolution}
\fi

\item لكل عدد من 1 إلى 10، اطبع العدد مرفوعًا لقوة نفسه (أي: $1^1, 2^2, 3^3, \ldots, 10^{10}$).
\ifwithsols
\begin{boxSolution}
\begin{english}
\begin{minted}{csharp}
for (int i = 1; i <= 10; i++)
{
    Console.WriteLine(Math.Pow(i, i));
}
\end{minted}
\end{english}
\end{boxSolution}
\clearpage
\fi

\item اكتب برنامجًا يستقبل من المستخدم عددًا صحيحًا $n$، ثم يستقبل منه $n$ أعداد صحيحة ويطبع مجموعها.
\ifwithsols
\begin{boxSolution}
\begin{english}
\begin{minted}{csharp}
int n = int.Parse(Console.ReadLine());
int sum = 0;
for (int i = 0; i < n; i++)
    sum += int.Parse(Console.ReadLine());
Console.WriteLine(sum);
\end{minted}
\end{english}
\end{boxSolution}
\fi

\end{enumerate}

\vspace{1cm}
\begin{flushleft}
أرجو لكم وقتًا ممتعًا.

الأستاذ محمود اغبارية.
\end{flushleft}

\end{document}
