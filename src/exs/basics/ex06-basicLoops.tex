\documentclass[14pt]{extarticle}
% Full article preamble (duplicated, no common file)
\usepackage{fontspec}
\usepackage[a4paper,top=2.4cm,bottom=2.4cm,left=2.3cm,right=2.3cm]{geometry}
\usepackage{polyglossia}
\usepackage{amsmath}
\usepackage{amssymb}
\usepackage{xcolor}
\usepackage{fancyhdr}
\usepackage{graphicx}
\usepackage{listings}
\usepackage[most]{tcolorbox}
\usepackage{pifont}
\usepackage{enumitem}
\usepackage{titlesec}
\usepackage[bottom]{footmisc}
\usepackage{titling}
\usepackage{minted}
\usepackage{etoolbox}
\usepackage{array}
\usepackage{extsizes}

\newfontfamily\emoji{Segoe UI Emoji}

\pagestyle{fancy}

\setmainlanguage[numerals=western]{arabic}
\setotherlanguage{english}
\newfontfamily\arabicfont[Script=Arabic]{Amiri}
\newfontfamily\arabicfonttt[Script=Arabic]{Courier New}

\lstset{
  language=[Sharp]C,
  numbers=left,
  stepnumber=1,
  numbersep=8pt,
  frame=single,
  basicstyle=\ttfamily\small,
  keywordstyle=\color{blue},
  stringstyle=\color{red},
  commentstyle=\color{green!50!black}
}

\newif\ifdetailed
\ifdefined\setdetailed
  \setdetailed
\fi

\newif\ifwithsols
\ifdefined\setwithsols
  \setwithsols
\fi

% unified tcolorboxes for articles
\tcbset{colback=white, colframe=black, fonttitle=\bfseries, boxrule=0.8pt}
\newtcolorbox{boxDef}[1][]{colback=blue!5!white,colframe=blue!75!black,
  title={{\emoji📘} تعريف\ifx\\#1\\\else ~#1\fi :}}
\newtcolorbox{boxExercise}[1][]{colback=cyan!5!white,colframe=cyan!70!black,
  title={{\emoji🧩} تمرين\ifx\\#1\\\else ~#1\fi :}}
\newtcolorbox{boxExample}[1][]{colback=yellow!5!white,colframe=orange!90!black,
  title={{\emoji📝} مثال\ifx\\#1\\\else ~#1\fi :}}
\newtcolorbox{boxNote}[1][]{colback=gray!10!white,colframe=black,
  title={{\emoji✨} ملاحظة\ifx\\#1\\\else ~#1\fi :}}
\newtcolorbox{boxAttention}[1][]{colback=magenta!10!white,colframe=magenta!80!black,
  title={{\emoji🔔} تنبيه\ifx\\#1\\\else ~#1\fi :}}
\newtcolorbox{boxWarning}[1][]{colback=red!5!white,colframe=red!75!black,
  title={{\emoji⚡} ملاحظة هامة\ifx\\#1\\\else ~#1\fi :}}
\newtcolorbox{boxSolution}[1][]{colback=green!5!white,colframe=green!60!black,
  title={{\emoji✅} حل\ifx\\#1\\\else ~#1\fi :}}
\newtcolorbox{boxSymbol}[1][]{colback=purple!5!white,colframe=purple!70!black,
  title={{\emoji🔣} رمز\ifx\\#1\\\else ~#1\fi :}}
\newtcolorbox{boxHint}[1][]{colback=teal!5!white,colframe=teal!60!black,
  title={{\emoji💡} تلميح\ifx\\#1\\\else ~#1\fi :}}


\tcbset{simplecode/.style={ colback=gray!5, colframe=black!50, boxrule=0.4pt, arc=2pt, left=4pt,right=4pt,top=4pt,bottom=4pt}}
\newenvironment{boxCode}{\begin{tcolorbox}[simplecode]}{\end{tcolorbox}}

\newcolumntype{C}[1]{>{\centering\arraybackslash}p{#1}}

% redefine spaces after titles
\makeatletter
\renewcommand{\@maketitle}{%
  \begin{center}
    {\huge \bfseries \@title \par}%
    \vskip 0.2em % space between title and author
    {\large \@author \par}%
    % \vskip 0.2em % space between author and date
    % {\normalsize \@date \par}%
  \end{center}
}
\makeatother

\fancyhf{} % clear default
\fancypagestyle{plain}{
  \fancyhf{}
  \fancyhead[L]{مدرسة التسامح الشاملة}
  % \fancyhead[L]{\includegraphics[height=1cm]{../../../images/logoTasamoh.png}}
  \fancyhead[R]{الأستاذ محمود اغبارية}
  \fancyfoot[C]{\thepage}
}

\fancyhead[L]{مدرسة التسامح الشاملة}
\fancyhead[R]{الأستاذ محمود اغبارية}
\fancyfoot[C]{\thepage}
% \date{\today}

\setcounter{tocdepth}{3} % only section subsection and subsubsection in TOC


% ----------------------


% \begin{document}

% \maketitle

% % \clearpage  % start TOC on a new page
% % \renewcommand{\contentsname}{جدول المحتويات}
% % \tableofcontents
% % \clearpage

% \part*{part 1} % the * prevents numbering
% \section*{مقدمة}
% \subsection*{مثال رياضي}
% \subsubsection*{مثال فرعي}
% \paragraph*{ paragraph 1}
% \subparagraph*{sub paragraph 1}

% \ifdetailed
% \begin{english}
% \begin{minted}{csharp}
% // C# Example
% \end{minted}
% \end{english}
% \fi

% OLD WAY
% \ifdetailed
% \begin{english}
% \begin{lstlisting}
% // C# Example
% \end{lstlisting}
% \end{english}
% \fi

% % \includegraphics[width=0.2\textwidth]{../../../images/DFAs/ex1_q1.png}



% \vspace{3cm}
% \begin{flushleft}
% أرجو لكم وقتًا ممتعًا.

% الأستاذ محمود اغبارية.
% \end{flushleft}


% \end{document}


\title{ورقة تمرن 6 للصف العاشر 10 - الحلقات}

\begin{document}

\maketitle
\thispagestyle{fancy}

\section{أسئلة على حلقة \textenglish{while}}
\begin{enumerate}[itemsep=1.5em]
    \item اطلب من المستخدم إدخال أعداد صحيحة، على البرنامج أن يطبع كل عدد يدخله المستخدم، إذا أدخل المستخدم العدد 0، يتوقف البرنامج.
    \item اطلب من المستخدم إدخال رقم بين 1 و 10. إذا أدخل رقمًا غير ذلك، اطبع رسالة خطأ واطلب رقمًا جديدًا حتى يُدخل رقمًا صحيحًا. في النهاية اطبع له عدد المرات التي أدخل فيها عددًا غير صحيح.
    \item اطلب من المستخدم رقمًا صحيحًا يقسم على 7، فإذا أدخل رقمًا لا يقسم على 7، كرر الطلب حتى يدخل رقمًا يقبل القسمة على 7.
    \item اقرأ أعدادًا من المستخدم، واجمعها جميعًا، وتوقف فقط عندما تصبح قيمة المجموع أكبر من 100.
    \item اطلب من المستخدم إدخال أعداد صحيحة حتى يدخل العدد 999. بعد التوقف، اطبع أكبر عدد قام بإدخاله قبل 999.
    \item
    \begin{enumerate}
      \item اكتب عملية خارجية تتلقى عددًا صحيحًا يمثّل كلمة السر. \\
    على العملية أن تستقبل من المستخدم كلمة مرور، إذا كانت خاطئة تستمر بالاستقبال حتى يدخل كلمة السر الصحيحة. \\
    على العملية أن تعيد عدد المحاولات الفاشلة التي حاولها المستخدم.
    \item
    في البرنامج الرئيسي: استدع العملية مع كلمة المرور 8264، ثم اطبع (في البرنامج الرئيسي) عدد المحاولات الفاشلة التي حاولها المستخدم.
    \item في البرنامج الرئيسي استقبل عددًا صحيحًا، واستدع العملية من البند أ مع كلمة المرور هذه، ثم اطبع عدد المحاولات الفاشلة.
\end{enumerate}
    \item أمامك مقطع البرنامج التالي:
    \begin{english}
    \begin{boxCode}
    \begin{minted}{csharp}
while(x != y)
{
     x += 1;
     y -= 2;
}
\end{minted}
    \end{boxCode}
    \end{english}

    \begin{enumerate}
        \item اعط مثالًا لعددين x و  y بحيث تتكرر الحلقة 3 مرات.
        \item اعط مثالًا لعددين x و  y بحيث تتكرر الحلقة مرة واحدة فقط.
        \item اعط مثالًا لعددين x و  y بحيث لا تتكرر الحلقة أبدا.
        \item اعط مثالًا لعددين x و  y بحيث تتكرر الحلقة الى ما لا نهاية من المرّات.
    \end{enumerate} 

\end{enumerate}

\clearpage
\section{أسئلة على حلقة \textenglish{for}}
\begin{enumerate}[itemsep=1.5em]
    \item اكتب حلقة \textenglish{for} تطبع الأعداد من 1 إلى 20.
    \item اقرأ عددًا من المستخدم $N$، ثم اطبع $N$ أسطر، في كل سطر كلمة \textenglish{Hello}.
    \item اكتب برنامجًا يطبع الأعداد من 10 إلى 1 تنازليًا.
    \item اطبع جميع الأعداد الزوجية بين 20 و 50.
    \item احسب مجموع الأعداد من 70 إلى 100.
    \item اطبع أول 12 مضاعفًا للعدد 7.
    \item اطبع تربيع الأعداد من 1 إلى 15.
    \item اكتب مقطع برنامج يطبع عدد الاعداد ثلاثية المنزلة التي تقسم على 7.
    \item لكل عدد من 1 إلى 10، اطبع العدد مرفوعًا لقوة نفسه (أي: $1^1, 2^2, 3^3, \ldots, 10^{10}$).
    \item اكتب برنامجًا يستقبل من المستخدم عددًا صحيحًا $n$، ثم يستقبل منه $n$ أعداد صحيحة ويطبع معدّلها.
    \item اكتب برنامجًا يستقبل من المستخدم عددًا صحيحا، ويطبع له كل عوامل هذا العدد (أي كل الأعداد التي يقسم عليها هذا العدد).
    \item اطبع أول 20 عددًا من المتتالية: $3, 6, 9, 12,\dots$
\end{enumerate}


\clearpage
\section{أسئلة حلقات عامة}
\begin{enumerate}[itemsep=1.5em]
    \item
    \begin{enumerate}
        \item اكتب عمليّة تتلقى عددين صحيحين، على العمليّة أن تعيد عدد الاعداد التي تقسم على 3 بينهما.
        \item في البرنامج الرئيسي، قم باستدعاء العملية مع العددين 20 و70, واطبع النتيجة.
    \end{enumerate}

    \item
    في إحدى مسابقات العاب القوى هناك مسابقة "العرض الشيق" بحيث يقوم كل متسابق بعرض مواهبه الرياضية في القفز والحركة. \\
     في هذه اللعبة هناك 7 حكام، كل حكم علامة للمتسابق بعد عرضه. \\
     العلامة النهائية التي يستحقها اللاعب تكون بحساب معدل 5 علامات فقط من بين 7 وذلك بعد شطب العلامة الكبرى والعلامة الصغرى. \\
      اكتب مقطع برنامج يستقبل علامات الحكام السبعة لأحد المتسابقين، على البرنامج أن يطبع معدله النهائي.

    \item اكتب عملية خارجية تتلقى عددًا صحيحًا، على العملية أن تطبع الأعداد من 1 إلى الرقم الذي تلقته وبجانب كل رقم تربيعه. \\
    مثلًا، إذا  تلقت العملية الرقم 5، فعليها أن تطبع:
    \begin{english}
    \begin{boxCode}
      1 1 \\
      2 4 \\
      3 9 \\
      4 16 \\
      5 25
    \end{boxCode}
    \end{english}

    \item اكتب عملية خارجية تتلقى عددًا وتطبع: عدد خاناته، ومجموع خاناته، ومعدّل خاناته. \\
                 \textbf{مثلًا:} إذا تقلت العملية العدد 762945، فإنّها تطبع: \\

    \begin{english}
    \begin{boxCode}
      Digits count = 6 \\
      Digits sum = 33 (5+4+9+2+6+7=33) \\
      Digits average = 5.5 (33 / 6 = 5.5)
    \end{boxCode}
    \end{english}

\clearpage

    \item \textbf{للتذكير} عدد أوّلي هو عدد يقسم فقط على نفسه وعلى 1.
    \begin{enumerate}
        \item اكتب عمليّة خارجية باسم \textenglish{IsPrime} تتلقى عددًا صحيحًا وتعيد \textenglish{true} إذا كان هذا العدد أوليًّا، وإلا فإنّها تعيد \textenglish{false}.
        \item في البرنامج الرئيسي، استدعِ العملية مع العدد 17 واطبع النتيجة. ومع العدد 20 واطبع النتيجة.
        \item اكتب عملية خارجية أخرى باسم \textenglish{PrintPrimes} تتلقى عددًا صحيحًا وتطبع كل الأعداد الأولية حتى هذا العدد. \\
                 \textbf{مثلًا:} إذا تقلت العملية العدد 20 فإنّها تطبع: 2، 3، 5، 7، 11، 13، 17، 19.
        \item في البرنامج الرئيسي استدعِ العملية مع الرقم 100.
    \end{enumerate}

    \item اكتب عمليّة خارجيّة تتلقى عددًا صحيحًا $x$ الذي يمثل الحد الأقصى للوزن الإجمالي الذي يمكن أن تحمله شاحنة ما بالكيلوغرام، على العملية أن تستقبل وزن كل صندوق يتم تحميله للشاحنة، على العملية أن تعيد عدد الصناديق التي تم تحميلها للشاحنة حتى امتلأت. (فكّر ما هو شرط الحلقة؟)

\clearpage

    \item متوالية فيبوناتشي معرفة بالشكل التالي:
    \begin{english}
        \begin{align*}
            a_1 &= 1 \\
            a_2 &= 1 \\
            a_n &= a_{n-1} + a_{n-2} \text{ , for } n > 2
        \end{align*}
    \end{english}
    مثلًا، أول 7 حدود من متوالية فيبوناتشي هي: $1, 1, 2, 3, 5, 8, 13$. \\
    \begin{enumerate}
      \item اكتب عملية خارجية تتلقى عددًا صحيحًا: $n$ وتعيد الحد الـ $n$-ي من متوالية فيبوناتشي.
      مثلًا: إذا تلقت العدد 3، فإنّها تعيد 2.

      \item اكتب عملية تتلقى عددًا صحيحًا $n$، وتطبع أول $n$ حدود من متوالية فيبوناتشي. \\
      عليك استخدام العملية من البند السابق. \\
      مثلًا، إذا تلقت العملية الرقم 8، فعليها أن يطبع أول 8 حدود من المتوالية، أي:
    \begin{english}
    \begin{boxCode}
    1 \\
    1 \\
    2 \\
    3 \\
    5 \\
    8 \\
    13 \\
    21
    \end{boxCode}
    \end{english}

      \item اكتب برنامجًا رئيسيًّا يستقبل عددًا صحيحًا، ويستدعي العملية من البند ب مع الرقم الذي أدخله المستخدم.
      \end{enumerate}

\end{enumerate}

\vspace{1cm}
\begin{flushleft}
أرجو لكم وقتًا ممتعًا.

الأستاذ محمود اغبارية.
\end{flushleft}

\end{document}



