\documentclass[14pt]{extarticle}
% Full article preamble (duplicated, no common file)
\usepackage{fontspec}
\usepackage[a4paper,top=2.4cm,bottom=2.4cm,left=2.3cm,right=2.3cm]{geometry}
\usepackage{polyglossia}
\usepackage{amsmath}
\usepackage{amssymb}
\usepackage{xcolor}
\usepackage{fancyhdr}
\usepackage{graphicx}
\usepackage{listings}
\usepackage[most]{tcolorbox}
\usepackage{pifont}
\usepackage{enumitem}
\usepackage{titlesec}
\usepackage[bottom]{footmisc}
\usepackage{titling}
\usepackage{minted}
\usepackage{etoolbox}
\usepackage{array}
\usepackage{extsizes}

\newfontfamily\emoji{Segoe UI Emoji}

\pagestyle{fancy}

\setmainlanguage[numerals=western]{arabic}
\setotherlanguage{english}
\newfontfamily\arabicfont[Script=Arabic]{Amiri}
\newfontfamily\arabicfonttt[Script=Arabic]{Courier New}

\lstset{
  language=[Sharp]C,
  numbers=left,
  stepnumber=1,
  numbersep=8pt,
  frame=single,
  basicstyle=\ttfamily\small,
  keywordstyle=\color{blue},
  stringstyle=\color{red},
  commentstyle=\color{green!50!black}
}

\newif\ifdetailed
\ifdefined\setdetailed
  \setdetailed
\fi

\newif\ifwithsols
\ifdefined\setwithsols
  \setwithsols
\fi

% unified tcolorboxes for articles
\tcbset{colback=white, colframe=black, fonttitle=\bfseries, boxrule=0.8pt}
\newtcolorbox{boxDef}[1][]{colback=blue!5!white,colframe=blue!75!black,
  title={{\emoji📘} تعريف\ifx\\#1\\\else ~#1\fi :}}
\newtcolorbox{boxExercise}[1][]{colback=cyan!5!white,colframe=cyan!70!black,
  title={{\emoji🧩} تمرين\ifx\\#1\\\else ~#1\fi :}}
\newtcolorbox{boxExample}[1][]{colback=yellow!5!white,colframe=orange!90!black,
  title={{\emoji📝} مثال\ifx\\#1\\\else ~#1\fi :}}
\newtcolorbox{boxNote}[1][]{colback=gray!10!white,colframe=black,
  title={{\emoji✨} ملاحظة\ifx\\#1\\\else ~#1\fi :}}
\newtcolorbox{boxAttention}[1][]{colback=magenta!10!white,colframe=magenta!80!black,
  title={{\emoji🔔} تنبيه\ifx\\#1\\\else ~#1\fi :}}
\newtcolorbox{boxWarning}[1][]{colback=red!5!white,colframe=red!75!black,
  title={{\emoji⚡} ملاحظة هامة\ifx\\#1\\\else ~#1\fi :}}
\newtcolorbox{boxSolution}[1][]{colback=green!5!white,colframe=green!60!black,
  title={{\emoji✅} حل\ifx\\#1\\\else ~#1\fi :}}
\newtcolorbox{boxSymbol}[1][]{colback=purple!5!white,colframe=purple!70!black,
  title={{\emoji🔣} رمز\ifx\\#1\\\else ~#1\fi :}}
\newtcolorbox{boxHint}[1][]{colback=teal!5!white,colframe=teal!60!black,
  title={{\emoji💡} تلميح\ifx\\#1\\\else ~#1\fi :}}


\tcbset{simplecode/.style={ colback=gray!5, colframe=black!50, boxrule=0.4pt, arc=2pt, left=4pt,right=4pt,top=4pt,bottom=4pt}}
\newenvironment{boxCode}{\begin{tcolorbox}[simplecode]}{\end{tcolorbox}}

\newcolumntype{C}[1]{>{\centering\arraybackslash}p{#1}}

% redefine spaces after titles
\makeatletter
\renewcommand{\@maketitle}{%
  \begin{center}
    {\huge \bfseries \@title \par}%
    \vskip 0.2em % space between title and author
    {\large \@author \par}%
    % \vskip 0.2em % space between author and date
    % {\normalsize \@date \par}%
  \end{center}
}
\makeatother

\fancyhf{} % clear default
\fancypagestyle{plain}{
  \fancyhf{}
  \fancyhead[L]{مدرسة التسامح الشاملة}
  % \fancyhead[L]{\includegraphics[height=1cm]{../../../images/logoTasamoh.png}}
  \fancyhead[R]{الأستاذ محمود اغبارية}
  \fancyfoot[C]{\thepage}
}

\fancyhead[L]{مدرسة التسامح الشاملة}
\fancyhead[R]{الأستاذ محمود اغبارية}
\fancyfoot[C]{\thepage}
% \date{\today}

\setcounter{tocdepth}{3} % only section subsection and subsubsection in TOC


% ----------------------


% \begin{document}

% \maketitle

% % \clearpage  % start TOC on a new page
% % \renewcommand{\contentsname}{جدول المحتويات}
% % \tableofcontents
% % \clearpage

% \part*{part 1} % the * prevents numbering
% \section*{مقدمة}
% \subsection*{مثال رياضي}
% \subsubsection*{مثال فرعي}
% \paragraph*{ paragraph 1}
% \subparagraph*{sub paragraph 1}

% \ifdetailed
% \begin{english}
% \begin{minted}{csharp}
% // C# Example
% \end{minted}
% \end{english}
% \fi

% OLD WAY
% \ifdetailed
% \begin{english}
% \begin{lstlisting}
% // C# Example
% \end{lstlisting}
% \end{english}
% \fi

% % \includegraphics[width=0.2\textwidth]{../../../images/DFAs/ex1_q1.png}



% \vspace{3cm}
% \begin{flushleft}
% أرجو لكم وقتًا ممتعًا.

% الأستاذ محمود اغبارية.
% \end{flushleft}


% \end{document}


\title{ورقة تمرن 6 للصف العاشر 10 - الحلقات}

\begin{document}

\maketitle
\thispagestyle{fancy}

\section{أسئلة على حلقة \textenglish{while}}
\begin{enumerate}[itemsep=2em]
    \item اطلب من المستخدم إدخال أعداد صحيحة، واستمر بطباعتها حتى يدخل العدد 0، ثم توقف.
    \item اطلب من المستخدم إدخال رقم بين 1 و 10. إذا أدخل رقمًا غير ذلك، اطبع رسالة خطأ واطلب رقمًا جديدًا حتى يُدخل رقمًا صحيحًا.
    \item اقرأ أعدادًا من المستخدم، واجمعها جميعًا، وتوقف فقط عندما تصبح قيمة المجموع أكبر من 100.
    \item نعرف كلمة سر = $1234$، اطلب من المستخدم إدخال كلمة المرور، إذا كانت خاطئة اطبع له رسالة مناسبة واستمر بالاستقبال حتى يدخل الكلمة الصحيحة. \\
    عندما يُدخل الكلمة الصحيحة اطبع له رسالة مناسبة وينتهي البرنامج.
    \item اطلب من المستخدم إدخال أعداد صحيحة حتى يدخل العدد 999. بعد التوقف، اطبع أكبر عدد قام بإدخاله قبل 999.
\end{enumerate}

\clearpage
\section{أسئلة على حلقة \textenglish{for}}
\begin{enumerate}[itemsep=2em]
    \item اكتب حلقة \textenglish{for} تطبع الأعداد من 1 إلى 20.
    \item اقرأ عددًا من المستخدم $N$، ثم اطبع $N$ أسطر، في كل سطر كلمة \textenglish{Hello}.
    \item اكتب برنامجًا يطبع الأعداد من 10 إلى 1 تنازليًا.
    \item اطبع جميع الأعداد الزوجية بين 1 و 50.
    \item احسب مجموع الأعداد من 1 إلى 100.
    \item لكل عدد من 1 إلى 10، اطبع العدد مرفوعًا لقوة نفسه (أي: $1^1, 2^2, 3^3, \ldots, 10^{10}$).
\end{enumerate}

\clearpage
\section{اختر الحلقة المناسبة}
\begin{enumerate}[itemsep=2em]
    \item اطبع أول 12 مضاعفًا للعدد 7.
    \item اطبع تربيع الأعداد من 1 إلى 15.
    \item اطلب من المستخدم إدخال رقم أكبر من 100 إن أدخل أقل من ذلك اطلبه من جديد.
    \item اقرأ 20 عددًا من المستخدم واطبع مجموعها.
    \item اطلب من المستخدم رقمًا صحيحًا يقسم على 7، فإذا أدخل رقمًا لا يقسم على 7، كرر الطلب حتى يدخل رقمًا يقبل القسمة على 7.
    \item اطبع أول 20 عددًا من المتتالية: $3, 6, 9, 12,\dots$
    \item اطبع ناتج ضرب $8 \times i$ لكل $i$ من 1 إلى 12.
    \item
    في إحدى مسابقات العاب القوى هناك مسابقة "العرض الشيق" بحيث يقوم كل متسابق بعرض مواهبه الرياضية في القفز والحركة. \\
     في هذه اللعبة هناك 7 حكام، كل حكم علامة للمتسابق بعد عرضه. \\
     العلامة النهائية التي يستحقها اللاعب تكون بحساب معدل 5 علامات فقط من بين 7 وذلك بعد شطب العلامة الكبرى والعلامة الصغرى. \\
      اكتب مقطع برنامج يستقبل علامات الحكام السبعة لأحد المتسابقين، على البرنامج أن يطبع معدله النهائي.

\end{enumerate}

\clearpage
\section{عمليات خارجية مع حلقات}
\begin{enumerate}[itemsep=2em]
    \item
    \begin{enumerate}
        \item اكتب عمليّة تتلقى عددين صحيحين، على العمليّة أن تعيد عدد الاعداد التي تقسم على 3 بينهما.
\item في البرنامج الرئيسي، قم باستدعاء العملية مع العددين 20 و70, واطبع النتيجة.

    \end{enumerate}
    \item
    \begin{enumerate}
      \item اكتب عملية خارجية تتلقى عددًا صحيحًا يمثّل كلمة السر. \\
    على العملية أن تستقبل من المستخدم كلمة مرور، إذا كانت خاطئة تستمر بالاستقبال حتى يدخل كلمة السر الصحيحة. \\
    على العملية أن تعيد عدد المحاولات الفاشلة التي حاولها المستخدم.
    \item
    في البرنامج الرئيسي: استدع العملية مع كلمة المرور 8264، ثم اطبع (في البرنامج الرئيسي) عدد المحاولات الفاشلة التي حاولها المستخدم.
    \item في البرنامج الرئيسي استقبل عددًا صحيحًا، واستدع العملية من البند أ مع كلمة المرور هذه، ثم اطبع عدد المحاولات الفاشلة.
    \end{enumerate}

    \item اكتب عملية خارجية تتلقى عددًا صحيحًا، على العملية أن تطبع الأعداد من 1 إلى الرقم الذي تلقته وبجانب كل رقم تربيعه. \\
    مثلًا، إذا أدخل تلقت العملية الرقم 5، فعليها أن تطبع:
    \begin{english}
    \begin{boxCode}
      1 1 \\
      2 4 \\
      3 9 \\
      4 16 \\
      5 25
    \end{boxCode}
    \end{english}

    \clearpage
    \item متوالية فيبوناتشي معرفة بالشكل التالي:
    \begin{english}
        \begin{align*}
            a_1 &= 1 \\
            a_2 &= 1 \\
            a_n &= a_{n-1} + a_{n-2} \text{ , for } n > 2
        \end{align*}
    \end{english}
    مثلًا، أول 7 حدود من متوالية فيبوناتشي هي: $1, 1, 2, 3, 5, 8, 13$. \\
    \begin{enumerate}
      \item اكتب عملية خارجية تتلقى عددًا صحيحًا: $n$ وتعيد الحد الـ $n$-ي من متوالية فيبوناتشي.
      مثلًا: إذا تلقت العدد 3، فإنّها تعيد 2.

      \item اكتب عملية تتلقى عددًا صحيحًا $n$، وتطبع أول $n$ حدود من متوالية فيبوناتشي. \\
      عليك استخدام العملية من البند السابق. \\
      مثلًا، إذا تلقت العملية الرقم 8، فعليها أن يطبع أول 8 حدود من المتوالية، أي:
    \begin{english}
    \begin{boxCode}
    1 \\
    1 \\
    2 \\
    3 \\
    5 \\
    8 \\
    13 \\
    21
    \end{boxCode}
    \end{english}

      \item اكتب برنامجًا رئيسيًّا يستقبل عددًا صحيحًا، ويستدعي العملية من البند ب مع الرقم الذي أدخله المستخدم.
      \end{enumerate}

\end{enumerate}

% \clearpage
% \section{حلّ كلّ سؤال من الأسئلة التالية مرتين: مرة بـ while ومرة بـ for}
% \begin{enumerate}
%     \item اطبع جميع الأعداد من 1 إلى العدد الذي يدخله المستخدم.
%     \item اطبع أول 10 حدود من متتالية حسابية: البداية $a$، الفارق $d$ (يقرأ من المستخدم).
%     \item اقرأ أعدادًا من المستخدم واطبع مجموعها حتى يصل المجموع إلى 150.
%     \item اطبع جميع الأعداد الزوجية بين 1 و $N$ حيث $N$ يقرأ من المستخدم.
%     \item اقرأ عددًا $N$ واطبع $N$ أسطر، في كل سطر $N$ نجوم.
% \end{enumerate}


\vspace{1cm}
\begin{flushleft}
أرجو لكم وقتًا ممتعًا.

الأستاذ محمود اغبارية.
\end{flushleft}

\end{document}



