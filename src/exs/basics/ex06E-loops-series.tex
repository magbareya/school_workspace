\documentclass[14pt]{extarticle}
% Full article preamble (duplicated, no common file)
\usepackage{fontspec}
\usepackage[a4paper,top=2.4cm,bottom=2.4cm,left=2.3cm,right=2.3cm]{geometry}
\usepackage{polyglossia}
\usepackage{amsmath}
\usepackage{amssymb}
\usepackage{xcolor}
\usepackage{fancyhdr}
\usepackage{graphicx}
\usepackage{listings}
\usepackage[most]{tcolorbox}
\usepackage{pifont}
\usepackage{enumitem}
\usepackage{titlesec}
\usepackage[bottom]{footmisc}
\usepackage{titling}
\usepackage{minted}
\usepackage{etoolbox}
\usepackage{array}
\usepackage{extsizes}

\newfontfamily\emoji{Segoe UI Emoji}

\pagestyle{fancy}

\setmainlanguage[numerals=western]{arabic}
\setotherlanguage{english}
\newfontfamily\arabicfont[Script=Arabic]{Amiri}
\newfontfamily\arabicfonttt[Script=Arabic]{Courier New}

\lstset{
  language=[Sharp]C,
  numbers=left,
  stepnumber=1,
  numbersep=8pt,
  frame=single,
  basicstyle=\ttfamily\small,
  keywordstyle=\color{blue},
  stringstyle=\color{red},
  commentstyle=\color{green!50!black}
}

\newif\ifdetailed
\ifdefined\setdetailed
  \setdetailed
\fi

\newif\ifwithsols
\ifdefined\setwithsols
  \setwithsols
\fi

% unified tcolorboxes for articles
\tcbset{colback=white, colframe=black, fonttitle=\bfseries, boxrule=0.8pt}
\newtcolorbox{boxDef}[1][]{colback=blue!5!white,colframe=blue!75!black,
  title={{\emoji📘} تعريف\ifx\\#1\\\else ~#1\fi :}}
\newtcolorbox{boxExercise}[1][]{colback=cyan!5!white,colframe=cyan!70!black,
  title={{\emoji🧩} تمرين\ifx\\#1\\\else ~#1\fi :}}
\newtcolorbox{boxExample}[1][]{colback=yellow!5!white,colframe=orange!90!black,
  title={{\emoji📝} مثال\ifx\\#1\\\else ~#1\fi :}}
\newtcolorbox{boxNote}[1][]{colback=gray!10!white,colframe=black,
  title={{\emoji✨} ملاحظة\ifx\\#1\\\else ~#1\fi :}}
\newtcolorbox{boxAttention}[1][]{colback=magenta!10!white,colframe=magenta!80!black,
  title={{\emoji🔔} تنبيه\ifx\\#1\\\else ~#1\fi :}}
\newtcolorbox{boxWarning}[1][]{colback=red!5!white,colframe=red!75!black,
  title={{\emoji⚡} ملاحظة هامة\ifx\\#1\\\else ~#1\fi :}}
\newtcolorbox{boxSolution}[1][]{colback=green!5!white,colframe=green!60!black,
  title={{\emoji✅} حل\ifx\\#1\\\else ~#1\fi :}}
\newtcolorbox{boxSymbol}[1][]{colback=purple!5!white,colframe=purple!70!black,
  title={{\emoji🔣} رمز\ifx\\#1\\\else ~#1\fi :}}
\newtcolorbox{boxHint}[1][]{colback=teal!5!white,colframe=teal!60!black,
  title={{\emoji💡} تلميح\ifx\\#1\\\else ~#1\fi :}}


\tcbset{simplecode/.style={ colback=gray!5, colframe=black!50, boxrule=0.4pt, arc=2pt, left=4pt,right=4pt,top=4pt,bottom=4pt}}
\newenvironment{boxCode}{\begin{tcolorbox}[simplecode]}{\end{tcolorbox}}

\newcolumntype{C}[1]{>{\centering\arraybackslash}p{#1}}

% redefine spaces after titles
\makeatletter
\renewcommand{\@maketitle}{%
  \begin{center}
    {\huge \bfseries \@title \par}%
    \vskip 0.2em % space between title and author
    {\large \@author \par}%
    % \vskip 0.2em % space between author and date
    % {\normalsize \@date \par}%
  \end{center}
}
\makeatother

\fancyhf{} % clear default
\fancypagestyle{plain}{
  \fancyhf{}
  \fancyhead[L]{مدرسة التسامح الشاملة}
  % \fancyhead[L]{\includegraphics[height=1cm]{../../../images/logoTasamoh.png}}
  \fancyhead[R]{الأستاذ محمود اغبارية}
  \fancyfoot[C]{\thepage}
}

\fancyhead[L]{مدرسة التسامح الشاملة}
\fancyhead[R]{الأستاذ محمود اغبارية}
\fancyfoot[C]{\thepage}
% \date{\today}

\setcounter{tocdepth}{3} % only section subsection and subsubsection in TOC


% ----------------------


% \begin{document}

% \maketitle

% % \clearpage  % start TOC on a new page
% % \renewcommand{\contentsname}{جدول المحتويات}
% % \tableofcontents
% % \clearpage

% \part*{part 1} % the * prevents numbering
% \section*{مقدمة}
% \subsection*{مثال رياضي}
% \subsubsection*{مثال فرعي}
% \paragraph*{ paragraph 1}
% \subparagraph*{sub paragraph 1}

% \ifdetailed
% \begin{english}
% \begin{minted}{csharp}
% // C# Example
% \end{minted}
% \end{english}
% \fi

% OLD WAY
% \ifdetailed
% \begin{english}
% \begin{lstlisting}
% // C# Example
% \end{lstlisting}
% \end{english}
% \fi

% % \includegraphics[width=0.2\textwidth]{../../../images/DFAs/ex1_q1.png}



% \vspace{3cm}
% \begin{flushleft}
% أرجو لكم وقتًا ممتعًا.

% الأستاذ محمود اغبارية.
% \end{flushleft}


% \end{document}


\ifwithsols
\title{حل ورقة تمرن 6 للصف العاشر 10 - قسم 5 \\ أسئلة متواليات}
\else
\title{ورقة تمرن 6 للصف العاشر 10 - قسم 5 \\ أسئلة متواليات}
\fi

\begin{document}

\maketitle
\thispagestyle{fancy}

\begin{enumerate}[itemsep=1.5em]

\item اطبع أول 20 عددًا من المتوالية: $3, 6, 9, 12,\dots$
\ifwithsols
\begin{boxSolution}
\begin{english}
\begin{minted}{csharp}
for (int i = 1; i <= 20; i++)
    Console.WriteLine(3 * i);
\end{minted}
\end{english}
\end{boxSolution}
\fi


\item
اكتب عملية تتلقى عددًا صحيحًا $n$، وتقوم بحساب وإرجاع مجموع السلسلة التالية: \\
$Sum = 1 + \frac{1}{2} + \frac{1}{3} + \dots + \frac{1}{n}$
\begin{boxExample}
\begin{itemize}
\item
إذا تلقّت 1 فإنّها تعيد 1.
\item
إذا تلقّت 2 فإنّها تعيد: $1 + \frac{1}{2} = 1.5$
\item
إذا تلقّت 3 فإنّها تعيد: $1 + \frac{1}{2} +  \frac{1}{3} = 1.833$
\item
إذا تلقّت 10 فإنّها تعيد: $1 + \frac{1}{2} +  \frac{1}{3} + \ldots + \frac{1}{10} = 2.9289$
\end{itemize}
\end{boxExample}
\ifwithsols
\begin{boxSolution}
\begin{english}
\begin{minted}{csharp}
public static double SeriesSum(int n)
{
    double sum = 0;
    for (int i = 1; i <= n; i++)
    {
        sum += 1.0 / i;
    }
    return sum;
}
\end{minted}
\end{english}
\end{boxSolution}
\clearpage
\fi


\item متوالية فيبوناتشي معرفة بالشكل التالي:
    \begin{english}
        \begin{align*}
            a_1 &= 1 \\
            a_2 &= 1 \\
            a_n &= a_{n-1} + a_{n-2} \text{ , for } n > 2
        \end{align*}
    \end{english}
    مثلًا، أول 7 حدود من متوالية فيبوناتشي هي: $1, 1, 2, 3, 5, 8, 13$. \\
    \begin{enumerate}
      \item اكتب عملية خارجية تتلقى عددًا صحيحًا: $n$ وتعيد الحد الـ $n$-ي من متوالية فيبوناتشي.
      مثلًا: إذا تلقت العدد 3، فإنّها تعيد 2.

      \item اكتب عملية تتلقى عددًا صحيحًا $n$، وتطبع أول $n$ حدود من متوالية فيبوناتشي. \\
      عليك استخدام العملية من البند السابق. \\
      مثلًا، إذا تلقت العملية الرقم 8، فعليها أن يطبع أول 8 حدود من المتوالية، أي:
    \begin{english}
    \begin{boxCode}
    1 \\
    1 \\
    2 \\
    3 \\
    5 \\
    8 \\
    13 \\
    21
    \end{boxCode}
    \end{english}

      \item اكتب برنامجًا رئيسيًّا يستقبل عددًا صحيحًا، ويستدعي العملية من البند ب مع الرقم الذي أدخله المستخدم.
      \end{enumerate}
\ifwithsols
\begin{boxSolution}
\begin{english}
\begin{minted}{csharp}
public static int Fib(int n)
{
    if (n <= 2)
        return 1;
    int a = 1, b = 1;
    for (int i = 3; i <= n; i++)
    {
        int c = a + b;
        a = b;
        b = c;
    }
    return b;
}
public static void PrintFirstNFibs(int n)
{
    for (int i = 1; i <= n; i++)
        Console.WriteLine(Fib(i));
}

private static void Main(string[] args)
{
    int n = int.Parse(Console.ReadLine());
    PrintFirstNFibs(n);
}
\end{minted}
\end{english}
\end{boxSolution}
\fi

\item
اكتب عملية تتلقى عددًا $n$ وتطبع أوّل $n$ حدود من سلسلة الأعداد:
$$1,\; 2,\; 4,\; 7,\; 11,\; 16,\ldots$$
نحصل على هذه المتوالية بأنّ نبدأ بـ 1، ثم نضيف 1، ثم نضيف 2، ثم نضيف 3، ثم نضيف 4 \ldots
\ifwithsols
\begin{boxSolution}
\begin{english}
\begin{minted}{csharp}
public static void PrintSequence(int n)
{
    int last = 1;
    int add = 1;
    for(int i = 1 ; i <= n ; i++)
    {
        Console.WriteLine(val);
        val += i;
    }
}
\end{minted}
\end{english}
\end{boxSolution}
\fi


\end{enumerate}

\vspace{1cm}
\begin{flushleft}
أرجو لكم وقتًا ممتعًا.

الأستاذ محمود اغبارية.
\end{flushleft}

\end{document}
