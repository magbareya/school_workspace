\documentclass[14pt]{extarticle}
% Full article preamble (duplicated, no common file)
\usepackage{fontspec}
\usepackage[a4paper,top=2.4cm,bottom=2.4cm,left=2.3cm,right=2.3cm]{geometry}
\usepackage{polyglossia}
\usepackage{amsmath}
\usepackage{amssymb}
\usepackage{xcolor}
\usepackage{fancyhdr}
\usepackage{graphicx}
\usepackage{listings}
\usepackage[most]{tcolorbox}
\usepackage{pifont}
\usepackage{enumitem}
\usepackage{titlesec}
\usepackage[bottom]{footmisc}
\usepackage{titling}
\usepackage{minted}
\usepackage{etoolbox}
\usepackage{array}
\usepackage{extsizes}

\newfontfamily\emoji{Segoe UI Emoji}

\pagestyle{fancy}

\setmainlanguage[numerals=western]{arabic}
\setotherlanguage{english}
\newfontfamily\arabicfont[Script=Arabic]{Amiri}
\newfontfamily\arabicfonttt[Script=Arabic]{Courier New}

\lstset{
  language=[Sharp]C,
  numbers=left,
  stepnumber=1,
  numbersep=8pt,
  frame=single,
  basicstyle=\ttfamily\small,
  keywordstyle=\color{blue},
  stringstyle=\color{red},
  commentstyle=\color{green!50!black}
}

\newif\ifdetailed
\ifdefined\setdetailed
  \setdetailed
\fi

\newif\ifwithsols
\ifdefined\setwithsols
  \setwithsols
\fi

% unified tcolorboxes for articles
\tcbset{colback=white, colframe=black, fonttitle=\bfseries, boxrule=0.8pt}
\newtcolorbox{boxDef}[1][]{colback=blue!5!white,colframe=blue!75!black,
  title={{\emoji📘} تعريف\ifx\\#1\\\else ~#1\fi :}}
\newtcolorbox{boxExercise}[1][]{colback=cyan!5!white,colframe=cyan!70!black,
  title={{\emoji🧩} تمرين\ifx\\#1\\\else ~#1\fi :}}
\newtcolorbox{boxExample}[1][]{colback=yellow!5!white,colframe=orange!90!black,
  title={{\emoji📝} مثال\ifx\\#1\\\else ~#1\fi :}}
\newtcolorbox{boxNote}[1][]{colback=gray!10!white,colframe=black,
  title={{\emoji✨} ملاحظة\ifx\\#1\\\else ~#1\fi :}}
\newtcolorbox{boxAttention}[1][]{colback=magenta!10!white,colframe=magenta!80!black,
  title={{\emoji🔔} تنبيه\ifx\\#1\\\else ~#1\fi :}}
\newtcolorbox{boxWarning}[1][]{colback=red!5!white,colframe=red!75!black,
  title={{\emoji⚡} ملاحظة هامة\ifx\\#1\\\else ~#1\fi :}}
\newtcolorbox{boxSolution}[1][]{colback=green!5!white,colframe=green!60!black,
  title={{\emoji✅} حل\ifx\\#1\\\else ~#1\fi :}}
\newtcolorbox{boxSymbol}[1][]{colback=purple!5!white,colframe=purple!70!black,
  title={{\emoji🔣} رمز\ifx\\#1\\\else ~#1\fi :}}
\newtcolorbox{boxHint}[1][]{colback=teal!5!white,colframe=teal!60!black,
  title={{\emoji💡} تلميح\ifx\\#1\\\else ~#1\fi :}}


\tcbset{simplecode/.style={ colback=gray!5, colframe=black!50, boxrule=0.4pt, arc=2pt, left=4pt,right=4pt,top=4pt,bottom=4pt}}
\newenvironment{boxCode}{\begin{tcolorbox}[simplecode]}{\end{tcolorbox}}

\newcolumntype{C}[1]{>{\centering\arraybackslash}p{#1}}

% redefine spaces after titles
\makeatletter
\renewcommand{\@maketitle}{%
  \begin{center}
    {\huge \bfseries \@title \par}%
    \vskip 0.2em % space between title and author
    {\large \@author \par}%
    % \vskip 0.2em % space between author and date
    % {\normalsize \@date \par}%
  \end{center}
}
\makeatother

\fancyhf{} % clear default
\fancypagestyle{plain}{
  \fancyhf{}
  \fancyhead[L]{مدرسة التسامح الشاملة}
  % \fancyhead[L]{\includegraphics[height=1cm]{../../../images/logoTasamoh.png}}
  \fancyhead[R]{الأستاذ محمود اغبارية}
  \fancyfoot[C]{\thepage}
}

\fancyhead[L]{مدرسة التسامح الشاملة}
\fancyhead[R]{الأستاذ محمود اغبارية}
\fancyfoot[C]{\thepage}
% \date{\today}

\setcounter{tocdepth}{3} % only section subsection and subsubsection in TOC


% ----------------------


% \begin{document}

% \maketitle

% % \clearpage  % start TOC on a new page
% % \renewcommand{\contentsname}{جدول المحتويات}
% % \tableofcontents
% % \clearpage

% \part*{part 1} % the * prevents numbering
% \section*{مقدمة}
% \subsection*{مثال رياضي}
% \subsubsection*{مثال فرعي}
% \paragraph*{ paragraph 1}
% \subparagraph*{sub paragraph 1}

% \ifdetailed
% \begin{english}
% \begin{minted}{csharp}
% // C# Example
% \end{minted}
% \end{english}
% \fi

% OLD WAY
% \ifdetailed
% \begin{english}
% \begin{lstlisting}
% // C# Example
% \end{lstlisting}
% \end{english}
% \fi

% % \includegraphics[width=0.2\textwidth]{../../../images/DFAs/ex1_q1.png}



% \vspace{3cm}
% \begin{flushleft}
% أرجو لكم وقتًا ممتعًا.

% الأستاذ محمود اغبارية.
% \end{flushleft}


% \end{document}


\ifwithsols
\title{حل وظيفة 12 - المصفوفات الأحادية - أساسيات}
\else
\title{وظيفة 12 - المصفوفات الأحادية - أساسيات}
\fi

\begin{document}

\maketitle
\thispagestyle{fancy}

\begin{enumerate}[itemsep=2em]

    % ======================================================
    % سؤال 1 - فكرة: طباعة مواقع العناصر الموجبة
    % ======================================================

    \item
    اكتب مقطع برنامج يطبع مواقع (الفهارس) جميع العناصر السالبة في مصفوفة أعداد صحيحة \textenglish{numbers}.\\
    \textbf{مثال:} إذا كانت المصفوفة \textenglish{\{5, -3, 8, -1, 0, -7, 2\}}، يطبع البرنامج: 1, 3, 5.
    \ifwithsols
    \begin{boxSolution}
    \begin{english}
    \begin{minted}{csharp}
for (int i = 0; i < numbers.Length; i++)
{
    if (numbers[i] < 0)
        Console.WriteLine(i);
}
    \end{minted}
    \end{english}
    \end{boxSolution}
    \fi

    % ======================================================
    % سؤال 2 - فكرة: استبدال قيم بشرط
    % ======================================================

    \item
    لديك مصفوفة أعداد صحيحة \textenglish{scores} تمثل درجات طلاب.\\
    اكتب مقطع برنامج يستبدل كل درجة أقل من 60 بالقيمة 60 (الحد الأدنى للنجاح).\\
    \textbf{مثال:} إذا كانت المصفوفة \textenglish{\{75, 45, 88, 52, 90, 30\}}، تصبح \textenglish{\{75, 60, 88, 60, 90, 60\}}.
    \ifwithsols
    \begin{boxSolution}
    \begin{english}
    \begin{minted}{csharp}
for (int i = 0; i < scores.Length; i++)
{
    if (scores[i] < 60)
        scores[i] = 60;
}
    \end{minted}
    \end{english}
    \end{boxSolution}
    \clearpage
    \fi

    % ======================================================
    % سؤال 3 - فكرة: حساب عدد الانتقالات من موجب لسالب
    % ======================================================

    \item
    اكتب مقطع برنامج يعد كم مرة تغيرت الإشارة من موجب إلى سالب أو من سالب إلى موجب بين عنصرين متجاورين في مصفوفة أعداد صحيحة \textenglish{data}.\\
    \textbf{مثال:} إذا كانت المصفوفة \textenglish{\{5, -3, -2, 4, -1, -6\}}، عدد التغييرات هو 3:
    \begin{itemize}
        \item من 5 (موجب) إلى $-3$ (سالب)
        \item من $-2$ (سالب) إلى 4 (موجب)
        \item من 4 (موجب) إلى $-1$ (سالب)
    \end{itemize}
    \textbf{ملاحظة:} افترض أنّه لا يوجد قيمة 0 في المصفوفة.
    \ifwithsols
    \begin{boxSolution}
    \begin{english}
    \begin{minted}{csharp}
int changes = 0;
for (int i = 0; i < data.Length - 1; i++)
{
    if ((data[i] > 0 && data[i + 1] < 0) ||
        (data[i] < 0 && data[i + 1] > 0))
    {
        changes++;
    }
}
Console.WriteLine(changes);
    \end{minted}
    \end{english}
    \end{boxSolution}
    \clearpage
    \fi

%     % ======================================================
%     % سؤال 5 - فكرة: إنشاء مصفوفة الفروقات
%     % ======================================================

%     \item
%     اكتب مقطع برنامج ينشئ مصفوفة جديدة \textenglish{differences} بطول أقل بواحد من مصفوفة أعداد صحيحة \textenglish{original}.\\
%     كل عنصر في المصفوفة الجديدة يساوي الفرق المطلق بين عنصرين متجاورين في المصفوفة الأصلية.\\
%     \textbf{مثال:} إذا كانت \textenglish{original = \{10, 3, 8, 12, 5\}}، تصبح \textenglish{differences = \{7, 5, 4, 7\}}.\\
%     (لأن: $|10-3|=7$, $|3-8|=5$, $|8-12|=4$, $|12-5|=7$)
%     \ifwithsols
%     \begin{boxSolution}
%     \begin{english}
%     \begin{minted}{csharp}
% int[] differences = new int[original.Length - 1];

% for (int i = 0; i < differences.Length; i++)
% {
%     differences[i] = Math.Abs(original[i] - original[i + 1]);
% }
%     \end{minted}
%     \end{english}
%     \end{boxSolution}
%     \clearpage
%     \fi

    % ======================================================
    % سؤال 7 - فكرة: فحص هل المصفوفة مرتبة
    % ======================================================

    \item
    اكتب مقطع برنامج يفحص هل مصفوفة أعداد صحيحة \textenglish{arr} مرتبة تصاعدياً (كل عنصر أصغر من أو يساوي العنصر الذي يليه).\\
    على البرنامج أن يطبع \textenglish{"Sorted"} إذا كانت مرتبة و\textenglish{"Not Sorted"} خلاف ذلك.\\
    \textbf{مثال:} \\
    - المصفوفة \textenglish{\{2, 5, 7, 7, 10\}} مرتبة.\\
    - المصفوفة \textenglish{\{2, 5, 3, 7\}} غير مرتبة.
    \ifwithsols
    \begin{boxSolution}
    \begin{english}
    \begin{minted}{csharp}
bool isSorted = true;

for (int i = 0; i < arr.Length - 1; i++)
{
    if (arr[i] > arr[i + 1])
    {
        isSorted = false;
        break;
    }
}

if (isSorted)
    Console.WriteLine("Sorted");
else
    Console.WriteLine("Not Sorted");
    \end{minted}
    \end{english}
    \end{boxSolution}
    \fi
    \clearpage

    \item
    اكتب مقطع برنامج ينشئ مصفوفة جديدة \textenglish{sums} بنفس طول مصفوفة أعداد صحيحة \textenglish{values}.\\
    كل عنصر في المصفوفة الجديدة يساوي مجموع العنصر نفسه مع جيرانه (العنصر قبله والعنصر بعده في المصفوفة الأصلية).\\
    العنصر الأول يُجمع مع نفسه والعنصر الذي بعده فقط. العنصر الأخير يُجمع مع نفسه والعنصر الذي قبله فقط.\\
    \textbf{مثال:} إذا كانت \textenglish{values = \{5, 3, 8, 2\}}:
    \begin{itemize}
        \item \textenglish{sums[0] = 5 + 3 = 8}
        \item \textenglish{sums[1] = 5 + 3 + 8 = 16}
        \item \textenglish{sums[2] = 3 + 8 + 2 = 13}
        \item \textenglish{sums[3] = 8 + 2 = 10}
    \end{itemize}
    النتيجة: \textenglish{sums = \{8, 16, 13, 10\}}
    \ifwithsols
    \begin{boxSolution}
    \begin{english}
    \begin{minted}{csharp}
int[] sums = new int[values.Length];

for (int i = 0; i < values.Length; i++)
{
    sums[i] = values[i];

    if (i > 0)
        sums[i] += values[i - 1];

    if (i < values.Length - 1)
        sums[i] += values[i + 1];
}
    \end{minted}
    \end{english}
    \end{boxSolution}
    \clearpage
    \fi

    % ======================================================
    % سؤال 7 - المرور من النهاية للبداية
    % ======================================================

    \item
    اكتب مقطع برنامج يطبع جميع الأعداد الأكبر من 50 في مصفوفة أعداد صحيحة \textenglish{prices} بترتيب عكسي (من الأخير إلى الأول).\\    \textbf{مثال:} إذا كانت المصفوفة \textenglish{\{30, 60, 45, 75, 55, 40, 80\}}، يطبع البرنامج: 80, 55, 75, 60 (بهذا الترتيب).
    \ifwithsols
    \begin{boxSolution}
    \begin{english}
    \begin{minted}{csharp}
for (int i = prices.Length - 1; i >= 0; i--)
{
    if (prices[i] > 50)
        Console.WriteLine(prices[i]);
}
    \end{minted}
    \end{english}
    \end{boxSolution}
    \clearpage
    \fi

    % ======================================================
    % سؤال 8 - المرور من النهاية للبداية - نسخ العناصر الفردية بالعكس
    % ======================================================

    \item
    اكتب مقطع برنامج ينشئ مصفوفة جديدة تحتوي فقط على الأعداد الفردية من مصفوفة أعداد صحيحة \textenglish{numbers}، لكن بترتيب عكسي (من الأخير إلى الأول). \\
    \textbf{مثال:} إذا كانت المصفوفة \textenglish{\{2, 7, 4, 9, 6, 3, 8, 5\}}، الأعداد الفردية بالترتيب الأصلي هي:\textenglish{7, 9, 3, 5} \\
    لكن نريدها بترتيب عكسي، فتصبح المصفوفة الجديدة: \textenglish{\{5, 3, 9, 7\}}. \\
    \textbf{تلميح:} أولاً عُدّ الأعداد الفردية، ثم امرر على المصفوفة من النهاية للبداية لملء المصفوفة الجديدة.
    \ifwithsols
    \begin{boxSolution}
    \begin{english}
    \begin{minted}{csharp}
int oddCount = 0;
for (int i = 0; i < numbers.Length; i++)
{
    if (numbers[i] % 2 != 0)
        oddCount++;
}

int[] oddReversed = new int[oddCount];
int index = 0;

for (int i = numbers.Length - 1; i >= 0; i--)
{
    if (numbers[i] % 2 != 0)
    {
        oddReversed[index] = numbers[i];
        index++;
    }
}
    \end{minted}
    \end{english}
    \end{boxSolution}
    \fi

\end{enumerate}

\vspace{1cm}
\begin{flushleft}
أرجو لكم عملا ممتعًا!

الأستاذ محمود اغبارية.
\end{flushleft}

\end{document}
