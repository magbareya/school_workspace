\documentclass[14pt]{extarticle}
% Full article preamble (duplicated, no common file)
\usepackage{fontspec}
\usepackage[a4paper,top=2.4cm,bottom=2.4cm,left=2.3cm,right=2.3cm]{geometry}
\usepackage{polyglossia}
\usepackage{amsmath}
\usepackage{amssymb}
\usepackage{xcolor}
\usepackage{fancyhdr}
\usepackage{graphicx}
\usepackage{listings}
\usepackage[most]{tcolorbox}
\usepackage{pifont}
\usepackage{enumitem}
\usepackage{titlesec}
\usepackage[bottom]{footmisc}
\usepackage{titling}
\usepackage{minted}
\usepackage{etoolbox}
\usepackage{array}
\usepackage{extsizes}

\newfontfamily\emoji{Segoe UI Emoji}

\pagestyle{fancy}

\setmainlanguage[numerals=western]{arabic}
\setotherlanguage{english}
\newfontfamily\arabicfont[Script=Arabic]{Amiri}
\newfontfamily\arabicfonttt[Script=Arabic]{Courier New}

\lstset{
  language=[Sharp]C,
  numbers=left,
  stepnumber=1,
  numbersep=8pt,
  frame=single,
  basicstyle=\ttfamily\small,
  keywordstyle=\color{blue},
  stringstyle=\color{red},
  commentstyle=\color{green!50!black}
}

\newif\ifdetailed
\ifdefined\setdetailed
  \setdetailed
\fi

\newif\ifwithsols
\ifdefined\setwithsols
  \setwithsols
\fi

% unified tcolorboxes for articles
\tcbset{colback=white, colframe=black, fonttitle=\bfseries, boxrule=0.8pt}
\newtcolorbox{boxDef}[1][]{colback=blue!5!white,colframe=blue!75!black,
  title={{\emoji📘} تعريف\ifx\\#1\\\else ~#1\fi :}}
\newtcolorbox{boxExercise}[1][]{colback=cyan!5!white,colframe=cyan!70!black,
  title={{\emoji🧩} تمرين\ifx\\#1\\\else ~#1\fi :}}
\newtcolorbox{boxExample}[1][]{colback=yellow!5!white,colframe=orange!90!black,
  title={{\emoji📝} مثال\ifx\\#1\\\else ~#1\fi :}}
\newtcolorbox{boxNote}[1][]{colback=gray!10!white,colframe=black,
  title={{\emoji✨} ملاحظة\ifx\\#1\\\else ~#1\fi :}}
\newtcolorbox{boxAttention}[1][]{colback=magenta!10!white,colframe=magenta!80!black,
  title={{\emoji🔔} تنبيه\ifx\\#1\\\else ~#1\fi :}}
\newtcolorbox{boxWarning}[1][]{colback=red!5!white,colframe=red!75!black,
  title={{\emoji⚡} ملاحظة هامة\ifx\\#1\\\else ~#1\fi :}}
\newtcolorbox{boxSolution}[1][]{colback=green!5!white,colframe=green!60!black,
  title={{\emoji✅} حل\ifx\\#1\\\else ~#1\fi :}}
\newtcolorbox{boxSymbol}[1][]{colback=purple!5!white,colframe=purple!70!black,
  title={{\emoji🔣} رمز\ifx\\#1\\\else ~#1\fi :}}
\newtcolorbox{boxHint}[1][]{colback=teal!5!white,colframe=teal!60!black,
  title={{\emoji💡} تلميح\ifx\\#1\\\else ~#1\fi :}}


\tcbset{simplecode/.style={ colback=gray!5, colframe=black!50, boxrule=0.4pt, arc=2pt, left=4pt,right=4pt,top=4pt,bottom=4pt}}
\newenvironment{boxCode}{\begin{tcolorbox}[simplecode]}{\end{tcolorbox}}

\newcolumntype{C}[1]{>{\centering\arraybackslash}p{#1}}

% redefine spaces after titles
\makeatletter
\renewcommand{\@maketitle}{%
  \begin{center}
    {\huge \bfseries \@title \par}%
    \vskip 0.2em % space between title and author
    {\large \@author \par}%
    % \vskip 0.2em % space between author and date
    % {\normalsize \@date \par}%
  \end{center}
}
\makeatother

\fancyhf{} % clear default
\fancypagestyle{plain}{
  \fancyhf{}
  \fancyhead[L]{مدرسة التسامح الشاملة}
  % \fancyhead[L]{\includegraphics[height=1cm]{../../../images/logoTasamoh.png}}
  \fancyhead[R]{الأستاذ محمود اغبارية}
  \fancyfoot[C]{\thepage}
}

\fancyhead[L]{مدرسة التسامح الشاملة}
\fancyhead[R]{الأستاذ محمود اغبارية}
\fancyfoot[C]{\thepage}
% \date{\today}

\setcounter{tocdepth}{3} % only section subsection and subsubsection in TOC


% ----------------------


% \begin{document}

% \maketitle

% % \clearpage  % start TOC on a new page
% % \renewcommand{\contentsname}{جدول المحتويات}
% % \tableofcontents
% % \clearpage

% \part*{part 1} % the * prevents numbering
% \section*{مقدمة}
% \subsection*{مثال رياضي}
% \subsubsection*{مثال فرعي}
% \paragraph*{ paragraph 1}
% \subparagraph*{sub paragraph 1}

% \ifdetailed
% \begin{english}
% \begin{minted}{csharp}
% // C# Example
% \end{minted}
% \end{english}
% \fi

% OLD WAY
% \ifdetailed
% \begin{english}
% \begin{lstlisting}
% // C# Example
% \end{lstlisting}
% \end{english}
% \fi

% % \includegraphics[width=0.2\textwidth]{../../../images/DFAs/ex1_q1.png}



% \vspace{3cm}
% \begin{flushleft}
% أرجو لكم وقتًا ممتعًا.

% الأستاذ محمود اغبارية.
% \end{flushleft}


% \end{document}


\ifwithsols
\title{حل ورقة تمرن 6 للصف العاشر 10 - قسم 3 \\ أسئلة تحليل العدد إلى خاناته}
\else
\title{ورقة تمرن 6 للصف العاشر 10 - قسم 3 \\ أسئلة تحليل العدد إلى خاناته}
\fi

\begin{document}

\maketitle
\thispagestyle{fancy}

\begin{enumerate}[itemsep=1.5em]

\item اكتب عملية خارجية تتلقى عددًا صحيحًا وتعيد عدد خاناته.

\item اكتب عملية خارجية تتلقى عددًا صحيحًا وتعيد مجموع خاناته.

\item اكتب عملية خارجية تتلقى عددًا موجبًا وتطبع: عدد خاناته، ومجموع خاناته، ومعدّل خاناته. \\
        \textbf{مثلًا:} إذا تلقت العملية العدد 762945، فإنّها تطبع: \\
        لا تستخدم العمليات من الأسئلة السابقة، اكتب حلقة \textenglish{while} واحدة فقط.

\begin{english}
\begin{boxCode}
    Digits count = 6 \\
    Digits sum = 33 (5+4+9+2+6+7=33) \\
    Digits average = 5.5 (33 / 6 = 5.5)
\end{boxCode}
\end{english}
\ifwithsols
\begin{boxSolution}
\begin{english}
\begin{minted}{csharp}
public static void PrintDigitsStats(int n)
{
    int count = 0, sum = 0;
    int m = n;
    while (m > 0)
    {
        sum += (m % 10);
        count++;
        m /= 10;
    }
    Console.WriteLine("Digits count = " + count);
    Console.WriteLine("Digits sum = " + sum);
    double avg = ((double) sum / count);
    Console.WriteLine("Digits average = " + avg);
}
\end{minted}
\end{english}
\end{boxSolution}
\fi

\item اكتب عملية خارجية تتلقى عددًا صحيحا، وتطبع عدد الخانات الزوجية فيه وعدد الخانات الفردية فيه.

\item
اكتب برنامجًا يستقبل من المستخدم أعدادًا صحيحة، ويعدّ الأعداد التي خانة العشرات فيها زوجية، ويطبعه. \\
ينتهي الاستقبال عندما يدخل المستخدم عددًا سالبًا.

\item اكتب برنامجًا يستقبل من المستخدم 100 عدد صحيح. في نهاية البرنامج عليه أن يطبع الرقم الذي له أكبر مجموع خانات (يمكنك الاستعانة بالعملية التي كتبتها في الأسئلة السابقة).

\item
\begin{enumerate}
    \item اكتب عملية خارجية تتلقى عددًا صحيحًا وتعيد قيمة أكبر خانة فيه. \\
    \textbf{مثلًا:} إذا تلقّت العدد $23481$ فإنّها تعيد $8$.

    \item اكتب برنامجًا يستقبل من المستخدم 100 عدد صحيح. في نهاية البرنامج عليه أن يطبع الرقم الذي ظهرت فيه أكبر خانة. \\
    إذا كان هناك أكثر من رقم ظهرت فيه أكبر خانة، فإنّه يطبع أكبر رقم من بينها.
\end{enumerate}

\clearpage
\item
\begin{enumerate}
\item اكتب عملية خارجية باسم \textenglish{SumSquares} تتلقى عددًا صحيحًا، وتعيد مجموع \textbf{تربيع} خاناته.
\item في البرنامج الرئيسي، استدعِ العملية مع الرقم 23.
\item هل هناك أعداد تساوي مجموع تربيع خاناتها بين 1 و 10000؟ ما هي؟
\end{enumerate}

\item
\begin{enumerate}
    \item اكتب عملية خارجية باسم \textenglish{AppendDigit} تتلقى عددًا صحيحًا، ورقمًا (عددا من منزلة واحدة). \\
    العملية تلصق المنزلة بالعدد الأوّل وتعيد النتيجة. \\
    \textbf{مثال:} إذا تلقت العملية العددين: $12$ والعدد $5$ فإنّها تعيد: $125$.
    \item اكتب برنامجًا يستقبل من المستخدم أرقامًا (من منزلة واحدة) ويكوّن عددًا صحيحًا من هذه الأرقام ويطبعه. \\
    ينتهي استقبال الأرقام عندما يدخل المستخدم عددًا سالبًا.
    \item اكتب برنامجًا يستقبل من المستخدم 7 أرقام، كل واحد من خانة واحدة، ويكوّن له رقمًا من أربعة منازل ويطبعه. \\
    الرقم الأول الذي يدخله المستخدم تكون خانة عشرات الآلاف، والرقم الأخير يكون الآحاد. \\
    \textbf{مثال:} إذا أدخل المستخدم الأرقام $-1,4,2,1,5$ فعلى البرنامج أن يطبع $4215$. \\
    لا حاجة للتحقق من صحة المدخلات.
\end{enumerate}

\end{enumerate}

\vspace{1cm}
\begin{flushleft}
أرجو لكم وقتًا ممتعًا.

الأستاذ محمود اغبارية.
\end{flushleft}

\end{document}
