\documentclass[14pt]{extarticle}
% Full article preamble (duplicated, no common file)
\usepackage{fontspec}
\usepackage[a4paper,top=2.4cm,bottom=2.4cm,left=2.3cm,right=2.3cm]{geometry}
\usepackage{polyglossia}
\usepackage{amsmath}
\usepackage{amssymb}
\usepackage{xcolor}
\usepackage{fancyhdr}
\usepackage{graphicx}
\usepackage{listings}
\usepackage[most]{tcolorbox}
\usepackage{pifont}
\usepackage{enumitem}
\usepackage{titlesec}
\usepackage[bottom]{footmisc}
\usepackage{titling}
\usepackage{minted}
\usepackage{etoolbox}
\usepackage{array}
\usepackage{extsizes}

\newfontfamily\emoji{Segoe UI Emoji}

\pagestyle{fancy}

\setmainlanguage[numerals=western]{arabic}
\setotherlanguage{english}
\newfontfamily\arabicfont[Script=Arabic]{Amiri}
\newfontfamily\arabicfonttt[Script=Arabic]{Courier New}

\lstset{
  language=[Sharp]C,
  numbers=left,
  stepnumber=1,
  numbersep=8pt,
  frame=single,
  basicstyle=\ttfamily\small,
  keywordstyle=\color{blue},
  stringstyle=\color{red},
  commentstyle=\color{green!50!black}
}

\newif\ifdetailed
\ifdefined\setdetailed
  \setdetailed
\fi

\newif\ifwithsols
\ifdefined\setwithsols
  \setwithsols
\fi

% unified tcolorboxes for articles
\tcbset{colback=white, colframe=black, fonttitle=\bfseries, boxrule=0.8pt}
\newtcolorbox{boxDef}[1][]{colback=blue!5!white,colframe=blue!75!black,
  title={{\emoji📘} تعريف\ifx\\#1\\\else ~#1\fi :}}
\newtcolorbox{boxExercise}[1][]{colback=cyan!5!white,colframe=cyan!70!black,
  title={{\emoji🧩} تمرين\ifx\\#1\\\else ~#1\fi :}}
\newtcolorbox{boxExample}[1][]{colback=yellow!5!white,colframe=orange!90!black,
  title={{\emoji📝} مثال\ifx\\#1\\\else ~#1\fi :}}
\newtcolorbox{boxNote}[1][]{colback=gray!10!white,colframe=black,
  title={{\emoji✨} ملاحظة\ifx\\#1\\\else ~#1\fi :}}
\newtcolorbox{boxAttention}[1][]{colback=magenta!10!white,colframe=magenta!80!black,
  title={{\emoji🔔} تنبيه\ifx\\#1\\\else ~#1\fi :}}
\newtcolorbox{boxWarning}[1][]{colback=red!5!white,colframe=red!75!black,
  title={{\emoji⚡} ملاحظة هامة\ifx\\#1\\\else ~#1\fi :}}
\newtcolorbox{boxSolution}[1][]{colback=green!5!white,colframe=green!60!black,
  title={{\emoji✅} حل\ifx\\#1\\\else ~#1\fi :}}
\newtcolorbox{boxSymbol}[1][]{colback=purple!5!white,colframe=purple!70!black,
  title={{\emoji🔣} رمز\ifx\\#1\\\else ~#1\fi :}}
\newtcolorbox{boxHint}[1][]{colback=teal!5!white,colframe=teal!60!black,
  title={{\emoji💡} تلميح\ifx\\#1\\\else ~#1\fi :}}


\tcbset{simplecode/.style={ colback=gray!5, colframe=black!50, boxrule=0.4pt, arc=2pt, left=4pt,right=4pt,top=4pt,bottom=4pt}}
\newenvironment{boxCode}{\begin{tcolorbox}[simplecode]}{\end{tcolorbox}}

\newcolumntype{C}[1]{>{\centering\arraybackslash}p{#1}}

% redefine spaces after titles
\makeatletter
\renewcommand{\@maketitle}{%
  \begin{center}
    {\huge \bfseries \@title \par}%
    \vskip 0.2em % space between title and author
    {\large \@author \par}%
    % \vskip 0.2em % space between author and date
    % {\normalsize \@date \par}%
  \end{center}
}
\makeatother

\fancyhf{} % clear default
\fancypagestyle{plain}{
  \fancyhf{}
  \fancyhead[L]{مدرسة التسامح الشاملة}
  % \fancyhead[L]{\includegraphics[height=1cm]{../../../images/logoTasamoh.png}}
  \fancyhead[R]{الأستاذ محمود اغبارية}
  \fancyfoot[C]{\thepage}
}

\fancyhead[L]{مدرسة التسامح الشاملة}
\fancyhead[R]{الأستاذ محمود اغبارية}
\fancyfoot[C]{\thepage}
% \date{\today}

\setcounter{tocdepth}{3} % only section subsection and subsubsection in TOC


% ----------------------


% \begin{document}

% \maketitle

% % \clearpage  % start TOC on a new page
% % \renewcommand{\contentsname}{جدول المحتويات}
% % \tableofcontents
% % \clearpage

% \part*{part 1} % the * prevents numbering
% \section*{مقدمة}
% \subsection*{مثال رياضي}
% \subsubsection*{مثال فرعي}
% \paragraph*{ paragraph 1}
% \subparagraph*{sub paragraph 1}

% \ifdetailed
% \begin{english}
% \begin{minted}{csharp}
% // C# Example
% \end{minted}
% \end{english}
% \fi

% OLD WAY
% \ifdetailed
% \begin{english}
% \begin{lstlisting}
% // C# Example
% \end{lstlisting}
% \end{english}
% \fi

% % \includegraphics[width=0.2\textwidth]{../../../images/DFAs/ex1_q1.png}



% \vspace{3cm}
% \begin{flushleft}
% أرجو لكم وقتًا ممتعًا.

% الأستاذ محمود اغبارية.
% \end{flushleft}


% \end{document}


\ifwithsols
\title{حل ورقة تمرن 6 للصف العاشر 10 - قسم 3 \\ أسئلة تحليل العدد إلى خاناته}
\else
\title{ورقة تمرن 6 للصف العاشر 10 - قسم 3 \\ أسئلة تحليل العدد إلى خاناته}
\fi

\begin{document}

\maketitle
\thispagestyle{fancy}

\begin{enumerate}[itemsep=1.5em]


\item
اكتب عملية خارجية تتلقى عددًا صحيحًا $n$ وتطبع من 1 حتى $n$ فقط الأعداد التي ينتهي رقمها بــ 5.
\ifwithsols
\begin{boxSolution}
\begin{english}
\begin{minted}{csharp}
public static void PrintEndingWith5(int n)
{
    for(int i = 1; i <= n; i++)
    {
        if (i % 10 == 5)
            Console.WriteLine(i);
    }
}
\end{minted}
\end{english}
\end{boxSolution}
\fi

\item
اكتب عملية تتلقى عددًا $n$ وتعيد مجموع كل الأعداد التي رقم آحادها 3 بين 1 و $n$.
\ifwithsols
\begin{boxSolution}
\begin{english}
\begin{minted}{csharp}
public static int SumEnding3(int n)
{
    int sum = 0;
    for(int i=1; i<=n; i++)
    {
        if (i % 10 == 3)
            sum += i;
    }
    return sum;
}
\end{minted}
\end{english}
\end{boxSolution}
\clearpage
\fi


\item
اكتب برنامجًا يستقبل من المستخدم 20 عددًا ويطبع عدد الأعداد التي كان رقم آحادها أكبر من رقم عشراتها.
\ifwithsols
\begin{boxSolution}
\begin{english}
\begin{minted}{csharp}
int cnt = 0, x;
for(int i=0; i<20; i++)
{
    x = int.Parse(Console.ReadLine());
    x = Math.Abs(x);
    int ones = x % 10;
    int tens = (x / 10) % 10;
    if (ones > tens)
        cnt++;
}
Console.WriteLine(cnt);
\end{minted}
\end{english}
\end{boxSolution}
\clearpage
\fi

\item
اكتب برنامجًا يستقبل من المستخدم أعدادًا صحيحة، ويعدّ الأعداد التي خانة العشرات فيها زوجية، ويطبعه. \\
ينتهي الاستقبال عندما يدخل المستخدم عددًا سالبًا.

\ifwithsols
\begin{boxSolution}[1]
\begin{english}
\begin{minted}{csharp}
int cnt = 0;
int n = int.Parse(Console.ReadLine());
while (n >= 0)
{
    int tens = (n / 10) % 10;
    if (tens % 2 == 0)
        cnt++;
    n = int.Parse(Console.ReadLine());
}
Console.WriteLine(cnt);
\end{minted}
\end{english}
\end{boxSolution}
\begin{boxSolution}[2، فكر لماذا هو صحيح؟]
\begin{english}
\begin{minted}{csharp}
int cnt = 0;
int n = int.Parse(Console.ReadLine());
while (n >= 0)
{
    n /= 10;
    if (n % 2 == 0)
        cnt++;
    n = int.Parse(Console.ReadLine());
}
Console.WriteLine(cnt);
\end{minted}
\end{english}
\end{boxSolution}
\clearpage
\fi

\item \label{item:digitCount} اكتب عملية خارجية تتلقى عددًا صحيحًا وتعيد عدد خاناته.
\ifwithsols
\begin{boxSolution}
\begin{english}
\begin{minted}{csharp}
public static int DigitsCount(int n)
{
    if(n == 0)
        return 1;
    int count = 0;
    while (n > 0)
    {
        count++;
        n /= 10;
    }
    return count;
}
\end{minted}
\end{english}
\end{boxSolution}
\fi

\item اكتب عملية خارجية تتلقى عددًا صحيحًا وتعيد مجموع خاناته.
\ifwithsols
\begin{boxSolution}
\begin{english}
\begin{minted}{csharp}
public static int DigitsSum(int n)
{
    int sum = 0;
    while (n > 0)
    {
        sum += (n % 10);
        n /= 10;
    }
    return sum;
}
\end{minted}
\end{english}
\end{boxSolution}
\clearpage
\fi

\item اكتب عملية تتلقى عددًا صحيحًا، وتعيد حاصل ضرب خاناته \\
مثال: إذا كان العدد 234، النتيجة تكون $2 \times 3 \times 4 = 24$.
\ifwithsols
\begin{boxSolution}
\begin{english}
\begin{minted}{csharp}
public static int ProductOfDigits(int n)
{
    if (n == 0)
        return 0;
    int prod = 1;
    while (n != 0)
    {
        prod *= (n % 10);
        n /= 10;
    }
    return prod;
}
\end{minted}
\end{english}
\end{boxSolution}
\clearpage
\fi

\item اكتب عملية خارجية تتلقى عددًا موجبًا وتطبع: عدد خاناته، ومجموع خاناته، ومعدّل خاناته. \\
لا تستخدم العمليات من الأسئلة السابقة، اكتب حلقة \textenglish{while} واحدة فقط. \\
        \textbf{مثلًا:} إذا تلقت العملية العدد 762945، فإنّها تطبع:

\begin{english}
\begin{boxCode}
    Digits count = 6 \\
    Digits sum = 33 (5+4+9+2+6+7=33) \\
    Digits average = 5.5 (33 / 6 = 5.5)
\end{boxCode}
\end{english}
\ifwithsols
\begin{boxSolution}
\begin{english}
\begin{minted}{csharp}
public static void PrintDigitsStats(int n)
{
    int count = 0, sum = 0;
    int m = n;
    while (m > 0)
    {
        sum += (m % 10);
        count++;
        m /= 10;
    }
    Console.WriteLine("Digits count = " + count);
    Console.WriteLine("Digits sum = " + sum);
    double avg = ((double) sum / count);
    Console.WriteLine("Digits average = " + avg);
}
\end{minted}
\end{english}
\end{boxSolution}
\clearpage
\fi

\item اكتب عملية خارجية تتلقى عددًا صحيحا، وتطبع عدد الخانات الزوجية فيه وعدد الخانات الفردية فيه.
\ifwithsols
\begin{boxSolution}
\begin{english}
\begin{minted}{csharp}
public static void PrintEvenOddDigitsCount(int n)
{
    int even = 0, odd = 0;
    if (n == 0)
        even = 1;
    while (n > 0)
    {
        int d = n % 10;
        if (d % 2 == 0)
            even++;
        else
            odd++;
        n /= 10;
    }
    Console.WriteLine("Even digits = " + even);
    Console.WriteLine("Odd digits = " + odd);
}
\end{minted}
\end{english}
\end{boxSolution}
\clearpage
\fi

\item اكتب برنامجًا يستقبل من المستخدم 100 عدد صحيح. في نهاية البرنامج عليه أن يطبع الرقم الذي له أكبر مجموع خانات (يمكنك الاستعانة بالعملية التي كتبتها في الأسئلة السابقة).
\ifwithsols
\begin{boxSolution}
\begin{english}
\begin{minted}{csharp}
int bestNum = 0, bestSum = -1;
for (int i = 0; i < 100; i++)
{
    int x = int.Parse(Console.ReadLine());
    int s = DigitsSum(x);
    if (s > bestSum)
    {
        bestSum = s;
        bestNum = x;
    }
}
Console.WriteLine(bestNum);
\end{minted}
\end{english}
\end{boxSolution}
\clearpage
\fi

\item
\begin{enumerate}
    \item اكتب عملية خارجية تتلقى عددًا صحيحًا وتعيد قيمة أكبر خانة فيه. \\
    \textbf{مثلًا:} إذا تلقّت العدد $23481$ فإنّها تعيد $8$.

    \item اكتب برنامجًا يستقبل من المستخدم 100 عدد صحيح. في نهاية البرنامج عليه أن يطبع الرقم الذي ظهرت فيه أكبر خانة. \\
    إذا كان هناك أكثر من رقم ظهرت فيه أكبر خانة، فإنّه يطبع أكبر رقم من بينها.
\end{enumerate}
\ifwithsols
\begin{boxSolution}
\begin{english}
\begin{minted}{csharp}
public static int MaxDigit(int n)
{
    if (n == 0)
        return 0;
    int md = 0;
    while (n > 0)
    {
        md = Math.Max(md, n % 10);
        n /= 10;
    }
    return md;
}

private static void Main(string[] args)
{
    int bestNum = 0, bestMax = -1;
    for (int i = 0; i < 100; i++)
    {
        int x = int.Parse(Console.ReadLine());
        int m = MaxDigit(x);
        if (m > bestMax)
        {
            bestMax = m;
            bestNum = x;
        }
        else if (m == bestMax && x > bestNum)
            bestNum = x;
    }
    Console.WriteLine(bestNum);
}
\end{minted}
\end{english}
\end{boxSolution}
\clearpage
\fi

\item
\begin{enumerate}
\item اكتب عملية خارجية باسم \textenglish{SumSquares} تتلقى عددًا صحيحًا، وتعيد مجموع \textbf{تربيع} خاناته. \\
\textbf{مثال:} إذا تلقت العملية العدد 123، فإنها تعيد: $1^2 + 2^2 + 3^2 = 1 + 4 + 9 = 14$.
\item في البرنامج الرئيسي، استدعِ العملية مع الرقم 23.
\item هل هناك أعداد تساوي مجموع تربيع خاناتها بين 1 و 10000؟ ما هي؟
\end{enumerate}
\ifwithsols
\begin{boxSolution}
\begin{english}
\begin{minted}{csharp}
public static int SumSquares(int n)
{
    int sum = 0;
    while (n > 0)
    {
        int d = n % 10;
        sum += d * d;
        n /= 10;
    }
    return sum;
}

private static void Main(string[] args)
{
    Console.WriteLine(SumSquares(23));

    for (int i = 1; i <= 10000; i++)
    {
        if (SumSquares(i) == i)
            Console.WriteLine(i);
    }
}
\end{minted}
\end{english}
\end{boxSolution}
\clearpage
\fi

\item نعرّف عددًا مميّزًا بأنّه عدد إذا رفعنا كل خانة من خاناته لقوة مساوية لعدد خاناته ثم جمعنا النتائج فإنّنا نحصل على العدد الأصلي. \\
\textbf{مثال:} العدد 153 هو عدد مميّز، لأنّه مكوّن من 3 خانات، و: \\
$1^3 + 5^3 + 3^3 = 1 + 125 + 27 = 153$
\begin{enumerate}
\item اكتب عملية خارجية باسم \textenglish{IsSpecial} تتلقى عددًا صحيحًا، وتعيد \textenglish{true} إذا كان العدد مميّزًا، و\textenglish{false} خلاف ذلك. \\
يمكنك الاستعانة بالعملية التي كتبتها في السؤال (\ref{item:digitCount}) لمعرفة عدد خانات العدد.
\item في البرنامج الرئيسي، اطبع الأعداد المميّزة من 1 إلى 10000. \\
عليك الاستعانة بالعملية التي كتبتها في الفرع السابق.
\end{enumerate}
\ifwithsols
\begin{boxSolution}
\begin{english}
\begin{minted}{csharp}
public static bool IsSpecial(int n)
{
    int k = DigitsCount(n);
    int m = n, sum = 0;
    while (m > 0)
    {
        int d = m % 10;
        sum += (int) Math.Pow(d, k);
        m /= 10;
    }
    if(sum == n)
        return true;
    else
        return false;
}

private static void Main(string[] args)
{
    for (int i = 1; i <= 10000; i++)
    {
        if (IsSpecial(i))
            Console.WriteLine(i);
    }
}
\end{minted}
\end{english}
\end{boxSolution}
\clearpage
\fi

\item
\begin{enumerate}
    \item اكتب عملية خارجية باسم \textenglish{AppendDigit} تتلقى عددًا صحيحًا، ورقمًا (عددا من منزلة واحدة). \\
    العملية تلصق المنزلة بالعدد الأوّل وتعيد النتيجة. \\
    \textbf{مثال:} إذا تلقت العملية العددين: $12$ والعدد $5$ فإنّها تعيد: $125$.
    \item اكتب برنامجًا يستقبل من المستخدم أرقامًا (من منزلة واحدة) ويكوّن عددًا صحيحًا من هذه الأرقام ويطبعه. \\
    ينتهي استقبال الأرقام عندما يدخل المستخدم عددًا سالبًا. \\
    \textbf{مثال:} إذا أدخل المستخدم الأرقام $-1,4,2,1,5$ فعلى البرنامج أن يطبع $5124$. \\
    عليك استخدام العملية من الفرع السابق. لا حاجة للتحقق من صحة المدخلات.
\end{enumerate}
\ifwithsols
\begin{boxSolution}
\begin{english}
\begin{minted}{csharp}
public static int AppendDigit(int n, int d)
{
    return n * 10 + d;
}

private static void Main(string[] args)
{
    int res = 0;
    int d = int.Parse(Console.ReadLine());
    while (d >= 0)
    {
        res = AppendDigit(res, d);
        d = int.Parse(Console.ReadLine());
    }
    Console.WriteLine(res);
}
\end{minted}
\end{english}
\end{boxSolution}
\clearpage
\fi

\item \label{item:appendDigitToLeft}
 اكتب عملية خارجية باسم \textenglish{AppendDigitToLeft} تتلقى عددًا صحيحًا، ورقمًا (عددا من منزلة واحدة). \\
العملية تلصق المنزلة بالعدد الأوّل \textbf{من اليسار} وتعيد النتيجة. \\
\textbf{مثال:} إذا تلقت العملية العددين: $12$ والعدد $5$ فإنّها تعيد: $512$.
\begin{boxHint}
    يمكنك الاستعانة بالعملية التي كتبتها في السؤال (\ref{item:digitCount}) لمعرفة عدد خانات العدد. \\
    لكي نحصل على $512$ من $5$ و $12$ نضرب الـ $5$ بـ $100$ (وهو $10$ مرفوعًا لعدد خانات الـ $12$) ثم نجمع الـ $12$.
\end{boxHint}
\ifwithsols
\begin{boxSolution}
\begin{english}
\begin{minted}{csharp}
public static int AppendDigitToLeft(int n, int d)
{
    int dc = DigitsCount(n);
    int p = (int) Math.Pow(10, dc);
    return d * p + n;
}
\end{minted}
\end{english}
\end{boxSolution}
\clearpage
\fi

% \item
% //TODO: هذا السؤال فيه خطأ، بحاجة لإعادة صياغة: إذا كانت إحدى خانات العدد 0 فإنّها لا تضاف للعدد الناتج. يمكن حل المشكلة عن طريق حلقات متداخلة.
% اكتب عملية تتلقى عددًا صحيحًا وتعيد عددًا صحيحًا جديدًا يحوي فقط الأرقام الزوجية منه.
% مثال: إذا تلقّت 835462 تُعيد 8462.
% \begin{boxHint}
%     يمكنك استخدام عملية \textenglish{AppendDigitToLeft} التي كتبتها في السؤال \ref{item:appendDigitToLeft} لبناء العدد الجديد.
% \end{boxHint}
% \ifwithsols
% \begin{boxSolution}
% \begin{english}
% \begin{minted}{csharp}
% public static int ExtractEvenDigits(int n)
% {
%     int res = 0;
%     while(n > 0)
%     {
%         int d = n % 10;
%         if (d % 2 == 0)
%             res = AppendDigitToLeft(res, d);
%         n /= 10;
%     }
%     return res;
% }
% \end{minted}
% \end{english}
% \end{boxSolution}
% \fi

\end{enumerate}

\importpdfpage{../../../bagrut_questions/basics/loops_both_2025_899371_3.pdf}{1}

\ifwithsols

\paragraph*{حل سؤال 3 امتحان 899371 سنة 2025}:
\bigskip

% // TODO:

\fi

\subsection*{سؤال 6 امتحان 899371 سنة 2024}

\insertFullImg{../../../bagrut_questions/basics/loops_both_2024_899371_6.png}


\noindent
\textbf{ج.} اكتبو تطبيقًا للعمليتين \textenglish{DigitSum} و \textenglish{DigitExists}

\clearpage
\ifwithsols
\begin{boxSolution}[سؤال 6 امتحان 899371 سنة 2024]
\begin{enumerate}[itemsep=1.5em, label=\alph*.]
\item
\begin{enumerate}[itemsep=1.5em, label=\textbf{(\arabic*)}]

\item .
\begin{boxCode}
\begin{english}
\begin{minted}{csharp}
public static int DeepSum(int n)
{
    int ds = DigitSum(n);
    while (ds >= 10)
    {
        ds = DigitSum(ds);
    }
    return ds;
}
\end{minted}
\end{english}
\end{boxCode}

\item .
\begin{boxCode}
\begin{english}
\begin{minted}{csharp}
public static bool IsCorrect()
{
    int LIMIT = 999999;
    int evenCnt = 0, oddCnt = 0;
    for(int i = 1; i <= LIMIT; i++)
    {
        int ds = DeepSum(i);
        if(ds % 2 == 0)
            evenCnt++;
        else
            oddCnt++;
    }

    if(evenCnt > oddCnt)
        return true;
    else
        return false;
}
\end{minted}
\end{english}
\end{boxCode}
\end{enumerate}
\end{enumerate}
\end{boxSolution}
\clearpage

\begin{boxSolution}[سؤال 6 امتحان 899371 سنة 2024 - فرغ ب]
\begin{enumerate}[itemsep=1.5em, label=\alph*., start=2]
\item .
\begin{boxCode}
\begin{english}
\begin{minted}{csharp}
public static bool InBoth(int num1, int num2)
{
    int ds1 = DeepSum(num1);
    int ds2 = DeepSum(num2);
    if(DigitExists(num1, ds2) && DigitExists(num2, ds1))
        return true;
    else
        return false;
}
\end{minted}
\end{english}
\end{boxCode}

\item .
\begin{boxCode}
\begin{english}
\begin{minted}{csharp}
public static int DigitSum(int num1)
{
    int sum = 0;
    while (num1 > 0)
    {
        sum += (num1 % 10);
        num1 /= 10;
    }
    return sum;
}

public static bool DigitExists(int num, int digit)
{
    while (num > 0)
    {
        if (num % 10 == digit)
            return true;
        num /= 10;
    }
    return false;
}
\end{minted}
\end{english}
\end{boxCode}
\end{enumerate}
\end{boxSolution}

\clearpage
\fi


% \vspace{1cm}
% \begin{flushleft}
% أرجو لكم وقتًا ممتعًا.

% الأستاذ محمود اغبارية.
% \end{flushleft}

\end{document}
