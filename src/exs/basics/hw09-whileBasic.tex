\documentclass[14pt]{extarticle}
% Full article preamble (duplicated, no common file)
\usepackage{fontspec}
\usepackage[a4paper,top=2.4cm,bottom=2.4cm,left=2.3cm,right=2.3cm]{geometry}
\usepackage{polyglossia}
\usepackage{amsmath}
\usepackage{amssymb}
\usepackage{xcolor}
\usepackage{fancyhdr}
\usepackage{graphicx}
\usepackage{listings}
\usepackage[most]{tcolorbox}
\usepackage{pifont}
\usepackage{enumitem}
\usepackage{titlesec}
\usepackage[bottom]{footmisc}
\usepackage{titling}
\usepackage{minted}
\usepackage{etoolbox}
\usepackage{array}
\usepackage{extsizes}

\newfontfamily\emoji{Segoe UI Emoji}

\pagestyle{fancy}

\setmainlanguage[numerals=western]{arabic}
\setotherlanguage{english}
\newfontfamily\arabicfont[Script=Arabic]{Amiri}
\newfontfamily\arabicfonttt[Script=Arabic]{Courier New}

\lstset{
  language=[Sharp]C,
  numbers=left,
  stepnumber=1,
  numbersep=8pt,
  frame=single,
  basicstyle=\ttfamily\small,
  keywordstyle=\color{blue},
  stringstyle=\color{red},
  commentstyle=\color{green!50!black}
}

\newif\ifdetailed
\ifdefined\setdetailed
  \setdetailed
\fi

\newif\ifwithsols
\ifdefined\setwithsols
  \setwithsols
\fi

% unified tcolorboxes for articles
\tcbset{colback=white, colframe=black, fonttitle=\bfseries, boxrule=0.8pt}
\newtcolorbox{boxDef}[1][]{colback=blue!5!white,colframe=blue!75!black,
  title={{\emoji📘} تعريف\ifx\\#1\\\else ~#1\fi :}}
\newtcolorbox{boxExercise}[1][]{colback=cyan!5!white,colframe=cyan!70!black,
  title={{\emoji🧩} تمرين\ifx\\#1\\\else ~#1\fi :}}
\newtcolorbox{boxExample}[1][]{colback=yellow!5!white,colframe=orange!90!black,
  title={{\emoji📝} مثال\ifx\\#1\\\else ~#1\fi :}}
\newtcolorbox{boxNote}[1][]{colback=gray!10!white,colframe=black,
  title={{\emoji✨} ملاحظة\ifx\\#1\\\else ~#1\fi :}}
\newtcolorbox{boxAttention}[1][]{colback=magenta!10!white,colframe=magenta!80!black,
  title={{\emoji🔔} تنبيه\ifx\\#1\\\else ~#1\fi :}}
\newtcolorbox{boxWarning}[1][]{colback=red!5!white,colframe=red!75!black,
  title={{\emoji⚡} ملاحظة هامة\ifx\\#1\\\else ~#1\fi :}}
\newtcolorbox{boxSolution}[1][]{colback=green!5!white,colframe=green!60!black,
  title={{\emoji✅} حل\ifx\\#1\\\else ~#1\fi :}}
\newtcolorbox{boxSymbol}[1][]{colback=purple!5!white,colframe=purple!70!black,
  title={{\emoji🔣} رمز\ifx\\#1\\\else ~#1\fi :}}
\newtcolorbox{boxHint}[1][]{colback=teal!5!white,colframe=teal!60!black,
  title={{\emoji💡} تلميح\ifx\\#1\\\else ~#1\fi :}}


\tcbset{simplecode/.style={ colback=gray!5, colframe=black!50, boxrule=0.4pt, arc=2pt, left=4pt,right=4pt,top=4pt,bottom=4pt}}
\newenvironment{boxCode}{\begin{tcolorbox}[simplecode]}{\end{tcolorbox}}

\newcolumntype{C}[1]{>{\centering\arraybackslash}p{#1}}

% redefine spaces after titles
\makeatletter
\renewcommand{\@maketitle}{%
  \begin{center}
    {\huge \bfseries \@title \par}%
    \vskip 0.2em % space between title and author
    {\large \@author \par}%
    % \vskip 0.2em % space between author and date
    % {\normalsize \@date \par}%
  \end{center}
}
\makeatother

\fancyhf{} % clear default
\fancypagestyle{plain}{
  \fancyhf{}
  \fancyhead[L]{مدرسة التسامح الشاملة}
  % \fancyhead[L]{\includegraphics[height=1cm]{../../../images/logoTasamoh.png}}
  \fancyhead[R]{الأستاذ محمود اغبارية}
  \fancyfoot[C]{\thepage}
}

\fancyhead[L]{مدرسة التسامح الشاملة}
\fancyhead[R]{الأستاذ محمود اغبارية}
\fancyfoot[C]{\thepage}
% \date{\today}

\setcounter{tocdepth}{3} % only section subsection and subsubsection in TOC


% ----------------------


% \begin{document}

% \maketitle

% % \clearpage  % start TOC on a new page
% % \renewcommand{\contentsname}{جدول المحتويات}
% % \tableofcontents
% % \clearpage

% \part*{part 1} % the * prevents numbering
% \section*{مقدمة}
% \subsection*{مثال رياضي}
% \subsubsection*{مثال فرعي}
% \paragraph*{ paragraph 1}
% \subparagraph*{sub paragraph 1}

% \ifdetailed
% \begin{english}
% \begin{minted}{csharp}
% // C# Example
% \end{minted}
% \end{english}
% \fi

% OLD WAY
% \ifdetailed
% \begin{english}
% \begin{lstlisting}
% // C# Example
% \end{lstlisting}
% \end{english}
% \fi

% % \includegraphics[width=0.2\textwidth]{../../../images/DFAs/ex1_q1.png}



% \vspace{3cm}
% \begin{flushleft}
% أرجو لكم وقتًا ممتعًا.

% الأستاذ محمود اغبارية.
% \end{flushleft}


% \end{document}


\title{ورقة تمرن 9 للصف العاشر 10 \\ حلقة \textenglish{while}}

\begin{document}
\maketitle
\thispagestyle{fancy}

\setlist[enumerate]{itemsep=0.8em, topsep=0.5em}

\begin{enumerate}[itemsep=2em]

% -------------------------------------------------------
\item
تتبع الكود التالي بجدول متابعة بحيث يحتوي على عمود لكل متغير، وعمود لكل شرط، وعمود للطباعة.
\begin{boxCode}
\begin{english}
\begin{minted}{csharp}
private static void Main(string[] args)
{
    int num = 100;
    while(num >= 93)
    {
        if(num % 2 == 0)
            Console.WriteLine(num);
        num --;
    }
}
\end{minted}
\end{english}
\end{boxCode}

\ifwithsols
\begin{boxSolution}
\begin{center}
\begin{tabular}{c|c|c|c}
\textenglish{num} & \textenglish{num >= 93} & \textenglish{num \% 2 == 0} & \textenglish{Console.WriteLine} \\
\hline
100 & \textenglish{True} & \textenglish{True} & 100 \\ \hline
99 & \textenglish{True} & \textenglish{False} & \textemdash \\ \hline
98 & \textenglish{True} & \textenglish{True} & 98 \\ \hline
97 & \textenglish{True} & \textenglish{False} & \textemdash \\ \hline
96 & \textenglish{True} & \textenglish{True} & 96 \\ \hline
95 & \textenglish{True} & \textenglish{False} & \textemdash \\ \hline
94 & \textenglish{True} & \textenglish{True} & 94 \\ \hline
93 & \textenglish{True} & \textenglish{False} & \textemdash \\ \hline
92 & \textenglish{False} & \textemdash & \textemdash \\ \hline
\end{tabular}
\end{center}
\end{boxSolution}
\clearpage
\fi

\item
صاحب دكانة صغيرة يبيع الأغراض في دكانه، بحيث يكون ربحه من كل غرض يبيعه هو $20\%$ من سعره.
مثلًا: إذا كان يبيع غرضًا ما بسعر $100$ شاقل، فهذا يعني أنّه ربح من هذا الغرض $20 = 0.2 \times 100$ شاقلًا.

اكتب برنامجًا لهذا التاجر، بحيث في كل مرة يشتري فيها زبون من دكانه غرضًا ما فإنّه يُدخل لك المبلغ الذي دفعه هذا الزبون. \\
ينتهي الإدخال عندما يُدخل التاجر القيمة $-1$. \\
على البرنامج أن يطبع للمستخدم ربحه في ذلك اليوم.

\begin{boxExample}
    إذا أدخل التاجر في أحد الأيام المدخلات التالية: \\
    \begin{english}
    \begin{minted}{text}
    100
    50
    200
    3
    -1
    \end{minted}
    \end{english}
    فإنّ البرنامج يجب أن يطبع له: $70.6$  لأنّ: \\
    $0.2 \times 100 + 0.2 \times 50 + 0.2 \times 200 + 0.2 \times 3 = 70.6$
\end{boxExample}

\ifwithsols
\begin{boxSolution}[1]
\begin{english}
\begin{minted}{csharp}
public static void Main()
{
    double profit = 0;
    double price = 0;
    while (price != -1)
    {
        price = double.Parse(Console.ReadLine());
        profit += 0.2 * price;
    }
    Console.WriteLine(profit);
}
\end{minted}
\end{english}
\end{boxSolution}
\begin{boxSolution}[2]
\begin{english}
\begin{minted}{csharp}
public static void Main()
{
    double sum = 0;
    double price = 0;
    while (price != -1)
    {
        price = double.Parse(Console.ReadLine());
        sum += price;
    }
    Console.WriteLine(sum * 0.2);
}
\end{minted}
\end{english}
\end{boxSolution}
\clearpage
\fi

\item
اكتب برنامجًا يستقبل علامات طلاب صف معيّن، ينتهي الاستقبال عند إدخال الرقم $-1$. \\
على البرنامج أن يطبع معدّل علامات طلاب الصف.
\begin{boxHint}
    ستحتاج إلى متغيّرين: متغير يحسب المجموع، ومتغير يحسب عدد الطلاب.
\end{boxHint}

\ifwithsols
\begin{boxSolution}
نستخدم مجموعًا وعدّادًا ثم نقسم للحصول على المعدل.
\begin{english}
\begin{minted}{csharp}
public static void Main()
{
    double sum = 0;
    int count = 0;
    double grade = 0;
    while (grade != -1)
    {
        grade = double.Parse(Console.ReadLine());
        sum += grade;
        count++;
    }
    Console.WriteLine(sum / count);
}
\end{minted}
\end{english}
\end{boxSolution}
\fi

\end{enumerate}
\end{document}
