\documentclass[14pt]{extarticle}
% Full article preamble (duplicated, no common file)
\usepackage{fontspec}
\usepackage[a4paper,top=2.4cm,bottom=2.4cm,left=2.3cm,right=2.3cm]{geometry}
\usepackage{polyglossia}
\usepackage{amsmath}
\usepackage{amssymb}
\usepackage{xcolor}
\usepackage{fancyhdr}
\usepackage{graphicx}
\usepackage{listings}
\usepackage[most]{tcolorbox}
\usepackage{pifont}
\usepackage{enumitem}
\usepackage{titlesec}
\usepackage[bottom]{footmisc}
\usepackage{titling}
\usepackage{minted}
\usepackage{etoolbox}
\usepackage{array}
\usepackage{extsizes}

\newfontfamily\emoji{Segoe UI Emoji}

\pagestyle{fancy}

\setmainlanguage[numerals=western]{arabic}
\setotherlanguage{english}
\newfontfamily\arabicfont[Script=Arabic]{Amiri}
\newfontfamily\arabicfonttt[Script=Arabic]{Courier New}

\lstset{
  language=[Sharp]C,
  numbers=left,
  stepnumber=1,
  numbersep=8pt,
  frame=single,
  basicstyle=\ttfamily\small,
  keywordstyle=\color{blue},
  stringstyle=\color{red},
  commentstyle=\color{green!50!black}
}

\newif\ifdetailed
\ifdefined\setdetailed
  \setdetailed
\fi

\newif\ifwithsols
\ifdefined\setwithsols
  \setwithsols
\fi

% unified tcolorboxes for articles
\tcbset{colback=white, colframe=black, fonttitle=\bfseries, boxrule=0.8pt}
\newtcolorbox{boxDef}[1][]{colback=blue!5!white,colframe=blue!75!black,
  title={{\emoji📘} تعريف\ifx\\#1\\\else ~#1\fi :}}
\newtcolorbox{boxExercise}[1][]{colback=cyan!5!white,colframe=cyan!70!black,
  title={{\emoji🧩} تمرين\ifx\\#1\\\else ~#1\fi :}}
\newtcolorbox{boxExample}[1][]{colback=yellow!5!white,colframe=orange!90!black,
  title={{\emoji📝} مثال\ifx\\#1\\\else ~#1\fi :}}
\newtcolorbox{boxNote}[1][]{colback=gray!10!white,colframe=black,
  title={{\emoji✨} ملاحظة\ifx\\#1\\\else ~#1\fi :}}
\newtcolorbox{boxAttention}[1][]{colback=magenta!10!white,colframe=magenta!80!black,
  title={{\emoji🔔} تنبيه\ifx\\#1\\\else ~#1\fi :}}
\newtcolorbox{boxWarning}[1][]{colback=red!5!white,colframe=red!75!black,
  title={{\emoji⚡} ملاحظة هامة\ifx\\#1\\\else ~#1\fi :}}
\newtcolorbox{boxSolution}[1][]{colback=green!5!white,colframe=green!60!black,
  title={{\emoji✅} حل\ifx\\#1\\\else ~#1\fi :}}
\newtcolorbox{boxSymbol}[1][]{colback=purple!5!white,colframe=purple!70!black,
  title={{\emoji🔣} رمز\ifx\\#1\\\else ~#1\fi :}}
\newtcolorbox{boxHint}[1][]{colback=teal!5!white,colframe=teal!60!black,
  title={{\emoji💡} تلميح\ifx\\#1\\\else ~#1\fi :}}


\tcbset{simplecode/.style={ colback=gray!5, colframe=black!50, boxrule=0.4pt, arc=2pt, left=4pt,right=4pt,top=4pt,bottom=4pt}}
\newenvironment{boxCode}{\begin{tcolorbox}[simplecode]}{\end{tcolorbox}}

\newcolumntype{C}[1]{>{\centering\arraybackslash}p{#1}}

% redefine spaces after titles
\makeatletter
\renewcommand{\@maketitle}{%
  \begin{center}
    {\huge \bfseries \@title \par}%
    \vskip 0.2em % space between title and author
    {\large \@author \par}%
    % \vskip 0.2em % space between author and date
    % {\normalsize \@date \par}%
  \end{center}
}
\makeatother

\fancyhf{} % clear default
\fancypagestyle{plain}{
  \fancyhf{}
  \fancyhead[L]{مدرسة التسامح الشاملة}
  % \fancyhead[L]{\includegraphics[height=1cm]{../../../images/logoTasamoh.png}}
  \fancyhead[R]{الأستاذ محمود اغبارية}
  \fancyfoot[C]{\thepage}
}

\fancyhead[L]{مدرسة التسامح الشاملة}
\fancyhead[R]{الأستاذ محمود اغبارية}
\fancyfoot[C]{\thepage}
% \date{\today}

\setcounter{tocdepth}{3} % only section subsection and subsubsection in TOC


% ----------------------


% \begin{document}

% \maketitle

% % \clearpage  % start TOC on a new page
% % \renewcommand{\contentsname}{جدول المحتويات}
% % \tableofcontents
% % \clearpage

% \part*{part 1} % the * prevents numbering
% \section*{مقدمة}
% \subsection*{مثال رياضي}
% \subsubsection*{مثال فرعي}
% \paragraph*{ paragraph 1}
% \subparagraph*{sub paragraph 1}

% \ifdetailed
% \begin{english}
% \begin{minted}{csharp}
% // C# Example
% \end{minted}
% \end{english}
% \fi

% OLD WAY
% \ifdetailed
% \begin{english}
% \begin{lstlisting}
% // C# Example
% \end{lstlisting}
% \end{english}
% \fi

% % \includegraphics[width=0.2\textwidth]{../../../images/DFAs/ex1_q1.png}



% \vspace{3cm}
% \begin{flushleft}
% أرجو لكم وقتًا ممتعًا.

% الأستاذ محمود اغبارية.
% \end{flushleft}


% \end{document}


\ifwithsols
\title{حل ورقة تمرين 8 - المصفوفات الأحادية \\ قسم 2 - أسئلة عكس، دمج، إزاحة}
\else
\title{ورقة تمرين 8 - المصفوفات الأحادية \\ قسم 2 - أسئلة عكس، دمج، إزاحة}
\fi

\begin{document}

\maketitle
\thispagestyle{fancy}

\begin{enumerate}[itemsep=2em]

    % ======================================================
    % 1. تبديل بسيط - أول وأخير
    % ======================================================

    \item
    اكتب مقطع برنامج يبدّل العنصر الأول مع العنصر الأخير في مصفوفة أعداد صحيحة \textenglish{arr}.\\
    \textbf{مثال:} إذا كانت المصفوفة \textenglish{\{5, 10, 15, 20\}}، تصبح بعد التبديل \textenglish{\{20, 10, 15, 5\}}.
    \ifwithsols
    \begin{boxSolution}
    \begin{english}
    \begin{minted}{csharp}
int temp = arr[0];
arr[0] = arr[arr.Length - 1];
arr[arr.Length - 1] = temp;
    \end{minted}
    \end{english}
    \end{boxSolution}
    \fi

    % ======================================================
    % 2. طباعة عكسية بسيطة
    % ======================================================

    \item
    اكتب مقطع برنامج يطبع عناصر مصفوفة أعداد صحيحة \textenglish{arr} بترتيب عكسي (من الأخير إلى الأول).\\
    على البرنامج أن يطبع كل عنصر بسطر منفصل.
    \ifwithsols
    \begin{boxSolution}
    \begin{english}
    \begin{minted}{csharp}
for (int i = arr.Length - 1; i >= 0; i--)
{
    Console.WriteLine(arr[i]);
}
    \end{minted}
    \end{english}
    \end{boxSolution}
    \clearpage
    \fi

    % ======================================================
    % 9. فحص التساوي بين مصفوفتين
    % ======================================================

    \item
    اكتب مقطع برنامج يفحص هل مصفوفتان من الأعداد الصحيحة \textenglish{arr1} و\textenglish{arr2} متساويتان (نفس الطول ونفس العناصر بنفس الترتيب).\\
    على البرنامج أن يطبع \textenglish{"Equal"} إذا كانتا متساويتين و\textenglish{"Not Equal"} خلاف ذلك.
    \ifwithsols
    \begin{boxSolution}
    \begin{english}
    \begin{minted}{csharp}
bool areEqual = true;

if (arr1.Length != arr2.Length)
    areEqual = false;
else
{
    for (int i = 0; i < arr1.Length; i++)
    {
        if (arr1[i] != arr2[i])
        {
            areEqual = false;
            break;
        }
    }
}

if (areEqual)
    Console.WriteLine("Equal");
else
    Console.WriteLine("Not Equal");
    \end{minted}
    \end{english}
    \end{boxSolution}
    \clearpage
    \fi

    % ======================================================
    % 3. مجموع مرحلي من النهاية إلى البداية
    % ======================================================

    \item
    اكتب مقطع برنامج يطبع المجموع المرحلي لعناصر مصفوفة أعداد صحيحة \textenglish{data} بترتيب عكسي (من الأخير إلى الأول).\\
    \textbf{مثال:} إذا كانت المصفوفة \textenglish{\{5, 10, 15, 20\}}، يطبع البرنامج:
    \begin{english}
    \begin{boxCode}
    20 \\
    35 (20 + 15) \\
    45 (20 + 15 + 10) \\
    60 (20 + 15 + 10 + 5)
    \end{boxCode}
    \end{english}
    \ifwithsols
    \begin{boxSolution}
    \begin{english}
    \begin{minted}{csharp}
int sum = 0;
for (int i = data.Length - 1; i >= 0; i--)
{
    sum += data[i];
    Console.WriteLine(sum);
}
    \end{minted}
    \end{english}
    \end{boxSolution}
    \fi

    % ======================================================
    % 4. نسخ بترتيب عكسي
    % ======================================================

    \item
    اكتب مقطع برنامج ينسخ مصفوفة أعداد صحيحة \textenglish{source} إلى مصفوفة جديدة \textenglish{reversed} بترتيب عكسي.\\
    \textbf{مثال:} إذا كانت \textenglish{source = \{10, 20, 30, 40\}}، تصبح \textenglish{reversed = \{40, 30, 20, 10\}}.
    \ifwithsols
    \begin{boxSolution}
    \begin{english}
    \begin{minted}{csharp}
int[] reversed = new int[source.Length];
for (int i = 0; i < source.Length; i++)
{
    reversed[i] = source[source.Length - 1 - i];
}
    \end{minted}
    \end{english}
    \end{boxSolution}
    \clearpage
    \fi

    % ======================================================
    % 5. تبديل عناصر متجاورة
    % ======================================================

    \item
    اكتب مقطع برنامج يبدّل كل عنصرين متجاورين في مصفوفة أعداد صحيحة \textenglish{numbers} (افترض أن طول المصفوفة زوجي).\\
    \textbf{مثال:} إذا كانت المصفوفة \textenglish{\{1, 2, 3, 4, 5, 6\}}، تصبح بعد التبديل \textenglish{\{2, 1, 4, 3, 6, 5\}}.
    \ifwithsols
    \begin{boxSolution}
    \begin{english}
    \begin{minted}{csharp}
for (int i = 0; i < numbers.Length - 1; i += 2)
{
    int temp = numbers[i];
    numbers[i] = numbers[i + 1];
    numbers[i + 1] = temp;
}
    \end{minted}
    \end{english}
    \end{boxSolution}
    \clearpage
    \fi

    % ======================================================
    % 6. فحص التماثل - بسيط
    % ======================================================

    \item
    اكتب مقطع برنامج يفحص هل مصفوفة أعداد صحيحة \textenglish{arr} متماثلة (أي العنصر الأول يساوي العنصر الأخير، والثاني يساوي قبل الأخير، وهكذا).\\
    على البرنامج أن يطبع \textenglish{"Palindrome"} إذا كانت متماثلة و\textenglish{"Not Palindrome"} خلاف ذلك.\\
    \textbf{مثال:} \textenglish{\{1, 2, 3, 2, 1\}} متماثلة، لكن \textenglish{\{1, 2, 3, 4\}} ليست متماثلة.
    \ifwithsols
    \begin{boxSolution}
    \begin{english}
    \begin{minted}{csharp}
bool isPalin = true;
for (int i = 0; i < arr.Length / 2; i++)
{
    if (arr[i] != arr[arr.Length - 1 - i])
    {
        isPalin = false;
        break;
    }
}

if (isPalin)
    Console.WriteLine("Palindrome");
else
    Console.WriteLine("Not Palindrome");
    \end{minted}
    \end{english}
    \end{boxSolution}
    \clearpage
    \fi

    \item
    لعب أمجد وسمير  عددًا من الجولات في لعبة فيديو. تم تسجيل نقاط كل جولة في مصفوفة \textenglish{amjadScores} لأمجد و\textenglish{samirScores} لسمير (كلاهما لعبا نفس عدد الجولات).\\
    اكتب مقطع برنامج يُنشئ مصفوفة نصوص جديدة، \textenglish{roundWinners}، تحتوي على اسم الفائز في كل جولة: "Amjad" إذا كانت نقطة أمجد أعلى، "Samir" إذا كانت نقطة سمير أعلى، و"Tie" إذا تعادلا.\\
    \textbf{مثال:} إذا كانت \textenglish{amjadScores = \{10, 20, 15\}} و\textenglish{samirScores = \{15, 15, 15\}}، تصبح \textenglish{roundWinners = \{"Samir", "Amjad", "Tie"\}}.
    \ifwithsols
    \begin{boxSolution}
    \begin{english}
    \begin{minted}{csharp}
string[] roundWinners = new string[amjadScores.Length];
for (int i = 0; i < amjadScores.Length; i++)
{
    if (amjadScores[i] > samirScores[i])
        roundWinners[i] = "Amjad";
    else if (amjadScores[i] < samirScores[i])
        roundWinners[i] = "Samir";
    else
        roundWinners[i] = "Tie";
}
    \end{minted}
    \end{english}
    \end{boxSolution}
    \clearpage
    \fi

    % ======================================================
    % 7. عكس ترتيب بدون مصفوفة إضافية
    % ======================================================

    \item
    اكتب مقطع برنامج يعكس ترتيب عناصر مصفوفة أعداد صحيحة \textenglish{arr} \textbf{داخل نفس المصفوفة} دون الاستعانة بمصفوفة إضافية (استخدم متغيراً مساعداً واحداً فقط للتبديل).\\
    \textbf{مثال:} إذا كانت \textenglish{arr = \{1, 2, 3, 4, 5\}}، تصبح \textenglish{arr = \{5, 4, 3, 2, 1\}}.
    \ifwithsols
    \begin{boxSolution}
    \begin{english}
    \begin{minted}{csharp}
for (int i = 0; i < arr.Length / 2; i++)
{
    int temp = arr[i];
    arr[i] = arr[arr.Length - 1 - i];
    arr[arr.Length - 1 - i] = temp;
}
    \end{minted}
    \end{english}
    \end{boxSolution}
    \fi

    % ======================================================
    % 10. إزاحة إلى اليمين
    % ======================================================

    \item
    اكتب مقطع برنامج يزيح جميع عناصر مصفوفة أعداد صحيحة \textenglish{arr} بمقدار خانة واحدة إلى اليمين.\\
    العنصر الأخير ينتقل إلى الخانة الأولى.\\
    \textbf{مثال:} إذا كانت \textenglish{arr = \{1, 2, 3, 4, 5\}}، تصبح \textenglish{arr = \{5, 1, 2, 3, 4\}}.
    \ifwithsols
    \begin{boxSolution}
    \begin{english}
    \begin{minted}{csharp}
int last = arr[arr.Length - 1];
for (int i = arr.Length - 1; i > 0; i--)
{
    arr[i] = arr[i - 1];
}
arr[0] = last;
    \end{minted}
    \end{english}
    \end{boxSolution}
    \fi

    % ======================================================
    % 11. دمج مصفوفتين - متتابع
    % ======================================================

    \clearpage
    \item
    اكتب مقطع برنامج يدمج مصفوفتين من الأعداد الصحيحة \textenglish{A} و\textenglish{B} في مصفوفة جديدة \textenglish{result} تحتوي على جميع عناصر \textenglish{A} متبوعة بجميع عناصر \textenglish{B}.\\
    \textbf{مثال:} إذا كانت \textenglish{A = \{1, 2, 3\}} و\textenglish{B = \{4, 5\}}، تصبح \textenglish{result = \{1, 2, 3, 4, 5\}}.
    \ifwithsols
    \begin{boxSolution}
    \begin{english}
    \begin{minted}{csharp}
int[] result = new int[A.Length + B.Length];

for (int i = 0; i < A.Length; i++)
{
    result[i] = A[i];
}

for (int i = 0; i < B.Length; i++)
{
    result[A.Length + i] = B[i];
}
    \end{minted}
    \end{english}
    \end{boxSolution}
    \clearpage
    \fi

    % ======================================================
    % 12. دمج مصفوفتين - بالتناوب
    % ======================================================

    \item
    اكتب مقطع برنامج يدمج مصفوفتين من الأعداد الصحيحة \textenglish{A} و\textenglish{B} (لهما نفس الطول) في مصفوفة جديدة \textenglish{result} تحتوي على عناصر من المصفوفتين بالتناوب (عنصر من \textenglish{A}، عنصر من \textenglish{B}، عنصر من \textenglish{A}، وهكذا).\\
    \textbf{مثال:} إذا كانت \textenglish{A = \{1, 3, 5\}} و\textenglish{B = \{2, 4, 6\}}، تصبح \textenglish{result = \{1, 2, 3, 4, 5, 6\}}.
    \ifwithsols
    \begin{boxSolution}
    \begin{english}
    \begin{minted}{csharp}
int[] result = new int[A.Length + B.Length];

for (int i = 0; i < A.Length; i++)
{
    result[2 * i] = A[i];
    result[2 * i + 1] = B[i];
}
    \end{minted}
    \end{english}
    \end{boxSolution}
    \clearpage
    \fi

    % ======================================================
    % 13. إزاحة إلى اليسار n مرات
    % ======================================================

    \item
    اكتب مقطع برنامج يزيح جميع عناصر مصفوفة أعداد صحيحة \textenglish{arr} بمقدار \textenglish{n} خانة إلى اليسار داخل نفس المصفوفة.\\
    العناصر التي تخرج من أول المصفوفة تنتقل إلى آخرها.\\
    \textbf{مثال:} إذا كانت المصفوفة \textenglish{\{1, 2, 3, 4, 5\}} و \textenglish{n = 2}، تصبح \textenglish{\{3, 4, 5, 1, 2\}}.
    \ifwithsols
    \begin{boxSolution}
    \begin{english}
    \begin{minted}{csharp}
n = n % arr.Length; // Handle cases where n > arr.Length
for (int shift = 0; shift < n; shift++)
{
    int first = arr[0];
    for (int i = 0; i < arr.Length - 1; i++)
    {
        arr[i] = arr[i + 1];
    }
    arr[arr.Length - 1] = first;
}
    \end{minted}
    \end{english}
    \end{boxSolution}
    \clearpage
    \fi

    % ======================================================
    % 14. نسخ عناصر بترتيب عكسي مع شرط
    % ======================================================

    \item
    اكتب مقطع برنامج ينسخ فقط الأعداد الزوجية من مصفوفة أعداد صحيحة \textenglish{source} إلى مصفوفة جديدة بترتيب عكسي.\\
    \textbf{مثال:} إذا كانت \textenglish{source = \{1, 2, 3, 4, 5, 6, 7, 8\}}، تحتوي المصفوفة الجديدة على \textenglish{\{8, 6, 4, 2\}}.\\
    \textbf{تلميح:} أولاً عُدّ الأعداد الزوجية لتعرف حجم المصفوفة الجديدة.
    \ifwithsols
    \begin{boxSolution}
    \begin{english}
    \begin{minted}{csharp}
// Count even numbers first
int evenCount = 0;
for (int i = 0; i < source.Length; i++)
{
    if (source[i] % 2 == 0)
        evenCount++;
}

// Create new array and fill it in reverse
int[] evenReversed = new int[evenCount];
int index = 0;
for (int i = source.Length - 1; i >= 0; i--)
{
    if (source[i] % 2 == 0)
    {
        evenReversed[index] = source[i];
        index++;
    }
}
    \end{minted}
    \end{english}
    \end{boxSolution}
    \clearpage
    \fi

    % ======================================================
    % 15. عكس مصفوفة في مجالات محددة
    % ======================================================

    \item
    اكتب مقطع برنامج يعكس ترتيب عناصر مصفوفة أعداد صحيحة \textenglish{arr} فقط بين المؤشرين \textenglish{start} و\textenglish{end} (شامل الطرفين) داخل نفس المصفوفة.\\
    \textbf{مثال:} إذا كانت \textenglish{arr = \{1, 2, 3, 4, 5, 6, 7\}} و\textenglish{start = 2} و\textenglish{end = 5}، تصبح \textenglish{arr = \{1, 2, 6, 5, 4, 3, 7\}}.\\
    (العناصر من المؤشر 2 إلى 5 هي \textenglish{\{3, 4, 5, 6\}} وبعد العكس تصبح \textenglish{\{6, 5, 4, 3\}}).
    \ifwithsols
    \begin{boxSolution}
    \begin{english}
    \begin{minted}{csharp}
int left = start;
int right = end;

while (left < right)
{
    int temp = arr[left];
    arr[left] = arr[right];
    arr[right] = temp;
    left++;
    right--;
}
    \end{minted}
    \end{english}
    \end{boxSolution}
    \fi

    \clearpage
    \item
    في مسابقة رياضية، يتنافس فريقان: الفريق الأزرق والفريق الأحمر. نقاط كل واحد من الفريقين مخزنة في مصفوفة: \textenglish{blueScores} تحتوي على نقاط الفريق الأزرق في كل جولة، و\textenglish{redScores} تحتوي على نقاط الفريق الأحمر في كل جولة (كلا الفريقين لعبا نفس عدد الجولات).\\
    يريد المنظمون إعداد تقرير يعرض النقاط بالتناوب: جولة الفريق الأزرق الأولى، ثم جولة الفريق الأحمر الأولى، ثم جولة الفريق الأزرق الثانية، وهكذا.\\
    اكتب مقطع برنامج يدمج المصفوفتين في مصفوفة جديدة \textenglish{combinedScores} بهذا الترتيب المتناوب.\\
    \textbf{مثال:} إذا كانت \textenglish{blueScores = \{10, 15, 20\}} و\textenglish{redScores = \{12, 18, 22\}}، تصبح \textenglish{combinedScores = \{10, 12, 15, 18, 20, 22\}}.
    \ifwithsols
    \begin{boxSolution}
    \begin{english}
    \begin{minted}{csharp}
int[] combinedScores = new int[blueScores.Length + redScores.Length];

for (int i = 0; i < blueScores.Length; i++)
{
    combinedScores[2 * i] = blueScores[i];
    combinedScores[2 * i + 1] = redScores[i];
}
    \end{minted}
    \end{english}
    \end{boxSolution}
    \fi

\end{enumerate}

\end{document}
