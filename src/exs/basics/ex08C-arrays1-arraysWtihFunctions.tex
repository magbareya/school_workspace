\documentclass[14pt]{extarticle}
% Full article preamble (duplicated, no common file)
\usepackage{fontspec}
\usepackage[a4paper,top=2.4cm,bottom=2.4cm,left=2.3cm,right=2.3cm]{geometry}
\usepackage{polyglossia}
\usepackage{amsmath}
\usepackage{amssymb}
\usepackage{xcolor}
\usepackage{fancyhdr}
\usepackage{graphicx}
\usepackage{listings}
\usepackage[most]{tcolorbox}
\usepackage{pifont}
\usepackage{enumitem}
\usepackage{titlesec}
\usepackage[bottom]{footmisc}
\usepackage{titling}
\usepackage{minted}
\usepackage{etoolbox}
\usepackage{array}
\usepackage{extsizes}

\newfontfamily\emoji{Segoe UI Emoji}

\pagestyle{fancy}

\setmainlanguage[numerals=western]{arabic}
\setotherlanguage{english}
\newfontfamily\arabicfont[Script=Arabic]{Amiri}
\newfontfamily\arabicfonttt[Script=Arabic]{Courier New}

\lstset{
  language=[Sharp]C,
  numbers=left,
  stepnumber=1,
  numbersep=8pt,
  frame=single,
  basicstyle=\ttfamily\small,
  keywordstyle=\color{blue},
  stringstyle=\color{red},
  commentstyle=\color{green!50!black}
}

\newif\ifdetailed
\ifdefined\setdetailed
  \setdetailed
\fi

\newif\ifwithsols
\ifdefined\setwithsols
  \setwithsols
\fi

% unified tcolorboxes for articles
\tcbset{colback=white, colframe=black, fonttitle=\bfseries, boxrule=0.8pt}
\newtcolorbox{boxDef}[1][]{colback=blue!5!white,colframe=blue!75!black,
  title={{\emoji📘} تعريف\ifx\\#1\\\else ~#1\fi :}}
\newtcolorbox{boxExercise}[1][]{colback=cyan!5!white,colframe=cyan!70!black,
  title={{\emoji🧩} تمرين\ifx\\#1\\\else ~#1\fi :}}
\newtcolorbox{boxExample}[1][]{colback=yellow!5!white,colframe=orange!90!black,
  title={{\emoji📝} مثال\ifx\\#1\\\else ~#1\fi :}}
\newtcolorbox{boxNote}[1][]{colback=gray!10!white,colframe=black,
  title={{\emoji✨} ملاحظة\ifx\\#1\\\else ~#1\fi :}}
\newtcolorbox{boxAttention}[1][]{colback=magenta!10!white,colframe=magenta!80!black,
  title={{\emoji🔔} تنبيه\ifx\\#1\\\else ~#1\fi :}}
\newtcolorbox{boxWarning}[1][]{colback=red!5!white,colframe=red!75!black,
  title={{\emoji⚡} ملاحظة هامة\ifx\\#1\\\else ~#1\fi :}}
\newtcolorbox{boxSolution}[1][]{colback=green!5!white,colframe=green!60!black,
  title={{\emoji✅} حل\ifx\\#1\\\else ~#1\fi :}}
\newtcolorbox{boxSymbol}[1][]{colback=purple!5!white,colframe=purple!70!black,
  title={{\emoji🔣} رمز\ifx\\#1\\\else ~#1\fi :}}
\newtcolorbox{boxHint}[1][]{colback=teal!5!white,colframe=teal!60!black,
  title={{\emoji💡} تلميح\ifx\\#1\\\else ~#1\fi :}}


\tcbset{simplecode/.style={ colback=gray!5, colframe=black!50, boxrule=0.4pt, arc=2pt, left=4pt,right=4pt,top=4pt,bottom=4pt}}
\newenvironment{boxCode}{\begin{tcolorbox}[simplecode]}{\end{tcolorbox}}

\newcolumntype{C}[1]{>{\centering\arraybackslash}p{#1}}

% redefine spaces after titles
\makeatletter
\renewcommand{\@maketitle}{%
  \begin{center}
    {\huge \bfseries \@title \par}%
    \vskip 0.2em % space between title and author
    {\large \@author \par}%
    % \vskip 0.2em % space between author and date
    % {\normalsize \@date \par}%
  \end{center}
}
\makeatother

\fancyhf{} % clear default
\fancypagestyle{plain}{
  \fancyhf{}
  \fancyhead[L]{مدرسة التسامح الشاملة}
  % \fancyhead[L]{\includegraphics[height=1cm]{../../../images/logoTasamoh.png}}
  \fancyhead[R]{الأستاذ محمود اغبارية}
  \fancyfoot[C]{\thepage}
}

\fancyhead[L]{مدرسة التسامح الشاملة}
\fancyhead[R]{الأستاذ محمود اغبارية}
\fancyfoot[C]{\thepage}
% \date{\today}

\setcounter{tocdepth}{3} % only section subsection and subsubsection in TOC


% ----------------------


% \begin{document}

% \maketitle

% % \clearpage  % start TOC on a new page
% % \renewcommand{\contentsname}{جدول المحتويات}
% % \tableofcontents
% % \clearpage

% \part*{part 1} % the * prevents numbering
% \section*{مقدمة}
% \subsection*{مثال رياضي}
% \subsubsection*{مثال فرعي}
% \paragraph*{ paragraph 1}
% \subparagraph*{sub paragraph 1}

% \ifdetailed
% \begin{english}
% \begin{minted}{csharp}
% // C# Example
% \end{minted}
% \end{english}
% \fi

% OLD WAY
% \ifdetailed
% \begin{english}
% \begin{lstlisting}
% // C# Example
% \end{lstlisting}
% \end{english}
% \fi

% % \includegraphics[width=0.2\textwidth]{../../../images/DFAs/ex1_q1.png}



% \vspace{3cm}
% \begin{flushleft}
% أرجو لكم وقتًا ممتعًا.

% الأستاذ محمود اغبارية.
% \end{flushleft}


% \end{document}


\ifwithsols
\title{حل ورقة تمرين 8 - المصفوفات الأحادية \\ قسم 3 - عمليات خارجية مع مصفوفات}
\else
\title{ورقة تمرين 8 - المصفوفات الأحادية \\ قسم 3 - عمليات خارجية مع مصفوفات}
\clearpage
\fi

\begin{document}

\maketitle
\thispagestyle{fancy}

\begin{enumerate}[itemsep=2em]

    % ======================================================
    % 1. GetMax - إيجاد أكبر قيمة
    % ======================================================

    \item
    اكتب عملية خارجية \textenglish{int GetMax(int[] arr)} تستقبل مصفوفة أعداد صحيحة وتعيد أكبر قيمة فيها.\\
    \textbf{مثال:} إذا كانت المصفوفة \textenglish{\{5, 12, 8, 20, 3\}}، تعيد العملية 20.
    \ifwithsols
    \begin{boxSolution}
    \begin{english}
    \begin{minted}{csharp}
public static int GetMax(int[] arr)
{
    int max = arr[0];
    for (int i = 1; i < arr.Length; i++)
    {
        if (arr[i] > max)
            max = arr[i];
    }
    return max;
}
    \end{minted}
    \end{english}
    \end{boxSolution}
    \clearpage
    \fi

    % ======================================================
    % 2. CountNegative - عد الأعداد السالبة
    % ======================================================

    \item
    اكتب عملية خارجية \textenglish{int CountNegative(int[] arr)} تستقبل مصفوفة أعداد صحيحة وتعيد عدد الأعداد السالبة فيها.\\
    \textbf{مثال:} في المصفوفة \textenglish{\{5, -3, 8, -7, 0, -1\}} يوجد 3 أعداد سالبة.
    \ifwithsols
    \begin{boxSolution}
    \begin{english}
    \begin{minted}{csharp}
public static int CountNegative(int[] arr)
{
    int count = 0;
    for (int i = 0; i < arr.Length; i++)
    {
        if (arr[i] < 0)
            count++;
    }
    return count;
}
    \end{minted}
    \end{english}
    \end{boxSolution}
    \clearpage
    \fi

    % ======================================================
    % 3. CountOccurrences - عد تكرار عدد
    % ======================================================

    \item
    اكتب عملية خارجية \textenglish{int CountOccurrences(int[] arr, int value)} تستقبل مصفوفة أعداد صحيحة وعدداً صحيحاً، وتعيد عدد مرات ظهور العدد في المصفوفة.\\
    \textbf{مثال:} في المصفوفة \textenglish{\{5, 12, 8, 12, 3, 12\}}، العدد 12 يظهر 3 مرات.
    \ifwithsols
    \begin{boxSolution}
    \begin{english}
    \begin{minted}{csharp}
public static int CountOccurrences(int[] arr, int value)
{
    int count = 0;
    for (int i = 0; i < arr.Length; i++)
    {
        if (arr[i] == value)
            count++;
    }
    return count;
}
    \end{minted}
    \end{english}
    \end{boxSolution}
    \clearpage
    \fi

    % ======================================================
    % 4. AreArraysEqual - فحص تساوي مصفوفتين
    % ======================================================

    \item
    اكتب عملية خارجية \textenglish{bool AreArraysEqual(int[] arr1, int[] arr2)} تستقبل مصفوفتين من الأعداد الصحيحة وتعيد \textenglish{true} إذا كانتا متساويتين (نفس الطول ونفس العناصر بنفس الترتيب)، وإلا تعيد \textenglish{false}.\\
    \textbf{مثال:}
    \begin{itemize}
        \item \textenglish{\{1, 2, 3\}} و \textenglish{\{1, 2, 3\}} $\rightarrow$ \textenglish{true}
        \item \textenglish{\{1, 2, 3\}} و \textenglish{\{1, 2, 4\}} $\rightarrow$ \textenglish{false}
    \end{itemize}
    \ifwithsols
    \begin{boxSolution}
    \begin{english}
    \begin{minted}{csharp}
public static bool AreArraysEqual(int[] arr1, int[] arr2)
{
    if (arr1.Length != arr2.Length)
        return false;

    for (int i = 0; i < arr1.Length; i++)
    {
        if (arr1[i] != arr2[i])
            return false;
    }

    return true;
}
    \end{minted}
    \end{english}
    \end{boxSolution}
    \clearpage
    \fi

    % ======================================================
    % 5. FindIndex - بحث عن قيمة وإعادة موقعها
    % ======================================================

    \item
    اكتب عملية خارجية \textenglish{int FindIndex(int[] arr, int val)} تعيد رقم الخانة (Index) لأول ظهور للقيمة \textenglish{val} في المصفوفة. إذا لم تجدها تعيد $-1$.\\    \textbf{مثال:} إذا كانت المصفوفة \textenglish{\{10, 20, 30, 40\}}:
    \begin{itemize}
        \item \textenglish{FindIndex(arr, 30)} تعيد 2
        \item \textenglish{FindIndex(arr, 50)} تعيد $-1$
    \end{itemize}
    \ifwithsols
    \begin{boxSolution}
    \begin{english}
    \begin{minted}{csharp}
public static int FindIndex(int[] arr, int val)
{
    for (int i = 0; i < arr.Length; i++)
    {
        if (arr[i] == val)
            return i;
    }
    return -1;
}
    \end{minted}
    \end{english}
    \end{boxSolution}
    \clearpage
\fi

    % ======================================================
    % 6. ReplaceNegativeWithZero - استبدال السالب بالصفر
    % ======================================================

    \item
    اكتب عملية خارجية \textenglish{void ReplaceNegativeWithZero(int[] arr)} تستقبل مصفوفة أعداد صحيحة وتعدلها بحيث تستبدل كل عدد سالب فيها بالصفر.\\    \textbf{مثال:} إذا كانت المصفوفة \textenglish{\{5, -3, 0, -7, 2\}}، تصبح بعد استدعاء العملية \textenglish{\{5, 0, 0, 0, 2\}}.
    \ifwithsols
    \begin{boxSolution}
    \begin{english}
    \begin{minted}{csharp}
public static void ReplaceNegativeWithZero(int[] arr)
{
    for (int i = 0; i < arr.Length; i++)
    {
        if (arr[i] < 0)
            arr[i] = 0;
    }
}
    \end{minted}
    \end{english}
    \end{boxSolution}
    \clearpage
\fi

    % ======================================================
    % 3. IsSorted - فحص الترتيب التصاعدي
    % ======================================================

    \item
    اكتب عملية خارجية \textenglish{bool IsSorted(int[] arr)} تعيد \textenglish{true} فقط إذا كانت عناصر المصفوفة مرتبة تصاعدياً بالكامل (من الأصغر للأكبر).\\    \textbf{مثال:}
    \begin{itemize}
        \item \textenglish{\{2, 5, 7, 9\}} $\leftarrow$ تعيد \textenglish{true}
        \item \textenglish{\{2, 5, 3, 9\}} $\leftarrow$ تعيد \textenglish{false}
    \end{itemize}
    \ifwithsols
    \begin{boxSolution}
    \begin{english}
    \begin{minted}{csharp}
public static bool IsSorted(int[] arr)
{
    for (int i = 0; i < arr.Length - 1; i++)
    {
        if (arr[i] > arr[i + 1])
            return false;
    }
    return true;
}
    \end{minted}
    \end{english}
    \end{boxSolution}
    \clearpage
\fi

    % ======================================================
    % 4. IsPalindrome - فحص التماثل
    % ======================================================

    \item
    اكتب عملية خارجية \textenglish{bool IsPalindrome(int[] arr)} تفحص هل مصفوفة الأعداد الصحيحة متناظرة (تقرأ من اليمين كاليسار).\\
    \textbf{مثال:}
    \begin{itemize}
        \item \textenglish{\{1, 2, 3, 2, 1\}} متناظرة $\leftarrow$ تعيد \textenglish{true}
        \item \textenglish{\{1, 2, 3, 4\}} غير متناظرة $\leftarrow$ تعيد \textenglish{false}
    \end{itemize}
    \ifwithsols
    \begin{boxSolution}
    \begin{english}
    \begin{minted}{csharp}
public static bool IsPalindrome(int[] arr)
{
    for (int i = 0; i < arr.Length / 2; i++)
    {
        if (arr[i] != arr[arr.Length - 1 - i])
            return false;
    }
    return true;
}
    \end{minted}
    \end{english}
    \end{boxSolution}
    \clearpage
\fi

    % ======================================================
    % 5. MergeArrays - دمج مصفوفتين متتابع
    % (نقل من ex08B)
    % ======================================================

    \item
    اكتب عملية خارجية \textenglish{int[] MergeArrays(int[] A, int[] B)} تستقبل مصفوفتين من الأعداد الصحيحة وتعيد مصفوفة جديدة تحتوي على جميع عناصر \textenglish{A} متبوعة بجميع عناصر \textenglish{B}.\\
    \textbf{مثال:} إذا كانت \textenglish{A = \{1, 2, 3\}} و\textenglish{B = \{4, 5\}}، تعيد العملية \textenglish{\{1, 2, 3, 4, 5\}}.
    \ifwithsols
    \begin{boxSolution}
    \begin{english}
    \begin{minted}{csharp}
public static int[] MergeArrays(int[] A, int[] B)
{
    int[] result = new int[A.Length + B.Length];

    for (int i = 0; i < A.Length; i++)
        result[i] = A[i];

    for (int i = 0; i < B.Length; i++)
        result[A.Length + i] = B[i];

    return result;
}
    \end{minted}
    \end{english}
    \end{boxSolution}
    \clearpage
\fi

    % ======================================================
    % 6. ReverseArray - عكس بدون مصفوفة إضافية
    % ======================================================

    \item
    اكتب عملية خارجية \textenglish{void ReverseArray(int[] arr)} لقلب ترتيب عناصر المصفوفة \textbf{بدون} استخدام مصفوفة إضافية.\\
    \textbf{مثال:} إذا كانت \textenglish{arr = \{1, 2, 3, 4, 5\}}، تصبح بعد استدعاء العملية \textenglish{arr = \{5, 4, 3, 2, 1\}}.
    \ifwithsols
    \begin{boxSolution}
    \begin{english}
    \begin{minted}{csharp}
public static void ReverseArray(int[] arr)
{
    for (int i = 0; i < arr.Length / 2; i++)
    {
        int temp = arr[i];
        arr[i] = arr[arr.Length - 1 - i];
        arr[arr.Length - 1 - i] = temp;
    }
}
    \end{minted}
    \end{english}
    \end{boxSolution}
    \fi
    \clearpage

    % ======================================================
    % 7. ReverseRange - عكس في مجال محدد
    % (نقل من ex08B)
    % ======================================================

    \item
    اكتب عملية خارجية \textenglish{void ReverseRange(int[] arr, int start, int end)} تعكس ترتيب عناصر المصفوفة فقط بين المؤشرين \textenglish{start} و\textenglish{end} (شامل الطرفين) داخل نفس المصفوفة.\\
    \textbf{مثال:} إذا كانت \textenglish{arr = \{1, 2, 3, 4, 5, 6, 7\}} و\textenglish{start = 2} و\textenglish{end = 5}، تصبح \textenglish{arr = \{1, 2, 6, 5, 4, 3, 7\}}.\\
    (العناصر من المؤشر 2 إلى 5 هي \textenglish{\{3, 4, 5, 6\}} وبعد العكس تصبح \textenglish{\{6, 5, 4, 3\}}).
    \ifwithsols
    \begin{boxSolution}
    \begin{english}
    \begin{minted}{csharp}
public static void ReverseRange(int[] arr, int start, int end)
{
    while (start < end)
    {
        int temp = arr[start];
        arr[start] = arr[end];
        arr[end] = temp;
        start++;
        end--;
    }
}
    \end{minted}
    \end{english}
    \end{boxSolution}
    \clearpage
\fi

    % ======================================================
    % 8. HasIncreasingSequence - فحص تسلسل متزايد
    % ======================================================

    \item
    اكتب عملية خارجية \textenglish{bool HasIncreasingSequence(int[] arr)} تعيد \textenglish{true} إذا وجدت في المصفوفة 3 أرقام متتالية (بجانب بعضها) مرتبة تصاعدياً بصرامة ($a < b < c$).\\
    \textbf{مثال:} في المصفوفة \textenglish{\{5, 2, 4, 6, 1\}} يوجد التسلسل \textenglish{\{2, 4, 6\}} فتعيد \textenglish{true}.
    \ifwithsols
    \begin{boxSolution}
    \begin{english}
    \begin{minted}{csharp}
public static bool HasIncreasingSequence(int[] arr)
{
    if (arr.Length < 3) return false;

    for (int i = 0; i < arr.Length - 2; i++)
    {
        if (arr[i] < arr[i + 1] && arr[i + 1] < arr[i + 2])
            return true;
    }
    return false;
}
    \end{minted}
    \end{english}
    \end{boxSolution}
    \clearpage
\fi

    % ======================================================
    % 9. CountSignChanges - عد تغيرات الإشارة
    % ======================================================

    \item
    اكتب عملية خارجية \textenglish{int CountSignChanges(int[] arr)} تعد عدد المرات التي تتغير فيها إشارة الرقم (من موجب لسالب أو العكس) بين كل عنصر والعنصر الذي يليه.\\
    \textbf{مثال:} في المصفوفة \textenglish{\{3, -2, -5, 4\}}:
    \begin{itemize}
        \item 3 و $-2$: موجب $\leftarrow$ سالب (تغيير)
        \item $-2$ و $-5$: سالب $\leftarrow$ سالب
        \item $-5$ و 4: سالب $\leftarrow$ موجب (تغيير)
    \end{itemize}
    عدد التغييرات هو 2.
    \ifwithsols
    \begin{boxSolution}
    \begin{english}
    \begin{minted}{csharp}
public static int CountSignChanges(int[] arr)
{
    int changes = 0;
    for (int i = 0; i < arr.Length - 1; i++)
    {
        if (arr[i] * arr[i + 1] < 0)
            changes++;
    }
    return changes;
}
    \end{minted}
    \end{english}
    \end{boxSolution}
    \clearpage
\fi

    % ======================================================
    % 10. LongestSequence - أطول تسلسل متكرر
    % ======================================================

    \item
    اكتب عملية خارجية \textenglish{int LongestSequence(int[] arr)} تعيد طول أطول تسلسل متصل من عناصر لها نفس القيمة.\\
    \textbf{مثال:} في المصفوفة \textenglish{\{1, 2, 2, 2, 3, 3\}} الناتج هو 3 (التسلسل \textenglish{\{2, 2, 2\}}).
    \ifwithsols
    \begin{boxSolution}
    \begin{english}
    \begin{minted}{csharp}
public static int LongestSequence(int[] arr)
{
    if (arr.Length == 0) return 0;

    int maxLen = 1;
    int currentLen = 1;

    for (int i = 1; i < arr.Length; i++)
    {
        if (arr[i] == arr[i - 1])
        {
            currentLen++;
            if (currentLen > maxLen)
                maxLen = currentLen;
        }
        else
        {
            currentLen = 1;
        }
    }
    return maxLen;
}
    \end{minted}
    \end{english}
    \end{boxSolution}
    \clearpage
\fi

    % ======================================================
    % 11. CenteredAverage - معدل بدون أكبر وأصغر
    % ======================================================

    \item
    اكتب عملية خارجية \textenglish{double CenteredAverage(int[] arr)} تحسب معدل الأعداد في المصفوفة بعد استثناء أكبر قيمة وأصغر قيمة لمرة واحدة.
    (افترض أن طول المصفوفة 3 على الأقل).\\
    \textbf{مثال:} في المصفوفة \textenglish{\{1, 5, 3, 8, 2\}} نستثني الأكبر (8) والأصغر (1)، فالمعدل هو $\frac{5+3+2}{3} = 3.33$.
    \ifwithsols
    \begin{boxSolution}
    \begin{english}
    \begin{minted}{csharp}
public static double CenteredAverage(int[] arr)
{
    int max = arr[0];
    int min = arr[0];
    int sum = 0;

    for (int i = 0; i < arr.Length; i++)
    {
        sum += arr[i];
        if (arr[i] > max) max = arr[i];
        if (arr[i] < min) min = arr[i];
    }

    sum = sum - max - min;
    return (double)sum / (arr.Length - 2);
}
    \end{minted}
    \end{english}
    \end{boxSolution}
    \clearpage
\fi

    % ======================================================
    % 12. MergeAlternating - دمج بالتناوب
    % (نقل من ex08B)
    % ======================================================

    \item
    اكتب عملية خارجية \textenglish{int[] MergeAlternating(int[] A, int[] B)} تستقبل مصفوفتين من الأعداد الصحيحة (لهما نفس الطول) وتعيد مصفوفة جديدة تحتوي على عناصر من المصفوفتين بالتناوب (عنصر من \textenglish{A}، عنصر من \textenglish{B}، وهكذا).\\
    \textbf{مثال:} إذا كانت \textenglish{A = \{1, 3, 5\}} و\textenglish{B = \{2, 4, 6\}}، تعيد العملية \textenglish{\{1, 2, 3, 4, 5, 6\}}.
    \ifwithsols
    \begin{boxSolution}
    \begin{english}
    \begin{minted}{csharp}
public static int[] MergeAlternating(int[] A, int[] B)
{
    int[] result = new int[A.Length + B.Length];

    for (int i = 0; i < A.Length; i++)
    {
        result[2 * i] = A[i];
        result[2 * i + 1] = B[i];
    }

    return result;
}
    \end{minted}
    \end{english}
    \end{boxSolution}
    \clearpage
\fi

    % ======================================================
    % 13. SecondMax - ثاني أكبر قيمة
    % ======================================================

    \item
    اكتب عملية خارجية \textenglish{int SecondMax(int[] arr)} تعيد ثاني أكبر قيمة في مصفوفة أعداد صحيحة.\\
    (افترض أن المصفوفة تحتوي على الأقل على عنصرين مختلفين).\\
    \textbf{مثال:} في المصفوفة \textenglish{\{3, 7, 2, 9, 5\}} ثاني أكبر قيمة هو 7.
    \ifwithsols
    \begin{boxSolution}
    \begin{english}
    \begin{minted}{csharp}
public static int SecondMax(int[] arr)
{
    int max = int.MinValue;
    int secondMax = int.MinValue;

    for (int i = 0; i < arr.Length; i++)
    {
        if (arr[i] > max)
        {
            secondMax = max;
            max = arr[i];
        }
        else if (arr[i] > secondMax && arr[i] != max)
        {
            secondMax = arr[i];
        }
    }
    return secondMax;
}
    \end{minted}
    \end{english}
    \end{boxSolution}
    \clearpage
\fi

    % ======================================================
    % 14. GetEvenReversed - نسخ الأعداد الزوجية بالعكس
    % (نقل من ex08B)
    % ======================================================

    \item
    اكتب عملية خارجية \textenglish{int[] GetEvenReversed(int[] source)} تستقبل مصفوفة أعداد صحيحة وتعيد مصفوفة جديدة تحتوي فقط على الأعداد الزوجية منها بترتيب عكسي.\\
    \textbf{مثال:} إذا كانت \textenglish{source = \{1, 2, 3, 4, 5, 6, 7, 8\}}، تعيد العملية \textenglish{\{8, 6, 4, 2\}}.
    \ifwithsols
    \begin{boxSolution}
    \begin{english}
    \begin{minted}{csharp}
public static int[] GetEvenReversed(int[] source)
{
    int evenCount = 0;
    for (int i = 0; i < source.Length; i++)
    {
        if (source[i] % 2 == 0)
            evenCount++;
    }

    int[] evenReversed = new int[evenCount];
    int index = 0;

    for (int i = source.Length - 1; i >= 0; i--)
    {
        if (source[i] % 2 == 0)
        {
            evenReversed[index] = source[i];
            index++;
        }
    }

    return evenReversed;
}
    \end{minted}
    \end{english}
    \end{boxSolution}
    \clearpage
\fi

    % ======================================================
    % 15. ShiftRight - إزاحة دائرية إلى اليمين
    % ======================================================

    \item
    اكتب عملية خارجية \textenglish{void ShiftRight(int[] arr, int n)} تقوم بإزاحة عناصر المصفوفة \textenglish{n} خطوات لليمين بشكل دائري (العناصر التي تخرج من النهاية تنتقل إلى البداية).\\
    \textbf{مثال:} إذا كانت \textenglish{arr = \{1, 2, 3, 4, 5\}} و\textenglish{n = 2}، تصبح \textenglish{arr = \{4, 5, 1, 2, 3\}}.
    \ifwithsols
    \begin{boxSolution}
    \begin{english}
    \begin{minted}{csharp}
public static void ShiftRight(int[] arr, int n)
{
    int len = arr.Length;
    if (len == 0) return;

    n = n % len;
    int[] temp = new int[len];

    for (int i = 0; i < len; i++)
    {
        temp[(i + n) % len] = arr[i];
    }

    for (int i = 0; i < len; i++)
    {
        arr[i] = temp[i];
    }
}
    \end{minted}
    \end{english}
    \end{boxSolution}
    \fi
\end{enumerate}

\end{document}
