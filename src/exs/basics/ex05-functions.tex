\documentclass[12pt]{article}
% Full article preamble (duplicated, no common file)
\usepackage{fontspec}
\usepackage[a4paper,top=2.4cm,bottom=2.4cm,left=2.3cm,right=2.3cm]{geometry}
\usepackage{polyglossia}
\usepackage{amsmath}
\usepackage{amssymb}
\usepackage{xcolor}
\usepackage{fancyhdr}
\usepackage{graphicx}
\usepackage{listings}
\usepackage[most]{tcolorbox}
\usepackage{pifont}
\usepackage{enumitem}
\usepackage{titlesec}
\usepackage[bottom]{footmisc}
\usepackage{titling}
\usepackage{minted}
\usepackage{etoolbox}
\usepackage{array}
\usepackage{extsizes}

\newfontfamily\emoji{Segoe UI Emoji}

\pagestyle{fancy}

\setmainlanguage[numerals=western]{arabic}
\setotherlanguage{english}
\newfontfamily\arabicfont[Script=Arabic]{Amiri}
\newfontfamily\arabicfonttt[Script=Arabic]{Courier New}

\lstset{
  language=[Sharp]C,
  numbers=left,
  stepnumber=1,
  numbersep=8pt,
  frame=single,
  basicstyle=\ttfamily\small,
  keywordstyle=\color{blue},
  stringstyle=\color{red},
  commentstyle=\color{green!50!black}
}

\newif\ifdetailed
\ifdefined\setdetailed
  \setdetailed
\fi

\newif\ifwithsols
\ifdefined\setwithsols
  \setwithsols
\fi

% unified tcolorboxes for articles
\tcbset{colback=white, colframe=black, fonttitle=\bfseries, boxrule=0.8pt}
\newtcolorbox{boxDef}[1][]{colback=blue!5!white,colframe=blue!75!black,
  title={{\emoji📘} تعريف\ifx\\#1\\\else ~#1\fi :}}
\newtcolorbox{boxExercise}[1][]{colback=cyan!5!white,colframe=cyan!70!black,
  title={{\emoji🧩} تمرين\ifx\\#1\\\else ~#1\fi :}}
\newtcolorbox{boxExample}[1][]{colback=yellow!5!white,colframe=orange!90!black,
  title={{\emoji📝} مثال\ifx\\#1\\\else ~#1\fi :}}
\newtcolorbox{boxNote}[1][]{colback=gray!10!white,colframe=black,
  title={{\emoji✨} ملاحظة\ifx\\#1\\\else ~#1\fi :}}
\newtcolorbox{boxAttention}[1][]{colback=magenta!10!white,colframe=magenta!80!black,
  title={{\emoji🔔} تنبيه\ifx\\#1\\\else ~#1\fi :}}
\newtcolorbox{boxWarning}[1][]{colback=red!5!white,colframe=red!75!black,
  title={{\emoji⚡} ملاحظة هامة\ifx\\#1\\\else ~#1\fi :}}
\newtcolorbox{boxSolution}[1][]{colback=green!5!white,colframe=green!60!black,
  title={{\emoji✅} حل\ifx\\#1\\\else ~#1\fi :}}
\newtcolorbox{boxSymbol}[1][]{colback=purple!5!white,colframe=purple!70!black,
  title={{\emoji🔣} رمز\ifx\\#1\\\else ~#1\fi :}}
\newtcolorbox{boxHint}[1][]{colback=teal!5!white,colframe=teal!60!black,
  title={{\emoji💡} تلميح\ifx\\#1\\\else ~#1\fi :}}


\tcbset{simplecode/.style={ colback=gray!5, colframe=black!50, boxrule=0.4pt, arc=2pt, left=4pt,right=4pt,top=4pt,bottom=4pt}}
\newenvironment{boxCode}{\begin{tcolorbox}[simplecode]}{\end{tcolorbox}}

\newcolumntype{C}[1]{>{\centering\arraybackslash}p{#1}}

% redefine spaces after titles
\makeatletter
\renewcommand{\@maketitle}{%
  \begin{center}
    {\huge \bfseries \@title \par}%
    \vskip 0.2em % space between title and author
    {\large \@author \par}%
    % \vskip 0.2em % space between author and date
    % {\normalsize \@date \par}%
  \end{center}
}
\makeatother

\fancyhf{} % clear default
\fancypagestyle{plain}{
  \fancyhf{}
  \fancyhead[L]{مدرسة التسامح الشاملة}
  % \fancyhead[L]{\includegraphics[height=1cm]{../../../images/logoTasamoh.png}}
  \fancyhead[R]{الأستاذ محمود اغبارية}
  \fancyfoot[C]{\thepage}
}

\fancyhead[L]{مدرسة التسامح الشاملة}
\fancyhead[R]{الأستاذ محمود اغبارية}
\fancyfoot[C]{\thepage}
% \date{\today}

\setcounter{tocdepth}{3} % only section subsection and subsubsection in TOC


% ----------------------


% \begin{document}

% \maketitle

% % \clearpage  % start TOC on a new page
% % \renewcommand{\contentsname}{جدول المحتويات}
% % \tableofcontents
% % \clearpage

% \part*{part 1} % the * prevents numbering
% \section*{مقدمة}
% \subsection*{مثال رياضي}
% \subsubsection*{مثال فرعي}
% \paragraph*{ paragraph 1}
% \subparagraph*{sub paragraph 1}

% \ifdetailed
% \begin{english}
% \begin{minted}{csharp}
% // C# Example
% \end{minted}
% \end{english}
% \fi

% OLD WAY
% \ifdetailed
% \begin{english}
% \begin{lstlisting}
% // C# Example
% \end{lstlisting}
% \end{english}
% \fi

% % \includegraphics[width=0.2\textwidth]{../../../images/DFAs/ex1_q1.png}



% \vspace{3cm}
% \begin{flushleft}
% أرجو لكم وقتًا ممتعًا.

% الأستاذ محمود اغبارية.
% \end{flushleft}


% \end{document}


\title{ورقة تمرن 5 للصف العاشر 10 - عمليات خارجية}

\begin{document}

\maketitle
\thispagestyle{fancy}

\ifwithsols
\begin{enumerate}[itemsep=3em]
\else
\begin{enumerate}
\fi



\begin{boxAttention}
عندما يٌطلب "عملية خارجية \textbf{تستقبل}" يجب ان يكون المستخدم يدخل قيمة للمتغير. أي عليك استعمال \texttt{Console.ReadLine()}. \\


أما عندما يُطلب منك "عملية خارجية \textbf{تتلقى}" فهذا يعني أنّها تتلقى المطلوب كبارمترات.
\end{boxAttention}

\item اكتب عملية خارجية \textbf{تستقبل} اسم المستخدم \textbf{وتطبع} اسمه.
\ifwithsols
\begin{boxSolution}
\begin{english}
\begin{minted}{csharp}
public static void PrintNameReceive()
{
    string name = Console.ReadLine();
    Console.WriteLine(name);
}
\end{minted}
\end{english}
\end{boxSolution}
\fi


\item اكتب عملية خارجية \textbf{تستقبل} اسم المستخدم \textbf{وتعيد} اسمه.
\ifwithsols
\begin{boxSolution}
\begin{english}
\begin{minted}{csharp}
public static string ReadNameReturn()
{
    string name = Console.ReadLine();
    return name;
}
\end{minted}
\end{english}
\end{boxSolution}
\fi


\item اكتب عملية خارجية \textbf{تتلقى} عددا صحيحًا \textbf{وتطبع} تربيعه
\ifwithsols
\begin{boxSolution}
\begin{english}
\begin{minted}{csharp}
public static void PrintSquare(int n)
{
    int sq = n * n;
    Console.WriteLine(sq);
}
\end{minted}
\end{english}
\end{boxSolution}
\fi


\item اكتب عملية خارجية \textbf{تتلقى} عددا صحيحًا \textbf{وتعيد} تربيعه
\ifwithsols
\begin{boxSolution}
\begin{english}
\begin{minted}{csharp}
public static int GetSquare(int n)
{
    return n * n;
}
\end{minted}
\end{english}
\end{boxSolution}
\fi


\item اكتب عملية خارجية \textbf{تتلقى} عددين عشريّين \textbf{وتعيد} معدلهما
\ifwithsols
\begin{boxSolution}
\begin{english}
\begin{minted}{csharp}
public static double AverageTwo(double a, double b)
{
    return (a + b) / 2.0;
}
\end{minted}
\end{english}
\end{boxSolution}
\fi


\ifwithsols
\begin{boxSolution}
\begin{english}
\begin{minted}{csharp}
public static double GetAverage(double a, double b)
{
    return (a + b) / 2.0;
}
\end{minted}
\end{english}
\end{boxSolution}
\fi


\item اكتب عملية خارجية \textbf{تتلقى} عددين عشريّين \textbf{وتعيد} معدلهما
\item اكتب عملية خارجية \textbf{تتلقى} عددين عشريّين \textbf{وتعيد} \textenglish{true} إذا كانا متساويين.
\ifwithsols
\begin{boxSolution}
\begin{english}
\begin{minted}{csharp}
public static bool AreEqual(double a, double b)
{
    return a == b;
}
\end{minted}
\end{english}
\end{boxSolution}
\fi


\item اكتب عملية خارجية \textbf{تتلقى} عددين صحيحين \textbf{وتعيد} \textenglish{true} إذا كان العدد الأول يقسم على الثاني بدون باقٍ.
\ifwithsols
\begin{boxSolution}
\begin{english}
\begin{minted}{csharp}
public static bool Divides(int a, int b)
{
    if (a == 0) return false;
    return (b % a) == 0;
}
\end{minted}
\end{english}
\end{boxSolution}
\fi


\item اكتب عملية خارجية \textbf{تتلقى} ثلاثة أعداد صحيحة \textbf{وتعيد} العدد الأكبر بين الثلاثة.
\ifwithsols
\begin{boxSolution}
\begin{english}
\begin{minted}{csharp}
public static int MaxOfThree(int a, int b, int c)
{
    return Math.Max(a, Math.Max(b, c));
}
\end{minted}
\end{english}
\end{boxSolution}
\fi


\item اكتب عملية خارجية \textbf{تتلقى} ثلاثة أعداد صحيحة \textbf{وتعيد} العدد الأصغر بين الثلاثة.
\ifwithsols
\begin{boxSolution}
\begin{english}
\begin{minted}{csharp}
public static int MinOfThree(int a, int b, int c)
{
    return Math.Min(a, Math.Min(b, c));
}
\end{minted}
\end{english}
\end{boxSolution}
\fi



\clearpage

\item
\begin{enumerate}
    \item اكتب عملية خارجية باسم \textenglish{AppendDigit} تتلقى عددين صحيحين. \\
    العملية تلصق العدد الثاني بالعدد الأوّل وتعيد النتيجة. \\
    \textbf{مثال:} إذا تلقت العملية العددين: $12$ والعدد $5$ فإنّها تعيد: $125$.
    \item اكتب برنامجًا يستقبل من المستخدم 4 أرقام، كل واحد من خانة واحدة، ويكوّن له رقمًا من أربعة منازل ويطبعه. \\
    الرقم الأول الذي يدخله المستخدم تكون خانة عشرات الآلاف، والرقم الأخير يكون الآحاد. \\
    \textbf{مثال:} إذا أدخل المستخدم الأرقام $4,2,1,5$ فعلى البرنامج أن يطبع $4215$. \\
    عليك استخدام العملية التي كتبتها في البند السابق. \\
    لا حاجة للتحقق من صحة المدخلات.
\ifwithsols
\begin{boxSolution}
\begin{english}
\begin{minted}{csharp}
public static int AppendDigit(int n, int d)
{
    return n * 10 + d;
}

// Reading four one-digit numbers and building the final number
int d1 = int.Parse(Console.ReadLine());
int d2 = int.Parse(Console.ReadLine());
int d3 = int.Parse(Console.ReadLine());
int d4 = int.Parse(Console.ReadLine());
int number = AppendDigit(AppendDigit(AppendDigit(d1, d2), d3), d4);
Console.WriteLine(number);
\end{minted}
\end{english}
\end{boxSolution}
\fi


\end{enumerate}

\clearpage
\item
\begin{center}
    \includegraphics[width=1.1\textwidth]{../../../images/basics_functions_bagrut_question.png}
\end{center}


% قرّر مجلس البلدية في مدينة معيّنة أن يمنح سكان المدينة تخفيضات من مدفوعات مختلفة حسب المعايير التالية:
% \begin{itemize}
%     \item الساكن في المدينة الذي عمره 70 فأصعد، يستحق تخفيضًا من الدفع للأرونا.
%     \item الساكن في المدينة الذي لديه 4 أولاد فأصعد، يستحق تخفيضًا من الدفع لرياض الأطفال.
%     \item الساكن في المدينة الذي مكان عمله في المدينة، يستحق تخفيضًا من الدفع مقابل صفّ (ركن) السيارة في المدينة.
% \end{itemize}

% المعلومات عن مدى استحقاق هذه التخفيضات مُمثّلة بواسطة عدد مكوَّن من ثلاثة أرقام، مركّب من الرقمين 1 و 2 فقط.
% الرقم 1 يشير إلى استحقاق التخفيض، والرقم 2 يشير إلى عدم استحقاق التخفيض.

% \begin{itemize}
%     \item الرقم الأوّل من اليسار (رقم المئات) يشير إلى الاستحقاق أو عدم الاستحقاق للتخفيض من الدفع للأرونا.
%     \item الرقم الأوسط (رقم العشرات) يشير إلى الاستحقاق أو عدم الاستحقاق للتخفيض من الدفع لرياض الأطفال.
%     \item الرقم الأخير (رقم الآحاد) يشير إلى الاستحقاق أو عدم الاستحقاق للتخفيض من الدفع لصفّ السيارة في المدينة.
% \end{itemize}

% مثال: الرقم $122$ يشير إلى الاستحقاق للتخفيض في الدفع للأرونا، وعدم الاستحقاق للتخفيض في الدفع لرياض الأطفال وفي الدفع لصفّ السيارة.

% اكتبوا عملية خارجية باسم \textenglish{Discounts} بلغة \textenglish{C\#} تتلقّى ثلاثة متغيّرات:
% \begin{itemize}
%     \item \textenglish{age} وهو عدد صحيح يشير إلى عمر الساكن في المدينة،
%     \item \textenglish{num} وهو عدد صحيح يشير إلى عدد الأولاد للساكن،
%     \item \textenglish{city} وهو قيمة بوليانية \textenglish{(true/false)} تشير إلى ما إذا كان مكان عمل الساكن هو في المدينة أم لا.
% \end{itemize}

% تُعيد العملية عددًا صحيحًا مكوَّنًا من ثلاثة أرقام يعبّر عن استحقاق الساكن للتخفيضات أعلاه.
% \ifdetailed
% \begin{boxExample}
% \begin{english}
% \begin{minted}{text}
% Input:
% age = 75, num = 2, city = false -> 122
% age = 30, num = 5, city = true  -> 211
% \end{minted}
% \end{english}
% \end{boxExample}
% \fi


\ifwithsols
\begin{boxSolution}
\begin{english}
\begin{minted}{csharp}
public static int Discounts(int age, int num, bool city)
{
    int hundreds = (age >= 70) ? 1 : 2;   // Arnona
    int tens     = (num >= 4)  ? 1 : 2;   // Kindergarten
    int ones     = (city)      ? 1 : 2;   // Parking
    return hundreds * 100 + tens * 10 + ones;
}
\end{minted}
\end{english}
\end{boxSolution}
\fi



\end{enumerate}%

\vspace{1cm}
\begin{flushleft}
أرجو لكم وقتًا ممتعًا.

الأستاذ محمود اغبارية.
\end{flushleft}

\end{document}
