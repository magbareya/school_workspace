\documentclass[14pt]{extarticle}
% Full article preamble (duplicated, no common file)
\usepackage{fontspec}
\usepackage[a4paper,top=2.4cm,bottom=2.4cm,left=2.3cm,right=2.3cm]{geometry}
\usepackage{polyglossia}
\usepackage{amsmath}
\usepackage{amssymb}
\usepackage{xcolor}
\usepackage{fancyhdr}
\usepackage{graphicx}
\usepackage{listings}
\usepackage[most]{tcolorbox}
\usepackage{pifont}
\usepackage{enumitem}
\usepackage{titlesec}
\usepackage[bottom]{footmisc}
\usepackage{titling}
\usepackage{minted}
\usepackage{etoolbox}
\usepackage{array}
\usepackage{extsizes}

\newfontfamily\emoji{Segoe UI Emoji}

\pagestyle{fancy}

\setmainlanguage[numerals=western]{arabic}
\setotherlanguage{english}
\newfontfamily\arabicfont[Script=Arabic]{Amiri}
\newfontfamily\arabicfonttt[Script=Arabic]{Courier New}

\lstset{
  language=[Sharp]C,
  numbers=left,
  stepnumber=1,
  numbersep=8pt,
  frame=single,
  basicstyle=\ttfamily\small,
  keywordstyle=\color{blue},
  stringstyle=\color{red},
  commentstyle=\color{green!50!black}
}

\newif\ifdetailed
\ifdefined\setdetailed
  \setdetailed
\fi

\newif\ifwithsols
\ifdefined\setwithsols
  \setwithsols
\fi

% unified tcolorboxes for articles
\tcbset{colback=white, colframe=black, fonttitle=\bfseries, boxrule=0.8pt}
\newtcolorbox{boxDef}[1][]{colback=blue!5!white,colframe=blue!75!black,
  title={{\emoji📘} تعريف\ifx\\#1\\\else ~#1\fi :}}
\newtcolorbox{boxExercise}[1][]{colback=cyan!5!white,colframe=cyan!70!black,
  title={{\emoji🧩} تمرين\ifx\\#1\\\else ~#1\fi :}}
\newtcolorbox{boxExample}[1][]{colback=yellow!5!white,colframe=orange!90!black,
  title={{\emoji📝} مثال\ifx\\#1\\\else ~#1\fi :}}
\newtcolorbox{boxNote}[1][]{colback=gray!10!white,colframe=black,
  title={{\emoji✨} ملاحظة\ifx\\#1\\\else ~#1\fi :}}
\newtcolorbox{boxAttention}[1][]{colback=magenta!10!white,colframe=magenta!80!black,
  title={{\emoji🔔} تنبيه\ifx\\#1\\\else ~#1\fi :}}
\newtcolorbox{boxWarning}[1][]{colback=red!5!white,colframe=red!75!black,
  title={{\emoji⚡} ملاحظة هامة\ifx\\#1\\\else ~#1\fi :}}
\newtcolorbox{boxSolution}[1][]{colback=green!5!white,colframe=green!60!black,
  title={{\emoji✅} حل\ifx\\#1\\\else ~#1\fi :}}
\newtcolorbox{boxSymbol}[1][]{colback=purple!5!white,colframe=purple!70!black,
  title={{\emoji🔣} رمز\ifx\\#1\\\else ~#1\fi :}}
\newtcolorbox{boxHint}[1][]{colback=teal!5!white,colframe=teal!60!black,
  title={{\emoji💡} تلميح\ifx\\#1\\\else ~#1\fi :}}


\tcbset{simplecode/.style={ colback=gray!5, colframe=black!50, boxrule=0.4pt, arc=2pt, left=4pt,right=4pt,top=4pt,bottom=4pt}}
\newenvironment{boxCode}{\begin{tcolorbox}[simplecode]}{\end{tcolorbox}}

\newcolumntype{C}[1]{>{\centering\arraybackslash}p{#1}}

% redefine spaces after titles
\makeatletter
\renewcommand{\@maketitle}{%
  \begin{center}
    {\huge \bfseries \@title \par}%
    \vskip 0.2em % space between title and author
    {\large \@author \par}%
    % \vskip 0.2em % space between author and date
    % {\normalsize \@date \par}%
  \end{center}
}
\makeatother

\fancyhf{} % clear default
\fancypagestyle{plain}{
  \fancyhf{}
  \fancyhead[L]{مدرسة التسامح الشاملة}
  % \fancyhead[L]{\includegraphics[height=1cm]{../../../images/logoTasamoh.png}}
  \fancyhead[R]{الأستاذ محمود اغبارية}
  \fancyfoot[C]{\thepage}
}

\fancyhead[L]{مدرسة التسامح الشاملة}
\fancyhead[R]{الأستاذ محمود اغبارية}
\fancyfoot[C]{\thepage}
% \date{\today}

\setcounter{tocdepth}{3} % only section subsection and subsubsection in TOC


% ----------------------


% \begin{document}

% \maketitle

% % \clearpage  % start TOC on a new page
% % \renewcommand{\contentsname}{جدول المحتويات}
% % \tableofcontents
% % \clearpage

% \part*{part 1} % the * prevents numbering
% \section*{مقدمة}
% \subsection*{مثال رياضي}
% \subsubsection*{مثال فرعي}
% \paragraph*{ paragraph 1}
% \subparagraph*{sub paragraph 1}

% \ifdetailed
% \begin{english}
% \begin{minted}{csharp}
% // C# Example
% \end{minted}
% \end{english}
% \fi

% OLD WAY
% \ifdetailed
% \begin{english}
% \begin{lstlisting}
% // C# Example
% \end{lstlisting}
% \end{english}
% \fi

% % \includegraphics[width=0.2\textwidth]{../../../images/DFAs/ex1_q1.png}



% \vspace{3cm}
% \begin{flushleft}
% أرجو لكم وقتًا ممتعًا.

% الأستاذ محمود اغبارية.
% \end{flushleft}


% \end{document}


\ifwithsols
\title{حل وظيفة 13 - المصفوفات الأحادية - عمليات خارجية}
\else
\title{وظيفة 13 - المصفوفات الأحادية - عمليات خارجية}
\fi

\begin{document}

\maketitle
\thispagestyle{fancy}

\begin{enumerate}[itemsep=2em]

    \item
    اكتب عملية خارجية \textenglish{int GetMin(int[] arr)} تتلقى مصفوفة أعداد صحيحة وتعيد أصغر قيمة فيها.\\
    \textbf{مثال:} في المصفوفة \textenglish{\{15, 3, 9, 21, 7\}} أصغر قيمة هي 3.
    \ifwithsols
    \begin{boxSolution}
    \begin{english}
    \begin{minted}{csharp}
public static int GetMin(int[] arr)
{
    int min = arr[0];
    for (int i = 1; i < arr.Length; i++)
    {
        if (arr[i] < min)
            min = arr[i];
    }
    return min;
}
    \end{minted}
    \end{english}
    \end{boxSolution}
    \clearpage
    \fi

    \item
    اكتب عملية خارجية \textenglish{int SumInRange(int[] arr, int min, int max)} تتلقى مصفوفة أعداد صحيحة وقيمتين (\textenglish{min} و\textenglish{max})،
    وتعيد مجموع جميع الأعداد التي تقع بين $[min, max]$، أي مجموع جميع الأعداد الأصغر أو تساوي \textenglish{max} وأيضًا أكبر أو تساوي \textenglish{min}.
    \begin{boxExample}
    في المصفوفة \textenglish{\{5, 15, 25, 8, 30, 12, 3\}} مع \textenglish{min = 10} و\textenglish{max = 20}:
    \begin{itemize}[itemsep=1pt]
        \item الأعداد في المجال: 15، 12
        \item المجموع: $15 + 12 = 27$
    \end{itemize}
    \end{boxExample}
    \ifwithsols
    \begin{boxSolution}
    \begin{english}
    \begin{minted}{csharp}
public static int SumInRange(int[] arr, int min, int max)
{
    int sum = 0;
    for (int i = 0; i < arr.Length; i++)
    {
        if (arr[i] >= min && arr[i] <= max)
            sum += arr[i];
    }
    return sum;
}
    \end{minted}
    \end{english}
    \end{boxSolution}
    \clearpage
    \fi

    \item
    اكتب عملية خارجية \textenglish{int[] GetMultiples(int[] arr, int factor)} تتلقى مصفوفة أعداد صحيحة وعدداً صحيحاً (\textenglish{factor})، وتعيد مصفوفة جديدة تحتوي فقط على الأعداد القابلة للقسمة على \textenglish{factor}.
    \begin{boxExample}
    في المصفوفة \textenglish{\{12, 14, 18, 22, 24\}} مع \textenglish{factor = 3}:
    \begin{itemize}[itemsep=1pt]
        \item العناصر القابلة للقسمة: 12، 18، 24
        \item النتيجة: \textenglish{\{12, 18, 24\}}
    \end{itemize}
    \end{boxExample}
    \begin{boxHint}
    عليك أولاً عدّ العناصر القابلة للقسمة على \textenglish{factor} لبناء مصفوفة بالحجم المناسب، ثم ملء هذه المصفوفة بالعناصر المطلوبة.
    \end{boxHint}
    \ifwithsols
    \begin{boxSolution}
    \begin{english}
    \begin{minted}{csharp}
public static int[] GetMultiples(int[] arr, int factor)
{
    int count = 0;
    for (int i = 0; i < arr.Length; i++)
    {
        if (arr[i] % factor == 0)
            count++;
    }

    int[] result = new int[count];
    int index = 0;
    for (int i = 0; i < arr.Length; i++)
    {
        if (arr[i] % factor == 0)
        {
            result[index] = arr[i];
            index++;
        }
    }

    return result;
}
    \end{minted}
    \end{english}
    \end{boxSolution}
    \fi
    \clearpage

    \item
    اكتب عملية خارجية \textenglish{bool HasConsecutivePair(int[] arr)} تتلقى مصفوفة أعداد صحيحة وتعيد \textenglish{true} إذا وجدت عنصرين متجاورين بحيث تكون قيمهما أعدادًا متتالية.
    \begin{boxExample}
    \begin{itemize}[itemsep=1pt]
        \item في المصفوفة \textenglish{\{5, 3, 4, 8, 2\}} $\leftarrow$ \textenglish{true} (لأن 3 و4 متتاليان)
        \item في المصفوفة \textenglish{\{10, 8, 6, 4\}} $\leftarrow$ \textenglish{false} (لا يوجد أعداد متتالية)
        \item في المصفوفة \textenglish{\{2, 4, 6, 8\}} $\leftarrow$ \textenglish{false} (لا يوجد أعداد متتالية)
    \end{itemize}
    \end{boxExample}
    \ifwithsols
    \begin{boxSolution}
    \begin{english}
    \begin{minted}{csharp}
public static bool HasConsecutivePair(int[] arr)
{
    for (int i = 0; i < arr.Length - 1; i++)
    {
        if (Math.Abs(arr[i + 1] - arr[i]) == 1)
            return true;
    }
    return false;
}
    \end{minted}
    \end{english}
    \end{boxSolution}
    \clearpage
    \fi

    \item
    اكتب عملية خارجية \textenglish{int FirstAboveAverage(int[] arr)} تتلقى مصفوفة أعداد صحيحة وتعيد موقع (index) أول عنصر أكبر أو يساوي معدل المصفوفة.
    \begin{boxExample}
    في المصفوفة \textenglish{\{10, 5, 20, 15, 8\}}:
    \begin{itemize}[itemsep=1pt]
        \item المعدل = $\frac{10+5+20+15+8}{5} = 11.6$
        \item أول عنصر أكبر من المعدل هو 20 في الموقع 2 $\leftarrow$ العملية تعيد 2
    \end{itemize}
    \end{boxExample}
    \ifwithsols
    \begin{boxSolution}
    \begin{english}
    \begin{minted}{csharp}
public static int FirstAboveAverage(int[] arr)
{
    int sum = 0;
    for (int i = 0; i < arr.Length; i++)
    {
        sum += arr[i];
    }

    double avg = (double)sum / arr.Length;

    for (int i = 0; i < arr.Length; i++)
    {
        if (arr[i] > avg)
            return i;
    }

    return -1;
}
    \end{minted}
    \end{english}
    \end{boxSolution}
    \clearpage
    \fi

    \item
    اكتب عملية خارجية \textenglish{int[] GetNumberDigits(int number)} تتلقى عددًا صحيحًا وتعيد مصفوفة تحتوي على خانات العدد: كل خانة في عنصر من عناصر المصفوفة.
    \begin{boxExample}
    للعدد 4729، المصفوفة الناتجة هي \textenglish{\{4, 7, 2, 9\}}.
    \end{boxExample}
    \begin{boxHint}
    عليك معرفة عدد خانات العدد ثم بناء مصفوفة بحجم مناسب، ثم املأ المصفوفة من اليمين إلى اليسار بخانات العدد.
    \end{boxHint}
    \ifwithsols
    \begin{boxSolution}
    \begin{english}
    \begin{minted}{csharp}
public static int[] GetNumberDigits(int number)
{
    int temp = number;
    int count = 0;
    while (temp > 0)
    {
        count++;
        temp /= 10;
    }

    int[] digits = new int[count];
    for (int i = count - 1; i >= 0; i--)
    {
        digits[i] = number % 10;
        number /= 10;
    }

    return digits;
}
    \end{minted}
    \end{english}
    \end{boxSolution}
    \fi

\end{enumerate}

\ifwithsols
\else
\vspace{1cm}
\begin{flushleft}
أرجو لكم عملاً ممتعًا!

الأستاذ محمود اغبارية.
\end{flushleft}
\fi

\end{document}
