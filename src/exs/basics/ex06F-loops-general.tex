\documentclass[14pt]{extarticle}
% Full article preamble (duplicated, no common file)
\usepackage{fontspec}
\usepackage[a4paper,top=2.4cm,bottom=2.4cm,left=2.3cm,right=2.3cm]{geometry}
\usepackage{polyglossia}
\usepackage{amsmath}
\usepackage{amssymb}
\usepackage{xcolor}
\usepackage{fancyhdr}
\usepackage{graphicx}
\usepackage{listings}
\usepackage[most]{tcolorbox}
\usepackage{pifont}
\usepackage{enumitem}
\usepackage{titlesec}
\usepackage[bottom]{footmisc}
\usepackage{titling}
\usepackage{minted}
\usepackage{etoolbox}
\usepackage{array}
\usepackage{extsizes}

\newfontfamily\emoji{Segoe UI Emoji}

\pagestyle{fancy}

\setmainlanguage[numerals=western]{arabic}
\setotherlanguage{english}
\newfontfamily\arabicfont[Script=Arabic]{Amiri}
\newfontfamily\arabicfonttt[Script=Arabic]{Courier New}

\lstset{
  language=[Sharp]C,
  numbers=left,
  stepnumber=1,
  numbersep=8pt,
  frame=single,
  basicstyle=\ttfamily\small,
  keywordstyle=\color{blue},
  stringstyle=\color{red},
  commentstyle=\color{green!50!black}
}

\newif\ifdetailed
\ifdefined\setdetailed
  \setdetailed
\fi

\newif\ifwithsols
\ifdefined\setwithsols
  \setwithsols
\fi

% unified tcolorboxes for articles
\tcbset{colback=white, colframe=black, fonttitle=\bfseries, boxrule=0.8pt}
\newtcolorbox{boxDef}[1][]{colback=blue!5!white,colframe=blue!75!black,
  title={{\emoji📘} تعريف\ifx\\#1\\\else ~#1\fi :}}
\newtcolorbox{boxExercise}[1][]{colback=cyan!5!white,colframe=cyan!70!black,
  title={{\emoji🧩} تمرين\ifx\\#1\\\else ~#1\fi :}}
\newtcolorbox{boxExample}[1][]{colback=yellow!5!white,colframe=orange!90!black,
  title={{\emoji📝} مثال\ifx\\#1\\\else ~#1\fi :}}
\newtcolorbox{boxNote}[1][]{colback=gray!10!white,colframe=black,
  title={{\emoji✨} ملاحظة\ifx\\#1\\\else ~#1\fi :}}
\newtcolorbox{boxAttention}[1][]{colback=magenta!10!white,colframe=magenta!80!black,
  title={{\emoji🔔} تنبيه\ifx\\#1\\\else ~#1\fi :}}
\newtcolorbox{boxWarning}[1][]{colback=red!5!white,colframe=red!75!black,
  title={{\emoji⚡} ملاحظة هامة\ifx\\#1\\\else ~#1\fi :}}
\newtcolorbox{boxSolution}[1][]{colback=green!5!white,colframe=green!60!black,
  title={{\emoji✅} حل\ifx\\#1\\\else ~#1\fi :}}
\newtcolorbox{boxSymbol}[1][]{colback=purple!5!white,colframe=purple!70!black,
  title={{\emoji🔣} رمز\ifx\\#1\\\else ~#1\fi :}}
\newtcolorbox{boxHint}[1][]{colback=teal!5!white,colframe=teal!60!black,
  title={{\emoji💡} تلميح\ifx\\#1\\\else ~#1\fi :}}


\tcbset{simplecode/.style={ colback=gray!5, colframe=black!50, boxrule=0.4pt, arc=2pt, left=4pt,right=4pt,top=4pt,bottom=4pt}}
\newenvironment{boxCode}{\begin{tcolorbox}[simplecode]}{\end{tcolorbox}}

\newcolumntype{C}[1]{>{\centering\arraybackslash}p{#1}}

% redefine spaces after titles
\makeatletter
\renewcommand{\@maketitle}{%
  \begin{center}
    {\huge \bfseries \@title \par}%
    \vskip 0.2em % space between title and author
    {\large \@author \par}%
    % \vskip 0.2em % space between author and date
    % {\normalsize \@date \par}%
  \end{center}
}
\makeatother

\fancyhf{} % clear default
\fancypagestyle{plain}{
  \fancyhf{}
  \fancyhead[L]{مدرسة التسامح الشاملة}
  % \fancyhead[L]{\includegraphics[height=1cm]{../../../images/logoTasamoh.png}}
  \fancyhead[R]{الأستاذ محمود اغبارية}
  \fancyfoot[C]{\thepage}
}

\fancyhead[L]{مدرسة التسامح الشاملة}
\fancyhead[R]{الأستاذ محمود اغبارية}
\fancyfoot[C]{\thepage}
% \date{\today}

\setcounter{tocdepth}{3} % only section subsection and subsubsection in TOC


% ----------------------


% \begin{document}

% \maketitle

% % \clearpage  % start TOC on a new page
% % \renewcommand{\contentsname}{جدول المحتويات}
% % \tableofcontents
% % \clearpage

% \part*{part 1} % the * prevents numbering
% \section*{مقدمة}
% \subsection*{مثال رياضي}
% \subsubsection*{مثال فرعي}
% \paragraph*{ paragraph 1}
% \subparagraph*{sub paragraph 1}

% \ifdetailed
% \begin{english}
% \begin{minted}{csharp}
% // C# Example
% \end{minted}
% \end{english}
% \fi

% OLD WAY
% \ifdetailed
% \begin{english}
% \begin{lstlisting}
% // C# Example
% \end{lstlisting}
% \end{english}
% \fi

% % \includegraphics[width=0.2\textwidth]{../../../images/DFAs/ex1_q1.png}



% \vspace{3cm}
% \begin{flushleft}
% أرجو لكم وقتًا ممتعًا.

% الأستاذ محمود اغبارية.
% \end{flushleft}


% \end{document}


\ifwithsols
\title{حل ورقة تمرن 6 للصف العاشر 10 - قسم 6 \\ أسئلة بجروت عن الحلقات}
\else
\title{ورقة تمرن 6 للصف العاشر 10 - قسم 6 \\ أسئلة بجروت عن الحلقات}
\fi

\begin{document}

\maketitle
\thispagestyle{fancy}

\begin{enumerate}[itemsep=1.5em]

\item
اكتب برنامجًا يستقبل 20 عددًا صحيحًا ويطبع مجموع كل عددين متتاليين.
أي إذا كانت المدخلات: a1, a2, a3, ... فعليه أن يطبع:
(a1 + a2), (a2 + a3), ... حتى النهاية.
\ifwithsols
\begin{boxSolution}
\begin{english}
\begin{minted}{csharp}
int prev = int.Parse(Console.ReadLine());
for(int i=1; i<20; i++)
{
    int x = int.Parse(Console.ReadLine());
    Console.WriteLine(prev + x);
    prev = x;
}
\end{minted}
\end{english}
\end{boxSolution}
\fi

\item
اكتب برنامجًا يستقبل من المستخدم 15 عددًا صحيحًا ويطبع عدد الأعداد التي تقسم على العدد الذي قبلها مباشرة بدون باقٍ.
\ifwithsols
\begin{boxSolution}
\begin{english}
\begin{minted}{csharp}
int prev = int.Parse(Console.ReadLine());
int cnt = 0;

for(int i=1; i<15; i++)
{
    int x = int.Parse(Console.ReadLine());
    if (prev != 0 && x % prev == 0)
        cnt++;
    prev = x;
}
Console.WriteLine(cnt);
\end{minted}
\end{english}
\end{boxSolution}
\fi


    \item
اكتب برنامجًا يستقبل 30 عددًا صحيحًا ويطبع عدد الأعداد السالبة المتتالية الأطول.
\ifwithsols
\begin{boxSolution}
\begin{english}
\begin{minted}{csharp}
int maxSeq = 0, curr = 0;

for (int i = 0; i < 30; i++)
{
    int x = int.Parse(Console.ReadLine());
    if (x < 0)
    {
        curr++;
        if (curr > maxSeq) maxSeq = curr;
    }
    else
        curr = 0;
}
Console.WriteLine(maxSeq);
\end{minted}
\end{english}
\end{boxSolution}
\fi

\item
اكتب برنامجًا يستقبل 15 عددًا صحيحًا ويطبع أطول سلسلة متزايدة متتالية.
\ifwithsols
\begin{boxSolution}
\begin{english}
\begin{minted}{csharp}
int prev = int.Parse(Console.ReadLine());
int currLen = 1, maxLen = 1;

for(int i = 1; i < 15; i++)
{
    int x = int.Parse(Console.ReadLine());
    if (x > prev)
        currLen++;
    else
        currLen = 1;
    if (currLen > maxLen)
        maxLen = currLen;
    prev = x;
}
Console.WriteLine(maxLen);
\end{minted}
\end{english}
\end{boxSolution}
\fi

\item
اكتب برنامجًا يستقبل 25 عددًا صحيحًا ويطبع عدد الأعداد التي تساوي مجموع الأرقام التي قبلها مباشرة (أي:
$x_i = x_{i-1} + x_{i-2}$ ).
\ifwithsols
\begin{boxSolution}
\begin{english}
\begin{minted}{csharp}
int a = int.Parse(Console.ReadLine());
int b = int.Parse(Console.ReadLine());
int cnt = 0;

for(int i = 2; i < 25; i++)
{
    int x = int.Parse(Console.ReadLine());
    if (x == a + b)
        cnt++;
    a = b;
    b = x;
}
Console.WriteLine(cnt);
\end{minted}
\end{english}
\end{boxSolution}
\fi

\item
اكتب برنامجًا يستقبل 12 عددًا صحيحًا ويفحص إذا ظهر رقم 5 ثلاث مرات متتالية.
في النهاية يطبع "نعم" أو "لا".
\ifwithsols
\begin{boxSolution}
\begin{english}
\begin{minted}{csharp}
int cnt = 0;
bool ok = false;

for(int i=0; i<12; i++)
{
    int x = int.Parse(Console.ReadLine());
    if (x == 5)
    {
        cnt++;
        if (cnt == 3) ok = true;
    }
    else
        cnt = 0;
}
if(ok)
    Console.WriteLine("Yes");
else
    Console.WriteLine("No");
\end{minted}
\end{english}
\end{boxSolution}
\fi

\item اكتب عملية تتلقى عددًا صحيحًا $x$. على العملية أن تجد وتعيد أصغر عدد يكون من قوى العدد 2 (Powers of 2) ولكنه أكبر تمامًا من $x$. \\
مثال: إذا كان المدخل 10، المخرج هو 16. إذا كان المدخل 50، المخرج 64.
\ifwithsols
\begin{boxSolution}
\begin{english}
\begin{minted}{csharp}
public static int NextPowerOf2(int x)
{
    int p = 1;
    while (p <= x)
    {
        p *= 2;
    }
    return p;
}
\end{minted}
\end{english}
\end{boxSolution}
\fi

\end{enumerate}

\subsection*{سؤال 7 امتحان 899222 سنة 2015}

\noindent
    \makebox[\textwidth][c]{
        \includegraphics[width=0.9\paperwidth,keepaspectratio]{ ../../../bagrut_questions/basics/loops_while_2015_899222_7.png }%
    }%

\ifwithsols
\begin{boxSolution}[سؤال 7 امتحان 899222 سنة 2015]
\begin{boxCode}
\begin{english}
\begin{minted}{csharp}
int price0 = int.Parse(Console.ReadLine());
int price1 = int.Parse(Console.ReadLine());
int price2 = int.Parse(Console.ReadLine());
int cnt0 = 0, cnt1 = 0, cnt2 = 0;

while (cnt0 < 500 && cnt1 < 500 && cnt2 < 500)
{
    string name = Console.ReadLine();
    int offerCode = int.Parse(Console.ReadLine());

    Console.WriteLine("Customer Name: " + name);
    if (offerCode == 0)
    {
        Console.WriteLine("Price: " + price0);
        cnt0++;
    }
    else if (offerCode == 1)
    {
        Console.WriteLine("Price: " + price1);
        cnt1++;
    }
    else if (offerCode == 2)
    {
        Console.WriteLine("Price: " + price2);
        cnt2++;
    }
}
\end{minted}
\end{english}
\end{boxCode}

\end{boxSolution}
\clearpage
\fi


\vspace{1cm}
\begin{flushleft}
أرجو لكم وقتًا ممتعًا.

الأستاذ محمود اغبارية.
\end{flushleft}

\end{document}
