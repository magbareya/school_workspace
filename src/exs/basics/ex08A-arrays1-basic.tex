\documentclass[14pt]{extarticle}
% Full article preamble (duplicated, no common file)
\usepackage{fontspec}
\usepackage[a4paper,top=2.4cm,bottom=2.4cm,left=2.3cm,right=2.3cm]{geometry}
\usepackage{polyglossia}
\usepackage{amsmath}
\usepackage{amssymb}
\usepackage{xcolor}
\usepackage{fancyhdr}
\usepackage{graphicx}
\usepackage{listings}
\usepackage[most]{tcolorbox}
\usepackage{pifont}
\usepackage{enumitem}
\usepackage{titlesec}
\usepackage[bottom]{footmisc}
\usepackage{titling}
\usepackage{minted}
\usepackage{etoolbox}
\usepackage{array}
\usepackage{extsizes}

\newfontfamily\emoji{Segoe UI Emoji}

\pagestyle{fancy}

\setmainlanguage[numerals=western]{arabic}
\setotherlanguage{english}
\newfontfamily\arabicfont[Script=Arabic]{Amiri}
\newfontfamily\arabicfonttt[Script=Arabic]{Courier New}

\lstset{
  language=[Sharp]C,
  numbers=left,
  stepnumber=1,
  numbersep=8pt,
  frame=single,
  basicstyle=\ttfamily\small,
  keywordstyle=\color{blue},
  stringstyle=\color{red},
  commentstyle=\color{green!50!black}
}

\newif\ifdetailed
\ifdefined\setdetailed
  \setdetailed
\fi

\newif\ifwithsols
\ifdefined\setwithsols
  \setwithsols
\fi

% unified tcolorboxes for articles
\tcbset{colback=white, colframe=black, fonttitle=\bfseries, boxrule=0.8pt}
\newtcolorbox{boxDef}[1][]{colback=blue!5!white,colframe=blue!75!black,
  title={{\emoji📘} تعريف\ifx\\#1\\\else ~#1\fi :}}
\newtcolorbox{boxExercise}[1][]{colback=cyan!5!white,colframe=cyan!70!black,
  title={{\emoji🧩} تمرين\ifx\\#1\\\else ~#1\fi :}}
\newtcolorbox{boxExample}[1][]{colback=yellow!5!white,colframe=orange!90!black,
  title={{\emoji📝} مثال\ifx\\#1\\\else ~#1\fi :}}
\newtcolorbox{boxNote}[1][]{colback=gray!10!white,colframe=black,
  title={{\emoji✨} ملاحظة\ifx\\#1\\\else ~#1\fi :}}
\newtcolorbox{boxAttention}[1][]{colback=magenta!10!white,colframe=magenta!80!black,
  title={{\emoji🔔} تنبيه\ifx\\#1\\\else ~#1\fi :}}
\newtcolorbox{boxWarning}[1][]{colback=red!5!white,colframe=red!75!black,
  title={{\emoji⚡} ملاحظة هامة\ifx\\#1\\\else ~#1\fi :}}
\newtcolorbox{boxSolution}[1][]{colback=green!5!white,colframe=green!60!black,
  title={{\emoji✅} حل\ifx\\#1\\\else ~#1\fi :}}
\newtcolorbox{boxSymbol}[1][]{colback=purple!5!white,colframe=purple!70!black,
  title={{\emoji🔣} رمز\ifx\\#1\\\else ~#1\fi :}}
\newtcolorbox{boxHint}[1][]{colback=teal!5!white,colframe=teal!60!black,
  title={{\emoji💡} تلميح\ifx\\#1\\\else ~#1\fi :}}


\tcbset{simplecode/.style={ colback=gray!5, colframe=black!50, boxrule=0.4pt, arc=2pt, left=4pt,right=4pt,top=4pt,bottom=4pt}}
\newenvironment{boxCode}{\begin{tcolorbox}[simplecode]}{\end{tcolorbox}}

\newcolumntype{C}[1]{>{\centering\arraybackslash}p{#1}}

% redefine spaces after titles
\makeatletter
\renewcommand{\@maketitle}{%
  \begin{center}
    {\huge \bfseries \@title \par}%
    \vskip 0.2em % space between title and author
    {\large \@author \par}%
    % \vskip 0.2em % space between author and date
    % {\normalsize \@date \par}%
  \end{center}
}
\makeatother

\fancyhf{} % clear default
\fancypagestyle{plain}{
  \fancyhf{}
  \fancyhead[L]{مدرسة التسامح الشاملة}
  % \fancyhead[L]{\includegraphics[height=1cm]{../../../images/logoTasamoh.png}}
  \fancyhead[R]{الأستاذ محمود اغبارية}
  \fancyfoot[C]{\thepage}
}

\fancyhead[L]{مدرسة التسامح الشاملة}
\fancyhead[R]{الأستاذ محمود اغبارية}
\fancyfoot[C]{\thepage}
% \date{\today}

\setcounter{tocdepth}{3} % only section subsection and subsubsection in TOC


% ----------------------


% \begin{document}

% \maketitle

% % \clearpage  % start TOC on a new page
% % \renewcommand{\contentsname}{جدول المحتويات}
% % \tableofcontents
% % \clearpage

% \part*{part 1} % the * prevents numbering
% \section*{مقدمة}
% \subsection*{مثال رياضي}
% \subsubsection*{مثال فرعي}
% \paragraph*{ paragraph 1}
% \subparagraph*{sub paragraph 1}

% \ifdetailed
% \begin{english}
% \begin{minted}{csharp}
% // C# Example
% \end{minted}
% \end{english}
% \fi

% OLD WAY
% \ifdetailed
% \begin{english}
% \begin{lstlisting}
% // C# Example
% \end{lstlisting}
% \end{english}
% \fi

% % \includegraphics[width=0.2\textwidth]{../../../images/DFAs/ex1_q1.png}



% \vspace{3cm}
% \begin{flushleft}
% أرجو لكم وقتًا ممتعًا.

% الأستاذ محمود اغبارية.
% \end{flushleft}


% \end{document}


\ifwithsols
\title{حل ورقة تمرين 8 - المصفوفات الأحادية \\ قسم 1 - أسئلة أساسية}
\else
\title{ورقة تمرين 8 - المصفوفات الأحادية \\ قسم 1 - أسئلة أساسية}
\fi

\begin{document}

\maketitle
\thispagestyle{fancy}

\begin{enumerate}[itemsep=2em]

    % ======================================================
    % التعريف والتهيئة
    % ======================================================

    \item
    أ) صرح عن مصفوفة أعداد صحيحة (\textenglish{int}) باسم \textenglish{numbers} تتسع لـ 10 عناصر.\\
    ب) صرح عن مصفوفة نصوص (\textenglish{string}) باسم \textenglish{colors} وامنحها القيم الأولية: \textenglish{"Red", "Green", "Blue"}.
    \ifwithsols
    \begin{boxSolution}
    \begin{english}
    \begin{minted}{csharp}
// A
int[] numbers = new int[10];

// B
string[] colors = { "Red", "Green", "Blue" };
    \end{minted}
    \end{english}
    \end{boxSolution}
    \fi

    \item
    عرف مصفوفة أعداد صحيحة طولها 10، ثم اكتب حلقة تكرار لملء جميع خاناتها بالقيمة $-1$.
    \ifwithsols
    \begin{boxSolution}
    \begin{english}
    \begin{minted}{csharp}
int[] arr = new int[10];
for (int i = 0; i < arr.Length; i++)
{
    arr[i] = -1;
}
    \end{minted}
    \end{english}
    \end{boxSolution}
    \clearpage
    \fi

    % ======================================================
    % الوصول والطباعة
    % ======================================================

    \item
    لديك مصفوفة أعداد صحيحة باسم \textenglish{data}. اكتب مقطع برنامج يطبع العنصر الأول والعنصر الأخير فيها. (افترض أن المصفوفة ليست فارغة).
    \ifwithsols
    \begin{boxSolution}
    \begin{english}
    \begin{minted}{csharp}
Console.WriteLine("First: " + data[0]);
Console.WriteLine("Last: " + data[data.Length - 1]);
    \end{minted}
    \end{english}
    \end{boxSolution}
    \fi

    \item
    اكتب حلقة لطباعة عناصر مصفوفة النصوص \textenglish{names} بترتيب عكسي (من آخر عنصر إلى أول عنصر).\\
    \textbf{مثال:} إذا كانت المصفوفة \textenglish{\{"A", "B", "C"\}} يطبع البرنامج \textenglish{C} ثم \textenglish{B} ثم \textenglish{A}.
    \ifwithsols
    \begin{boxSolution}
    \begin{english}
    \begin{minted}{csharp}
for (int i = names.Length - 1; i >= 0; i--)
{
    Console.WriteLine(names[i]);
}
    \end{minted}
    \end{english}
    \end{boxSolution}
    \fi

    \item
    اكتب مقطع برنامج يطبع القيم الموجودة في الخانات الزوجية فقط (0, 2, 4...) لمصفوفة أعداد صحيحة \textenglish{arr}.
    \ifwithsols
    \begin{boxSolution}
    \begin{english}
    \begin{minted}{csharp}
for (int i = 0; i < arr.Length; i += 2)
{
    Console.WriteLine(arr[i]);
}
    \end{minted}
    \end{english}
    \end{boxSolution}
    \clearpage
    \fi

    % ======================================================
    % التعديل والعمليات الحسابية
    % ======================================================

    \item
    لديك مصفوفة درجات طلاب \textenglish{scores} (أعداد صحيحة). اكتب مقطع برنامج يضيف 5 درجات لكل طالب في المصفوفة (تعديل القيم داخل نفس المصفوفة).
    \ifwithsols
    \begin{boxSolution}
    \begin{english}
    \begin{minted}{csharp}
for (int i = 0; i < scores.Length; i++)
{
    scores[i] += 5;
}
    \end{minted}
    \end{english}
    \end{boxSolution}
    \fi

    \item
    اكتب مقطع برنامج يجمع عناصر مصفوفتين من الأعداد الصحيحة \textenglish{A} و \textenglish{B} (لهما نفس الطول) ويخزن الناتج في مصفوفة جديدة \textenglish{C}.\\
    بحيث يكون: $C[i] = A[i] + B[i]$.
    \ifwithsols
    \begin{boxSolution}
    \begin{english}
    \begin{minted}{csharp}
int[] C = new int[A.Length];
for (int i = 0; i < A.Length; i++)
{
    C[i] = A[i] + B[i];
}
    \end{minted}
    \end{english}
    \end{boxSolution}
    \clearpage
    \fi

    % ======================================================
    % البحث والعد والإحصاء
    % ======================================================

    \item
    اكتب مقطع برنامج يحسب عدد العناصر الموجبة في مصفوفة الأعداد الصحيحة \textenglish{numbers}.
    \ifwithsols
    \begin{boxSolution}
    \begin{english}
    \begin{minted}{csharp}
int count = 0;
for (int i = 0; i < numbers.Length; i++)
{
    if (numbers[i] > 0)
        count++;
}
Console.WriteLine("Positive count: " + count);
    \end{minted}
    \end{english}
    \end{boxSolution}
    \fi

    \item
    اكتب مقطع برنامج يحسب مجموع العناصر الموجودة في الخانات الفردية لمصفوفة أعداد صحيحة \textenglish{arr}.
    \ifwithsols
    \begin{boxSolution}
    \begin{english}
    \begin{minted}{csharp}
int sum = 0;
for (int i = 1; i < arr.Length; i += 2)
{
    sum += arr[i];
}
Console.WriteLine("Sum at odd indices: " + sum);
    \end{minted}
    \end{english}
    \end{boxSolution}
    \clearpage
    \fi

    \item
    اكتب مقطع برنامج يعد كم عنصراً في مصفوفة العلامات \textenglish{grades} يقع ضمن المجال $[60, 80]$ (أي أكبر أو يساوي 60 وأصغر أو يساوي 80).
    \ifwithsols
    \begin{boxSolution}
    \begin{english}
    \begin{minted}{csharp}
int count = 0;
for (int i = 0; i < grades.Length; i++)
{
    if (grades[i] >= 60 && grades[i] <= 80)
        count++;
}
Console.WriteLine("Count in range [60-80]: " + count);
    \end{minted}
    \end{english}
    \end{boxSolution}
    \fi

    \item
    اكتب مقطع برنامج يبحث عن \textbf{أول} عدد سالب في مصفوفة أعداد صحيحة \textenglish{arr}.
    إذا وجده يطبع قيمته وموقعه ثم يتوقف عن البحث.
    \ifwithsols
    \begin{boxSolution}
    \begin{english}
    \begin{minted}{csharp}
for (int i = 0; i < arr.Length; i++)
{
    if (arr[i] < 0)
    {
        Console.WriteLine($"Found negative {arr[i]} at index {i}");
        break; // Stop loop
    }
}
    \end{minted}
    \end{english}
    \end{boxSolution}
    \clearpage
    \fi

    \item
    اكتب مقطع برنامج يفحص هل العدد \textenglish{50} موجود داخل المصفوفة \textenglish{numbers} أم لا، ويطبع \textenglish{"Found"} أو \textenglish{"Not Found"}.
    \ifwithsols
    \begin{boxSolution}
    \begin{english}
    \begin{minted}{csharp}
bool found = false;
for (int i = 0; i < numbers.Length; i++)
{
    if (numbers[i] == 50)
    {
        found = true;
        break;
    }
}
if (found) Console.WriteLine("Found");
else Console.WriteLine("Not Found");
    \end{minted}
    \end{english}
    \end{boxSolution}
    \clearpage
    \fi

    \item
    اكتب مقطع برنامج يحسب ويطبع الفرق بين أكبر قيمة وأصغر قيمة في مصفوفة أعداد صحيحة \textenglish{data}.\\
    \textbf{مثال:} إذا كانت المصفوفة تحتوي على \textenglish{\{3, 10, 6, 1\}}، الفرق هو $10 - 1 = 9$.
    \ifwithsols
    \begin{boxSolution}
    \begin{english}
    \begin{minted}{csharp}
if (data.Length > 0) {
    int max = data[0];
    int min = data[0];
    for (int i = 1; i < data.Length; i++)
    {
        if (data[i] > max) max = data[i];
        if (data[i] < min) min = data[i];
    }
    Console.WriteLine("Difference: " + (max - min));
}
    \end{minted}
    \end{english}
    \end{boxSolution}
    \clearpage
    \fi

    \item
    اكتب برنامجاً يقرأ 10 أعداد صحيحة من المستخدم ويطبع له كل الأعداد الأكبر من معدل الـ 10 أعداد كلها.\\
    \textbf{مثال:} إذا أدخل المستخدم الأعداد: 5, 10, 15, 20, 25, 30, 35, 40, 45, 50، يكون المعدل هو $27.5$، والأعداد الأكبر من المعدل هي: 30, 35, 40, 45, 50.
    \ifwithsols
    \begin{boxSolution}
    \begin{english}
    \begin{minted}{csharp}
int[] userNums = new int[10];
int sum = 0;

for (int i = 0; i < userNums.Length; i++)
{
    userNums[i] = int.Parse(Console.ReadLine());
    sum += userNums[i];
}

double avg = (double)sum / userNums.Length;

for (int i = 0; i < userNums.Length; i++)
{
    if(userNums[i] > avg)
        Console.WriteLine(userNums[i]);
}
    \end{minted}
    \end{english}
    \end{boxSolution}
    \fi

\end{enumerate}

\end{document}
