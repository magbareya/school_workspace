\documentclass[14pt]{extarticle}
% Full article preamble (duplicated, no common file)
\usepackage{fontspec}
\usepackage[a4paper,top=2.4cm,bottom=2.4cm,left=2.3cm,right=2.3cm]{geometry}
\usepackage{polyglossia}
\usepackage{amsmath}
\usepackage{amssymb}
\usepackage{xcolor}
\usepackage{fancyhdr}
\usepackage{graphicx}
\usepackage{listings}
\usepackage[most]{tcolorbox}
\usepackage{pifont}
\usepackage{enumitem}
\usepackage{titlesec}
\usepackage[bottom]{footmisc}
\usepackage{titling}
\usepackage{minted}
\usepackage{etoolbox}
\usepackage{array}
\usepackage{extsizes}

\newfontfamily\emoji{Segoe UI Emoji}

\pagestyle{fancy}

\setmainlanguage[numerals=western]{arabic}
\setotherlanguage{english}
\newfontfamily\arabicfont[Script=Arabic]{Amiri}
\newfontfamily\arabicfonttt[Script=Arabic]{Courier New}

\lstset{
  language=[Sharp]C,
  numbers=left,
  stepnumber=1,
  numbersep=8pt,
  frame=single,
  basicstyle=\ttfamily\small,
  keywordstyle=\color{blue},
  stringstyle=\color{red},
  commentstyle=\color{green!50!black}
}

\newif\ifdetailed
\ifdefined\setdetailed
  \setdetailed
\fi

\newif\ifwithsols
\ifdefined\setwithsols
  \setwithsols
\fi

% unified tcolorboxes for articles
\tcbset{colback=white, colframe=black, fonttitle=\bfseries, boxrule=0.8pt}
\newtcolorbox{boxDef}[1][]{colback=blue!5!white,colframe=blue!75!black,
  title={{\emoji📘} تعريف\ifx\\#1\\\else ~#1\fi :}}
\newtcolorbox{boxExercise}[1][]{colback=cyan!5!white,colframe=cyan!70!black,
  title={{\emoji🧩} تمرين\ifx\\#1\\\else ~#1\fi :}}
\newtcolorbox{boxExample}[1][]{colback=yellow!5!white,colframe=orange!90!black,
  title={{\emoji📝} مثال\ifx\\#1\\\else ~#1\fi :}}
\newtcolorbox{boxNote}[1][]{colback=gray!10!white,colframe=black,
  title={{\emoji✨} ملاحظة\ifx\\#1\\\else ~#1\fi :}}
\newtcolorbox{boxAttention}[1][]{colback=magenta!10!white,colframe=magenta!80!black,
  title={{\emoji🔔} تنبيه\ifx\\#1\\\else ~#1\fi :}}
\newtcolorbox{boxWarning}[1][]{colback=red!5!white,colframe=red!75!black,
  title={{\emoji⚡} ملاحظة هامة\ifx\\#1\\\else ~#1\fi :}}
\newtcolorbox{boxSolution}[1][]{colback=green!5!white,colframe=green!60!black,
  title={{\emoji✅} حل\ifx\\#1\\\else ~#1\fi :}}
\newtcolorbox{boxSymbol}[1][]{colback=purple!5!white,colframe=purple!70!black,
  title={{\emoji🔣} رمز\ifx\\#1\\\else ~#1\fi :}}
\newtcolorbox{boxHint}[1][]{colback=teal!5!white,colframe=teal!60!black,
  title={{\emoji💡} تلميح\ifx\\#1\\\else ~#1\fi :}}


\tcbset{simplecode/.style={ colback=gray!5, colframe=black!50, boxrule=0.4pt, arc=2pt, left=4pt,right=4pt,top=4pt,bottom=4pt}}
\newenvironment{boxCode}{\begin{tcolorbox}[simplecode]}{\end{tcolorbox}}

\newcolumntype{C}[1]{>{\centering\arraybackslash}p{#1}}

% redefine spaces after titles
\makeatletter
\renewcommand{\@maketitle}{%
  \begin{center}
    {\huge \bfseries \@title \par}%
    \vskip 0.2em % space between title and author
    {\large \@author \par}%
    % \vskip 0.2em % space between author and date
    % {\normalsize \@date \par}%
  \end{center}
}
\makeatother

\fancyhf{} % clear default
\fancypagestyle{plain}{
  \fancyhf{}
  \fancyhead[L]{مدرسة التسامح الشاملة}
  % \fancyhead[L]{\includegraphics[height=1cm]{../../../images/logoTasamoh.png}}
  \fancyhead[R]{الأستاذ محمود اغبارية}
  \fancyfoot[C]{\thepage}
}

\fancyhead[L]{مدرسة التسامح الشاملة}
\fancyhead[R]{الأستاذ محمود اغبارية}
\fancyfoot[C]{\thepage}
% \date{\today}

\setcounter{tocdepth}{3} % only section subsection and subsubsection in TOC


% ----------------------


% \begin{document}

% \maketitle

% % \clearpage  % start TOC on a new page
% % \renewcommand{\contentsname}{جدول المحتويات}
% % \tableofcontents
% % \clearpage

% \part*{part 1} % the * prevents numbering
% \section*{مقدمة}
% \subsection*{مثال رياضي}
% \subsubsection*{مثال فرعي}
% \paragraph*{ paragraph 1}
% \subparagraph*{sub paragraph 1}

% \ifdetailed
% \begin{english}
% \begin{minted}{csharp}
% // C# Example
% \end{minted}
% \end{english}
% \fi

% OLD WAY
% \ifdetailed
% \begin{english}
% \begin{lstlisting}
% // C# Example
% \end{lstlisting}
% \end{english}
% \fi

% % \includegraphics[width=0.2\textwidth]{../../../images/DFAs/ex1_q1.png}



% \vspace{3cm}
% \begin{flushleft}
% أرجو لكم وقتًا ممتعًا.

% الأستاذ محمود اغبارية.
% \end{flushleft}


% \end{document}


\ifwithsols
\title{حل ورقة تمرين 8 - المصفوفات الأحادية \\ قسم 4 - أسئلة عامة}
\else
\title{ورقة تمرين 8 - المصفوفات الأحادية \\ قسم 4 - أسئلة عامة}
\clearpage
\fi

\begin{document}

\maketitle
\thispagestyle{fancy}

\begin{enumerate}[itemsep=2em]

    \item
    اكتب عملية خارجية \textenglish{int SumPositive(int[] arr)} تتلقى مصفوفة أعداد صحيحة وتعيد مجموع الأعداد الموجبة فقط (الأكبر من صفر).\\
    \textbf{مثال:} إذا تلقت المصفوفة \textenglish{\{5, -3, 8, -2, 0, 10\}}، تعيد $5 + 8 + 10 = 23$.
    \ifwithsols
    \begin{boxSolution}
    \begin{english}
    \begin{minted}{csharp}
public static int SumPositive(int[] arr)
{
    int sum = 0;
    for (int i = 0; i < arr.Length; i++)
    {
        if (arr[i] > 0)
            sum += arr[i];
    }
    return sum;
}
    \end{minted}
    \end{english}
    \end{boxSolution}
    \fi

    \item
    أراد أحمد كتابة رسالة سرية لصديقه. قرر أن يكتب كلمتين: الأولى هي الرسالة الحقيقية والثانية هي رسالة زائفة للتمويه. ثم قام بدمج الكلمتين بحيث يأخذ حرفاً من الأولى ثم حرفاً من الثانية، وهكذا بالتناوب.\\
    اكتب عملية خارجية \textenglish{string MergeAlternating(string s1, string s2)} تتلقى نصين \textbf{بنفس الطول} وتعيد نصاً جديداً يدمج الحرفين بالتناوب.\\
    \textbf{مثال:} إذا تلقت \textenglish{"HELP"} و\textenglish{"XOXO"}، تعيد \textenglish{"HXEOLXPO"}.
    \ifwithsols
    \begin{boxSolution}
    \begin{english}
    \begin{minted}{csharp}
public static string MergeAlternating(string s1, string s2)
{
    string result = "";
    for (int i = 0; i < s1.Length; i++)
    {
        result += s1[i];
        result += s2[i];
    }
    return result;
}
    \end{minted}
    \end{english}
    \end{boxSolution}
    \clearpage
    \fi

    \item
    اكتب عملية خارجية \textenglish{int[] CreateFrequencyArray(int[] arr)} تتلقى مصفوفة أعداد صحيحة (تحتوي فقط على الأرقام من 0 إلى 9) وتعيد مصفوفة بطول 10، كل خانة فيها تحتوي على عدد مرات ظهور الرقم المقابل لها.\\
    \textbf{مثال:} إذا تلقت المصفوفة \textenglish{\{5, 2, 5, 8, 2, 5, 0\}}، تعيد:\\
    \textenglish{\{1, 0, 2, 0, 0, 3, 0, 0, 1, 0\}}\\
    (الصفر ظهر مرة، الواحد لم يظهر، الاثنان ظهر مرتين، ..., الخمسة ظهرت 3 مرات، ..., الثمانية ظهرت مرة)
    \ifwithsols
    \begin{boxSolution}
    \begin{english}
    \begin{minted}{csharp}
public static int[] CreateFrequencyArray(int[] arr)
{
    int[] frequency = new int[10];

    for (int i = 0; i < arr.Length; i++)
    {
        frequency[arr[i]]++;
    }

    return frequency;
}
    \end{minted}
    \end{english}
    \end{boxSolution}
    \fi

    \item
    اكتب برنامجاً لمكتب أرصاد جوية يستقبل 30 درجة حرارة (لمدة شهر).
    على البرنامج حساب ومقارنة معدل درجات الحرارة في \textbf{النصف الأول} من الشهر (الأيام 0-14) مع معدل درجات الحرارة في \textbf{النصف الثاني} (الأيام 15-29)، وطباعة رسالة توضح أي النصفين كان معدله أعلى.
    \ifwithsols
    \begin{boxSolution}
    \begin{english}
    \begin{minted}{csharp}
double sum1 = 0, sum2 = 0;
int half = 15; // 30 / 2

// Assuming array 'temps' is already filled
for (int i = 0; i < 30; i++)
{
    if (i < half)
        sum1 += temps[i]; // First half
    else
        sum2 += temps[i]; // Second half
}

double avg1 = sum1 / half;
double avg2 = sum2 / half; // Or remaining days

Console.WriteLine($"Avg1: {avg1}, Avg2: {avg2}");
if (avg1 > avg2) Console.WriteLine("First half was hotter");
else if (avg2 > avg1) Console.WriteLine("Second half was hotter");
else Console.WriteLine("Both halves were equal");
    \end{minted}
    \end{english}
    \end{boxSolution}
    \fi

    \clearpage
    \item
    في تطبيق للمطاعم، يقيّم الزبائن المطعم بتقييم من 0 إلى 5. \\
    يعتبر المطعم "ممتازًا" إذا تحقق \textbf{أحد هذه الشروط على الأقل}:
    \begin{itemize}
        \item حصل على تقييم 5 من أحد الزبائن على الأقل، ولم يحصل على تقييم 0 أبدًا.
        \item متوسط التقييمات هو $4.0$ أو أعلى.
    \end{itemize}
    \textbf{أمثلة:}
    \begin{itemize}
        \item إذا كانت التقييمات \textenglish{\{1, 2, 3, 4, 5\}}، يعتبر المطعم ممتازًا (لأن هناك تقييم 5 ولم يحصل على تقييم 0).
        \item إذا كانت التقييمات \textenglish{\{5, 4, 4, 5, 5, 5, 5, 0\}}، يعتبر المطعم ممتازًا، لأنّ معدل التقييمات هو $\frac{5 + 4 + 4 + 5 + 5 + 5 + 5 + 0}{8} = 4.125$، وهو أعلى من 4.0.
        \item إذا كانت التقييمات \textenglish{\{0, 5, 5, 4\}}، لا يعتبر المطعم ممتازًا لأنّه لم يتحقق أي من الشرطين (هناك تقييم 5، لكن هناك أيضًا تقييم 0).
        \item إذا كانت التقييمات \textenglish{\{3, 3, 3\}}، لا يعتبر المطعم ممتازًا.
    \end{itemize}

    اكتبوا عملية خارجية \textenglish{IsExcellent} تتلقى مصفوفة أعداد صحيحة تمثل تقييمات الزبائن (من 0 إلى 5) وتعيد \textenglish{true} إذا كان المطعم ممتازًا، و\textenglish{false} خلاف ذلك.
    \ifwithsols
    \begin{boxSolution}
    \begin{english}
    \begin{minted}{csharp}
public static bool IsExcellent(int[] ratings)
{
    bool hasFive = false;
    bool hasZero = false;
    int sum = 0;
    for (int i = 0; i < ratings.Length; i++)
    {
        if (ratings[i] == 5)
            hasFive = true;
        if (ratings[i] == 0)
            hasZero = true;
        sum += ratings[i];
    }
    double average = (double)sum / ratings.Length;
    if ((hasFive && !hasZero) || average >= 4.0)
        return true;
    else
        return false;
}
    \end{minted}
    \end{english}
    \end{boxSolution}
    \fi

    \item
    معطى ثلاث مصفوفات تخزن بيانات كتب:
    \begin{itemize}
        \item \textenglish{names}: مصفوفة نصوص لأسماء الكتب.
        \item \textenglish{prices}: مصفوفة أعداد عشرية تمثل سعر كل كتاب.
        \item \textenglish{years}: مصفوفة أعداد صحيحة لسنة إصدار كل كتاب.
    \end{itemize}
    بحيث \textenglish{names[i]}، \textenglish{prices[i]}، و\textenglish{years[i]} تمثل بيانات نفس الكتاب \textenglish{i}.
    \begin{enumerate}[label=\alph*.]
        \item
    اكتب دالة تتلقى هذه المصفوفات الثلاث، وتطبع \textbf{اسم} الكتاب الذي له أكبر سعر.
    \ifwithsols
    \begin{boxSolution}
    \begin{english}
    \begin{minted}{csharp}
public static void PrintMostExpensiveBook(string[] names, double[] prices)
{
    double maxPrice = prices[0];
    int maxIndex = 0;

    for (int i = 1; i < prices.Length; i++)
    {
        if (prices[i] > maxPrice)
        {
            maxPrice = prices[i];
            maxIndex = i;
        }
    }

    Console.WriteLine("Most expensive book: " + names[maxIndex]);
}
    \end{minted}
    \end{english}
    \end{boxSolution}
    \fi

    \item
    اكتب دالة تتلقى المصفوفات وسنة معينة \textenglish{year}، وتقوم بطباعة أسماء جميع الكتب التي صدرت \textbf{بعد} هذه السنة.
    \ifwithsols
    \begin{boxSolution}
    \begin{english}
    \begin{minted}{csharp}
public static void PrintRecentBooks(string[] names, int[] years, int year)
{
    Console.WriteLine($"Books published after {year}:");
    bool found = false;

    for (int i = 0; i < years.Length; i++)
    {
        if (years[i] > year)
        {
            Console.WriteLine(names[i]);
            found = true;
        }
    }

    if (!found) Console.WriteLine("No books found.");
}
    \end{minted}
    \end{english}
    \end{boxSolution}
    \fi
\end{enumerate}

\end{enumerate}

\end{document}