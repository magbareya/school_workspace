\documentclass[14pt]{extarticle}
% Full article preamble (duplicated, no common file)
\usepackage{fontspec}
\usepackage[a4paper,top=2.4cm,bottom=2.4cm,left=2.3cm,right=2.3cm]{geometry}
\usepackage{polyglossia}
\usepackage{amsmath}
\usepackage{amssymb}
\usepackage{xcolor}
\usepackage{fancyhdr}
\usepackage{graphicx}
\usepackage{listings}
\usepackage[most]{tcolorbox}
\usepackage{pifont}
\usepackage{enumitem}
\usepackage{titlesec}
\usepackage[bottom]{footmisc}
\usepackage{titling}
\usepackage{minted}
\usepackage{etoolbox}
\usepackage{array}
\usepackage{extsizes}

\newfontfamily\emoji{Segoe UI Emoji}

\pagestyle{fancy}

\setmainlanguage[numerals=western]{arabic}
\setotherlanguage{english}
\newfontfamily\arabicfont[Script=Arabic]{Amiri}
\newfontfamily\arabicfonttt[Script=Arabic]{Courier New}

\lstset{
  language=[Sharp]C,
  numbers=left,
  stepnumber=1,
  numbersep=8pt,
  frame=single,
  basicstyle=\ttfamily\small,
  keywordstyle=\color{blue},
  stringstyle=\color{red},
  commentstyle=\color{green!50!black}
}

\newif\ifdetailed
\ifdefined\setdetailed
  \setdetailed
\fi

\newif\ifwithsols
\ifdefined\setwithsols
  \setwithsols
\fi

% unified tcolorboxes for articles
\tcbset{colback=white, colframe=black, fonttitle=\bfseries, boxrule=0.8pt}
\newtcolorbox{boxDef}[1][]{colback=blue!5!white,colframe=blue!75!black,
  title={{\emoji📘} تعريف\ifx\\#1\\\else ~#1\fi :}}
\newtcolorbox{boxExercise}[1][]{colback=cyan!5!white,colframe=cyan!70!black,
  title={{\emoji🧩} تمرين\ifx\\#1\\\else ~#1\fi :}}
\newtcolorbox{boxExample}[1][]{colback=yellow!5!white,colframe=orange!90!black,
  title={{\emoji📝} مثال\ifx\\#1\\\else ~#1\fi :}}
\newtcolorbox{boxNote}[1][]{colback=gray!10!white,colframe=black,
  title={{\emoji✨} ملاحظة\ifx\\#1\\\else ~#1\fi :}}
\newtcolorbox{boxAttention}[1][]{colback=magenta!10!white,colframe=magenta!80!black,
  title={{\emoji🔔} تنبيه\ifx\\#1\\\else ~#1\fi :}}
\newtcolorbox{boxWarning}[1][]{colback=red!5!white,colframe=red!75!black,
  title={{\emoji⚡} ملاحظة هامة\ifx\\#1\\\else ~#1\fi :}}
\newtcolorbox{boxSolution}[1][]{colback=green!5!white,colframe=green!60!black,
  title={{\emoji✅} حل\ifx\\#1\\\else ~#1\fi :}}
\newtcolorbox{boxSymbol}[1][]{colback=purple!5!white,colframe=purple!70!black,
  title={{\emoji🔣} رمز\ifx\\#1\\\else ~#1\fi :}}
\newtcolorbox{boxHint}[1][]{colback=teal!5!white,colframe=teal!60!black,
  title={{\emoji💡} تلميح\ifx\\#1\\\else ~#1\fi :}}


\tcbset{simplecode/.style={ colback=gray!5, colframe=black!50, boxrule=0.4pt, arc=2pt, left=4pt,right=4pt,top=4pt,bottom=4pt}}
\newenvironment{boxCode}{\begin{tcolorbox}[simplecode]}{\end{tcolorbox}}

\newcolumntype{C}[1]{>{\centering\arraybackslash}p{#1}}

% redefine spaces after titles
\makeatletter
\renewcommand{\@maketitle}{%
  \begin{center}
    {\huge \bfseries \@title \par}%
    \vskip 0.2em % space between title and author
    {\large \@author \par}%
    % \vskip 0.2em % space between author and date
    % {\normalsize \@date \par}%
  \end{center}
}
\makeatother

\fancyhf{} % clear default
\fancypagestyle{plain}{
  \fancyhf{}
  \fancyhead[L]{مدرسة التسامح الشاملة}
  % \fancyhead[L]{\includegraphics[height=1cm]{../../../images/logoTasamoh.png}}
  \fancyhead[R]{الأستاذ محمود اغبارية}
  \fancyfoot[C]{\thepage}
}

\fancyhead[L]{مدرسة التسامح الشاملة}
\fancyhead[R]{الأستاذ محمود اغبارية}
\fancyfoot[C]{\thepage}
% \date{\today}

\setcounter{tocdepth}{3} % only section subsection and subsubsection in TOC


% ----------------------


% \begin{document}

% \maketitle

% % \clearpage  % start TOC on a new page
% % \renewcommand{\contentsname}{جدول المحتويات}
% % \tableofcontents
% % \clearpage

% \part*{part 1} % the * prevents numbering
% \section*{مقدمة}
% \subsection*{مثال رياضي}
% \subsubsection*{مثال فرعي}
% \paragraph*{ paragraph 1}
% \subparagraph*{sub paragraph 1}

% \ifdetailed
% \begin{english}
% \begin{minted}{csharp}
% // C# Example
% \end{minted}
% \end{english}
% \fi

% OLD WAY
% \ifdetailed
% \begin{english}
% \begin{lstlisting}
% // C# Example
% \end{lstlisting}
% \end{english}
% \fi

% % \includegraphics[width=0.2\textwidth]{../../../images/DFAs/ex1_q1.png}



% \vspace{3cm}
% \begin{flushleft}
% أرجو لكم وقتًا ممتعًا.

% الأستاذ محمود اغبارية.
% \end{flushleft}


% \end{document}


\title{ورقة تمرن 7 للصف العاشر 10 \\ جمل شرطية متداخلة وجدول متابعة}

\begin{document}
\maketitle
\thispagestyle{fancy}

\setlist[enumerate]{itemsep=0.8em, topsep=0.5em}

\begin{enumerate}[itemsep=2em]

% -------------------------------------------------------
\item
أمامك البرنامج التالي:
\begin{boxCode}
\begin{english}
\begin{minted}{csharp}
private static void Main(string[] args)
{
    int count = 0;
    int grade = int.Parse(Console.ReadLine());
    if (grade >= 70)
        count++;
    grade = int.Parse(Console.ReadLine());
    if (grade >= 70)
        count++;
    grade = int.Parse(Console.ReadLine());
    if (grade >= 70)
        count++;
    grade = int.Parse(Console.ReadLine());
    if (grade >= 70)
        count++;
    Console.WriteLine(count);
}
\end{minted}
\end{english}
\end{boxCode}

\begin{enumerate}
\item اكتب جدول متابعة للكود أعلاه بحيث يحتوي على عمود لكل متغير، وعمود لكل شرط، وعمود للطباعة، عندما تكون المدخلات هي: $87, 77, 68, 93$.
\item ما هي وظيفة البرنامج أعلاه؟
\item أعط مثالًا لمدخلات ينتج عنها طباعة $0$.
\item أعط مثالًا لمدخلات ينتج عنها طباعة $4$.
\end{enumerate}


\clearpage
\item
أمامك البرنامج التالي:
\begin{boxCode}
\begin{english}
\begin{minted}{csharp}
private static void Main(string[] args)
{
    int sum = 0;
    int num = int.Parse(Console.ReadLine());
    sum += num;
    num = int.Parse(Console.ReadLine());
    sum += num;
    num = int.Parse(Console.ReadLine());
    sum += num;
    num = int.Parse(Console.ReadLine());
    sum += num;
    Console.WriteLine(sum);
}
\end{minted}
\end{english}
\end{boxCode}

\begin{enumerate}
\item اكتب جدول متابعة للكود أعلاه بحيث يحتوي على عمود لكل متغير، وعمود لكل شرط، وعمود للطباعة، عندما تكون المدخلات هي: $12, 17, 3, 8$.
\item ما هي وظيفة البرنامج أعلاه؟
\item أعط مثالًا لمدخلات ينتج عنها طباعة $4$.
\end{enumerate}


\clearpage
\item
أمامك البرنامج التالي:
\begin{boxCode}
\begin{english}
\begin{minted}{csharp}
private static void Main(string[] args)
{
    Console.WriteLine("Please enter 4 positive numbers:");
    int m = 0;
    int n = int.Parse(Console.ReadLine());
    if (n > m)
        m = n;
    n = int.Parse(Console.ReadLine());
    if (n > m)
        m = n;
    n = int.Parse(Console.ReadLine());
    if (n > m)
        m = n;
    n = int.Parse(Console.ReadLine());
    if (n > m)
        m = n;
    Console.WriteLine(m);
}
\end{minted}
\end{english}
\end{boxCode}

\begin{enumerate}
\item اكتب جدول متابعة للكود أعلاه بحيث يحتوي على عمود لكل متغير، وعمود لكل شرط، وعمود للطباعة، عندما تكون المدخلات هي: $23, 11, 34, 40$.
\item ما هي وظيفة البرنامج أعلاه؟
\item أعط مثالًا لمدخلات ينتج عنها طباعة $7$.
\end{enumerate}

\clearpage
\item
أمامك البرنامج التالي:
\begin{boxCode}
\begin{english}
\begin{minted}{csharp}
private static void Main(string[] args)
{
    int n = int.Parse(Console.ReadLine());
    int m = int.Parse(Console.ReadLine());
    if (n == 0 || m == 0)
        Console.WriteLine("1");
    else
    {
        if (n % m == 0)
            Console.WriteLine("2");
        Console.WriteLine("3");
        if (m % n == 0)
            Console.WriteLine("4");
        else
            Console.WriteLine("5");
    }
}
\end{minted}
\end{english}
\end{boxCode}

\begin{enumerate}
\item
املأ الجدول التالي بناءً على المدخلات المختلفة: لكل زوج من القيم $n$ و $m$،
اكتب \textenglish{true} أو \textenglish{false} في الأعمدة الخاصة بالشروط الثلاثة،
وضع علامة صح ($\checkmark$) في الأعمدة التي ستتمّ طباعتها عند تشغيل البرنامج مع هذه المدخلات.

\begin{tabular}{|c|c|c|c|c|c|c|c|c|c|}
\hline
\textenglish{n} & \textenglish{m} &
$n==0 || m==0$ &
$n \% m==0$ &
$m \% n==0$ &
1 &
2 &
3 &
4 &
5 \\ \hline
$0$ & $5$ &  &  &  &  &  &  &  &  \\ \hline
$7$ & $0$ &  &  &  &  &  &  &  &  \\ \hline
$4$ & $2$ &  &  &  &  &  &  &  &  \\ \hline
$2$ & $4$ &  &  &  &  &  &  &  &  \\ \hline
$3$ & $3$ &  &  &  &  &  &  &  &  \\ \hline
$5$ & $7$ &  &  &  &  &  &  &  &  \\ \hline
$8$ & $4$ &  &  &  &  &  &  &  &  \\ \hline
$9$ & $3$ &  &  &  &  &  &  &  &  \\ \hline
$6$ & $2$ &  &  &  &  &  &  &  &  \\ \hline
$2$ & $6$ &  &  &  &  &  &  &  &  \\ \hline
\end{tabular}

\item هل يمكنك إعطاء قيم تجعل البرنامج يطبع الأرقام 1 و 2 معًا (سواء مع طباعة أرقام أخرى أو لا)؟
\item هل يمكنك إعطاء قيم تجعل البرنامج يطبع الأرقام 4 و 5 معًا (سواء مع طباعة أرقام أخرى أو لا)؟
\item هل يمكنك إعطاء قيم بحيث \textbf{لا} يطبع البرنامج الرقم 1 \textbf{ولا} الرقم 3؟

\end{enumerate}

\clearpage
\item
اكتب برنامجًا يستقبل من المستخدم 4 أعداد صحيحة، ويطبع أصغر عدد من بينها. \\
\begin{boxWarning}
على البرنامج أن يحتوي فقط على متغيّرين اثنين: $min$ و $num$ كلاهما من نمط عدد صحيح (\textenglish{int}).
\end{boxWarning}


\item
اكتب برنامجًا يستقبل من المستخدم 4 أعداد صحيحة، ويطبع كم عددًا زوجيًّا كان من بينها. \\
\textbf{مثال:} إذا كانت الأعداد المدخلة هي $3, 4, 7, 8$ فإن البرنامج يجب أن يطبع $2$، لأنّه يوجد فقط عددان زوجيان ($4$ و $8$). \\
\begin{boxWarning}
على البرنامج أن يحتوي فقط على متغيّرين اثنين: $count$ و $n$ كلاهما من نمط عدد صحيح (\textenglish{int}).
\end{boxWarning}


\item
اكتب برنامجًا يستقبل من المستخدم 4 أعداد عشريّة، ويطبع معدّلها (معدّل الأعداد = مجموعها تقسيم عددها). \\
\begin{boxWarning}
على البرنامج أن يحتوي فقط على متغيّرين اثنين: $average$ و $num$ كلاهما من نمط عدد عشري (\textenglish{double}).
\end{boxWarning}

\clearpage
\item
اكتب برنامجًا يقرأ من المستخدم 4 أعداد صحيحة ويطبع \textbf{مجموع} الأرقام الفردية فقط. \\
\textbf{مثال:} إذا كانت الأعداد المدخلة هي $5, 2, 7, 4$ فإن البرنامج يجب أن يطبع $12$، لأنّ مجموع الأعداد الفردية ($5 + 7$) يساوي $12$.
\begin{boxWarning}
على البرنامج أن يحتوي فقط على متغيّرين اثنين: $sum$ و $n$ كلاهما من نمط عدد صحيح (\textenglish{int}).
\end{boxWarning}


\item
اكتب برنامجًا يقرأ من المستخدم 4 أعداد صحيحة ويطبع \textbf{مجموع} الأرقام الموجبة على حدة، والسالبة على حدة. \\
\textbf{مثال:} إذا كانت الأعداد المدخلة هي $5, -2, 7, -4$ فإن البرنامج يجب أن يطبع $12$ للموجبة ($5 + 7$) و $-6$ للسالبة ($-2 + -4$).
\begin{boxWarning}
على البرنامج أن يحتوي فقط على ثلاثة متغيّرات: $posSum$ و $negSum$ و $n$ كلها من نمط عدد صحيح (\textenglish{int}).
\end{boxWarning}

\clearpage
\item
اكتب مقطع برنامج يستقبل من المستخدم عددًا صحيحًا، ويطبع:
\begin{itemize}
\item "موجب زوجي" إذا كان العدد موجبًا وزوجيًّا.
\item "موجب فردي" إذا كان العدد موجبًا وفرديًّا.
\item "سالب" إذا كان العدد سالبًا.
\item "صفر" إذا كان العدد صفرًا.
\end{itemize}
\begin{boxWarning}
ممنوع استخدام \textenglish{\&\&} أو \textenglish{||} في الشروط.
\end{boxWarning}

\item
للتذكير: رأينا في تمرين سابق كيف نحسب مؤشر كتلة الجسم (BMI) باستخدام المعادلة التالية:
\[\text{BMI} = \frac{\text{الوزن بالكيلوغرام}}{(\text{الطول بالمتر})^2}\]

وباستخدام مؤشر كتلة الجسم يمكننا تصنيف حالة الشخص الصحية كما في الجدول التالي:
\begin{center}
\begin{tabular}{|c|c|}
\hline
\textbf{التصنيف} & \textbf{القيمة (BMI)} \\ \hline
نقص في الوزن & $\text{BMI} < 18.5$ \\ \hline
وزن طبيعي & $18.5 \leq \text{BMI} < 25$ \\ \hline
زيادة في الوزن & $25 \leq \text{BMI} < 30$ \\ \hline
سمنة من الدرجة الأولى & $30 \leq \text{BMI} < 35$ \\ \hline
سمنة من الدرجة الثانية & $35 \leq \text{BMI} < 40$ \\ \hline
سمنة مفرطة (درجة ثالثة) & $\text{BMI} \geq 40$ \\ \hline
\end{tabular}
\end{center}

\begin{enumerate}
    \item اكتب عملية خارجية باسم \textenglish{CalculateBMI} تتلقى الوزن (بالكيلوغرام) والطول (بالمتر) وتُعيد قيمة مؤشر كتلة الجسم (BMI). \\
    \textbf{مثال:} إذا تلقت الوزن $80$ كيلوغرام والطول $1.8$ متر تعيد $24.22$.

    \item اكتب برنامجًا يستقبل من المستخدم الوزن والطول، ويحسب مؤشر كتلة الجسم باستخدام العملية السابقة، ثم يطبع القيمة التي أرجعتها العملية.

    \item اكتب عملية خارجية إضافية باسم \textenglish{BMICategory} تتلقى قيمة مؤشر كتلة الجسم (BMI) وتطبع التصنيف المناسب حسب الجدول أعلاه. \\
    \textbf{مثال:} إذا تلقت قيمة $24.22$ تطبع "وزن طبيعي".

    \item عدّل للبرنامج السابق (من بند ب) بحيث يستدعي العملية الجديدة بعد حساب مؤشر كتلة الجسم، ليطبع التصنيف المناسب أيضًا.

    \item اكتب عملية خارجية إضافية باسم \textenglish{BMI} تتلقى الوزن والطول، وتطبع التصنيف المناسب حسب الجدول أعلاه، واستدعها من البرنامج الرئيسي. \\
    \textbf{مثال:} إذا تلقت الوزن $80$ كيلوغرام والطول $1.8$ متر تطبع "وزن طبيعي".
    \begin{boxHint}
         استخدم العمليتين السابقتين داخل هذه العملية.
    \end{boxHint}
\end{enumerate}

% \item
% \begin{enumerate}
%     \item اكتب عملية خارجية باسم \textenglish{AverageTwo} تتلقى عددين صحيحين وتُعيد معدلهما الحسابي (أي مجموعهما تقسيم 2). \\
%     \textbf{مثال:} إذا تلقت العددين $8$ و $6$ تعيد $7$.

%     \item اكتب برنامجًا يستقبل من المستخدم عددين صحيحَين ويطبع له معدّلهما باستخدام العملية السابقة.

%     \item اكتب برنامجًا يطلب من المستخدم 4 علامات، ويحسب معدلها باستخدام العملية السابقة أكثر من مرة. \\
%     \begin{boxHint}
%          استخدم العملية لحساب معدل كل زوج، ثم معدل الناتجين.  \\
%          \textbf{مثال:} إذا كانت العلامات المدخلة هي $80, 90, 70, 60$ فإن البرنامج يحسب أولًا معدل $80$ و $90$، ثم معدل $70$ و $60$، وأخيرًا يحسب معدل الناتجين ويطبعه. \\
%         \end{boxHint}
% \end{enumerate}

\clearpage
\item
\begin{enumerate}
    \item اكتب عملية خارجية باسم \textenglish{ToSeconds} تتلقى ثلاث قيم: عدد الساعات، عدد الدقائق، وعدد الثواني. \\
    على العملية أن تُعيد عدد الثواني الكلي الناتج عن تحويل الساعات والدقائق والثواني إلى ثوانٍ. \\
    \begin{boxExample}
         إذا تلقت العملية $1$، $2$، $30$، فعليها أن تُعيد $3750$ لأنّ \\ $1 \times 3600 + 2 \times 60 + 30 = 3750$.
    \end{boxExample}

    \item اكتب برنامجًا يطلب من المستخدم إدخال عدد الساعات والدقائق والثواني، ثم يستدعي العملية ويطبع الناتج الذي أعادته.

    \item اكتب برنامجًا يستقبل من المستخدم وقتيْن (ساعات، دقائق، ثوانٍ لكل وقت). \\
    على البرنامج أن يطبع \textenglish{Equal} إذا كان الوقتان متساويين من حيث عدد الثواني، و \textenglish{Not Equal} إذا لم يكونا متساويين.
    \begin{boxExample}
         إذا كانت المدخلات هي $1$، $30$، $0$ للوقت الأول، و $1$، $29$، $60$ للوقت الثاني، فإن البرنامج يطبع \textenglish{Equal} لأنّ كلا الوقتين يساوي $5400$ ثانية.
    \end{boxExample}
\end{enumerate}


% \begin{boxCode}
% \begin{english}
% \begin{minted}{csharp}

% \end{minted}
% \end{english}
% \end{boxCode}

\end{enumerate}
\end{document}
