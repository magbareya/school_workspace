\documentclass[14pt]{extarticle}
% Full article preamble (duplicated, no common file)
\usepackage{fontspec}
\usepackage[a4paper,top=2.4cm,bottom=2.4cm,left=2.3cm,right=2.3cm]{geometry}
\usepackage{polyglossia}
\usepackage{amsmath}
\usepackage{amssymb}
\usepackage{xcolor}
\usepackage{fancyhdr}
\usepackage{graphicx}
\usepackage{listings}
\usepackage[most]{tcolorbox}
\usepackage{pifont}
\usepackage{enumitem}
\usepackage{titlesec}
\usepackage[bottom]{footmisc}
\usepackage{titling}
\usepackage{minted}
\usepackage{etoolbox}
\usepackage{array}
\usepackage{extsizes}

\newfontfamily\emoji{Segoe UI Emoji}

\pagestyle{fancy}

\setmainlanguage[numerals=western]{arabic}
\setotherlanguage{english}
\newfontfamily\arabicfont[Script=Arabic]{Amiri}
\newfontfamily\arabicfonttt[Script=Arabic]{Courier New}

\lstset{
  language=[Sharp]C,
  numbers=left,
  stepnumber=1,
  numbersep=8pt,
  frame=single,
  basicstyle=\ttfamily\small,
  keywordstyle=\color{blue},
  stringstyle=\color{red},
  commentstyle=\color{green!50!black}
}

\newif\ifdetailed
\ifdefined\setdetailed
  \setdetailed
\fi

\newif\ifwithsols
\ifdefined\setwithsols
  \setwithsols
\fi

% unified tcolorboxes for articles
\tcbset{colback=white, colframe=black, fonttitle=\bfseries, boxrule=0.8pt}
\newtcolorbox{boxDef}[1][]{colback=blue!5!white,colframe=blue!75!black,
  title={{\emoji📘} تعريف\ifx\\#1\\\else ~#1\fi :}}
\newtcolorbox{boxExercise}[1][]{colback=cyan!5!white,colframe=cyan!70!black,
  title={{\emoji🧩} تمرين\ifx\\#1\\\else ~#1\fi :}}
\newtcolorbox{boxExample}[1][]{colback=yellow!5!white,colframe=orange!90!black,
  title={{\emoji📝} مثال\ifx\\#1\\\else ~#1\fi :}}
\newtcolorbox{boxNote}[1][]{colback=gray!10!white,colframe=black,
  title={{\emoji✨} ملاحظة\ifx\\#1\\\else ~#1\fi :}}
\newtcolorbox{boxAttention}[1][]{colback=magenta!10!white,colframe=magenta!80!black,
  title={{\emoji🔔} تنبيه\ifx\\#1\\\else ~#1\fi :}}
\newtcolorbox{boxWarning}[1][]{colback=red!5!white,colframe=red!75!black,
  title={{\emoji⚡} ملاحظة هامة\ifx\\#1\\\else ~#1\fi :}}
\newtcolorbox{boxSolution}[1][]{colback=green!5!white,colframe=green!60!black,
  title={{\emoji✅} حل\ifx\\#1\\\else ~#1\fi :}}
\newtcolorbox{boxSymbol}[1][]{colback=purple!5!white,colframe=purple!70!black,
  title={{\emoji🔣} رمز\ifx\\#1\\\else ~#1\fi :}}
\newtcolorbox{boxHint}[1][]{colback=teal!5!white,colframe=teal!60!black,
  title={{\emoji💡} تلميح\ifx\\#1\\\else ~#1\fi :}}


\tcbset{simplecode/.style={ colback=gray!5, colframe=black!50, boxrule=0.4pt, arc=2pt, left=4pt,right=4pt,top=4pt,bottom=4pt}}
\newenvironment{boxCode}{\begin{tcolorbox}[simplecode]}{\end{tcolorbox}}

\newcolumntype{C}[1]{>{\centering\arraybackslash}p{#1}}

% redefine spaces after titles
\makeatletter
\renewcommand{\@maketitle}{%
  \begin{center}
    {\huge \bfseries \@title \par}%
    \vskip 0.2em % space between title and author
    {\large \@author \par}%
    % \vskip 0.2em % space between author and date
    % {\normalsize \@date \par}%
  \end{center}
}
\makeatother

\fancyhf{} % clear default
\fancypagestyle{plain}{
  \fancyhf{}
  \fancyhead[L]{مدرسة التسامح الشاملة}
  % \fancyhead[L]{\includegraphics[height=1cm]{../../../images/logoTasamoh.png}}
  \fancyhead[R]{الأستاذ محمود اغبارية}
  \fancyfoot[C]{\thepage}
}

\fancyhead[L]{مدرسة التسامح الشاملة}
\fancyhead[R]{الأستاذ محمود اغبارية}
\fancyfoot[C]{\thepage}
% \date{\today}

\setcounter{tocdepth}{3} % only section subsection and subsubsection in TOC


% ----------------------


% \begin{document}

% \maketitle

% % \clearpage  % start TOC on a new page
% % \renewcommand{\contentsname}{جدول المحتويات}
% % \tableofcontents
% % \clearpage

% \part*{part 1} % the * prevents numbering
% \section*{مقدمة}
% \subsection*{مثال رياضي}
% \subsubsection*{مثال فرعي}
% \paragraph*{ paragraph 1}
% \subparagraph*{sub paragraph 1}

% \ifdetailed
% \begin{english}
% \begin{minted}{csharp}
% // C# Example
% \end{minted}
% \end{english}
% \fi

% OLD WAY
% \ifdetailed
% \begin{english}
% \begin{lstlisting}
% // C# Example
% \end{lstlisting}
% \end{english}
% \fi

% % \includegraphics[width=0.2\textwidth]{../../../images/DFAs/ex1_q1.png}



% \vspace{3cm}
% \begin{flushleft}
% أرجو لكم وقتًا ممتعًا.

% الأستاذ محمود اغبارية.
% \end{flushleft}


% \end{document}


\title{ورقة تمرن 8 للصف العاشر 10 - حلقات متداخلة}

\begin{document}

\maketitle
\thispagestyle{fancy}

\section{الحلقات المتداخلة}
\begin{enumerate}
    \item استخدم \textenglish{for} داخل  \textenglish{for} لطباعة الجدول التالي:
    \begin{align*}
      &1\ 2\ 3\ 4\ 5 \\
      &1\ 2\ 3\ 4\ 5 \\
      &1\ 2\ 3\ 4\ 5
    \end{align*}
    \item اطبع مصفوفة نجوم $4 \times 6$ (4 أسطر، كل سطر 6 نجوم).
    % \item  \textenglish{for} داخل  \textenglish{while}: اقرأ 5 أعداد، وإذا كان أحدها سالبًا أعد طلبه حتى يدخل عددًا صحيحًا.
    \item استخدم \textenglish{for} داخل  \textenglish{for} لطباعة جميع الأزواج $(i,j)$ حيث $i=1..3$ و $j=1..4$.
    \item استخدم \textenglish{while} داخل  \textenglish{for}: اطلب عددًا موجبًا، وإذا كان صفرًا أو سالبًا أعد طلبه، وبعدها اطبع جدول ضربه حتى ×10.
    \item استخدم \textenglish{for} داخل  \textenglish{for} لطباعة مثلث النجوم:
    \begin{english}
    \begin{align*}
    * \\
    ** \\
    *** \\
    ****
    \end{align*}
\end{english}
    \item استخدم \textenglish{for داخل while}: اقرأ عدد الأسطر، وإذا كان سلبيًا اطلبه مرة أخرى، ثم أطبع من 1 حتى طول السطر في كل سطر.
    \item استخدم \textenglish{for داخل for} لطباعة جميع الأزواج $(i,j)$ حيث $j>i$ و $i,j$ من 1 إلى 5.
    \item استخدم \textenglish{while داخل for}: اطبع الأعداد من 1 إلى 5، لكن قبل كل عدد اطلب من المستخدم إدخال ``y'' للتأكيد.
    \item استخدم \textenglish{for داخل while}: اقرأ أعدادًا حتى يدخل المستخدم عددًا سالبًا، ولكل عدد موجب اطبع كلمة ``Hello'' بعدد ذلك العدد.

    \item اكتب عملية تتلقى عددًا $n$ وتطبع مثلثًا من الأرقام كما في الشكل (للمدخل 4):
\begin{english}
\begin{verbatim}
1
1 2
1 2 3
1 2 3 4
\end{verbatim}
\end{english}
\ifwithsols
\begin{boxSolution}
\begin{english}
\begin{minted}{csharp}
public static void PrintNumTriangle(int n)
{
    for (int i = 1; i <= n; i++)
    {
        for (int j = 1; j <= i; j++)
        {
            Console.Write(j + " ");
        }
        Console.WriteLine();
    }
}
\end{minted}
\end{english}
\end{boxSolution}
\fi
\end{enumerate}

\vspace{1cm}
\begin{flushleft}
أرجو لكم وقتًا ممتعًا.

الأستاذ محمود اغبارية.
\end{flushleft}

\end{document}



