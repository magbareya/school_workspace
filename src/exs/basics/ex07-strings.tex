\documentclass[14pt]{extarticle}
% Full article preamble (duplicated, no common file)
\usepackage{fontspec}
\usepackage[a4paper,top=2.4cm,bottom=2.4cm,left=2.3cm,right=2.3cm]{geometry}
\usepackage{polyglossia}
\usepackage{amsmath}
\usepackage{amssymb}
\usepackage{xcolor}
\usepackage{fancyhdr}
\usepackage{graphicx}
\usepackage{listings}
\usepackage[most]{tcolorbox}
\usepackage{pifont}
\usepackage{enumitem}
\usepackage{titlesec}
\usepackage[bottom]{footmisc}
\usepackage{titling}
\usepackage{minted}
\usepackage{etoolbox}
\usepackage{array}
\usepackage{extsizes}

\newfontfamily\emoji{Segoe UI Emoji}

\pagestyle{fancy}

\setmainlanguage[numerals=western]{arabic}
\setotherlanguage{english}
\newfontfamily\arabicfont[Script=Arabic]{Amiri}
\newfontfamily\arabicfonttt[Script=Arabic]{Courier New}

\lstset{
  language=[Sharp]C,
  numbers=left,
  stepnumber=1,
  numbersep=8pt,
  frame=single,
  basicstyle=\ttfamily\small,
  keywordstyle=\color{blue},
  stringstyle=\color{red},
  commentstyle=\color{green!50!black}
}

\newif\ifdetailed
\ifdefined\setdetailed
  \setdetailed
\fi

\newif\ifwithsols
\ifdefined\setwithsols
  \setwithsols
\fi

% unified tcolorboxes for articles
\tcbset{colback=white, colframe=black, fonttitle=\bfseries, boxrule=0.8pt}
\newtcolorbox{boxDef}[1][]{colback=blue!5!white,colframe=blue!75!black,
  title={{\emoji📘} تعريف\ifx\\#1\\\else ~#1\fi :}}
\newtcolorbox{boxExercise}[1][]{colback=cyan!5!white,colframe=cyan!70!black,
  title={{\emoji🧩} تمرين\ifx\\#1\\\else ~#1\fi :}}
\newtcolorbox{boxExample}[1][]{colback=yellow!5!white,colframe=orange!90!black,
  title={{\emoji📝} مثال\ifx\\#1\\\else ~#1\fi :}}
\newtcolorbox{boxNote}[1][]{colback=gray!10!white,colframe=black,
  title={{\emoji✨} ملاحظة\ifx\\#1\\\else ~#1\fi :}}
\newtcolorbox{boxAttention}[1][]{colback=magenta!10!white,colframe=magenta!80!black,
  title={{\emoji🔔} تنبيه\ifx\\#1\\\else ~#1\fi :}}
\newtcolorbox{boxWarning}[1][]{colback=red!5!white,colframe=red!75!black,
  title={{\emoji⚡} ملاحظة هامة\ifx\\#1\\\else ~#1\fi :}}
\newtcolorbox{boxSolution}[1][]{colback=green!5!white,colframe=green!60!black,
  title={{\emoji✅} حل\ifx\\#1\\\else ~#1\fi :}}
\newtcolorbox{boxSymbol}[1][]{colback=purple!5!white,colframe=purple!70!black,
  title={{\emoji🔣} رمز\ifx\\#1\\\else ~#1\fi :}}
\newtcolorbox{boxHint}[1][]{colback=teal!5!white,colframe=teal!60!black,
  title={{\emoji💡} تلميح\ifx\\#1\\\else ~#1\fi :}}


\tcbset{simplecode/.style={ colback=gray!5, colframe=black!50, boxrule=0.4pt, arc=2pt, left=4pt,right=4pt,top=4pt,bottom=4pt}}
\newenvironment{boxCode}{\begin{tcolorbox}[simplecode]}{\end{tcolorbox}}

\newcolumntype{C}[1]{>{\centering\arraybackslash}p{#1}}

% redefine spaces after titles
\makeatletter
\renewcommand{\@maketitle}{%
  \begin{center}
    {\huge \bfseries \@title \par}%
    \vskip 0.2em % space between title and author
    {\large \@author \par}%
    % \vskip 0.2em % space between author and date
    % {\normalsize \@date \par}%
  \end{center}
}
\makeatother

\fancyhf{} % clear default
\fancypagestyle{plain}{
  \fancyhf{}
  \fancyhead[L]{مدرسة التسامح الشاملة}
  % \fancyhead[L]{\includegraphics[height=1cm]{../../../images/logoTasamoh.png}}
  \fancyhead[R]{الأستاذ محمود اغبارية}
  \fancyfoot[C]{\thepage}
}

\fancyhead[L]{مدرسة التسامح الشاملة}
\fancyhead[R]{الأستاذ محمود اغبارية}
\fancyfoot[C]{\thepage}
% \date{\today}

\setcounter{tocdepth}{3} % only section subsection and subsubsection in TOC


% ----------------------


% \begin{document}

% \maketitle

% % \clearpage  % start TOC on a new page
% % \renewcommand{\contentsname}{جدول المحتويات}
% % \tableofcontents
% % \clearpage

% \part*{part 1} % the * prevents numbering
% \section*{مقدمة}
% \subsection*{مثال رياضي}
% \subsubsection*{مثال فرعي}
% \paragraph*{ paragraph 1}
% \subparagraph*{sub paragraph 1}

% \ifdetailed
% \begin{english}
% \begin{minted}{csharp}
% // C# Example
% \end{minted}
% \end{english}
% \fi

% OLD WAY
% \ifdetailed
% \begin{english}
% \begin{lstlisting}
% // C# Example
% \end{lstlisting}
% \end{english}
% \fi

% % \includegraphics[width=0.2\textwidth]{../../../images/DFAs/ex1_q1.png}



% \vspace{3cm}
% \begin{flushleft}
% أرجو لكم وقتًا ممتعًا.

% الأستاذ محمود اغبارية.
% \end{flushleft}


% \end{document}


\ifwithsols
\title{حل ورقة تمرن 7 - النصوص (Strings)}
\else
\title{ورقة تمرن 7 - النصوص (Strings)}
\fi

\begin{document}

\maketitle
\thispagestyle{fancy}

\begin{enumerate}[itemsep=2em]

    % ------------------------------------------------------
    % 1. [New] Write vs WriteLine (Trace)
    % ------------------------------------------------------
    \item
    ما هو ناتج مقطع الكود التالي؟
    \begin{boxCode}
    \begin{english}
    \begin{minted}{csharp}
Console.Write("A");
Console.WriteLine("B");
Console.Write("C");
    \end{minted}
    \end{english}
    \end{boxCode}
    \ifwithsols
    \begin{boxSolution}
    \begin{english}
    \begin{verbatim}
AB
C
    \end{verbatim}
    \end{english}
    الشرح: "A" تطبع ويبقى المؤشر، "B" تطبع وينزل سطر، "C" تطبع في السطر الجديد.
    \end{boxSolution}
    \fi

    % ------------------------------------------------------
    % 2. [Old] Basics: Length and Indexing
    % ------------------------------------------------------
    \item
    اكتب برنامجاً يستقبل كلمة من المستخدم، ويطبع ما يلي:
    \begin{itemize}
        \item طول الكلمة.
        \item الحرف الأول من الكلمة.
        \item الحرف الأخير من الكلمة.
    \end{itemize}
    \textbf{مثال:} إذا كانت الكلمة \textenglish{"Hello"}، يطبع الطول 5، الحرف الأول \textenglish{'H'}، والحرف الأخير \textenglish{'o'}.

    \ifwithsols
    \begin{boxSolution}
    \begin{english}
    \begin{minted}{csharp}
string s = Console.ReadLine();
Console.WriteLine("Length: " + s.Length);
Console.WriteLine("First: " + s[0]);
Console.WriteLine("Last: " + s[s.Length - 1]);
    \end{minted}
    \end{english}
    \end{boxSolution}
    \fi

    % ------------------------------------------------------
    % 3. [New] Contains (char overload)
    % ------------------------------------------------------
    \item
    اكتب برنامجاً يستقبل كلمة سر، ويفحص هل تحتوي على الرمز \textenglish{'\#'}. \\
    \textbf{مثال:} للمُدخل \textenglish{"pass\#123"} يطبع \textenglish{"Contains hash"}.
    \ifwithsols
    \begin{boxSolution}
    \begin{english}
    \begin{minted}{csharp}
string pass = Console.ReadLine();
if (pass.Contains('#'))
{
    Console.WriteLine("Contains hash");
}
else
{
    Console.WriteLine("No hash");
}
    \end{minted}
    \end{english}
    \end{boxSolution}
    \fi

    % ------------------------------------------------------
    % 4. [Old] Contains (string overload) - Email
    % ------------------------------------------------------
    \item
    اكتب برنامجاً يستقبل عنوان بريد إلكتروني، ويفحص هل يحتوي على \textenglish{"@"} وهل يحتوي على \textenglish{"."}. \\
    \textbf{مثال:} للمدخل \textenglish{"student@school.com"} يطبع \textenglish{"Valid"}.
    \ifwithsols
    \begin{boxSolution}
    \begin{english}
    \begin{minted}{csharp}
string email = Console.ReadLine();
if (email.Contains("@") && email.Contains("."))
    Console.WriteLine("Valid");
else
    Console.WriteLine("Invalid");
    \end{minted}
    \end{english}
    \end{boxSolution}
    \fi

    % ------------------------------------------------------
    % 5. [New] IndexOf (string overload)
    % ------------------------------------------------------
    \item
    اكتب برنامجاً يستقبل جملة، ويطبع الموقع (Index) الذي تبدأ عنده كلمة \textenglish{"cat"}.
    إذا لم تكن موجودة يطبع $-1$. \\
    \textbf{مثال:} للجملة \textenglish{"The cat sat"} يطبع \textenglish{4}.
    \ifwithsols
    \begin{boxSolution}
    \begin{english}
    \begin{minted}{csharp}
string sentence = Console.ReadLine();
int index = sentence.IndexOf("cat");
Console.WriteLine(index);
    \end{minted}
    \end{english}
    \end{boxSolution}
    \fi

    % ------------------------------------------------------
    % 6. [Old] ToLower/ToUpper
    % ------------------------------------------------------
    \item
    اكتب برنامجاً يسأل المستخدم عن عاصمة مصر.
    إذا أجاب \textenglish{"Cairo"} (بأي حالة أحرف، كبيرة أو صغيرة) يطبع \textenglish{"Correct"}. \\
    \textbf{مثال:} إذا أدخل المستخدم \textenglish{"CAIRO"} أو \textenglish{"cairo"} تكون الإجابة صحيحة.
    \ifwithsols
    \begin{boxSolution}
    \begin{english}
    \begin{minted}{csharp}
string ans = Console.ReadLine();
if (ans.ToLower() == "cairo")
    Console.WriteLine("Correct");
else
    Console.WriteLine("Wrong");
    \end{minted}
    \end{english}
    \end{boxSolution}
    \fi

    % ------------------------------------------------------
    % 7. [Old] Substring (Start only - First Half)
    % ------------------------------------------------------
    \item
    اكتب عملية خارجية تتلقى نصاً وتطبع النصف الأول منه. \\
    \textbf{مثال:} إذا تلقت العملية \textenglish{"Computer"}، فإنها تطبع \textenglish{"Comp"}.
    \ifwithsols
    \begin{boxSolution}
    \begin{english}
    \begin{minted}{csharp}
public static void PrintFirstHalf(string text)
{
    Console.WriteLine(text.Substring(0, text.Length / 2));
}
    \end{minted}
    \end{english}
    \end{boxSolution}
    \fi

    % ------------------------------------------------------
    % 8. [New] Substring (Start only - Last 3 chars)
    % ------------------------------------------------------
    \item
    اكتب برنامجاً يستقبل كلمة (طولها 3 على الأقل)، ويطبع آخر 3 حروف منها. \\
    \textbf{مثال:} للكلمة \textenglish{"School"} يطبع \textenglish{"ool"}.
    \ifwithsols
    \begin{boxSolution}
    \begin{english}
    \begin{minted}{csharp}
string s = Console.ReadLine();
if (s.Length >= 3)
{
    string last3 = s.Substring(s.Length - 3);
    Console.WriteLine(last3);
}
    \end{minted}
    \end{english}
    \end{boxSolution}
    \fi

    % ------------------------------------------------------
    % 9. [New] Substring (Start, Length)
    % ------------------------------------------------------
    \item
    اكتب برنامجاً يستقبل رقم هاتف بصيغة \textenglish{050-1234567}.
    على البرنامج استخراج وطباعة مقدمة الهاتف (أول 3 أرقام). \\
    \textbf{مثال:} للمدخل \textenglish{052-9999999} يطبع \textenglish{052}.
    \ifwithsols
    \begin{boxSolution}
    \begin{english}
    \begin{minted}{csharp}
string phone = Console.ReadLine();
string prefix = phone.Substring(0, 3);
Console.WriteLine(prefix);
    \end{minted}
    \end{english}
    \end{boxSolution}
    \fi

    % ------------------------------------------------------
    % 11. [Old] Loop: Count specific char
    % ------------------------------------------------------
    \item
    اكتب عملية خارجية تتلقى نصاً وحرفاً (\textenglish{char})، وتعيد عدد مرات ظهور هذا الحرف داخل النص. \\
    \textbf{مثال:} إذا تلقت العملية النص \textenglish{"banana"} والحرف \textenglish{'a'}، فإنها تعيد 3.
    \ifwithsols
    \begin{boxSolution}
    \begin{english}
    \begin{minted}{csharp}
public static int CountChar(string text, char c)
{
    int count = 0;
    for (int i = 0; i < text.Length; i++)
    {
        if (text[i] == c)
            count++;
    }
    return count;
}
    \end{minted}
    \end{english}
    \end{boxSolution}
    \fi

    % ------------------------------------------------------
    % 12. [Old] Parsing (IndexOf + Substring) - Challenge
    % ------------------------------------------------------
    \item
    اكتب برنامجاً يستقبل الاسم الكامل (الاسم الشخصي واسم العائلة بينهما مسافة)، ويطبع "أهلاً" مع الاسم الشخصي فقط. \\
    \textbf{مثال:} للمدخل \textenglish{"Majd Ahmad"} يطبع \textenglish{"Hello Majd"}.
    \ifwithsols
    \begin{boxSolution}
    \begin{english}
    \begin{minted}{csharp}
string fullName = Console.ReadLine();
int spaceIndex = fullName.IndexOf(" ");
if (spaceIndex != -1)
{
    string firstName = fullName.Substring(0, spaceIndex);
    Console.WriteLine("Hello " + firstName);
}
    \end{minted}
    \end{english}
    \end{boxSolution}
    \fi
\end{enumerate}

\clearpage

\subsection*{سؤال 6 امتحان 899222 سنة 2015}
\addcontentsline{toc}{subsection}{سؤال 6 امتحان 899222 سنة 2015}

\importpdfpage{ ../../../bagrut_questions/basics/loops_for_2015_899222_6.pdf }{1}

\clearpage
\ifwithsols
\begin{boxSolution}[سؤال 6 امتحان 899222 سنة 2015]
\begin{enumerate}[itemsep=1.5em, label=\alph*.]
    \item // TODO:

\end{enumerate}

\begin{boxCode}
\begin{english}
\begin{minted}{csharp}
    // TODO
\end{minted}
\end{english}
\end{boxCode}

\end{boxSolution}
\clearpage
\fi

\importpdfpage{../../../bagrut_questions/basics/loops_while_2011_899222_6.pdf}{1}

\ifwithsols

\paragraph*{حل سؤال 6 امتحان 899222 سنة 2011}:
\bigskip

% // TODO:

\fi


\end{document}
