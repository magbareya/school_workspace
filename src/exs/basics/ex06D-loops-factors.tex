\documentclass[14pt]{extarticle}
% Full article preamble (duplicated, no common file)
\usepackage{fontspec}
\usepackage[a4paper,top=2.4cm,bottom=2.4cm,left=2.3cm,right=2.3cm]{geometry}
\usepackage{polyglossia}
\usepackage{amsmath}
\usepackage{amssymb}
\usepackage{xcolor}
\usepackage{fancyhdr}
\usepackage{graphicx}
\usepackage{listings}
\usepackage[most]{tcolorbox}
\usepackage{pifont}
\usepackage{enumitem}
\usepackage{titlesec}
\usepackage[bottom]{footmisc}
\usepackage{titling}
\usepackage{minted}
\usepackage{etoolbox}
\usepackage{array}
\usepackage{extsizes}

\newfontfamily\emoji{Segoe UI Emoji}

\pagestyle{fancy}

\setmainlanguage[numerals=western]{arabic}
\setotherlanguage{english}
\newfontfamily\arabicfont[Script=Arabic]{Amiri}
\newfontfamily\arabicfonttt[Script=Arabic]{Courier New}

\lstset{
  language=[Sharp]C,
  numbers=left,
  stepnumber=1,
  numbersep=8pt,
  frame=single,
  basicstyle=\ttfamily\small,
  keywordstyle=\color{blue},
  stringstyle=\color{red},
  commentstyle=\color{green!50!black}
}

\newif\ifdetailed
\ifdefined\setdetailed
  \setdetailed
\fi

\newif\ifwithsols
\ifdefined\setwithsols
  \setwithsols
\fi

% unified tcolorboxes for articles
\tcbset{colback=white, colframe=black, fonttitle=\bfseries, boxrule=0.8pt}
\newtcolorbox{boxDef}[1][]{colback=blue!5!white,colframe=blue!75!black,
  title={{\emoji📘} تعريف\ifx\\#1\\\else ~#1\fi :}}
\newtcolorbox{boxExercise}[1][]{colback=cyan!5!white,colframe=cyan!70!black,
  title={{\emoji🧩} تمرين\ifx\\#1\\\else ~#1\fi :}}
\newtcolorbox{boxExample}[1][]{colback=yellow!5!white,colframe=orange!90!black,
  title={{\emoji📝} مثال\ifx\\#1\\\else ~#1\fi :}}
\newtcolorbox{boxNote}[1][]{colback=gray!10!white,colframe=black,
  title={{\emoji✨} ملاحظة\ifx\\#1\\\else ~#1\fi :}}
\newtcolorbox{boxAttention}[1][]{colback=magenta!10!white,colframe=magenta!80!black,
  title={{\emoji🔔} تنبيه\ifx\\#1\\\else ~#1\fi :}}
\newtcolorbox{boxWarning}[1][]{colback=red!5!white,colframe=red!75!black,
  title={{\emoji⚡} ملاحظة هامة\ifx\\#1\\\else ~#1\fi :}}
\newtcolorbox{boxSolution}[1][]{colback=green!5!white,colframe=green!60!black,
  title={{\emoji✅} حل\ifx\\#1\\\else ~#1\fi :}}
\newtcolorbox{boxSymbol}[1][]{colback=purple!5!white,colframe=purple!70!black,
  title={{\emoji🔣} رمز\ifx\\#1\\\else ~#1\fi :}}
\newtcolorbox{boxHint}[1][]{colback=teal!5!white,colframe=teal!60!black,
  title={{\emoji💡} تلميح\ifx\\#1\\\else ~#1\fi :}}


\tcbset{simplecode/.style={ colback=gray!5, colframe=black!50, boxrule=0.4pt, arc=2pt, left=4pt,right=4pt,top=4pt,bottom=4pt}}
\newenvironment{boxCode}{\begin{tcolorbox}[simplecode]}{\end{tcolorbox}}

\newcolumntype{C}[1]{>{\centering\arraybackslash}p{#1}}

% redefine spaces after titles
\makeatletter
\renewcommand{\@maketitle}{%
  \begin{center}
    {\huge \bfseries \@title \par}%
    \vskip 0.2em % space between title and author
    {\large \@author \par}%
    % \vskip 0.2em % space between author and date
    % {\normalsize \@date \par}%
  \end{center}
}
\makeatother

\fancyhf{} % clear default
\fancypagestyle{plain}{
  \fancyhf{}
  \fancyhead[L]{مدرسة التسامح الشاملة}
  % \fancyhead[L]{\includegraphics[height=1cm]{../../../images/logoTasamoh.png}}
  \fancyhead[R]{الأستاذ محمود اغبارية}
  \fancyfoot[C]{\thepage}
}

\fancyhead[L]{مدرسة التسامح الشاملة}
\fancyhead[R]{الأستاذ محمود اغبارية}
\fancyfoot[C]{\thepage}
% \date{\today}

\setcounter{tocdepth}{3} % only section subsection and subsubsection in TOC


% ----------------------


% \begin{document}

% \maketitle

% % \clearpage  % start TOC on a new page
% % \renewcommand{\contentsname}{جدول المحتويات}
% % \tableofcontents
% % \clearpage

% \part*{part 1} % the * prevents numbering
% \section*{مقدمة}
% \subsection*{مثال رياضي}
% \subsubsection*{مثال فرعي}
% \paragraph*{ paragraph 1}
% \subparagraph*{sub paragraph 1}

% \ifdetailed
% \begin{english}
% \begin{minted}{csharp}
% // C# Example
% \end{minted}
% \end{english}
% \fi

% OLD WAY
% \ifdetailed
% \begin{english}
% \begin{lstlisting}
% // C# Example
% \end{lstlisting}
% \end{english}
% \fi

% % \includegraphics[width=0.2\textwidth]{../../../images/DFAs/ex1_q1.png}



% \vspace{3cm}
% \begin{flushleft}
% أرجو لكم وقتًا ممتعًا.

% الأستاذ محمود اغبارية.
% \end{flushleft}


% \end{document}


\ifwithsols
\title{حل ورقة تمرن 6 للصف العاشر 10 - قسم 4 \\ أسئلة تحليل العدد إلى عوامله}
\else
\title{ورقة تمرن 6 للصف العاشر 10 - قسم 4 \\ أسئلة تحليل العدد إلى عوامله}
\fi

\begin{document}

\maketitle
\thispagestyle{fancy}

\begin{enumerate}[itemsep=1.5em]

\item اكتب برنامجًا يستقبل من المستخدم عددًا صحيحا، ويطبع له كل عوامل هذا العدد (أي كل الأعداد التي يقسم عليها هذا العدد).
\ifwithsols
\begin{boxSolution}
\begin{english}
\begin{minted}{csharp}
int n = int.Parse(Console.ReadLine());
for (int d = 1; d <= n; d++)
{
    if (n % d == 0)
        Console.WriteLine(d);
}
\end{minted}
\end{english}
\end{boxSolution}
\fi

\item \textbf{للتذكير} عدد أوّلي هو عدد يقسم فقط على نفسه وعلى 1.
    \begin{enumerate}
        \item اكتب عمليّة خارجية باسم \textenglish{IsPrime} تتلقى عددًا صحيحًا وتعيد \textenglish{true} إذا كان هذا العدد أوليًّا، وإلا فإنّها تعيد \textenglish{false}.
        \item في البرنامج الرئيسي، استدعِ العملية مع العدد 17 واطبع النتيجة. ومع العدد 20 واطبع النتيجة.
        \item اكتب عملية خارجية أخرى باسم \textenglish{PrintPrimes} تتلقى عددًا صحيحًا وتطبع كل الأعداد الأولية حتى هذا العدد. \\
                 \textbf{مثلًا:} إذا تلقت العملية العدد 20 فإنّها تطبع: 2، 3، 5، 7، 11، 13، 17، 19.
        \item في البرنامج الرئيسي استدعِ العملية مع الرقم 100.
    \end{enumerate}
\ifwithsols
\begin{boxSolution}
\begin{english}
\begin{minted}{csharp}
public static bool IsPrime(int n)
{
    if (n < 2)
        return false;
    for (int d = 2; d * d <= n; d++)
    {
        if (n % d == 0)
            return false;
    }
    return true;
}

public static void PrintPrimes(int n)
{
    for (int i = 2; i <= n; i++)
    {
        if (IsPrime(i))
            Console.WriteLine(i);
    }
}

private static void Main(string[] args)
{
    Console.WriteLine(IsPrime(17));
    Console.WriteLine(IsPrime(20));
    PrintPrimes(100);
}
\end{minted}
\end{english}
\end{boxSolution}
\fi

\item اكتب عملية تتلقى عددًا صحيحًا، وتطبع كل عوامله \textbf{الأوّلية}. (أي كل الأعداد الأولية التي يقسم عليها هذا العدد) \\
يمكنك الاستعانة بالعملية \textenglish{IsPrime} التي كتبتها في السؤال السابق.

\item اكتب عملية خارجية باسم \textenglish{IsPrimeSquare} تتلقى عددًا صحيحًا وتعيد \textenglish{true} إذا كان العدد الذي تلقته عبارة عن تربيع عدد أوّلي. وإلا فإنّها تعيد \textenglish{false} \\
\textbf{مثلًا:} تعيد العملية \textenglish{true} إذا تلقّت 25. وتعيد \textenglish{false} إذا تلقت 30.

\end{enumerate}

\vspace{1cm}
\begin{flushleft}
أرجو لكم وقتًا ممتعًا.

الأستاذ محمود اغبارية.
\end{flushleft}

\end{document}
