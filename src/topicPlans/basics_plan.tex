\documentclass[12pt]{article}
\usepackage{fontspec}
\usepackage{polyglossia}
\usepackage{amsmath}
\usepackage{amssymb}
\usepackage{xcolor}
\usepackage{fancyhdr}
\usepackage{graphicx}
\usepackage{listings}
\usepackage{geometry}

\pagestyle{fancy}

\setmainlanguage{arabic}
\setotherlanguage{english}
\newfontfamily\arabicfont[Script=Arabic]{Amiri}


\lstset{
  language=[Sharp]C,
  numbers=left,
  stepnumber=1,
  numbersep=8pt,
  frame=single,
  basicstyle=\ttfamily\small,
  keywordstyle=\color{blue},
  stringstyle=\color{red},
  commentstyle=\color{green!50!black}
}

\newif\ifdetailed

\geometry{a4paper, margin=2.5cm}

\fancyhf{} % clear default
\fancypagestyle{plain}{
  \fancyhf{}
  \fancyhead[L]{مدرسة التسامح الشاملة}
  \fancyhead[R]{الأستاذ محمود اغبارية}
  \fancyfoot[C]{\thepage}
}
\title{قائمة مواد أساسيات علوم الحاسوب بلغة C\# حسب خطة وزارة المعارف}
\fancyhead[L]{مدرسة التسامح الشاملة}
\fancyhead[R]{الأستاذ محمود اغبارية}
\fancyfoot[C]{\thepage}

\begin{document}
\maketitle
\renewcommand{\contentsname}{جدول المحتويات}
\tableofcontents
\clearpage

\section{المقدمة ومصطلحات أساسية}

\subsection{مقدمة عن الخوارزميات والبرمجة}

انظر عرض "مقدمة" الذي أعددته.

بعده نبدأ بالبرمجة.

\begin{itemize}
\item البدء خطوة بخطوة إلى أن ينشئ الطالب مشروعًا جديدًا في Visual Studio
\item التعريف بمبنى المشروع العام
\item التعريف بمبنى الملف $\mathtt{Program.cs}$ العام
\end{itemize}

\subsection{كتابة إلى الشاشة، وقراءة من المستخدم}

\begin{itemize}
    \item نشرح بأقل الممكن طريقة طباعة متغير على الشاشة أو أي نص
    \item نشرح بأقل الممكن طريقة استقبال عدد من المستخدم
\end{itemize}

\section{المتغيرات}

\begin{enumerate}
    \item  المتغيرات: تعريفها وإعطاؤها قيمة (الاسم والنوع لا يتغيران، القيمة تتغير)
    \item  الأنواع الأساسية للمتغيرات،
    \item  \textit{قوانين أسماء المتغيرات المسموحة}
    \item  \textbf{أهمية اختيار اسم معبّر للمتغير}
    \item  خاصة: $\mathtt{int}$ و $\mathtt{double}$ ومجالات قيم كل نوع (القيم بالبرمجة محدودة وليس كما في العالم غير محدودة)
    \item  العمليات الحسابية على المتغيرات:
    \begin{itemize}
        \item جمع
        \item طرح
        \item ضرب
        \item قسمة
        \item باقي
        \item ترتيب العمليات الحسابية (مع أقواس وبدون)
    \end{itemize}
    \item  تحويل بين الأنواع  $\mathtt{int}$ و $\mathtt{double}$:
    \item تحويل صريح
    \item تحويل ضمني (عند قسمة $\mathtt{int}$ على $\mathtt{double}$ مثلا)
    \item  تعريف المتغير وإعطاؤه قيمة في نفس السطر أو في سطرين مختلفين.
    \item  إعطاء قيمة للمتغير هو نسخ وليس نقل للقيمة
\end{enumerate}

\section{مقدمة قصيرة عن الفئات}

\subsection{الفئات}
\begin{itemize}
    \item تعريف المصطلحات: فئة، كائن، خاصية، عملية
    \item قراءة ملف فئة وفهمه
    \item إنشاء كائن من فئة معينة
    \item استدعاء عمليات فئة معينة
\end{itemize}


\subsection{استدعاء عمليات}

استدعاء عمليات $\mathtt{static}$ من فئة خارجية: $\mathtt{ClassName.MethodName()}$ بأقل تفصيل ممكن (التعامل مع $\mathtt{Math}$ كمجموعة عمليات كتبها شخص آخر ونحن نستدعيها.)

\subsection{عمليات الفئة $\mathtt{Math}$}
\begin{itemize}
    \item $\mathtt{Math.Abs}$
    \item $\mathtt{Math.Max}$
    \item $\mathtt{Math.Min}$
    \item $\mathtt{Math.Pow}$
    \item $\mathtt{Math.Round}$
    \item $\mathtt{Math.Sqrt}$
\end{itemize}

\subsection{الفئة $\mathtt{Random}$ واستخداماتها}

//TODO

\subsection{قراءة ملف فئة وفهمه}
\begin{itemize}
    \item قراء ملف فئة وفهمه
    \item قراءة برنامج يستخدم هذه الفئة وفهمه
    \item تغيير هذا البرنامج للحصول على النتيجة المطلوبة
\end{itemize}

\subsection{التعامل مع متغير النص $\mathtt{String}$}

\begin{itemize}
    \item ربط نصين ببعضهما
    \item تنسيق النص مع متغيرات $\mathtt{\$}$
    \item $\mathtt{int.Parse(Console.ReadLine());}$
    \item $\mathtt{double.Parse(Console.ReadLine());}$
    \item \textbf{موضوع $\mathtt{String}$ ليس مطلوبًا لأول وحدتين، المطلوب فقط التعامل الأساسي معه لأجل الطباعة واستقبال المتغيرات}
\end{itemize}

\subsection{العمليات الخارجية}


\begin{itemize}
    \item مثال يبين أهمية تقسيم البرنامج إلى عمليات صغيرة:
    \begin{itemize}
        \item وجود الأخطاء أسرع
        \item فحص الكود (testing) أسهل
        \item يمنع من تكرار الكود، ويمكننا من إعادة استخدامه
        \item أسهل للقراءة
        \item حل المشكلة يصبح أسهل
    \end{itemize}
    \item مبنى تعريف العملية واستدعاؤها:
    \begin{itemize}
        \item ختم العملية
        \item القيمة المرجعة
        \item جسم العملية (بين أقواس  مجعّدة)
    \end{itemize}
    \item تعريف عمليات خاصة ($\mathtt{private}$):
    \begin{itemize}
        \item لا تستقبل متغيرات ولا ترجع قيمة (تحسب شيئا وتطبعه مثلا)
        \item استقبال متغيرات في العملية الخارجية
        \item إرجاع قيمة في العملية الخارجية
        \item استعداء العمليات مع قيمة المتغير \textenglish{\texttt{Call by value}}، توضيح لا علاقة بين $\mathtt{x}$ في $\mathtt{Main}$ و $\mathtt{x}$ في العملية الخارجية
        \item نحن نجري العمليات على \textbf{قيمة} المتغير وليس على المتغير نفسه.
    \end{itemize}
    \item تمرين الطالب على ترتيب برنامج وتقسيمه لعمليات صغيرة
    \item شرح دمج القيمة المرجعة من العملية في تعبير رياضي:
    \begin{itemize}
        \item مثال لتعبير حسابي أحد حدوده قيمة مرجعة من عملية
        \item مثال لاستدعاء عملية وأحد البرامترات هو تعبير حسابي
        \item مثال لاستدعاء عملية وأحد البرامترات هو القيمة المرجعة من عملية أخرى (inline)
    \end{itemize}
\end{itemize}

\subsection{جدول المتابعة}

\begin{itemize}
    \item الجدول يحتوي على عمود لكل متغير نريد متابعته.
    \item يحتوي أيضًا على عمود للمخرجات
    \item يحتوي على عمود للشروط (عندما نتعلمها)
    \item إذا كان أحد المتغيرات هو كائن، فالعمود الخاص به يشير إلى ما يبيّن قيم خصائصه.
\end{itemize}

\subsection{معالجة أخطاء الكتابة، التشغيل، أخطاء منطقية}

برأيي هذا حسب الحاجة، ربما نكون قد غطيناه خلال العمل السابق.

\section{الشرط}

\section{الحلقات}

\section{مباني معطيات بسيطة}

\section{كائنات وفئات}

\end{document}
