\documentclass[12pt]{article}
\usepackage{fontspec}
\usepackage{polyglossia}
\usepackage{amsmath}
\usepackage{amssymb}
\usepackage{xcolor}
\usepackage{fancyhdr}
\usepackage{graphicx}
\usepackage{listings}
\usepackage{geometry}

\pagestyle{fancy}

\setmainlanguage{arabic}
\setotherlanguage{english}
\newfontfamily\arabicfont[Script=Arabic]{Amiri}


\lstset{
  language=[Sharp]C,
  numbers=left,
  stepnumber=1,
  numbersep=8pt,
  frame=single,
  basicstyle=\ttfamily\small,
  keywordstyle=\color{blue},
  stringstyle=\color{red},
  commentstyle=\color{green!50!black}
}

\newif\ifwithcode

\geometry{a4paper, margin=2.5cm}

\fancyhf{} % clear default
\fancypagestyle{plain}{
  \fancyhf{}
  \fancyhead[L]{مدرسة التسامح الشاملة}
  \fancyhead[R]{الأستاذ محمود اغبارية}
  \fancyfoot[C]{\thepage}
}
\title{قائمة مواد أساسيات علوم الحاسوب بلغة C\# حسب خطة وزارة المعارف}
\fancyhead[L]{مدرسة التسامح الشاملة}
\fancyhead[R]{الأستاذ محمود اغبارية}
\fancyfoot[C]{\thepage}

\begin{document}
\maketitle
\renewcommand{\contentsname}{جدول المحتويات}
\tableofcontents
\clearpage

\section{مقدّمة}

\begin{itemize}
    \item موضوع النماذج الحسابية هو موضوع نظري، لا يوجد فيه برمجة.
    \item هدف الموضوع هو وصف الحاسوب بطريقة رياضية حتى نستطيع فحص حدود إمكانياته.
    \item هل يستطيع الحاسوب حل أي مشكلة حسابية؟ (الجواب لا طبعا)
    \item قبل أن نصل إلى نموذج الحاسوب (ماكنة تيورنج) نتعرف على نماذج أضعف من الحاسوب (قدرتها الحسابية أقل من قدرة الحاسوب)
\end{itemize}

\end{document}
